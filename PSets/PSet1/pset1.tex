\documentclass[../psets.tex]{subfiles}

\pagestyle{main}
\renewcommand{\leftmark}{Problem Set \thesection}

\begin{document}




\section{Rings, Subrings, and Ring Homomorphisms}
\begin{enumerate}
    \item \marginnote{1/11:}Let $R$ be a ring with identity. Show that $R$ is a singleton if and only if $0_R=1_R$.
    \begin{proof}
        Suppose first that $R$ is a singleton. Let $x\in R$ be the sole element in $R$. Since $(R,+)$ is a group (necessarily the trivial group due to order), we know that $x=0_R$. Since $R$ is a ring with identity, $x$ must be said identity, i.e., we know that $x=1_R$. Therefore, by transitivity, $0_R=1_R$, as desired.\par
        Now suppose that $0_R=1_R$. Pick $x,y\in R$ arbitrary. Then we have that
        \begin{equation*}
            x = 1_R\times x
            = 0_R\times x
            = 0_R
        \end{equation*}
        and the same for $y$. Thus, by transitivity, $x=y$. Since any two elements of $R$ are equal, $R$ must be a singleton, as desired.
    \end{proof}
\end{enumerate}


\subsection*{Products}
\begin{enumerate}[resume]
    \item Let $X,Y$ be sets and let $R$ be a ring. Recall that pointwise addition and multiplication turns $R^X$ and $R^Y$ into rings. Let $f:X\to Y$ be a function. Define $f^*:R^Y\to R^X$ by $f^*(g)=g\circ f$ for all $g:Y\to R$. Prove that $f^*$ is a ring homomorphism.
    \begin{proof}
        To prove that $f^*$ is a ring homomorphism, it will suffice to check that $f^*(g_1+g_2)=f^*(g_1)+f^*(g_2)$ and $f^*(g_1\times g_2)=f^*(g_1)\times f^*(g_2)$ for all $g_1,g_2\in R^Y$, and $f^*(1_{R^Y})=1_{R^X}$. Let's begin.\par\smallskip
        Let $g_1,g_2\in R^Y$ be arbitrary. Then we have for any $x\in X$ that
        \begin{align*}
            [f^*(g_1+g_2)](x) &= [(g_1+g_2)\circ f](x)\\
            &= (g_1+g_2)(f(x))\\
            &= g_1(f(x))+g_2(f(x))\\
            &= (g_1\circ f)(x)+(g_2\circ f)(x)\\
            &= [f^*(g_1)](x)+[f^*(g_2)](x)\\
            &= [f^*(g_1)+f^*(g_2)](x)
        \end{align*}
        as desired.\par
        Let $g_1,g_2\in R^Y$ be arbitrary. Then we have for any $x\in X$ that
        \begin{align*}
            [f^*(g_1\times g_2)](x) &= [(g_1\times g_2)\circ f](x)\\
            &= (g_1\times g_2)(f(x))\\
            &= g_1(f(x))\times g_2(f(x))\\
            &= (g_1\circ f)(x)\times(g_2\circ f)(x)\\
            &= [f^*(g_1)](x)\times[f^*(g_2)](x)\\
            &= [f^*(g_1)\times f^*(g_2)](x)
        \end{align*}
        as desired.\par
        Let $1_{R^Y}:Y\to R$ denote the identity of $R^Y$, that is, the constant function evaluating to $1_R$ at every $y\in Y$. Then for any $x\in X$,
        \begin{equation*}
            [f^*(1_{R^Y})](x) = (1_{R^Y}\circ f)(x)
            = 1_{R^Y}(f(x))
            = 1_R
        \end{equation*}
        where the last equality holds by the definition of $1_{R^Y}$ since $f(x)\in Y$. Thus, since $f^*(1_{R^Y}):X\to R$ sends every $x\in X$ to $1_R$, it must be equal to $1_{R^X}$ by the definition of the latter, as desired.
    \end{proof}
    \item Let $Y\subset X$. Define $\phi:R^Y\to R^X$ by the following rule: For any function $g:Y\to R\in R^Y$, let $\phi(g):X\to R$ send
    \begin{equation*}
        x \mapsto
        \begin{cases}
            g(x) & x\in Y\\
            0 & x\notin Y
        \end{cases}
    \end{equation*}
    State whether the assertions (i) and (ii) below are \emph{true} or \emph{false}. No proof required.\par
    \emph{Warning}: Make sure to use the definitions of "ring homomorphism" and "subring" from class!
    \begin{enumerate}
        \item $\phi$ is a ring homomorphism.
        \begin{proof}[Answer]
            False\footnote{$\phi(1_{R^Y})\neq 1_{R^X}$ if $Y\subsetneq X$.}.
        \end{proof}
        \item The image of $\phi$ is a subring of $R^X$.
        \begin{proof}[Answer]
            False\footnote{$\phi(R^Y)$ does not contain an identity unless $Y=X$.}.
        \end{proof}
    \end{enumerate}
    \item For any ring $R$, define the set $\Delta(R)$ by
    \begin{equation*}
        \Delta(R) = \{(a,a):a\in R\}
    \end{equation*}
    Note that $\Delta(R)$ is a subring of $R\times R$. Prove that if $B$ is a subring of $\Q\times\Q$ that contains $\Delta(\Q)$, then $B$ is either $\Delta(\Q)$ or $\Q\times\Q$.
    \begin{proof}
        % If $X\in\Delta(\Q)$, then $\Delta(\Q)[X]=\Delta(\Q)$. If $X\notin\Delta(\Q)$, then $\Delta(\Q)...$

        % $\Q\to\Q\times\Q$ is a ring homomorphism.


        We divide into two cases ($B=\Delta(\Q)$ and $B\neq\Delta(\Q)$). In the first case, we are immediately done. In the second case, start with the observation that if $\Delta(\Q)\subsetneq B$, then there exists $x\in B$ such that $x\notin\Delta(\Q)$. It follows from class that the smallest subring of $\Q\times\Q$ containing $\Delta(\Q)$ and $x\notin\Delta(\Q)$ is $\Delta(\Q)[x]$. Thus, showing that $\Delta(\Q)[x]=\Q\times\Q$ will complete the proof.\par
        We proceed via a bidirectional inclusion proof. Suppose first that $p\in\Delta(\Q)[x]$. Each term $a_ix^i$ in $p$ is the finite product of elements of $\Q\times\Q$, and thus is an element of $\Q\times\Q$ itself (since $\Q\times\Q$ is a closed ring). It follows that $p$ is the finite sum of elements of $\Q\times\Q$ and hence is also an element of $\Q\times\Q$, as desired. Now suppose that $(q_1,q_2)\in\Q\times\Q$. Let $x=(x_1,x_2)$. Then\footnote{Derivation: Solve $(a,a)+(b,b)(x_1,x_2)=(q_1,q_2)$. Geometrically, this problem is equivalent to identifying $\Delta(\Q)$ with the subspace $y=x$ of $\R^2$ and noting that we only need one additional linearly independent element $(x_1,x_2)$ where $x_1\neq x_2$ to allow us to reach every other point in $\R^2$.}
        \begin{align*}
            (q_1,q_2) &= \left( \frac{q_2x_1-q_1x_2}{x_1-x_2}+\frac{q_1-q_2}{x_1-x_2}\cdot x_1,\frac{q_2x_1-q_1x_2}{x_1-x_2}+\frac{q_1-q_2}{x_1-x_2}\cdot x_2 \right)\\
            &= \underbrace{\left( \frac{q_2x_1-q_1x_2}{x_1-x_2},\frac{q_2x_1-q_1x_2}{x_1-x_2} \right)}_{a_0}+\underbrace{\left( \frac{q_1-q_2}{x_1-x_2},\frac{q_1-q_2}{x_1-x_2} \right)}_{a_1}\vphantom{)}\cdot(x_1,x_2)\\
            &\in \Delta(\Q)[x]
        \end{align*}
        as desired. Note that $a_0,a_1$ defined above are elements of $\Delta(\Q)$ since $x_1-x_2\neq 0$ by hypothesis for this element not in $\Delta(\Q)$.
    \end{proof}
\end{enumerate}


\subsection*{Basic Properties}
\begin{enumerate}[start=7]
    \item Let $f:R_1\to R_2$ be a ring homomorphism, and let $R_3$ be a subring of $R_2$. Prove that $f^{-1}(R_3)$ is a subring of $R_1$.
    \begin{proof}
        To prove that $f^{-1}(R_3)\subset R_1$ is a subring, it will suffice to show that it is closed under addition, multiplication, and additive inverses, and that $1_{R_1}\in f^{-1}(R_3)$. Let's begin.\par\smallskip
        Let $a,b\in f^{-1}(R_3)$ be arbitrary. Then $f(a),f(b)\in R_3$. It follows that $f(a)+f(b)\in R_3$, hence $f(a+b)\in R_3$ since $f(a+b)=f(a)+f(b)$. Therefore, $a+b\in f^{-1}(R_3)$, as desired.\par
        An analogous argument holds for closure under multiplication.\par
        Let $a\in f^{-1}(R_3)$ be arbitrary. Then $f(a)\in R_3$. It follows that $-f(a)\in R_3$, hence $f(-a)\in R_3$ since $f:(R_1,+)\to(R_2,+)$ being a group homomorphism means that
        \begin{align*}
            f(0) &= 0\\
            f(a+(-a)) &= 0\\
            f(a)+f(-a) &= 0\\
            -f(a)+f(a)+f(-a) &= -f(a)+0\\
            f(-a) &= -f(a)
        \end{align*}
        Therefore, $-a\in f^{-1}(R_3)$, as desired.\par
        Since $f$ is a ring homomorphism, $f(1_{R_1})=1_{R_2}$. Since $R_3$ is a subring of $R_2$, $1_{R_2}\in R_3$. Therefore, $1_{R_1}\in f^{-1}(R_3)$, as desired.
    \end{proof}
    \stepcounter{enumi}
    \item Show that $A\cap B$ is a subring of $R$ if both $A,B$ are subrings of $R$.
    \begin{proof}
        Suppose $A,B\subset R$ are subrings. To prove that $A\cap B$ is a subring, it will suffice to show that it is closed under addition, multiplication, and additive inverses, and that $1_R\in A\cap B$. Let's begin.\par\smallskip
        Let $a,b\in A\cap B$ be arbitrary. Then $a,b\in A$ and $a,b\in B$. It follows from the closure of $A$ under addition (resp. multiplication, additive inverses) that $a+b,ab,-a\in A$. Analogously, $a+b,ab,-a\in B$. Therefore, $a+b,ab,-a\in A\cap B$, as desired.\par
        Since $A,B$ are subrings, $1_R\in A,B$. Therefore, $1_R\in A\cap B$, as desired.
    \end{proof}
\end{enumerate}
Recall the following lemma from MATH 25700: Let $(A,+)$ be an abelian group, and let $a\in A$. Then there is a unique group homomorphism $f:\Z\to A$ such that $f(1)=a$. Additionally, $f(n)=na$ for all $n\in\Z$.
\begin{enumerate}[resume]
    \item Let $1_R$ denote the multiplicative identity of a ring $R$. The above lemma then defines $na\in R$ for every $a\in R$ and $n\in\Z$. In particular, we define $n_R=n(1_R)$ for every integer $n\in\Z$. Prove that $n_R\cdot a=na$ for every $a\in R$ and $n\in\Z$.
    \begin{proof}
        % Lemma: There is a unique group homomorphism $f:\Z\to R$ such that $f(1)=1_R$. Additionally, $f(n)=n(1_R)...$


        Let $a\in R$ and $n\in\Z$ be arbitrary. We divide into three cases ($n>0$, $n=0$, and $n<0$). If $n>0$, then we have by iterating the distributive law that
        \begin{equation*}
            n_R\cdot a = (\underbrace{1_R+\cdots+1_R}_{n\text{ times}})\cdot a
            = \underbrace{(1_R\cdot a)+\cdots+(1_R\cdot a)}_{n\text{ times}}
            = \underbrace{a+\cdots+a}_{n\text{ times}}
            = na
        \end{equation*}
        as desired. If $n=0$, then $n_R=0(1_R)=0_R$. Thus,
        \begin{equation*}
            n_R\cdot a = 0_R\cdot a = 0 = 0a = na
        \end{equation*}
        as desired. If $n<0$, then $n_R=-1\cdot(-n_R)$, where $-n_R>0$. Thus, apply case 1 and factor the $-1$ back in at the end.
    \end{proof}
    \item With notation as above, show that $f:\Z\to R$ given by $f(n)=n_R$ is a ring homomorphism.
    \begin{proof}
        To prove that $f$ is a ring homomorphism, it will suffice to check that $f(n+m)=f(n)+f(m)$ and $f(nm)=f(n)f(m)$ for all $n,m\in\Z$, and $f(1)=1_R$. Let's begin.\par
        Let $n,m\in\Z$ be arbitrary. Then
        \begin{align*}
            f(n+m) &= (n+m)_R\\
            &= (n+m)\cdot 1_R\\
            &= \underbrace{1_R+\cdots+1_R}_{n+m\text{ times}}\\
            &= \underbrace{1_R+\cdots+1_R}_{n\text{ times}}+\underbrace{1_R+\cdots+1_R}_{m\text{ times}}\\
            &= n(1_R)+m(1_R)\\
            &= n_R+m_R\\
            &= f(n)+f(m)
        \end{align*}
        as desired. Note that this only treats the case $n,m>0$; all other would have to be addressed in extended casework, similar to what was done in Exercise 1.10.\par
        Let $n,m\in\Z$ be arbitrary. Then
        \begin{align*}
            f(nm) &= (nm)_R\\
            &= (nm)\cdot 1_R\\
            &= \sum_{i=1}^{nm}1_R\\
            &= \sum_{i=1}^n\sum_{i=1}^m1_R\\
            &= \sum_{i=1}^nm(1_R)\\
            &= n\cdot m(1_R)\\
            &= n_R\cdot m(1_R)\tag*{Problem 1.10}\\
            &= n_R\cdot m_R\\
            &= f(n)f(m)
        \end{align*}
        as desired. Same as before with the extra casework for negative numbers\footnote{For full credit in this problem, I would have to show more of this casework.}.\par
        By definition, $f$ is the unique homomorphism sending $1\mapsto 1_R$, as desired.
    \end{proof}
\end{enumerate}
The commutativity of a ring is required for all the identities of high school algebra. The next two problems (1.12 and 1.13) are instances.
\begin{enumerate}[resume]
    \item Prove that the following are equivalent.
    \begin{enumerate}
        \item $R$ is a commutative ring.
        \item $(a+b)(a-b)=a^2-b^2$ for all $a,b\in R$.
        \item $(a+b)^2=a^2+2ab+b^2$ for all $a,b\in R$.
    \end{enumerate}
    \begin{proof}
        {\color{white}hi}\\
        \underline{$\text{(i)}\Rightarrow\text{(ii)}$}: Suppose $R$ is a commutative ring, and let $a,b\in R$ be arbitrary. Then by the ring axioms (e.g., distributive law, etc.),
        \begin{equation*}
            (a+b)(a-b) = a(a+(-b))+b(a+(-b))
            = aa+a(-b)+ba+b(-b)
            = a^2-ab+ab-b^2
            = a^2-b^2
        \end{equation*}
        as desired.\par
        \underline{$\text{(ii)}\Rightarrow\text{(iii)}$}: Suppose $(a+b)(a-b)=a^2-b^2$ for all $a,b\in R$. Then
        \begin{align*}
            a^2-b^2 &= a^2-ab+ba-b^2\\
            ab &= ba
        \end{align*}
        Thus,
        \begin{equation*}
            (a+b)^2 = (a+b)(a+b)
            = a(a+b)+b(a+b)
            = aa+ab+ba+bb
            = aa+ab+ab+bb
            = a^2+2ab+b^2
        \end{equation*}
        as desired.\par
        \underline{$\text{(iii)}\Rightarrow\text{(i)}$}: Suppose $(a+b)^2=a^2+2ab+b^2$ for all $a,b\in R$. Let $a,b\in R$ be arbitrary. Then
        \begin{align*}
            a^2+ab+ab+b^2 &= a^2+ab+ba+b^2\\
            ab &= ba
        \end{align*}
        so $a,b$ commute. Therefore, $R$ is commutative, as desired.
    \end{proof}
    \stepcounter{enumi}
    \item For this problem, you only have to state whether each of the nine assertions $\text{(i)},\dots,\text{(ix)}$ is \emph{true} or \emph{false}. No proofs are required.\par
    Given sets $X,Y$, the set of all functions $f:Y\to X$ is denoted by $X^Y$. Let $(A,+)$ be an abelian group. Given functions $f,g:Y\to A$, define $f+g:Y\to A$ by pointwise addition, i.e., let
    \begin{equation*}
        (f+g)(y) = f(y)+g(y)
    \end{equation*}
    for all $y\in Y$.
    \begin{enumerate}
        \item The above binary operation $+$ on $A^Y$ gives $A^Y$ the structure of an abelian group.
        \begin{proof}[Answer]
            True.
        \end{proof}
    \end{enumerate}
    For (ii) and (iii) below, we continue with $Y=A$ where $(A,+)$ is an abelian group. In an attempt to give $A^A$ the structure of a ring --- for functions $f,g:A\to A$ --- we take $\circ$ as the second binary operation. Here, $(f\circ g)(a)=f(g(a))$ for all $a\in A$.
    \begin{enumerate}[resume]
        \item The right distributive law, i.e., $(f+g)\circ h=f\circ h+g\circ h$ holds for all functions $f,g,h:A\to A$.
        \begin{proof}[Answer]
            True.
        \end{proof}
        \item The left distributive law, i.e., $f\circ(g+h)=f\circ g+f\circ h$ holds for all functions $f,g,h:A\to A$.
        \begin{proof}[Answer]
            False.
        \end{proof}
        \item The identity function $\id_A:A\to A$ given by $\id_A(a)=a$ for all $a\in A$ satisfies
        \begin{equation*}
            \id_A\circ f = f = f\circ\id_A
        \end{equation*}
        for all $f:A\to A$.
        \begin{proof}[Answer]
            True.
        \end{proof}
    \end{enumerate}
    If you have solved the above problems correctly, you would have seen that $(A^A,+,\circ)$ is \emph{not} a ring. In an endeavor to produce a ring employing the same binary operations $+$ and $\circ$, we replace $A^A$ by its subset $\End(A)=\{f:A\to A:f\text{ is a group homomorphism}\}$.
    \begin{enumerate}[resume]
        \item For $f,g\in\End(A)$, both $f+g$ and $f\circ g$ belong to $\End(A)$.
        \begin{proof}[Answer]
            True.
        \end{proof}
        \item The left and right distributive laws hold for $(\End(A),+,\circ)$.
        \begin{proof}[Answer]
            True.
        \end{proof}
        \item $(\End(A),+,\circ)$ is a ring (with two-sided multiplicative identity).
        \begin{proof}[Answer]
            True.
        \end{proof}
        \item $(\End(A),+,\circ)$ is a commutative ring for all abelian groups $(A,+)$.
        \begin{proof}[Answer]
            False\footnote{Counterexample: Let $K$ denote the Klein 4-group. Define $f,g\in\End(K)$ by $(x,y)\mapsto(0,x)$ and $(x,y)\mapsto(0,y)$, respectively. Then $f,g$ are group homomorphisms, but $(f\circ g)(1,0)=(0,0)\neq(0,1)=(g\circ f)(1,0)$, so $f\circ g\neq g\circ f$, as desired.}.
        \end{proof}
        \item If $A=\Z\times\Z$, then $\End(A)$ is isomorphic to the ring of $2\times 2$ matrices with integer coefficients.
        \begin{proof}[Answer]
            True\footnote{Since matrices are linear transformations, they are group homomorphisms. On the other hand, any $f\in\End(A)$ respects addition (as a homomorphism) and scalar multiplication (since $af=f+\cdots+f$ $a$ times for any $a\in\Z$). Thus, any endomorphism on $\Z\times\Z$ is a linear transformation and hence has a matrix representation.}.
        \end{proof}
    \end{enumerate}
\end{enumerate}




\end{document}