\documentclass[../psets.tex]{subfiles}

\pagestyle{main}
\renewcommand{\leftmark}{Problem Set \thesection}
\setcounter{section}{6}

\begin{document}




\section{Modules Over PIDs}
\begin{enumerate}
    \item \marginnote{2/24:}\textbf{Uniqueness of the rational canonical form.} Let $I_1\subset I_2\subset\cdots$ be a sequence of ideals in a PID $R$. Assume that there is some natural number $N$ such that $I_N=R$. Thus, if $I_i=(a_i)$, we have $a_{i+1}\mid a_i$ for all $i$ and $1=a_N=a_{N+1}=\cdots$. Let $M_i=R/I_i$, and let $M=M_1\oplus M_2\oplus\cdots$. For a prime $p$ of $R$ and for $k\geq 0$, we see that $p^kM/p^{k+1}M$ is a module over the \emph{field} $R/(p)$, and is therefore a vector space over $R/(p)$. Denote by $d(p,k)$ its dimension. Define $n_i(p)$ to be the greatest nonnegative integer such that $I_i\subset(p^{n_i})$ --- equivalently, $n_i(p)$ is the power of $p$ that occurs in the factorization of $a_i$. However, $a_i=0$ (equivalently $I_i=0$) is a possibility, in which case we put $n_i(p)=\infty$.
    \begin{enumerate}
        \item Prove that the sequence $d(p,0),d(p,1),\dots$ determine the sequence $n_1(p),n_2(p),\dots$.
        \item Deduce that if $M\cong N$ where $N=N_1\oplus N_2\oplus\cdots$ and $N_i=R/J_i$ for an increasing sequence of ideals $J_1\subset J_2\subset\cdots$, then $I_n=J_n$ for all $n\in\N$.
    \end{enumerate}
    \item Let $K$ be the fraction field of the PID $R$. We regard $K$ as an $R$-module and regard $R\subset K$ as an $R$-submodule.
    \begin{enumerate}
        \item Show that $K/R$ is a torsion $R$-module.
        \item We have shown that every torsion $R$-module is the direct sum of its $p$-primary components. The $p$-primary component of $K/R$ is $S/R$, where $S$ is an $R$-submodule of $K$. Do you recognize $S$? \emph{Hint}: You encountered it in fourth week.
    \end{enumerate}
    \item Given subrings $A,B$ of a ring $C$, it is not true that $A+B$ is a subring in general. But here is an example where it is indeed a subring: Let $C=F(X)$ where $F$ is a field, let $A=F[X]$, let $\alpha\in F$, and let $B$ be the image of the unique ring homomorphism $\phi:F[T]\to F(X)$ such that $\phi(c)=c$ for all $c\in F$ and $\phi(T)=(X-a)^{-1}$. Prove that\dots
    \begin{enumerate}
        \item $A\cap B=F$;
        \item $A+B$ equals the subring $S$ of the previous problem, where $R=F[X]$ and $p=(X-\alpha)$.
    \end{enumerate}
    \item Let $R$ be a commutative ring. The \textbf{derivative} (of $f=a_0+a_1X+\cdots+a_nX^n\in R[X]$), denoted by $f'$, is defined by $f'(X)=a_1+2a_2X+\cdots+na_nX^{n-1}$. Assume that $R$ is a subring of a commutative ring $A$. Let $M$ be an $A$-module. An \textbf{$\bm{R}$-derivation} (of $A$ with values in $M$) is a function $D:A\to M$ that satisfies\dots
    \begin{enumerate}[label={(\arabic*)}]
        \item $D(a+b)=D(a)+D(b)$ for all $a,b\in A$;
        \item $D(ab)=aD(b)+bD(a)$ for all $a,b\in A$;
        \item $D(c)=0$ for all $c\in R$.
    \end{enumerate}
    Prove that $D(f)=f'$ is an $R$-derivation $D$ of $R[X]$ with values in $R[X]$ that satisfies $D(X)=1$.
    \item 
    \begin{enumerate}
        \item Let $a\in R$ and let $f\in R[X]$, where $R$ is a commutative ring. $a$ is said to be a \textbf{root} (resp. \textbf{repeated root}) of $f$ if $f$ is a multiple of $(X-a)$ (resp. $(X-a)^n$ for some $n\in\N$). Prove that $f(a)=f'(a)=0$ iff $f$ is a multiple of $(X-a)$.
        \item Let $F$ be a subfield of a field $E$. Let $a\in E$ and let $f\in F[X]$. Show that if $a$ is a repeated root of $f$, then there is some $g\in F[X]$ such that\dots
        \begin{enumerate}[label={(\arabic*)}]
            \item $\deg(g)>0$;
            \item Both $f$ and $f'$ are multiples of $g$ in $F[X]$.
        \end{enumerate}
    \end{enumerate}
    \item This is essentially a repetition of the last problem from HW6 but by a slightly different method.\par
    Let $F[X]_{<m}$ be the collection of $a\in F[X]$ such that $\deg(a)<m$. Let $f,g\in F[X]$ be polynomials of degrees $d$ and $e$, respectively. Define $T:F[X]_{<e}\oplus F[X]_{<d}\to F[X]_{<d+e}$ by $T(a,b)=af+bg$. Note that $T$ is a linear transformation of $F$-vector spaces, with domain and target of the same dimension.
    \begin{enumerate}
        \item Deduce that $\gcd(f,g)=1$ iff every $h\in F[X]$ with $\deg(h)<d+e$ can be expressed as $af+bg$ for some $a,b\in F[X]$ satisfying $\deg(a)<e$ and $\deg(b)<d$.
        \item The \textbf{resultant} (of $f,g$), denoted by $\Ress(f,g)$, is the determinant of $T$. To define the latter, one requires a basis for the source and target. In particular,
        \begin{equation*}
            (1,0),(X,0),\dots,(X^{e-1},0),(0,1),(0,X),\dots,(0,X^{d-1})
        \end{equation*}
        is the basis for $F[X]_{<e}\oplus F[X]_{<d}$ and
        \begin{equation*}
            1,X,\dots,X^{d+e-1}
        \end{equation*}
        is the basis for $F[X]_{<d+e}$.\par
        Deduce that $\gcd(f,g)=1$ iff $\Ress(f,g)\neq 0$.
    \end{enumerate}
    \item Given an $R$-module $M$ and $a\in R$, denote by $a_M:M\to M$ the function $a_M(m)=am$ for all $m\in M$. Now consider $M=R/(p^2)\oplus R/(p)$ where $R$ is a PID and $p\in R$ is a prime. Let $N$ be a submodule of $M$ which has the property that $T(N)\subset M$ for every $R$-module self-isomorphism $T:M\to M$. Prove that $N$ is one of the following four submodules: $0,M,pM,\ker(p_M)$. \emph{Note}: The above problem is also valid for $(R/(p^2))^m\oplus(R/(p))^n$.
\end{enumerate}




\end{document}