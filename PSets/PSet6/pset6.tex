\documentclass[../psets.tex]{subfiles}

\pagestyle{main}
\renewcommand{\leftmark}{Problem Set \thesection}
\setcounter{section}{5}

\begin{document}




\section{Getting Comfortable With Modules}
All modules considered are left modules. Given $A$-modules $M,N$, the set of all $A$-module homomorphisms from $M\to N$ is denoted by $\Hom_A(M,N)$. It is an additive abelian group.
\begin{enumerate}
    \item \marginnote{2/17:}Let $M$ be an $A$-module and let $e:M\to M$ be an $A$-module homomorphism satisfying $e\circ e=e$. We have shown that both $e(M)$ and $\ker(e)$ are submodules of $M$. %DO NOT WRITE A SOLUTION.
    \begin{enumerate}
        \item Prove that $\phi:e(M)\oplus\ker(e)\to M$ given by $\phi(v,w)=v+w$ for all $v\in e(M)$, $w\in\ker(e)$ is an isomorphism of $A$-modules.
        \item Define $P:e(M)\oplus\ker(e)\to e(M)\oplus\ker(e)$ by $P(v,w)=(v,0)$ for all $(v,w)\in e(M)\oplus\ker(e)$. Prove that $P=\phi^{-1}\circ e\circ\phi$.
    \end{enumerate}
    \item Let $f:M\to N$ and $g:N\to M$ be $A$-module homomorphisms such that $g(f(m))=m$ for all $m\in M$. Prove that $H:M\oplus\ker(g)\to N$ given by $H(m,n)=f(m)+n$ for all $m\in M$, $n\in\ker(g)$ is an isomorphism of $A$-modules.
    \item Let $\phi:A\to B$ be a ring homomorphism, and let $M$ be a $B$-module. Show that $\cdot:A\times M\to M$ defined by
    \begin{equation*}
        (a,m) \mapsto \phi(a)m
    \end{equation*}
    for all $a\in A$, $m\in M$ gives $M$ the structure of an $A$-module.\par
    In particular, every $B$-module $M$ has the structure of an $A$ module for every subring $A$ of $B$.\par
    A very important application of this observation ($F[X]$-modules) is discussed on \textcite[340]{bib:DummitFoote}; it will be all-important later on in this course. %DO NOT WRITE A SOLUTION.
    \item Let $K$ be the fraction field of an integral domain $R$. Let $V$ and $W$ be $K$-modules (i.e., vector spaces over the field $K$). The preceding problem shows that $V$ and $W$ are also $R$-modules in a natural manner.\par
    Prove that every $R$-module homomorphism $f:V\to W$ is also a $K$-module homomorphism (it has to be shown that $f(av)=af(v)$ for all $a\in K$, $v\in V$).
    \item With $K,R,V,W$ as in the preceding problem, let $M$ be an $R$-submodule of $V$. Assume that for every $v\in V$, there is a nonzero $a\in R$ such that $av\in M$. Let $f:M\to W$ be an $R$-module homomorphism. Prove that $f$ extends in a unique manner to a $K$-module homomorphism $F:V\to W$.
    \item We have shown in class that every $A$-module homomorphism $T:A^n\to M$ (where $M$ is an $A$-module) is given by
    \begin{equation*}
        T(a_1,\dots,a_n) = a_1v_1+\cdots+a_nv_n
    \end{equation*}
    for all $(a_1,\dots,a_n)\in A^n$ and some $v_1,\dots,v_n\in M$. This gives a bijection between $\Hom_A(A^n,M)$ and $M^n$.\par
    Now let $c=(c_1,\dots,c_n)\in A^n$. We have the $A$-submodule $Ac=\{ac:a\in A\}$ of $A^n$ and the quotient module $A^n/Ac$. Show that there is a bijection from the set of $A$-module homomorphisms $S:A^n/Ac\to M$ and a certain additive subgroup $G$ of $M^n$. Describe $G$ explicitly.\par
    \emph{Hint}: Given $S$, consider the composite $A^n\to A^n/Ac\xrightarrow{S}M$.
    \item Let $c=(c_1,\dots,c_n)\in A^n$. Assume that the \emph{right} ideal $c_1A+\cdots+c_nA$ equals $A$ itself.
    \begin{enumerate}
        \item Prove that there is a left $A$-module homomorphism $g:A^n\to A$ such that $g(c)=1$.
        \item Deduce that there is an isomorphism $A\oplus\ker(g)\to A^n$ of left $A$-modules. \emph{Hint}: Problem 6.2.
    \end{enumerate}
    \item Assume that $A$ is a commutative ring. Prove that if $M$ is an $A$-module such that $M\oplus A\cong A^2$, then there is an $A$-module isomorphism $A\to M$.
    \item Let $R$ be a commutative ring. Assume that there are $x,y,z\in R$ such that $x^2+y^2+z^2=1$. Define $f:R^3\to R$ by $f(a,b,c)=ax+by+cz$. Let $M=\ker(f)$.\par
    Prove that there is an $R$-module isomorphism $M\oplus R\to R^3$.\par
    \emph{Note}: However, $M$ need not be isomorphic to $R^2$. For example, if $R=\R[X,Y,Z]/(X^2+Y^2+Z^2-1)$ and $x,y,z$ are $\bar{X},\bar{Y},\bar{Z}$, respectively, here $M$ is not isomorphic to $R^2$. This is saying that the tangent bundle of the two-sphere is nontrivial. It is proved using Algebraic Topology, but purely algebraic proofs exist.
    \item Prove that every (left) $A$-module homomorphism from $A$ to itself is right multiplication by $a$, denoted by $r_a:A\to A$, for a unique $a\in A$. %DO NOT WRITE A SOLUTION.
    \item Let $R$ be a commutative ring. Show that if $T:M\to N$ is a homomorphism of $R$-modules and if $a\in R$, then $S:M\to N$ given by $S(m)=aT(m)$ for all $m\in M$ is also an $R$-module homomorphism. Deduce that $\Hom_R(M,N)$ has the structure of an $R$-module.
    \item Give an example of a PID $A$ and an $A$-submodule $M'$ of an $A$-module $M$ such that $M$ and $M'\oplus(M/M')$ are not isomorphic to each other (as $A$-modules).\par
    \emph{Note}: If $A$ is a field, then there is an isomorphism $M\to M'\oplus(M/M')$. In class, it was shown that there is such an isomorphism if $M/M'$ is isomorphic to $A^n$ for some $n=0,1,2,\dots$.
    \item Let $f,g\in F[X]$ be polynomials of degrees $d$ and $e$, respectively, where $F$ is a field. Assume that $\gcd(f,g)=1$. Prove that there is a unique pair $a,b\in F[X]$ such that
    \begin{align*}
        af+bg &= 1&
        \deg(a) &< e&
        \deg(b) &< d
    \end{align*}
    \emph{Hint}: One already knows that there exist $a,b$ satisfying $af+bg=1$, but the $a,b$ satisfying this equation are far from being unique. Given $a,b$, first find \emph{all} $a',b'$ satisfying $a'f+b'g=1$. After this, you will see that the problem is easily solved.\par
    \emph{Note}: There is also a different constructive method of finding the desired $a,b$ that relies on determinants and resultants.
\end{enumerate}




\end{document}