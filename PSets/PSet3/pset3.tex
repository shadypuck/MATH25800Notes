\documentclass[../psets.tex]{subfiles}

\pagestyle{main}
\renewcommand{\leftmark}{Problem Set \thesection}
\setcounter{section}{2}

\begin{document}




\section{Properties of Ideals}
When solving a particular problem $Y$, you may appeal to the result of any problem $X$ that has occurred before $Y$ (earlier problem sheets included) whether or not you submitted a solution of problem $X$.
\begin{enumerate}
    \item \marginnote{1/25:}How many maximal ideals does the ring $\Z/a\Z$ possess under the following conditions?
    \begin{enumerate}[label={(\roman*)}]
        \item $a=81$.
        \item $a=44$.
        \item $a=42$.
    \end{enumerate}
    \item Given ring homomorphisms $f:R\to A$ and $g:R\to B$, check that $h(v)=(f(v),g(v))$ for all $v\in R$ gives a ring homomorphism $h:R\to A\times B$.
    \item In particular, let $I_1,I_2$ be ideals of a commutative ring $R$, and let $\pi_i:R\to R/I_i$ ($i=1,2$) be canonical surjections. Consider the ring homomorphism $h:R\to(R/I_1)\times(R/I_2)$ given by $h(a)=(\pi_1(a),\pi_2(a))$ for all $a\in R$.
    \begin{enumerate}[label={(\roman*)}]
        \item Describe $\ker(h)$ in terms of $I_1,I_2$.
        \item Prove that $A\Longrightarrow B\Longrightarrow C\Longrightarrow A$.
        \begin{enumerate}[label={(\Alph*)}]
            \item $h$ is a surjection.
            \item $(0,1)$ is in the image of $h$.
            \item $I_1+I_2=R$.
        \end{enumerate}
        \item Assume that $I_1+I_2=R$. Prove that $I_1I_2=I_1\cap I_2$. Deduce that $\phi:R/(I_1I_2)\to(R/I_1)\times(R/I_2)$ is an isomorphism.
    \end{enumerate}
    \item Prove that a nonzero ideal $I\subset F[[X]]$, where $F$ is a field, is the principal ideal generated by $X^n$ for some $n\geq 0$. (This is a continuation of Exercise 7.2.3c of \textcite{bib:DummitFoote}, addressed in HW2 Q2.2.)
    \item Recall that $R[X,Y]:=R[X][Y]$. Regard $R$ as a subring of $R[X,Y]$. Let $R$ be a commutative ring.\par
    The \textbf{universal property of $\bm{R[X,Y]}$} states: Let $A$ be commutative. Given a ring homomorphism $\alpha:R\to A$ and $x,y\in A$, prove that there is a unique ring homomorphism $\beta:R[X,Y]\to A$ that satisfies $\beta(c)=\alpha(c)$ for all $c\in R$, $\beta(X)=x$, and $\beta(Y)=y$.\par
    Deduce this statement from the universal property of $R[X]$.
    \item 
    \begin{enumerate}[label={(\roman*)}]
        \item For any $a\in R$, we may define the ring homomorphism $\phi:R[X]\to R$ by $\phi(f(X))=f(a)$. Prove that $\ker\phi$ is a principal ideal, and find a generator of this ideal.
        \item Let $g\in R[X]$. Define $\phi:R[X,Y]\to R[X]$ by $\phi(f(X,Y))=f(X,g(X))$. Prove that $\ker\phi$ is a principal ideal, and find a generator of this ideal.
    \end{enumerate}
    \item Let $a,b$ be elements of $R$ a commutative ring, and let $a$ be a unit of $R$. Consider the ring homomorphism $\phi:R[X]\to R[X]$ given by $\phi(f)=f(aX+b)$. Prove that $\phi$ is an isomorphism. \emph{Hint}: It's inverse can be written down explicitly.
    \item Let $R$ be an integral domain. Prove that every isomorphism $\phi:R[X]\to R[X]$ that satisfies $\phi(c)=c$ for all $c\in R$ is of the type given in Q3.7.
    \item 
    \begin{enumerate}[label={(\roman*)}]
        \item Exercise 7.1.11 of \textcite{bib:DummitFoote}: Prove that if $R$ is an integral domain and $x^2=1$ for some $x\in R$, then $x=\pm 1$.
        \item Deduce that $\{a^2\mid 0\neq a\in\F_p\}$ has cardinality $(p-1)/2$. Here, $p$ is an odd prime, and $\F_p$ is the field of cardinality $p$.
    \end{enumerate}
    \item Prove that there are exactly four rings of cardinality $p^2$, where $p$ is a prime ($p=2$ is included). Identify which of them is a field, which is a product of two fields, and find a nonzero nilpotent in both of the remaining cases.\par
    \emph{Hint}: First show that there are only two possibilities for the characteristic of such a ring. If the characteristic is an odd prime $p$, show that there is some $\theta$ in the ring with the two properties: (i) $\theta^2\in\F_p$ and (ii) $1,\theta$ form a basis for the given ring viewed as an $\F_p$ vector space. Now apply the previous problem.
\end{enumerate}




\end{document}