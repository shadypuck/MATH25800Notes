\documentclass[../psets.tex]{subfiles}

\pagestyle{main}
\renewcommand{\leftmark}{Problem Set \thesection}
\stepcounter{section}

\begin{document}




\section{Ideals and Vector Spaces}
\subsection*{Problems from the Textbook}
\begin{enumerate}
    \item \marginnote{1/18:}Exercise 7.1.9 of \textcite{bib:DummitFoote}: For a fixed element $a\in R$, define
    \begin{equation*}
        C(a) = \{r\in R\mid ra=ar\}
    \end{equation*}
    Prove that $C(a)$ is a subring of $R$ containing $a$. Prove that the center of $R$ is the intersection of the subrings $C(a)$ over all $a\in R$.
    \item Exercise 7.2.3(b-c) of \textcite{bib:DummitFoote}: Define the set $R[[X]]$ of \textbf{formal power series} in the indeterminate $X$ with coefficients from $R$ to be all formal infinite sums
    \begin{equation*}
        \sum_{n=0}^\infty a_nx^n = a_0+a_1x+a_2x^2+a_3x^3+\cdots
    \end{equation*}
    Define addition and multiplication of power series in the same way as for power series with real or complex coefficients, i.e., extend polynomial addition and multiplication to power series as though they were "polynomials of infinite degree:"
    \begin{align*}
        \left( \sum_{n=0}^\infty a_nx^n \right)+\left( \sum_{n=0}^\infty b_nx^n \right) &= \sum_{n=0}^\infty(a_n+b_n)x^n\\
        \left( \sum_{n=0}^\infty a_nx^n \right)\times\left( \sum_{n=0}^\infty b_nx^n \right) &= \sum_{n=0}^\infty\left( \sum_{k=0}^na_kb_{n-k} \right)x^n
    \end{align*}
    (The term "formal" is used here to indicate that convergence is not considered, so that formal power series need not represent functions on $R$.)
    \begin{enumerate}[start=2,label={\textbf{(\alph*)}}]
        \item Show that $1-x$ is a unit in $R[[X]]$ with inverse $1+x+x^2+\cdots$.
        \item Prove that $\sum_{n=0}^\infty a_nx^n$ is a unit in $R[[X]]$ iff $a_0$ is a unit in $R$.
    \end{enumerate}
    \item Exercise 7.3.24 of \textcite{bib:DummitFoote}: Let $\varphi:R\to S$ be a ring homomorphism.
    \begin{enumerate}[label={\textbf{(\alph*)}}]
        \item Prove that if $J$ is an ideal of $S$, then $\varphi^{-1}(J)$ is an ideal of $R$. Apply this to the special case when $R$ is a subring of $S$ and $\varphi$ is the inclusion homomorphism to deduce that if $J$ is an ideal of $S$, then $J\cap R$ is an ideal of $R$.
        \item Prove that if $\varphi$ is surjective and $I$ is an ideal of $R$, then $\varphi(I)$ is an ideal of $S$. Give an example where this fails if $\varphi$ is not surjective.
    \end{enumerate}
    \item Exercise 7.4.27 of \textcite{bib:DummitFoote}: Let $R$ be a commutative ring with $1\neq 0$. Prove that if $a$ is a nilpotent element of $R$, then $1-ab$ is a unit for all $b\in R$.
    \item Exercise 7.4.33 of \textcite{bib:DummitFoote}: Let $R$ be the ring of all continuous functions from the closed interval $[0,1]$ to $\R$, and for each $c\in[0,1]$, let $M_c=\{f\in R\mid f(c)=0\}$. (Recall that $M_c$ was shown to be a maximal ideal of $R$.)
    \begin{enumerate}[label={\textbf{(\alph*)}}]
        \item Prove that if $M$ is any maximal ideal of $R$, then there is a real number $c\in[0,1]$ such that $M=M_c$.
        \item Prove that if $b,c$ are distinct points in $[0,1]$, then $M_b\neq M_c$.
        \item Prove that $M_c$ is not equal to the principal ideal generated by $x-c$.
        \item Prove that $M_c$ is not a finitely generated ideal.
    \end{enumerate}
\end{enumerate}
The preceding exercise shows that there is a bijection between the \emph{points} of the closed interval $[0,1]$ and the set of \emph{maximal ideals} in the ring $R$ of all continuous functions on $[0,1]$ given by $c\leftrightarrow M_c$. For any subset $X\subset\R$ or, more generally, for any completely regular topological space $X$, the map $c\mapsto M_c$ is an injection from $X$ to the set of maximal ideals of $R$, where $R$ is the ring of all bounded, continuous, real-valued functions on $X$ and $M_c$ is the maximal ideal of functions that vanish at $c$. Let $\beta(X)$ be the set of maximal ideals of $R$. One can put a topology on $\beta(X)$ in such a way that if we identify $X$ with its image in $\beta(X)$, then $X$ (in its given topology) becomes a subspace of $\beta(X)$. Moreover, $\beta(X)$ is a compact space under this topology and is called the \textbf{Stone-\u{C}ech compactification} of $X$.
\begin{enumerate}[resume]
    \item Let $R$ be the ring of all continuous functions from $\R$ to $\R$, and for each $c\in\R$, let $M_c$ be the maximal ideal $\{f\in R\mid f(c)=0\}$.
    \begin{enumerate}[label={\textbf{(\alph*)}}]
        \item Let $I$ be the collection of functions $f\in R$ with \textbf{compact support} (i.e., $f(x)=0$ for $|x|$ sufficiently large). Prove that $I$ is an ideal of $R$ that is not a prime ideal.
        \item Let $M$ be a maximal ideal of $R$ containing $I$ (properly, by part (a)). Prove that $M\neq M_c$ for any $c\in\R$ (refer to the preceding exercise).
    \end{enumerate}
\end{enumerate}


\subsection*{Custom Questions}
The first problem below is analogous to Corollary 3 on \textcite[228]{bib:DummitFoote}, where it is shown that any finite integral domain is a field.
\begin{enumerate}[resume]
    \item Let $R$ be a commutative ring, and $F$ be a subring of $R$ that is a field. Then $R$ acquires the structure of a vector space over the field $F$. Assume now that $R$ is a finite dimensional vector space over $F$. Show that if $R$ is an integral domain, then $R$ is a field.
    \item Give an example to show that the hypothesis of finite dimensionality cannot be dropped in the previous problem.
    \item Let $V$ be a finite dimensional vector space over a field $F$, and let $\End_F(V)$ denote the set of linear transformations $T:V\to V$.
    \begin{enumerate}[label={(\alph*)}]
        \item Let $W\subset V$ be a linear subspace. Show that $\{T\in\End_F(V):T(W)=0\}$ is a left ideal of the ring $\End_F(V)$.
        \item Let $T:V\to V$ be a linear transformation, and let $W=\ker(T)$. Show that the left ideal generated by $T$ is $\{S\in\End_F(V):S(W)=0\}$.
        \item Show that $\{T\in\End(V):T(V)\subset W\}$ is a right ideal of $\End_F(V)$.
        \item Show that if $\im(T)=W$, then the right ideal of $\End_F(V)$ generated by $T$ is $\{S\in\End_F(V):S(V)\subset W\}$.
    \end{enumerate}
    \item Prove that if $T$ is in the center of $\End_F(V)$, then there is some $c\in F$ such that $Tv=cv$ for all $v\in V$.
\end{enumerate}




\end{document}