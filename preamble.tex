\usepackage[margin=1in]{geometry}
\usepackage{csquotes}
\usepackage{fancyhdr}
\usepackage{marginnote}
\usepackage{enumitem}
\usepackage{scrextend}
\usepackage[bottom]{footmisc}
\usepackage{xr}
\usepackage[style=apa]{biblatex}
\usepackage{tikz}
\usepackage{float}
\usepackage{amsmath,amssymb,amsthm,amsbsy}
\usepackage{bm,nicematrix,physics,empheq}
\usepackage[hidelinks]{hyperref}
\usepackage{subfiles}

\MakeOuterQuote{"}

\fancypagestyle{main}{
    \fancyhf{}
    \fancyhead[L]{\leftmark}
    \fancyhead[R]{MATH 25800}
    \fancyfoot[R]{Labalme\ \thepage}
}
\fancypagestyle{plain}{
    \fancyhead{}
    \renewcommand{\headrulewidth}{0pt}
}

\reversemarginpar

\setitemize[3]{label={\scriptsize$\blacksquare$}}
\setitemize[4]{label={\tikz[scale=0.06,baseline={(0,-0.14)}]{
    \draw [line width=0.3pt] (0,1) -- (1.2,0) -- (0,-1) -- (3.5,0) -- cycle;
    \fill (1.2,0) -- (0,-1) -- (3.5,0);
}}}

\deffootnotemark{\textsuperscript{\textup{[}\thefootnotemark\textup{]}}}
\deffootnote[2.1em]{0em}{0em}{\textsuperscript{\thefootnote}}

\addbibresource{\subfix{../main.bib}}
\DefineBibliographyStrings{english}{bibliography={References}}

\usetikzlibrary{arrows}

\DeclareMathOperator{\Fun}{Fun}
\DeclareMathOperator{\id}{id}
\DeclareMathOperator{\End}{End}
\DeclareMathOperator{\evl}{ev}
\DeclareMathOperator{\diag}{diag}
\DeclareMathOperator{\im}{im}
\DeclareMathOperator{\lcm}{lcm}
\DeclareMathOperator{\chr}{char}
\DeclareMathOperator{\Frac}{Frac}
\DeclareMathOperator{\Aut}{Aut}
\DeclareMathOperator{\Hom}{Hom}
\DeclareMathOperator{\Tor}{Tor}
\DeclareMathOperator{\Ann}{Ann}

\theoremstyle{definition}
\newtheorem{proposition}{Proposition}
\newtheorem{corollary}[proposition]{Corollary}
\newtheorem{theorem}[proposition]{Theorem}

\newcounter{bookch}
\numberwithin{proposition}{bookch}

\NiceMatrixOptions{cell-space-limits=1pt}

\renewcommand{\ev}{\evl}

\newcommand{\N}{\mathbb{N}}
\newcommand{\Z}{\mathbb{Z}}
\newcommand{\Q}{\mathbb{Q}}
\newcommand{\R}{\mathbb{R}}
\newcommand{\C}{\mathbb{C}}
\newcommand{\F}{\mathbb{F}}
\newcommand{\Zg}{{\Z_{\geq 0}}}
\newcommand{\Rg}{{\R_{\geq 0}}}

\newcommand{\gen}[1]{\left\langle{#1}\right\rangle}