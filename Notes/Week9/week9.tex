\documentclass[../notes.tex]{subfiles}

\pagestyle{main}
\renewcommand{\chaptermark}[1]{\markboth{\chaptername\ \thechapter\ (#1)}{}}
\setcounter{chapter}{8}

\begin{document}




\chapter{Extension Topics}
\section{Intro to the Langlands Program}
\begin{itemize}
    \item \marginnote{2/27:}Sometime before 2:30 PM today, Nori will post an exam syllabus that will also put an upper bound on the types of questions he will ask.
    \begin{itemize}
        \item "There's only so much you can cover in a 2-hour exam on an 8-week course."
    \end{itemize}
    \item We now begin on some --- in Nori's opinion --- very interesting mathematics.
    \item Let $f\in\Z[X]$ be irreducible, monic, and of degree $d$.
    \item \textbf{Split} (prime for $f$): A prime number $p$ for which
    \begin{equation*}
        \bar{f} = \prod_{i=1}^d(X-a_i) \in \F_p[X]
    \end{equation*}
    \item \textbf{Langlands program}: The name for the overall problem, "which primes are split for a given $f$?"
    \begin{itemize}
        \item Gauss answers this for degree 2 polynomials using quadratic reciprocity.
        \item There has been a lot of progress since then: See Artin's reciprocity law.
        \item This is a major unsolved problem.
    \end{itemize}
    \item Example: $X^2+1$.
    \begin{itemize}
        \item Informally: If we go modulo a prime, does this factor or not?
        \item Formally: For which primes $p$ does there exist $m\in\Z$ such that $m^2\equiv -1\bmod p$.
        \item Answer: $m$ exists if and only if $p\equiv 1\bmod 4$.
        \item Proving this: Let $p$ be an odd prime. Let $x\in\F_p-\{0\}$. Let $S(x)=\{x,-x,1/x,-1/x\}$ be the stabilizer. We either have $\{x,-x\}\cap\{1/x,-1/x\}=\emptyset$ or both elements. Thus, we either have $x=\pm 1$ or $x^2=-1$. It follows that $|S(x)|=4$ except when $\{1,-1\}$ or $\{\alpha,-\alpha\}$ with $\alpha\in\F_p$ satisfies $\alpha^2=-1$.
        \item $p-1\equiv 2\bmod 4$.
        \item Thus, we're partitioning the set into elements of multiplicity 4.
    \end{itemize}
    \item We'll skip considering the Gaussian integers.
    \item Consider the $d$ square-free integers for $d\neq 1$. Let $R_d=\Z\oplus\Z\sqrt{d}\cong\Z[X]/(X^2-d)$.
    \begin{itemize}
        \item If $d\equiv 2,3\bmod 4$, no bueno.
        \item If $d\equiv 1\mod 4$, then $R_d=\Z\oplus\Z\theta$, where
        \begin{equation*}
            \theta = \frac{1+\sqrt{d}}{2}
        \end{equation*}
        \item All of these rings have an automorphic ring homomorphism $\sigma:R_d\to R_d$ defined by
        \begin{equation*}
            a+b\sqrt{d} \mapsto a-b\sqrt{d}
        \end{equation*}
        \item Recall the norm $N(a+b\sqrt{d})=|(a+b\sqrt{d})(a-b\sqrt{d})|=|a^2-b^2d|$.
        \item Let $I\subset R_d$ be a nonzero ideal. If $\alpha\in I$ nonzero, then $|\alpha\sigma\alpha|=N(\alpha)\in I$.
        \item Suppose $m\in\N$. Then $R_d/mR_d=\Z/(m)\oplus\sqrt{d}\Z/(m)$ has $m^2$ elements.
        \item In particular, $R_d/I$ is finite as the quotient of a finite ring $R_d/R_dN(\alpha)$ (as implied by the fact that $I$ is nonzero).
        \item Let $P\subset R_d$ be a nonzero prime ideal. We have just shown that $P\cap\Z\neq 0$. It follows that if $m\in\N$ and $m\in P$, then $p_1\cdots p_r$ implies some $p_i\in P$.
        \item There exists a unique prime number $p$ such that $p\in P$.
        \item Fix $p$. Search for all $P$ prime ideals of $R_d$ such that $p\in P$, i.e., $(p)\subset P\subset R_d$, i.e., $P/(p)$ is a prime ideal of $R_d/(p)$.
        \item Recall that
        \begin{equation*}
            R_d/(p) = \F_p\oplus\F_p\sqrt{d} \cong \F_p[X]/(X^2-d)
        \end{equation*}
        \item Case 1: $p\nmid d$ and $p\neq 2$.
        \begin{itemize}
            \item Case 1(a): There exists an integer $m\in\Z$ such that $m^2\equiv d\bmod p$.
            \item Case 1(b): No integer $m\in\Z$ exists such that $m^2\equiv d\bmod p$.
        \end{itemize}
        \item Case 2: $p\mid d$.
    \end{itemize}
    \item We now treat each case above individually.
    \item Case 2.
    \begin{itemize}
        \item Let $P$ be unique and $P=(p,\sqrt{d})=\sigma P$.
        \item We have $P\sigma P=(p,\sqrt{d})(p,\sqrt{d})=(p^2,d,p\sqrt{d})\subset(p)$.
        \item Even in $\Z$, $\gcd_\Z(p^2,d)=p$ (because $p\mid d$ and $p^2\nmid d$; the latter claim follows because $d$ is square-free).
        \item This implies that $p\in P\sigma P$, which implies that $(p)=P\sigma P$.
    \end{itemize}
    \item Case 1b.
    \begin{itemize}
        \item $X^2-d$ is irreducible in $\F_p[X]$.
        \item Thus, $P=(p)$.
        \item It follows that $P=\sigma P$ and hence $P\sigma P=(p^2)$.
    \end{itemize}
    \item Case 1a.
    \begin{itemize}
        \item There exists an $m\in\Z$ such that $m^2\equiv d\bmod p$.
        \item Let $P=(p,m-\sqrt{d})$, $\sigma P=(p,m+\sqrt{d})$.
        \item There exists exactly two prime ideals $P$.
        \item Thus, $P\sigma P=(p^2,m^2-d,p(m-\sqrt{d}),p(m+\sqrt{d}))\subset(p)$.
        \item Adding the last two generators together, we obtain $(p^2,2mp)\in P\sigma P$. But since $p\nmid m$ and $p$ ??, we know that
        \begin{equation*}
            \gcd_\Z(p^2,2mp) = p
        \end{equation*}
    \end{itemize}
    \item It follows that $P\sigma P=(p)\sim(p^2)$ in all cases.
    \item We now consider the $p=2$ case.
    \begin{itemize}
        \item Let $R=\Z\oplus\Z\sqrt{d}$. Let $\varepsilon=0,1$ and $\varepsilon=\varepsilon^2$.
        \item Case 1(a): Does not exist; $\F_2[X]/(X^2-\varepsilon)$ and $\F_2[X]/((X-\varepsilon)^2)$.
        \item Case 2: $2\mid d$ and $4\nmid d$. It follows that $P$ is unique and equal to $(2,\sqrt{d})$. We have $p^2=(2)$.
        \item Case 1(b): $p=2$ and $2\nmid d$. Let $\F_2[X]/(X^2-1)$. We have a unique $P$ and $P=(2,1-\sqrt{d})=\sigma P=(2,1+\sqrt{d})$.
        \item Let $P^2=P\sigma P=(4,1-d,2(1-\sqrt{d}))=(2)$ if $d\equiv 3\bmod 4$. Note that $P\sigma P$ is \emph{not} principal if $d\equiv 1\bmod 4$.
        \item Consider (example): $F[X^2,X^3]\subset F[X]$.
        \item If $R_d=\Z+\Z\theta$ and $d\equiv 1\bmod 8$, then there exists $P\neq\sigma P$ with $P\sigma P=(2)$.
        \item If $d\equiv 5\bmod 8$, then $P=(2)$.
    \end{itemize}
    \item Next lecture: Dedekind domains.
    \begin{itemize}
        \item Every nonzero ideal can be written as a product of nonzero not necessarily unique prime ideals.
        \item Next best thing to a PID.
    \end{itemize}
    \item Theorem: $\Z[\sqrt{-1}]$ is a Euclidean domain.
    \begin{proof}
        Given $g,f$, we want $g=qf+r$ in $\Z[\sqrt{-1}]$ with $N(r)<N(f)$.\par
        Technique: Go outside the integers into $\Q(\sqrt{-1})$. This is a field. Consider $g/f\in\Q(\sqrt{-1})$. Choose the closest lattice point in $\Z[\sqrt{-1}]$ to $g/f\in\Q(\sqrt{-1})$, visualized as a complex plane and complex lattice subset. This makes $g/f=q+c$ where $q\in\Z[\sqrt{-1}]$. Let $c=\alpha+\beta\sqrt{-1}$. Then $|\alpha|\leq 1/2$, $|\beta|\leq 1/2$, and $N(\alpha+i\beta)=\alpha^2+\beta^2\leq 1/4+1/4=1/2$.\par
        It follows that $g\in\Z[\sqrt{-1}]$ equals $qf+(fc)$, where $qf\in\Z[\sqrt{-1}]$ and $fc=r\in\Z[\sqrt{-1}]$. Moreover, $N(r)=N(f)N(c)\leq 1/2N(f)$.
    \end{proof}
    \item The same proof applies to $\Z[\sqrt{-1}]$, $\Z[\sqrt{2}]$, $\Z[\sqrt{3}]$, $\Z[(1+\sqrt{-3})/2]$, $\Z[(1+\sqrt{p})/2]$, and in fact all Euclidean domains.
\end{itemize}



\section{Factorization of Ideals}
\begin{itemize}
    \item \marginnote{3/1:}Nori proves that any ideal, perhaps under certain conditions, can be factored into prime ideals.
\end{itemize}



\section{Office Hours (Nori)}
\begin{itemize}
    \item \marginnote{3/7:}Computing the JCF and RCF is going to be on the midterm.
    \item Gauss's lemma may well be on the final. There is a difference between Gauss's lemma in class and in the textbook; the textbook version is a corollary of the real one (which Nori presented).
    \begin{itemize}
        \item To clarify on the final, specify which version of Gauss's lemma you are using.
    \end{itemize}
    \item One question like the universal properties of polynomial rings stuff from the first midterm.
    \item $R'=R/(f)$ looks a lot like a final question.
    \item We'll never have to decide which (JCF or RCF) to use; we'll only be asked, "what is the JCF/RCF of this?"
    \item RCF: We'll be able to stop after writing down the invariant factors.
\end{itemize}



\section{Final Review Sheet}
\begin{itemize}
    \item See midterm review sheet for everything rings-related.
    \item \textbf{Gauss's lemma}: Reducible polynomials in $R[X]$ are reducible in $R[X]$.
    \item Dividing the first and last coefficients of the polynomial by monomials.
    \item \textbf{Left $\bm{A}$-module}: An abelian group $(M,+)$ equipped with a binary operation $\cdot:A\times M\to M$ satisfying the following constraints.
    \begin{enumerate}
        \item $a(v_1+v_2)=av_1+av_2$.
        \item $(a+b)v=av+bv$.
        \item $a(bv)=(ab)v$.
        \item $1_Av=v$.
    \end{enumerate}
    \item Alternatively: $\rho:A\to\End(M)$ satisfies:
    \begin{enumerate}
        \item $\rho(a)$ is a group homomorphism from $M\to M$.
        \item $\rho(a+b)=\rho(a)+\rho(b)$.
        \item $\rho(a)\rho(b)=\rho(ab)$.
        \item $\rho(1_A)=1_{\End(M)}$.
    \end{enumerate}
    \item \textbf{Module homomorphism}: A group homomorphism that commutes with scalar multiplication, i.e., $T(av)=aT(v)$.
    \item $T:A^n\to M$ is defined by the action of $T$ on the $e_i$'s.
    \item \textbf{Module isomorphism}: A bijective module homomorphism.
    \item Quotient module exist in a natural way.
    \item FIT for modules.
    \item Let $R$ be a PID; then every $R$-submodule of $R^n$ is isomorphic to $R^m$ for some $0\leq m\leq n$.
    \item If $M/M'\cong A^n$, then $M'\oplus A^n\cong M$.
    \item \textbf{$\bm{R}$-algebra}: A pair $(A,f)$, where $A$ is a ring, $f:R\to A$ is a ring homomorphism, and $R$ is a commutative ring, such that $f(R)\subset Z(A)$.
    \begin{itemize}
        \item $A$ is an $R$-module under $r\cdot a=f(r)\cdot a$.
    \end{itemize}
    \item \textbf{$\bm{R}$-algebra homomorphism}: A ring homomorphism $\varphi:A\to B$ such that $\varphi(r\cdot a)=r\cdot\varphi(a)$.
    \item $\Hom_R(M,M)$ is a ring under componentwise addition and composition; it is an $R$-algebra when $R$ is commutative.
    \item \textbf{Poset}.
    \item \textbf{Maximal} ($f\in P$): An element $f\in P$ a poset such that for all $q\in P$, the statement $q>f$ is false.
    \item \textbf{Chain}: A totally ordered subset of a poset.
    \item \textbf{Zorn's lemma}: If $P$ is a poset such that $P\neq 0$ and every chain $C\subset P$ has an upper bound, then $P$ has a maximal element.
    \item Corollary: Every nonzero finitely generated $A$-module $M$ has a maximal submodule.
    \item Corollary: Every nonzero commutative ring $R$ has a maximal ideal.
    \item \textbf{Torsion module}: An $R$-module such that for all $m\in M$, there exists a nonzero $a\in R$ such that $am=0$.
    \item \textbf{Torsion-free module}: An $R$-module $M$ such that for all nonzero $m\in M$ and for all nonzero $a\in R$, $am\neq 0$.
    \item \textbf{Torsion element}: An element $m\in M$ for which there exists a nonzero $a\in R$ such that $am=0$.
    \item $\bm{\Tor(M)}$: The set of all torsion elements in $M$.
    \item \textbf{$\bm{p}$-primary} (module): An $R$-module $M$ such that for all $m\in M$, there exists $k\geq 0$ for which $p^km=0$, where $p\in R$ is prime.
    \item \textbf{$\bm{p}$-primary} (component): The submodule of a module $M$ consisting of all $m\in M$ such that $p^km=0$ for some $k\in\Zg$ and $p\in R$ prime.
    \item All finitely generated torsion-free $R$-modules are isomorphic to $R^n$.
    \item $\Tor(M)$ is an $R$-submodule of $M$.
    \item $M/\Tor(M)$ is torsion-free.
    \begin{itemize}
        \item To prove that something is torsion-free, it suffices to prove that every torsion element is zero.
    \end{itemize}
    \item Every finitely generated $R$-module has a free submodule.
    \item Finitely generated torsion-free modules $M$ over a PID $R$ are isomorphic to $R^h$.
    \item If $M$ is a finitely generated $R$-module over a PID, then $M\cong\Tor(M)\oplus R^h$, where $\Tor(M)$ is finitely generated.
    \item Finitely generated $R$-module are isomorphic iff they have the same rank and their torsion components are isomorphic.
    \item The direct sum of the $p$-primary components is equal to $\Tor(M)$.
    \item Every finitely generated $p$-primary module is the direct sum of the cyclic submodule $Re_i$.
    \item \textbf{RCF}: Let $R$ be a PID. Then every finitely generated $R$-torsion module is isomorphic to $R/(a_1)\oplus\cdots\oplus R/(a_\ell)$ where $a_\ell\mid a_{\ell-1}\mid\cdots\mid a_1$.
    \item Every finitely generated $R$-module, where $R$ is a PID, is isomorphic to $R/I_1\oplus R/I_2\oplus\cdots$ for a unique increasing sequence of ideals $I_1\subset I_2\subset\cdots$ which have the property that $I_n=R$ for some $n$.
    \item Vector spaces $(V,T)$ as $F[X]$-modules under $\rho:F[X]\to\End_F(V)$ defined by $X\mapsto T$.
    \item \textbf{Minimal polynomial}: The polynomial that generates $\ker(\rho)$, an ideal of the PID $F[X]$.
    \item Cyclic vectors and linear dependence, defining the minimal polynomial by $g(X)=X^k-(a_{k-1}X^{k-1}+\cdots+a_0)$.
\end{itemize}




\end{document}