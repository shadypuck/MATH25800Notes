\documentclass[../notes.tex]{subfiles}

\pagestyle{main}
\renewcommand{\chaptermark}[1]{\markboth{\chaptername\ \thechapter\ (#1)}{}}
\setcounter{chapter}{2}
\setcounter{proposition}{14}
\setcounter{bookch}{7}

\begin{document}




\chapter{Intro to Ring Types}
\section{Intro to Chapters 8-9}
\begin{itemize}
    \item \marginnote{1/18:}Moving onto Chapter 8 today.
    \item Friday: Rings of fractions (more than what's in the book; under lesser hypotheses).
    \begin{itemize}
        \item Def get notes!
    \end{itemize}
    \item The Chinese Remainder Theorem is at least partially in HW3.
    \item Today: A leisurely introduction to Chapter 8, as well as Spring Quarter content (which is the most interesting part of the Honors Algebra sequence).
    \item For the next three weeks or more, all rings will be assumed to be commutative.
    \begin{itemize}
        \item Excepting matrix rings, which may still appear in exercises.
    \end{itemize}
    \item At this point, we define $\deg(f)=-\infty$ where $f$ is the zero polynomial.
    \begin{itemize}
        \item We do this so that $\deg(fg)=\deg(f)+\deg(g)$ still holds.
    \end{itemize}
    \item Euclidean algorithm for monic polynomials: Let $f\in R[X]$ be a monic polynomial of degree $d\geq 0$, and let $h\in R[X]$. Then there exists a unique pair $q,r\in R[X]$ such that\dots
    \begin{enumerate}
        \item $h=qf+r$;
        \item $\deg(r)<\deg(f)$.
    \end{enumerate}
    \begin{proof}
        We tackle uniqueness first, and then existence.\par
        \underline{Uniqueness}: Suppose $h=q_1f+r_1=q_2f+r_2$, where $\deg(r_i)<d$ ($i=1,2$). We have that
        \begin{equation*}
            (q_1-q_2)f = q_1f-q_2f = r_2-r_1
        \end{equation*}
        Now suppose for the sake of contradiction that $q_1-q_2\neq 0$. We know that
        \begin{equation*}
            \deg(r_2-r_1) = \deg[(q_1-q_2)f]
            = \deg(q_1-q_2)+d
            \geq d
        \end{equation*}
        But since $\deg(r_i)<d$ ($i=1,2$), we have that $\deg(r_2-r_1)<d$, a contradiction. Thus, $q_1-q_2=0$. It follows easily that $0=r_2-r_1$. Therefore, $(q_1,r_1)=(q_2,r_2)$, as desired.\par
        \underline{Existence}: If $\deg(h)<d$, then put $q=0$ and $r=h$. We now induct on $\deg(h)$, starting from $d$. Our base case is already taken care of via the statement on $\deg(h)<d$. Now suppose using strong induction that we have proven the claim for all nonnegative integers $n<\deg(h)$. Let
        \begin{equation*}
            h(X) = a_0+\cdots+a_eX^e
        \end{equation*}
        where $a_e\neq 0$ and $e\geq d$ by hypothesis. Let
        \begin{equation*}
            f(X) = b_0+\cdots+b_{d-1}X^{d-1}+X^d
        \end{equation*}
        Define $g(X)$ by
        \begin{equation*}
            g = h-a_eX^{e-d}f
        \end{equation*}
        It follows that $\deg(g)<e$, so we may apply the induction hypothesis at this point. We learn from it that there exist $q,r$ such that $g=qf+r$ with $\deg(r)<d$. Therefore, we can deduce that
        \begin{equation*}
            h = (a_eX^{e-d}+q)f+r
        \end{equation*}
        as desired.
    \end{proof}
    \item Notes on the Euclidean algorithm.
    \begin{itemize}
        \item Think long polynomial division from high school.
    \end{itemize}
    \item Example.
    \begin{itemize}
        \item Let $a\in R$ and $f=X-a$ be a monic polynomial. Let $h\in R[X]$ be arbitrary. Then applying the theorem,
        \begin{equation*}
            h(X) = q(X)(X-a)+r
        \end{equation*}
        \item $\deg(r)<1=\deg(f)$ implies that $r$ is a constant, and hence $r\in R$.
        \item Moreover,
        \begin{align*}
            h(a) &= q(a)(a-a)+r\\
            r &= h(a)
        \end{align*}
        implying that
        \begin{equation*}
            h(X)-h(a) = q(X)(X-a)
        \end{equation*}
        for arbitrary polynomials $h$.
    \end{itemize}
    \item Corollary: Let $a\in R$. $\{h\in R[X]\mid h(a)=0\}$ is the \textbf{principal ideal generated by $\bm{X-a}$}.
    \item \textbf{Ideal generated by $\bm{b\in B}$}. \emph{Denoted by} $\bm{Bb}$, $\bm{(b)}$.
    \item Corollary: Let $f\in R[X]$ be monic of degree $d$. Then
    \begin{equation*}
        \{g\in R[X]\mid \deg(g)<d\} \hookrightarrow R[X]
        \twoheadrightarrow R[X]/(f)
    \end{equation*}
    and, in particular,
    \begin{equation*}
        \{g\in R[X]\mid \deg(g)<d\} \cong R[X]/(f)
    \end{equation*}
    as groups (in particular, \emph{not} as rings).
    \begin{proof}
        % One-to-one is obvious.\par
        % We know that if $h\in(f)$, then either $h=0$ or $\deg(h)\geq\deg(f)$ (since $f$ is monic). Injective: $\ker(\text{composite})=0$. The composite is the bijective map, i.e., the one after the injective map. Onto: Every element of $R[X]/(f)$ is the image of some $h\in R[X]$. $h=qf+r$, $h$ and $h-qf$ gives rise to some element of $R[X]/(f)$. Note that $h-qf=r$.

        The existence of the first two maps is obvious (they are just instances of the canonical injection and surjection, respectively).\par\smallskip
        We now verify that the last two sets are in bijective correspondence. Define a map $\varphi$ between them via the canonical surjection (note that since the domain of $\varphi$ is not $R[X]$, we will still have to verify surjectivity here). As established previously, $\varphi$ is well defined.\par
        To prove that $\varphi$ is injective, it will suffice to show that $\ker\varphi=0$. Let $h$ be an arbitrary polynomial in $R[X]$ with $\deg(h)<d$. Suppose $\varphi(h)=0=0+(f)=(f)$. Then $h\in(f)$. It follows that either $h=0$ or $\deg(h)\geq\deg(f)=d$. But as an element of the domain $\deg(h)<d$ by hypothesis. Therefore, $h=0$, as desired.\par
        To prove that $\varphi$ is surjective, it will suffice to show that for every $h+(f)\in R[X]/(f)$, there exists $r\in R[X]$ with $\deg(r)<d$ such that $\varphi(r)=h+(f)$. Let $h+(f)\in R[X]/(f)$ be arbitrary. By the Euclidean algorithm, $h=qf+r$ for some $q,r\in R[X]$ where $\deg(r)<\deg(f)=d$. Moreover, since $r=h+(-q)f$, $r\in h+(f)$ and hence $h+(f)=r+(f)$. Therefore, since $r$ is in the domain of $\varphi$ (as it has degree less than $d$), $\varphi(r)=r+(f)=h+(f)$, as desired.
    \end{proof}
    \item $R[X]$ is also a vector space with $1,X,X^2,\dots$ as the basis.
    \item We have that
    \begin{equation*}
        \{g\in R[X]\mid \deg(g)<d\} = \{a_0+\cdots+a_{d-1}X^{d-1}\mid a_0,\dots,a_{d-1}\in R\}
    \end{equation*}
    \begin{itemize}
        \item As an abelian group (ignoring multiplication), this set is group isomorphic to $(R^d,+)$.
    \end{itemize}
    \item Revisiting the creation of $\C$ from $\R$.
    \begin{itemize}
        \item We can use quotient rings to solve $X^2+1=0$.
        \item In particular, the equation $X^2+1=0$ does not have a solution in $\R[X]$. However, it does have a solution in $\R[X]/(X^2+1)$, as we will see presently.
        \item Consider the function described in the above corollary, sending $\R\hookrightarrow\R[X]\twoheadrightarrow\R[X]/(X^2+1)$. Let $\bar{X}:=X+(X^2+1)\in\R[X]/(X^2+1)$ denote the image of $X$ in $\R[X]/(X^2+1)$ under the second map. It follows that in this new ring,
        \begin{align*}
            \bar{X}^2+1 &= [X+(X^2+1)]\cdot[X+(X^2+1)]+[1+(X^2+1)]\\
            &= [X^2+1]+(X^2+1)\\
            &= 0+(X^2+1)\\
            &= 0
        \end{align*}
        as desired.
        \item Additionally, the elements of this ring are of the form $a_0+a_1\bar{X}$ ($a_0,a_1\in\R$) by the above corollary. As per the rules of addition and multiplication in quotient rings, our addition and multiplication in this ring are
        \begin{align*}
            (a_0+a_1\bar{X})+(b_0+b_1\bar{X}) &= (a_0+b_0)+(a_1+b_1)\bar{X}\\
            (a_0+a_1\bar{X})\cdot(b_0+b_1\bar{X}) &= (a_0b_0-a_1b_1)+(a_0b_1+a_1b_0)\bar{X}
        \end{align*}
        \begin{itemize}
            \item For addition, we expect componentwise.
            \item For multiplication, we apply the distributive law, and then reduce our final element mod $X^2+1$ using the fact that $\bar{X}^2=-1$ so $a_1b_1\bar{X}^2=-a_1b_1$.
        \end{itemize}
        \item Thus, since they have isomorphic sets of elements and identical operations,
        \begin{equation*}
            \R[X]/(X^2+1) \cong \C
        \end{equation*}
        \item Note that $\R[X]/(X^2+1)\cong\R[i]$, where $i=\sqrt{-1}$. In other words, we can look at the elements of $\R[X]/(X^2+1)$ as complex numbers, or as polynomials in $i$. The two concepts are equivalent since any polynomial in $i$ reduces to a complex number via the $i$-cycle as follows.
        \begin{align*}
            \sum_{j=0}^\infty a_ji^j &= a_0+a_1i+a_2i^2+a_3i^3+a_4i^4+a_5i^5+\cdots\\
            &= a_0+a_1i-a_2-a_3i+a_4+a_5i-\cdots\\
            &= (a_0-a_2+a_4-\cdots)+(a_1-a_3+a_5-\cdots)i\\
            &= \left( \sum_{j=0}^\infty a_{2j} \right)+\left( \sum_{j=0}^\infty a_{2j+1} \right)i
        \end{align*}
    \end{itemize}
    \item However, this construction renders $\C$ as just one particular special case of interest in a far more general construction.
    \begin{itemize}
        \item Specifically, $\C$ is the special case that takes $f=X^2+1$ as the divisor.
    \end{itemize}
    \item Indeed, we may create a ring in which the root of any polynomial $f\in R[X]$ exists.
    \begin{itemize}
        \item For the sake of simplicity, let $f$ be monic of degree $d$. Let $A=R[X]/(f)$. Then as per the corollary, $R\hookrightarrow R[X]\twoheadrightarrow A$.
        \item Once again, we let $\bar{X}$ be the image of $X$ under the second map. $f(X)\mapsto f(\bar{X})=0$, as desired.
        \item In analogy to the last line above,
        \begin{equation*}
            R[X]/(f) \cong R[\bar{X}]
        \end{equation*}
        for any $\bar{X}$ satisfying $f(\bar{X})=0$.
    \end{itemize}
    \item Additional examples.
    \begin{enumerate}
        \item Take $R=\Z$, $f(X)=2$. Then $\Z\hookrightarrow\Z[X]\twoheadrightarrow\Z[X]/(2)$.
        \begin{itemize}
            \item $(2)$ is the set of all polynomials with even integer coefficients. Thus, any polynomial with even integer coefficients in $\Z[X]$ will be projected down to zero, and any polynomial containing any odd coefficients will correspond to a coset in which all polynomials with odd terms in the same places are lumped together.
            \item Essentially, reducing occurs termwise and is modulo 2 based on the coefficients. For example,
            \begin{equation*}
                5+2X+4X^2+7X^4+(2) = 1+1X^4+(2)
            \end{equation*}
            since $4+2X+4X^2+6X^4\in(2)$ and
            \begin{equation*}
                5+2X+4X^2+7X^4 = 1+1X^4+4+2X+4X^2+6X^4
            \end{equation*}
            \item Thus, $\Z[X]/(2)\cong\Z/2\Z[X]$.
            \item What is $\bar{X}$ in this set?? It must be some integer??
        \end{itemize}
        \item Take $R=\Z$ and $f(X)=2X+3$. Then we have $\Z[X]/(2X+3)$.
        \begin{itemize}
            \item $X\mapsto\bar{X}$ and $2\bar{X}+3=0$, so $\bar{X}=-3/2$.
            \item Just like $i\notin\R$, $-3/2\notin\Z$.
            \item We still have $\Z[X]/(2X+3)\cong\Z[-3/2]$.
            \begin{itemize}
                \item In other words, $\Z[X]/(2X+3)$ is the set of all "polynomials" in $-3/2$ with integer coefficients, which is just equal to
                \begin{equation*}
                    \{a/2^n\mid a\in 3\Z\}
                \end{equation*}
                which is the diadic rationals with numerator equal to a multiple of 3.
            \end{itemize}
            % \item In another more literal perspective, $\Z[X]/(2X+3)$ is the set of all remainders (which will be just numbers since $\deg(r)<\deg(2X+3)=1$) we get after dividing a polynomial with integer coefficients by $(2X+3)$. It just happens that the set of all such remainders is the diadic rationals.
            \item This construction will be integral to Spring Quarter.
        \end{itemize}
    \end{enumerate}
    \item Question/exercise: Let $\alpha\in R$. Then $R[X]/R[X]\alpha\cong(R/R\alpha)[X]$.
    \item Is it that dividing by a polynomial of degree 0 puts a constraint on the coefficients whereas dividing by a polynomial of degree greater than zero puts a constraint on the variable??
    \item \textbf{Principal ideal domain}: A commutative ring $R$ that is an integral domain and for which every ideal is principal. \emph{Also known as} \textbf{PID}.
    \item There is a useful explanation of something on Chapter 8, page 2 of \textcite{bib:DummitFoote}.
    \item Theorem: Let $F$ be a field. Then $F[X]$ is a PID.
    \begin{proof}
        We have proven previously that $F$ an integral domain implies $F[X]$ is an integral domain.\par
        Let $I\subset F[X]$ be a nonzero ideal. Let
        \begin{equation*}
            d = \min\{\deg(g)\mid g\in I,\ g\neq 0\}
        \end{equation*}
        Pick $g\in I$ such that $\deg(g)=d$. We have that $g=a_0+\cdots+a_dX^d$, $a_d\neq 0$, $a_d^{-1}\in F$. Let $f=a_d^{-1}g\in I$ (as guaranteed by the presence of $g\in I$). Let $h\in I$. Then the EA produces $q,r$ such that $h=qf+r$ with $\deg(r)<d$. We know that $h,f\in I$. Thus, $h-qf=I$. It follows by the definition of $d$ that $r=0$. Therefore, $h\in(f)$.
    \end{proof}
    \item Callum will lecture on Friday.
    \item Feedback on the HW.
    \begin{itemize}
        \item Most people seem to think that the HW is at a reasonable level of difficulty.
        \item The third one should be more challenging.
    \end{itemize}
\end{itemize}



\section{Rings of Fractions}
\begin{itemize}
    \item \marginnote{1/20:}This lecture will cover material from Sections 7.5 and 15.4 of \textcite{bib:DummitFoote}.
    \item Defining $\Q$.
    \begin{itemize}
        \item Rigorously, we define $\Q$ as a subset of $(\Z\times\Z)\setminus\{(a,0)\mid a\in\Z\}$. In particular, we let $\Q$ be the set of equivalence classes in $\Z\times\Z$ under the equivalence relation
        \begin{equation*}
            \frac{a}{b} = \frac{c}{d}
            \quad\Longleftrightarrow\quad
            ad-bc = 0
        \end{equation*}
        where $a/b$ denotes $(a,b)\in\Z\times\Z$.
        \item Addition on $\Q$:
        \begin{equation*}
            \frac{a_1}{b_1}+\frac{a_2}{b_2} = \frac{a_1b_2+a_2b_1}{b_1b_2}
        \end{equation*}
        \begin{itemize}
            \item This makes $(\Q,+)$ an abelian group with identity $0=0/c$ for any $c\neq 0$.
        \end{itemize}
        \item Multiplication on $\Q$:
        \begin{equation*}
            \frac{a_1}{b_1}\cdot\frac{a_2}{b_2} = \frac{a_1a_2}{b_1b_2}
        \end{equation*}
        \begin{itemize}
            \item This makes $(\Q,+,\cdot)$ a ring with identity $1=1/1=d/d$ for any $d\neq 0$.
        \end{itemize}
    \end{itemize}
    \item Notice the similarities between the above approach and the definition of $\C$ from $\R$ in Lecture 2.1.
    \item It follows from the definition that $\Q$ is also a field: For any $a/b\in\Q$, $a/b\cdot b/a=1$.
    \item We can generalize this construction to any commutative ring $R$.
    \begin{itemize}
        \item As in $\Q$, we may only be able to take the "quotient" of certain elements of $R$ by certain other elements of $R$. For example, $a/0$ does not make sense in $\Q$. Thus, we first define a subset of $R$ called $D$: $D$ contains elements which can act as \underline{d}enominators. The properties of $D$ are motivated by the properties of denominators in $\Q$. In particular\dots
        \item Let $D\subset R$ be such that $1_R\in D$, $0_R\notin D$, $D$ has no zero divisors, and $D$ is closed under multiplication (that is, $b,d\in D$ implies $bd\in D$).
        \begin{itemize}
            \item We need $1_R\in D$ so that all of the elements $a\in R$ appear in the related ring of fractions as $a/1_R$.
            \item We can't have $0_R\in D$ because you cannot divide by zero.
            \item We can't have any zero divisors in $D$ because then during addition or multiplication, as defined above, the sum or product could have zero in the denominator.
            \item We need closure under multiplication so that the sums and products defined above are well-defined.
        \end{itemize}
        \item With these constraints on $D$, we can define the \textbf{ring of fractions}.
    \end{itemize}
    \item $\bm{\sim}$: The equivalence relation on a product ring $(A\times B,+,\cdot)$ defined as follows. \emph{Given by}
    \begin{equation*}
        (a_1,b_1) \sim (a_2,b_2)
        \quad\Longleftrightarrow\quad
        a_1\cdot b_2-a_2\cdot b_1 = 0
    \end{equation*}
    \item Exercise: Confirm that $\sim$ is an equivalence relation.
    \item Just as taking the quotient of a group by a normal subgroup or a ring by an ideal yields a partition of the original object where all elements in any set in the partition are related by the substructure, taking the quotient of a set by an equivalence relation yields a partition of that set into classes called \emph{equivalence} classes.
    \begin{itemize}
        \item Thus, when we write $(A\times B)/\sim$, we refer to the set of equivalence classes of $A\times B$ under $\sim$.
    \end{itemize}
    \item \textbf{Ring of fractions} (of $D$ with respect to $R$): The set defined as follows, under the operations defined as follows. \emph{Denoted by} $\bm{D^{-1}R}$. \emph{Given by}
    \begin{equation*}
        D^{-1}R = \{(x,t)\mid x\in R,\ t\in D\}/\sim
    \end{equation*}
    \begin{enumerate}
        \item Addition:
        \begin{equation*}
            \frac{x_1}{t_1}+\frac{x_2}{t_2} = \frac{x_1t_2+x_2t_1}{t_1t_2}
        \end{equation*}
        \begin{itemize}
            \item Let $0_{D^{-1}R}=0/1$.
            \begin{itemize}
                \item Note that because of the way $0/1$ is defined (i.e., as an equivalence class), we no longer need to say $0/1=0/d$ for all $d\in D$ since all $0/d$ are included in $0/1$. In fact, at this point, $0/d$ is just an alternate name for the set $0/1$.
            \end{itemize}
            \item It follows from the above definition that $-(x/t)=-x/t$.
        \end{itemize}
        \item Multiplication:
        \begin{equation*}
            \frac{x_1}{t_1}\cdot\frac{x_2}{t_2} = \frac{x_1x_2}{t_1t_2}
        \end{equation*}
        \begin{itemize}
            \item Let $1_{D^{-1}R}=1/1$.
        \end{itemize}
    \end{enumerate}
    \item Notes on the ring of fractions.
    \begin{itemize}
        \item Notice how the notation is a nice alternative to the (already taken) $R/D$.
        \item Notation: Write $x/t$ for the equivalence class $[(x,t)]$.
    \end{itemize}
    \item Proposition: $D^{-1}R$ is a ring as defined above.
    \begin{proof}
        There are three steps needed: (1) check that $+,\times$ are well defined; (2) check that $(D^{-1}R,+)$ is an abelian group; and (3) check that $\times$ is an associative, commutative, and distributive operation with an identity.
    \end{proof}
    \item \textbf{Field of fractions} (of $R$): The set $D^{-1}R$ where $R$ is an integral domain and $D=R\setminus\{0\}$. \emph{Also known as} \textbf{quotient field}. \emph{Denoted by} $\bm{\Frac R}$.
    \begin{itemize}
        \item Inverses are given by
        \begin{equation*}
            \left( \frac{a}{b} \right)^{-1} = \frac{b}{a}
        \end{equation*}
        for all nonzero elements $a/b\in\Frac R$ (i.e., all elements for which $a,b\neq 0$).
    \end{itemize}
    \item Example: Let $R$ be an integral domain, and let $f\in R$ not be nilpotent. Take $D=\{1,f,f^2,\dots\}$. Then $R_f=D^{-1}R$.
    \begin{itemize}
        \item Example: If $R=\Z$ and $f=2$, then $R_2=\{a/b\in\Q\mid b=2^n\}$. Recall that these are the diadic rationals.
    \end{itemize}
    \item Example: Let $R=\Z$ and $D=\{a\in\Z:2\nmid a\}$. Then $D^{-1}R=\{a/b\in\Q:2\nmid b\}$.
    \item Besides the last two examples, the only nontrivial ideal of $\Q$ left is $(2^n)$.
    \begin{itemize}
        \item Do I have this statement right??
    \end{itemize}
    \item If $R$ is an integral domain, then $\Frac(R[X])$ is the set of all rational functions with coefficients in $R$.
    \item We have a canonical injection $\iota:R\to D^{-1}R$ defined by $x\mapsto x/1$.
    \item Theorem (universal property of the ring of fractions):
    \begin{figure}[H]
        \centering
        \begin{tikzpicture}[scale=1.6]
            \small
            \node (R)  at (0,1) {$R$};
            \node (DR) at (0,0) {$D^{-1}R$};
            \node (S)  at (1,1) {$S$};
    
            \footnotesize
            \draw [right hook->] (R)  -- node[left]            {$\iota$}           (DR);
            \draw [->]           (DR) -- node[below right=-2pt]{$\tilde{\varphi}$} (S);
            \draw [->]           (R)  -- node[above]           {$\varphi$}         (S);
        \end{tikzpicture}
        \caption{Decomposition of a ring homomorphism using $D^{-1}R$.}
        \label{fig:fracDecomp}
    \end{figure}
    \begin{enumerate}[label={(\arabic*)}]
        \item $\iota:R\to D^{-1}R$ is an injective ring homomorphism.
        \item If $\varphi:R\to S$ is a ring homomorphism such that $\varphi(r)$ is a unit in $S$ for all $r\in D$, then there exists a unique ring homomorphism $\tilde{\varphi}:D^{-1}R\to S$ such that $\tilde{\varphi}\circ\iota=\varphi$ (see Figure \ref{fig:fracDecomp}).
        \item If $\varphi$ is injective, then so is $\tilde{\varphi}$.
    \end{enumerate}
    \begin{proof}
        (1) is easy.\par\smallskip
        We address (2) in two parts.\par
        \underline{Existence}: Define $\tilde{\varphi}(x/t)=\varphi(x)\varphi(t)^{-1}$.\par
        \underline{Uniqueness}: Suppose that there exists $\rho:D^{-1}R\to S$ such that $\rho\circ\iota=\varphi$. Then $\varphi(x)=(\rho\circ\iota)(x)=\rho(x/1)$. This result combined with the fact that $\rho$ is a ring homomorphism implies that
        \begin{equation*}
            1 = \rho(\tfrac{1}{1})
            = \rho(\tfrac{t}{1})\rho(\tfrac{1}{t})
            = \varphi(t)\rho(\tfrac{1}{t})
        \end{equation*}
        It follows since $\varphi(D)\subset S^\times$ by hypothesis that if $t\in D$, then $\rho(1/t)=\varphi(t)^{-1}$. Therefore,
        \begin{equation*}
            \rho(\tfrac{x}{t}) = \rho(\tfrac{x}{1})\rho(\tfrac{1}{t})
            = \varphi(x)\varphi(t)^{-1}
            = \tilde{\varphi}(\tfrac{x}{t})
        \end{equation*}\smallskip
        We now address (3).\par
        Suppose that $\varphi$ is injective. To prove that $\tilde{\varphi}$ is injective, it will suffice to show that $\ker\tilde{\varphi}=0$. Let $x/t\in\ker\tilde{\varphi}$ be arbitrary. Then $\tilde{\varphi}(\tfrac{x}{t})=0$. It follows by the definition of $\tilde{\varphi}$ that $\varphi(x)\varphi(t)^{-1}=0$. Since $\varphi(t)$ is a unit by hypothesis and hence nonzero, it must be that $\varphi(x)=0$. Additionally, as a ring homomorphism, $\varphi(0)=0$. Combining the last two results, we have by transitivity that $\varphi(x)=\varphi(0)$. Thus, since $\varphi$ is injective, $x=0$. It follows that $x/t=0/t$, so $\ker\tilde{\varphi}=0$, as desired.
    \end{proof}
\end{itemize}



\section{Chapter 7: Introduction to Rings}
\emph{From \textcite{bib:DummitFoote}.}
\subsection*{Section 7.5: Rings of Fractions}
\begin{itemize}
    \item \marginnote{1/30:}Let $R$ be a \emph{commutative} ring throughout this section.
    \item Review of how zero divisors are similar to units in some ways and dissimilar in other ways.
    \item "The aim of this section is to prove that a commutative ring $R$ is always a subring of a larger ring $Q$ in which every nonzero element of $R$ that is not a zero divisor is a unit in $Q$" \parencite[260]{bib:DummitFoote}.
    \begin{itemize}
        \item If $R$ is an integral domain, $Q$ will be its \textbf{field of fractions} or \textbf{quotient field}.
    \end{itemize}
    \item Review of the construction and properties of $\Q$.
    \item Why we can't include zeroes or zero divisors in the denominators.
    \begin{itemize}
        \item Suppose $b$ is a zero or zero divisor such that $bd=0$.
        \item If we allow $b$ as a denominator, then
        \begin{equation*}
            d = \frac{d}{1}
            = \frac{bd}{d}
            = \frac{0}{b}
            = 0
        \end{equation*}
        \item Thus, there is a certain "collapsing," and we cannot expect that $R$ appears as a natural subring of this "ring of fractions."
    \end{itemize}
    \item Why we must have closure under multiplication for the denominators.
    \begin{itemize}
        \item Review from class.
    \end{itemize}
    \item "The main result of this section shows that these two restrictions are sufficient to construct a ring of fractions for $R$. Note that this theorem includes the construction of $\Q$ from $\Z$ as a special case" \parencite[261]{bib:DummitFoote}.
    \begin{theorem}\label{trm:7.15}
        Let $R$ be a commutative ring. Let $D$ be any nonempty subset of $R$ that does not contain 0, does not contain any zero divisors, and is closed under multiplication (i.e., $ab\in D$ for all $a,b\in D$). Then there is a commutative ring $Q$ with $1$ such that $Q$ contains $R$ as a subring and every element of $D$ is a unit in $Q$. The ring $Q$ has the following additional properties.
        \begin{enumerate}
            \item Every element of $\Q$ is of the form $rd^{-1}$ for some $r\in R$ and $d\in D$. In particular, if $D=R\setminus\{0\}$, then $Q$ is a field.
            \item (Uniqueness of $Q$) The ring $Q$ is the "smallest" ring containing $R$ in which all elements of $D$ become units in the following sense. Let $S$ be any commutative ring with identity and let $\varphi:R\to S$ be any injective ring homomorphism such that $\varphi(d)$ is a unit in $S$ for every $d\in D$. Then there is an injective homomorphism $\Phi:Q\to S$ such that $\Phi|_R=\varphi$. In other words, any ring containing an isomorphic copy of $R$ in which all the elements of $D$ become units must also contain an isomorphic copy of $Q$.
        \end{enumerate}
        \begin{proof}
            Given.\par
            Same as in class: A general construction of $Q$, confirmation of its properties, and then the steps of the analogous theorem. Very well written, though, should I need additional insight in the future!
        \end{proof}
    \end{theorem}
    \item Theorem 36 in Section 15.4 generalizes Theorem \ref{trm:7.15} by allowing $D$ to contain zero and/or zero divisors.
    \item Definition of the \textbf{ring of fractions} and \textbf{field of fractions}.
    \item \textbf{Subfield generated by $\bm{A}$}: The subfield of $F$ equal to the intersection of all subfields of $F$ containing $A$, where $A$ is some subset of a field $F$.
    \item The subfield generated by $A$ is the smallest subfield of $F$ containing $A$.
    \item The smallest field containing an integral domain $R$ is its field of fractions.
    \begin{corollary}\label{cly:7.16}
        Let $R$ be an integral domain and let $Q$ be the field of fractions of $R$. If a field $F$ contains a subring $R'$ isomorphic to $R$, then the subfield of $F$ generated by $R'$ is isomorphic to $Q$.
        \begin{proof}
            Given.
        \end{proof}
    \end{corollary}
    \item Examples.
    \begin{enumerate}
        \item $\Frac F\cong F$ for any field $F$.
        \item $\Frac\Z=\Q$.
        \begin{itemize}
            \item Quadratic integer rings from Section 7.1 are brought up again.
        \end{itemize}
        \item $\Frac(2\Z)=\Q$.
        \begin{itemize}
            \item Notice how an identity "appears" in the field of fractions.
        \end{itemize}
        \item The \textbf{rational functions}.
        \begin{itemize}
            \item $\Frac(R[X])$ contains $\Frac(R)$.
            \item $\Frac(R[X])=\Frac(R)(X)$.
            \begin{itemize}
                \item Example: We have that
                \begin{equation*}
                    \Frac(\Z[X]) = \Frac(\Q[X])
                    = \Q(X)
                    = \Frac(\Z)(X)
                \end{equation*}
                \item We can easily see this since if $p(X)/q(X)\in\Frac(\Q[X])$, then there exists $N\in\Z$ such that $Np(X),Nq(X)$ both have integer coefficients (pick, for example, $N$ to be the common denominator of all the coefficients in $p(X),q(X)$). Then $p(X)/q(X)=Np(X)/Nq(X)\in\Frac(\Z[X])$, as desired.
            \end{itemize}
        \end{itemize}
        \item $R_d=R[1/d]=D^{-1}R$, where $D=\{1,d,d^2,d^3,\dots\}$.
    \end{enumerate}
    \item \textbf{Rational functions} (in $X$ over $R$): The field of fractions of the polynomial ring $R[X]$, where $R$ is an integral domain and hence $R[X]$ is an integral domain. \emph{Denoted by} $\bm{\Frac(R[X])}$.
    \item \textbf{Field of rational functions}: The rational functions in $X$ over a field $F$. \emph{Denoted by} $\bm{F(x)}$.
\end{itemize}




\end{document}