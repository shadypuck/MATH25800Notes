\documentclass[../notes.tex]{subfiles}

\pagestyle{main}
\renewcommand{\chaptermark}[1]{\markboth{\chaptername\ \thechapter\ (#1)}{}}
\setcounter{chapter}{2}
\setcounter{proposition}{14}

\begin{document}




\chapter{???}
\section{Intro to Chapters 8-9}
\begin{itemize}
    \item \marginnote{1/18:}Moving onto Chapter 8 today.
    \item Friday: Rings of fractions (more than what's in the book; under lesser hypotheses).
    \begin{itemize}
        \item Def get notes!
    \end{itemize}
    \item The Chinese Remainder Theorem is at least partially in HW3.
    \item Today: A leisurely introduction to Chapter 8, as well as Spring Quarter content (which is the most interesting part of the Honors Algebra sequence).
    \item For the next three weeks or more, all rings will be assumed to be commutative.
    \begin{itemize}
        \item Excepting matrix rings, which may still appear in exercises.
    \end{itemize}
    \item At this point, we define $\deg(f)=-\infty$ where $f$ is the zero polynomial.
    \begin{itemize}
        \item We do this so that $\deg(fg)=\deg(f)+\deg(g)$ still holds.
    \end{itemize}
    \item Euclidean algorithm for monic polynomials: Let $f\in R[X]$ be a monic polynomial of degree $d\geq 0$, and let $h\in R[X]$. Then there exists a unique pair $q,r\in R[X]$ such that\dots
    \begin{enumerate}
        \item $h=qf+r$;
        \item $\deg(r)<\deg(f)$.
    \end{enumerate}
    \begin{proof}
        We tackle uniqueness first, and then existence.\par
        \underline{Uniqueness}: Suppose $h=q_1f+r_1=q_2f+r_2$, where $\deg(r_i)<d$ ($i=1,2$). We have that
        \begin{equation*}
            (q_1-q_2)f = q_1f-q_2f = r_2-r_1
        \end{equation*}
        Now suppose for the sake of contradiction that $q_1-q_2\neq 0$. We know that
        \begin{equation*}
            \deg(r_2-r_1) = \deg[(q_1-q_2)f]
            = \deg(q_1-q_2)+d
            \geq d
        \end{equation*}
        But since $\deg(r_i)<d$ ($i=1,2$), we have that $\deg(r_2-r_1)<d$, a contradiction. Thus, $q_1-q_2=0$. It follows easily that $0=r_2-r_1$. Therefore, $(q_1,r_1)=(q_2,r_2)$, as desired.\par
        \underline{Existence}: If $\deg(h)<d$, then put $q=0$ and $r=h$. We now induct on $\deg(h)$, starting from $d$. Our base case is already taken care of via the statement on $\deg(h)<d$. Now suppose using strong induction that we have proven the claim for all nonnegative integers $n<\deg(h)$. Let
        \begin{equation*}
            h(X) = a_0+\cdots+a_eX^e
        \end{equation*}
        where $a_e\neq 0$ and $e\geq d$ by hypothesis. Let
        \begin{equation*}
            f(X) = b_0+\cdots+b_{d-1}X^{d-1}+X^d
        \end{equation*}
        Define $g(X)$ by
        \begin{equation*}
            g = h-a_eX^{e-d}f
        \end{equation*}
        It follows that $\deg(g)<e$, so we may apply the induction hypothesis at this point. We learn from it that there exist $q,r$ such that $g=qf+r$ with $\deg(r)<d$. Therefore, we can deduce that
        \begin{equation*}
            h = (a_eX^{e-d}+q)f+r
        \end{equation*}
        as desired.
    \end{proof}
    \item Notes on the Euclidean algorithm.
    \begin{itemize}
        \item Think long polynomial division from high school.
    \end{itemize}
    \item Example.
    \begin{itemize}
        \item Let $a\in R$ and $f=X-a$ be a monic polynomial. Let $h\in R[X]$ be arbitrary. Then applying the theorem,
        \begin{equation*}
            h(X) = q(X)(X-a)+r
        \end{equation*}
        \item $\deg(r)<1=\deg(f)$ implies that $r$ is a constant, and hence $r\in R$.
        \item Moreover,
        \begin{align*}
            h(a) &= q(a)(a-a)+r\\
            r &= h(a)
        \end{align*}
        implying that
        \begin{equation*}
            h(X)-h(a) = q(X)(X-a)
        \end{equation*}
        for arbitrary polynomials $h$.
    \end{itemize}
    \item Corollary: Let $a\in R$. $\{h\in R[X]\mid h(a)=0\}$ is the \textbf{principal ideal generated by $\bm{X-a}$}.
    \item \textbf{Ideal generated by $\bm{b\in B}$}. \emph{Denoted by} $\bm{Bb}$, $\bm{(b)}$.
    \item Corollary: Let $f\in R[X]$ be monic of degree $d$. Then
    \begin{equation*}
        \{g\in R[X]\mid \deg(g)<d\} \hookrightarrow R[X]
        \twoheadrightarrow R[X]/(f)
    \end{equation*}
    and, in particular,
    \begin{equation*}
        \{g\in R[X]\mid \deg(g)<d\} \cong R[X]/(f)
    \end{equation*}
    as groups (in particular, \emph{not} as rings).
    \begin{proof}
        % One-to-one is obvious.\par
        % We know that if $h\in(f)$, then either $h=0$ or $\deg(h)\geq\deg(f)$ (since $f$ is monic). Injective: $\ker(\text{composite})=0$. The composite is the bijective map, i.e., the one after the injective map. Onto: Every element of $R[X]/(f)$ is the image of some $h\in R[X]$. $h=qf+r$, $h$ and $h-qf$ gives rise to some element of $R[X]/(f)$. Note that $h-qf=r$.

        The existence of the first two maps is obvious (they are just instances of the canonical injection and surjection, respectively).\par\smallskip
        We now verify that the last two sets are in bijective correspondence. Define a map $\varphi$ between them via the canonical surjection (note that since the domain of $\varphi$ is not $R[X]$, we will still have to verify surjectivity here). As established previously, $\varphi$ is well defined.\par
        To prove that $\varphi$ is injective, it will suffice to show that $\ker\varphi=0$. Let $h$ be an arbitrary polynomial in $R[X]$ with $\deg(h)<d$. Suppose $\varphi(h)=0=0+(f)=(f)$. Then $h\in(f)$. It follows that either $h=0$ or $\deg(h)\geq\deg(f)=d$. But as an element of the domain $\deg(h)<d$ by hypothesis. Therefore, $h=0$, as desired.\par
        To prove that $\varphi$ is surjective, it will suffice to show that for every $h+(f)\in R[X]/(f)$, there exists $r\in R[X]$ with $\deg(r)<d$ such that $\varphi(r)=h+(f)$. Let $h+(f)\in R[X]/(f)$ be arbitrary. By the Euclidean algorithm, $h=qf+r$ for some $q,r\in R[X]$ where $\deg(r)<\deg(f)=d$. Moreover, since $r=h+(-q)f$, $r\in h+(f)$ and hence $h+(f)=r+(f)$. Therefore, since $r$ is in the domain of $\varphi$ (as it has degree less than $d$), $\varphi(r)=r+(f)=h+(f)$, as desired.
    \end{proof}
    \item $R[X]$ is also a vector space with $1,X,X^2,\dots$ as the basis.
    \item We have that
    \begin{equation*}
        \{g\in R[X]\mid \deg(g)<d\} = \{a_0+\cdots+a_{d-1}X^{d-1}\mid a_0,\dots,a_{d-1}\in R\}
    \end{equation*}
    \begin{itemize}
        \item As an abelian group (ignoring multiplication), this set is group isomorphic to $(R^d,+)$.
    \end{itemize}
    \item Revisiting the creation of $\C$ from $\R$.
    \begin{itemize}
        \item We can use quotient rings to solve $X^2+1=0$.
        \item In particular, the equation $X^2+1=0$ does not have a solution in $\R[X]$. However, it does have a solution in $\R[X]/(X^2+1)$, as we will see presently.
        \item Consider the function described in the above corollary, sending $\R\hookrightarrow\R[X]\twoheadrightarrow\R[X]/(X^2+1)$. Let $\bar{X}:=X+(X^2+1)\in\R[X]/(X^2+1)$ denote the image of $X$ in $\R[X]/(X^2+1)$ under the second map. It follows that in this new ring,
        \begin{align*}
            \bar{X}^2+1 &= [X+(X^2+1)]\cdot[X+(X^2+1)]+[1+(X^2+1)]\\
            &= [X^2+1]+(X^2+1)\\
            &= 0+(X^2+1)\\
            &= 0
        \end{align*}
        as desired.
        \item Additionally, the elements of this ring are of the form $a_0+a_1\bar{X}$ ($a_0,a_1\in\R$) by the above corollary. As per the rules of addition and multiplication in quotient rings, our addition and multiplication in this ring are
        \begin{align*}
            (a_0+a_1\bar{X})+(b_0+b_1\bar{X}) &= (a_0+b_0)+(a_1+b_1)\bar{X}\\
            (a_0+a_1\bar{X})\cdot(b_0+b_1\bar{X}) &= (a_0b_0-a_1b_1)+(a_0b_1+a_1b_0)\bar{X}
        \end{align*}
        \begin{itemize}
            \item For addition, we expect componentwise.
            \item For multiplication, we apply the distributive law, and then reduce our final element mod $X^2+1$ using the fact that $\bar{X}^2=-1$ so $a_1b_1\bar{X}^2=-a_1b_1$.
        \end{itemize}
        \item Thus, since they have isomorphic sets of elements and identical operations,
        \begin{equation*}
            \R[X]/(X^2+1) \cong \C
        \end{equation*}
        \item Note that $\R[X]/(X^2+1)\cong\R[i]$, where $i=\sqrt{-1}$. In other words, we can look at the elements of $\R[X]/(X^2+1)$ as complex numbers, or as polynomials in $i$. The two concepts are equivalent since any polynomial in $i$ reduces to a complex number via the $i$-cycle as follows.
        \begin{align*}
            \sum_{j=0}^\infty a_ji^j &= a_0+a_1i+a_2i^2+a_3i^3+a_4i^4+a_5i^5+\cdots\\
            &= a_0+a_1i-a_2-a_3i+a_4+a_5i-\cdots\\
            &= (a_0-a_2+a_4-\cdots)+(a_1-a_3+a_5-\cdots)i\\
            &= \left( \sum_{j=0}^\infty a_{2j} \right)+\left( \sum_{j=0}^\infty a_{2j+1} \right)i
        \end{align*}
    \end{itemize}
    \item However, this construction renders $\C$ as just one particular special case of interest in a far more general construction.
    \begin{itemize}
        \item Specifically, $\C$ is the special case that takes $f=X^2+1$ as the divisor.
    \end{itemize}
    \item Indeed, we may create a ring in which the root of any polynomial $f\in R[X]$ exists.
    \begin{itemize}
        \item For the sake of simplicity, let $f$ be monic of degree $d$. Let $A=R[X]/(f)$. Then as per the corollary, $R\hookrightarrow R[X]\twoheadrightarrow A$.
        \item Once again, we let $\bar{X}$ be the image of $X$ under the second map. $f(X)\mapsto f(\bar{X})=0$, as desired.
        \item In analogy to the last line above,
        \begin{equation*}
            R[X]/(f) \cong R[\bar{X}]
        \end{equation*}
        for any $\bar{X}$ satisfying $f(\bar{X})=0$.
    \end{itemize}
    \item Additional examples.
    \begin{enumerate}
        \item Take $R=\Z$, $f(X)=2$. Then $\Z\hookrightarrow\Z[X]\twoheadrightarrow\Z[X]/(2)$.
        \begin{itemize}
            \item $(2)$ is the set of all polynomials with even integer coefficients. Thus, any polynomial with even integer coefficients in $\Z[X]$ will be projected down to zero, and any polynomial containing any odd coefficients will correspond to a coset in which all polynomials with odd terms in the same places are lumped together.
            \item Essentially, reducing occurs termwise and is modulo 2 based on the coefficients. For example,
            \begin{equation*}
                5+2X+4X^2+7X^4+(2) = 1+1X^4+(2)
            \end{equation*}
            since $4+2X+4X^2+6X^4\in(2)$ and
            \begin{equation*}
                5+2X+4X^2+7X^4 = 1+1X^4+4+2X+4X^2+6X^4
            \end{equation*}
            \item Thus, $\Z[X]/(2)\cong\Z/2\Z[X]$.
            \item What is $\bar{X}$ in this set?? It must be some integer??
        \end{itemize}
        \item Take $R=\Z$ and $f(X)=2X+3$. Then we have $\Z[X]/(2X+3)$.
        \begin{itemize}
            \item $X\mapsto\bar{X}$ and $2\bar{X}+3=0$, so $\bar{X}=-3/2$.
            \item Just like $i\notin\R$, $-3/2\notin\Z$.
            \item We still have $\Z[X]/(2X+3)\cong\Z[-3/2]$.
            \begin{itemize}
                \item In other words, $\Z[X]/(2X+3)$ is the set of all "polynomials" in $-3/2$ with integer coefficients, which is just equal to
                \begin{equation*}
                    \{a/2^n\mid a\in 3\Z\}
                \end{equation*}
                which is the diadic rationals with numerator equal to a multiple of 3.
            \end{itemize}
            % \item In another more literal perspective, $\Z[X]/(2X+3)$ is the set of all remainders (which will be just numbers since $\deg(r)<\deg(2X+3)=1$) we get after dividing a polynomial with integer coefficients by $(2X+3)$. It just happens that the set of all such remainders is the diadic rationals.
            \item This construction will be integral to Spring Quarter.
        \end{itemize}
    \end{enumerate}
    \item Question/exercise: Let $\alpha\in R$. Then $R[X]/R[X]\alpha\cong(R/R\alpha)[X]$.
    \item Is it that dividing by a polynomial of degree 0 puts a constraint on the coefficients whereas dividing by a polynomial of degree greater than zero puts a constraint on the variable??
    \item \textbf{Principal ideal domain}: A commutative ring $R$ that is an integral domain and for which every ideal is principal. \emph{Also known as} \textbf{PID}.
    \item There is a useful explanation of something on Chapter 8, page 2 of \textcite{bib:DummitFoote}.
    \item Theorem: Let $F$ be a field. Then $F[X]$ is a PID.
    \begin{proof}
        We have proven previously that $F$ an integral domain implies $F[X]$ is an integral domain.\par
        Let $I\subset F[X]$ be a nonzero ideal. Let
        \begin{equation*}
            d = \min\{\deg(g)\mid g\in I,\ g\neq 0\}
        \end{equation*}
        Pick $g\in I$ such that $\deg(g)=d$. We have that $g=a_0+\cdots+a_dX^d$, $a_d\neq 0$, $a_d^{-1}\in F$. Let $f=a_d^{-1}g\in I$ (as guaranteed by the presence of $g\in I$). Let $h\in I$. Then the EA produces $q,r$ such that $h=qf+r$ with $\deg(r)<d$. We know that $h,f\in I$. Thus, $h-qf=I$. It follows by the definition of $d$ that $r=0$. Therefore, $h\in(f)$.
    \end{proof}
    \item Callum will lecture on Friday.
    \item Feedback on the HW.
    \begin{itemize}
        \item Most people seem to think that the HW is at a reasonable level of difficulty.
        \item The third one should be more challenging.
    \end{itemize}
\end{itemize}




\end{document}