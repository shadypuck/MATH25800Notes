\documentclass[../notes.tex]{subfiles}

\pagestyle{main}
\renewcommand{\chaptermark}[1]{\markboth{\chaptername\ \thechapter\ (#1)}{}}
\setcounter{chapter}{7}

\begin{document}




\chapter{???}
\section{Linear Algebra Review and Rational Canonical Form}
\begin{itemize}
    \item \marginnote{2/20:}Nori's change of heart.
    \begin{itemize}
        \item We've all seen linear algebra; thus, we'll speedrun it and then do exterior algebra and determinants. That's where we'll finish.
    \end{itemize}
    \item The following is part 1 of a linear algebra course.
    \item Let $F$ be a field.
    \item \textbf{Vector space}: An $F$-module.
    \item \textbf{Linearly independent} (subset $S\subset V$): Same definition we're familiar with.
    \item \textbf{Spanning} (subset $S\subset V$): A subset $S$ of $V$ that is a set of generators of $V$.
    \item $S$ is a \textbf{basis} implies that $S$ generates $V$ and is linearly independent.
    \item Every linearly independent subset of $V$ can be extended to a basis.
    \item Every spanning set $S$ contains a basis.
    \begin{itemize}
        \item Any maximal linearly independent subset of $S$ is a basis.
    \end{itemize}
    \item $S_1,S_2$ are bases for $V$ implies that $|S_1|=|S_2|$.
    \begin{itemize}
        \item The replacement theorem in \textcite{bib:DummitFoote} is a good way to prove this.
    \end{itemize}
    \item We are now done with part 1; this is part 2 of a linear algebra course.
    \item Let $T:V\to V$ be a linear transformation.
    \item Let $A$ be a ring. What is an $A[X]$-module $M$?
    \begin{itemize}
        \item It is an abelian group $(M,+)$ and a ring homomorphism $\rho:A[X]\to\End(M,+)$.
        \item Since $A\hookrightarrow A[X]$, $\rho|_A$ turns $M$ into an $A$-module.
        \item Since $aX=Xa$, $\rho(a)\rho(X)=\rho(X)\rho(a)$.
        \item But since we consider $M$ to be a module, we write $a:=\rho(a)$: Thus, $a\rho(X)m=\rho(X)am$ for all $m\in M$.
        \item Note that $\rho(X)\in\End_A(M)$ (which is the set of all $A$-module homomorphisms).
        \item Additionally, $\rho(X):M\to M$ is an $A$-module homomorphism.
    \end{itemize}
    \item Put $\rho(X)=T$. Thus, an $A[X]$-module is a pair $(M,T)$, where $M$ is an $A$-module and $T\in\End_A(M)$.
    \item Conversely, such $(M,T)$ gives rise to an $A[X]$-module.
    \begin{itemize}
        \item In particular, the action is
        \begin{equation*}
            \left( \sum_{n=0}^\ell a_nX^n \right)m = \sum_{n=0}^\ell a_nT^nm
        \end{equation*}
    \end{itemize}
    \item Take $A=F$ a field concerned with $(V,T)$ where $V$ is any $F$-vector space and $T:V\to V$ is a linear transformation.
    \begin{itemize}
        \item This induces a module over $F[X]$.
    \end{itemize}
    \item $V$ finite dimensional induces $\rho:F[X]\to\End_F(V)\cong M_n(F)$ defined by $X\mapsto T$.
    \begin{itemize}
        \item $\rho(X)=T$ and $\rho(c)=c$ for all $c\in F$.
    \end{itemize}
    \item Let $n^2$ be the dimension of the $F$-vector space??
    \item Then $\ker(\rho)=(f)$ for some $f$ be monic of degree $d\leq n^2$.
    \begin{figure}[H]
        \centering
        \begin{tikzpicture}[xscale=2,yscale=1.6]
            \small
            \node (R)  at (0,1) {$F[X]$};
            \node (DR) at (0,0) {$F[X]/(f)$};
            \node (S)  at (1,1) {$\End_F(V)$};
    
            \footnotesize
            \draw [->]           (R)  --                                     (DR);
            \draw [right hook->] (DR) -- node[above left=-2pt]{$\bar{\rho}$} (S);
            \draw [->]           (R)  -- node[above]          {$\rho$}       (S);
        \end{tikzpicture}
        \caption{$F[X]$-module actions.}
        \label{fig:fXmodAction}
    \end{figure}
    \begin{itemize}
        \item We have the constraint on the degree of $f$ by the isomorphism from Lecture 3.1.
    \end{itemize}
    \item \textbf{Minimal polynomial} (of $T$): The polynomial $f$ that generates $\ker(\rho)$.
    \begin{itemize}
        \item In particular, $V$ is a torsion $F[X]$-module $(f\cdot g)$.
    \end{itemize}
    \item \textbf{Cyclic vector}: A vector $v\in V$ belonging to $(V,T)$ such that $v,Tv,T^2v,\dots$ spans $V$.
    \item Assume $v,Tv,T^2v,\dots,T^{k-1}v$ are linearly independent, but $v,Tv,\dots,T^kv$ are not.
    \begin{itemize}
        \item Then
        \begin{equation*}
            T^kv = a_0v+a_1Tv+\cdots+a_{k-1}T^{k-1}v
        \end{equation*}
        where all $a_i\in F$ and not all $a_i=0$.
        \item It follows that $T^mk\in\gen{v,Tv,\dots,T^{k-1}v}=W$ a vector space.
        \item Let $g(X)=X^k-(a_{k-1}X^{k-1}+\cdots+a_1X+a_0)$. Then $g(T)v=0$. This implies that $g$ is the minimal polynomial of $T$.
        \item It follows that $T^hg(T)v=0$. Thus, $g(T)T^hv=0$ for all $h$.
        \item Lastly, it follows that $g(T)w=0$ for all $w\in W$.
        \item Assume $v$ is a cyclic vector. Then $W=V$. It follows that $g(T)v=0$ for all $v\in V$.
        \item The original assumption posits that no polynomial of degree less than or equal to $k-1$ can annihilate $v$.
    \end{itemize}
    \item Consider $V=F[X]/(f)$. Let $\deg(f)=d$, let $T:V\to V$, and let $T$ be the "multiply by $X$" linear transformation. It follows that if $v_i=\overline{X^{i-1}}$ ($i=1,\dots,d$), then
    \begin{equation*}
        Tv_i = v_{i+1}
    \end{equation*}
    for $i=1,\dots,d-1$ and
    \begin{equation*}
        Tv_d = -(a_0v_1+a_1v_2+\cdots+a_{d-1}v_d)
    \end{equation*}
    \begin{itemize}
        \item If $d=3$, then we have
        \begin{equation*}
            M(T) =
            \begin{pmatrix}
                0 & 0 & -a_0\\
                1 & 0 & -a_1\\
                0 & 1 & -a_2\\
            \end{pmatrix}
        \end{equation*}
        \item The above matrix is called the \textbf{companion matrix} of $f$ (monic of degree 3).
    \end{itemize}
    \item \textbf{Rational canonical form}: The form $(V,T)$ given by
    \begin{equation*}
        F[X]/(f_1)\oplus\cdots\oplus F[X]/(f_s)
    \end{equation*}
    where $f_2\mid f_1,\dots,f_s\mid f_{s-1}$ and $\deg(f_s)>0$.
    \begin{itemize}
        \item When $V=0$, then $s=0$. In this case, $f_1$ is the minimal polynomial of $T$.
        \item The form consisting of a block diagonal matrix of companion matrices.
    \end{itemize}
    \item \textbf{Jordan canonical form}:
    \begin{itemize}
        \item Has to do with $p$-primary components!
    \end{itemize}
    \item There's one more canonical form, too.
    \item Since no one knows what canonical forms are and we very much need them for what Nori was planning to do, Nori will change his plans. No tensors in the last week, either.
    \item $p$-primary components: When $p=X-a$, $a\in F$.
    \item $(V,T)$ is \textbf{$p$-primary} if there exists an $n$ such that $(T-a)^nv=0$ for all $v\in V$.
    \item $1_V:V\to V$ is the identity.
    \item $a\cdot 1_V=a_V:V\to V$.
    \item $(T-a_v)^n=0\in\End_F(V)$.
    \item We're now doing generalized eigenspaces ?? lol.
    \item The $p$-primary component is as the generalized $a$-eigenspace.
    \begin{itemize}
        \item $(T-a)v=0$, i.e., $Tv=av$ is the $a$-eigenspace; the eigenspaces are components of the generalized eigenspaces.
    \end{itemize}
    \item Let $V=F[X]/(X-a)^n$. Let $v_1=1$, $v_2=\overline{X-a}$, \dots, $v_n=\overline{(X-a)^{n-1}}$.
    \begin{itemize}
        \item We know that $X(X-a)^r=(X-a+a)(X-a)^r=(X-a)^{r+1}+a(X-a)^r$.
        \item Nori writes Jordan blocks as
        \begin{equation*}
            \begin{pmatrix}
                a & 0 & 0 & 0\\
                1 & a & 0 & 0\\
                0 & 1 & a & 0\\
                0 & 0 & 1 & a\\
            \end{pmatrix}
        \end{equation*}
        not with 1's in the superdiagonal.
        \begin{itemize}
            \item Thus, the \emph{last} generalized eigenvector is an eigenvector here, instead of the \emph{first}.
        \end{itemize}
    \end{itemize}
\end{itemize}



\section{Office Hours (Nori)}
\begin{itemize}
    \item Midterm: We never covered the universal property of a quotient in class, did we?
    \begin{itemize}
        \item That's the special lemma from last office hours.
    \end{itemize}
    \item PSet 7: 7.3 and 7.4 typos.
    \item Wednesday lecture?
    \begin{itemize}
        \item Seventh week summary will suffice.
    \end{itemize}
    \item What do you need us to know about the rational canonical form? Should I still read \textcite{bib:DummitFoote}, Section 12.2 or is that no longer necessary?
    \begin{itemize}
        \item Nori will probably push ahead with 12.2. Thus I should read it. He's not sure what he'll do beyond that, though, since he doesn't want to jam tensors into the last week.
        \item I will need tensor products for representation theory, regardless, so if I want to take it, I should self-study it.
        \item No chance tensor products will be covered next quarter.
        \item Serre is a terrific mathematician whose wife is a super chemist, and that's why he wrote his book on representation theory (and wrote it in a less terse manner than usual).
        \item No tensors means no exterior algebra, too.
        \item Nori hasn't read any of \textcite{bib:DummitFoote}.
        \item The transfer theory of groups arises in a later chapter, and that's important for representation theory, though.
    \end{itemize}
    \item Nori doesn't think any teacher pays attention to what courses are supposed to cover as stated in the course catalog.
    \begin{itemize}
        \item We will never do modules, multilinear and quadratic forms.
        \item $p$-adic field and Galois theory.
        \item Nori thinks the proof of Theorem \ref{trm:12.4} is very difficult to follow for a first-timer.
        \item Solvable groups were supposed to be a MATH 25700 topic, but got cut because of 9-week quarters.
        \item Syclotomic fields have applications to the representation theory of finite groups; there are theorems of representation theory that you need syclotomic fields to prove.
        \item Emil Artin: Galois Theory is worth looking up.
        \item Gauss and constructions of 17-gons also needs syclotomic fields.
    \end{itemize}
\end{itemize}




\end{document}