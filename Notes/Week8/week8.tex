\documentclass[../notes.tex]{subfiles}

\pagestyle{main}
\renewcommand{\chaptermark}[1]{\markboth{\chaptername\ \thechapter\ (#1)}{}}
\setcounter{chapter}{7}

\begin{document}




\chapter{???}
\section{Linear Algebra Review and Rational Canonical Form}
\begin{itemize}
    \item \marginnote{2/20:}Nori's change of heart.
    \begin{itemize}
        \item We've all seen linear algebra; thus, we'll speedrun it and then do exterior algebra and determinants. That's where we'll finish.
    \end{itemize}
    \item The following is part 1 of a linear algebra course.
    \item Let $F$ be a field.
    \item \textbf{Vector space}: An $F$-module.
    \item \textbf{Linearly independent} (subset $S\subset V$): Same definition we're familiar with.
    \item \textbf{Spanning} (subset $S\subset V$): A subset $S$ of $V$ that is a set of generators of $V$.
    \item $S$ is a \textbf{basis} implies that $S$ generates $V$ and is linearly independent.
    \item Every linearly independent subset of $V$ can be extended to a basis.
    \item Every spanning set $S$ contains a basis.
    \begin{itemize}
        \item Any maximal linearly independent subset of $S$ is a basis.
    \end{itemize}
    \item $S_1,S_2$ are bases for $V$ implies that $|S_1|=|S_2|$.
    \begin{itemize}
        \item The replacement theorem in \textcite{bib:DummitFoote} is a good way to prove this.
    \end{itemize}
    \item We are now done with part 1; this is part 2 of a linear algebra course.
    \item Let $T:V\to V$ be a linear transformation.
    \item Let $A$ be a ring. What is an $A[X]$-module $M$?
    \begin{itemize}
        \item It is an abelian group $(M,+)$ and a ring homomorphism $\rho:A[X]\to\End(M,+)$.
        \item Since $A\hookrightarrow A[X]$, $\rho|_A$ turns $M$ into an $A$-module.
        \item Since $aX=Xa$, $\rho(a)\rho(X)=\rho(X)\rho(a)$.
        \item But since we consider $M$ to be a module, we write $a:=\rho(a)$: Thus, $a\rho(X)m=\rho(X)am$ for all $m\in M$.
        \item Note that $\rho(X)\in\End_A(M)$ (which is the set of all $A$-module homomorphisms).
        \item Additionally, $\rho(X):M\to M$ is an $A$-module homomorphism.
    \end{itemize}
    \item Put $\rho(X)=T$. Thus, an $A[X]$-module is a pair $(M,T)$, where $M$ is an $A$-module and $T\in\End_A(M)$.
    \item Conversely, such $(M,T)$ gives rise to an $A[X]$-module.
    \begin{itemize}
        \item In particular, the action is
        \begin{equation*}
            \left( \sum_{n=0}^\ell a_nX^n \right)m = \sum_{n=0}^\ell a_nT^nm
        \end{equation*}
    \end{itemize}
    \item Take $A=F$ a field concerned with $(V,T)$ where $V$ is any $F$-vector space and $T:V\to V$ is a linear transformation.
    \begin{itemize}
        \item This induces a module over $F[X]$.
    \end{itemize}
    \item $V$ finite dimensional induces $\rho:F[X]\to\End_F(V)\cong M_n(F)$ defined by $X\mapsto T$.
    \begin{itemize}
        \item $\rho(X)=T$ and $\rho(c)=c$ for all $c\in F$.
    \end{itemize}
    \item Let $n^2$ be the dimension of the $F$-vector space??
    \item Then $\ker(\rho)=(f)$ for some $f$ be monic of degree $d\leq n^2$.
    \begin{figure}[H]
        \centering
        \begin{tikzpicture}[xscale=2,yscale=1.6]
            \small
            \node (R)  at (0,1) {$F[X]$};
            \node (DR) at (0,0) {$F[X]/(f)$};
            \node (S)  at (1,1) {$\End_F(V)$};
    
            \footnotesize
            \draw [->]           (R)  --                                     (DR);
            \draw [right hook->] (DR) -- node[above left=-2pt]{$\bar{\rho}$} (S);
            \draw [->]           (R)  -- node[above]          {$\rho$}       (S);
        \end{tikzpicture}
        \caption{$F[X]$-module actions.}
        \label{fig:fXmodAction}
    \end{figure}
    \begin{itemize}
        \item We have the constraint on the degree of $f$ by the isomorphism from Lecture 3.1.
    \end{itemize}
    \item \textbf{Minimal polynomial} (of $T$): The polynomial $f$ that generates $\ker(\rho)$.
    \begin{itemize}
        \item In particular, $V$ is a torsion $F[X]$-module $(f\cdot g)$.
    \end{itemize}
    \item \textbf{Cyclic vector}: A vector $v\in V$ belonging to $(V,T)$ such that $v,Tv,T^2v,\dots$ spans $V$.
    \item Assume $v,Tv,T^2v,\dots,T^{k-1}v$ are linearly independent, but $v,Tv,\dots,T^kv$ are not.
    \begin{itemize}
        \item Then
        \begin{equation*}
            T^kv = a_0v+a_1Tv+\cdots+a_{k-1}T^{k-1}v
        \end{equation*}
        where all $a_i\in F$ and not all $a_i=0$.
        \item It follows that $T^mk\in\gen{v,Tv,\dots,T^{k-1}v}=W$ a vector space.
        \item Let $g(X)=X^k-(a_{k-1}X^{k-1}+\cdots+a_1X+a_0)$. Then $g(T)v=0$. This implies that $g$ is the minimal polynomial of $T$.
        \item It follows that $T^hg(T)v=0$. Thus, $g(T)T^hv=0$ for all $h$.
        \item Lastly, it follows that $g(T)w=0$ for all $w\in W$.
        \item Assume $v$ is a cyclic vector. Then $W=V$. It follows that $g(T)v=0$ for all $v\in V$.
        \item The original assumption posits that no polynomial of degree less than or equal to $k-1$ can annihilate $v$.
    \end{itemize}
    \item Consider $V=F[X]/(f)$. Let $\deg(f)=d$, let $T:V\to V$, and let $T$ be the "multiply by $X$" linear transformation. It follows that if $v_i=\overline{X^{i-1}}$ ($i=1,\dots,d$), then
    \begin{equation*}
        Tv_i = v_{i+1}
    \end{equation*}
    for $i=1,\dots,d-1$ and
    \begin{equation*}
        Tv_d = -(a_0v_1+a_1v_2+\cdots+a_{d-1}v_d)
    \end{equation*}
    \begin{itemize}
        \item If $d=3$, then we have
        \begin{equation*}
            M(T) =
            \begin{pmatrix}
                0 & 0 & -a_0\\
                1 & 0 & -a_1\\
                0 & 1 & -a_2\\
            \end{pmatrix}
        \end{equation*}
        \item The above matrix is called the \textbf{companion matrix} of $f$ (monic of degree 3).
    \end{itemize}
    \item \textbf{Rational canonical form}: The form $(V,T)$ given by
    \begin{equation*}
        F[X]/(f_1)\oplus\cdots\oplus F[X]/(f_s)
    \end{equation*}
    where $f_2\mid f_1,\dots,f_s\mid f_{s-1}$ and $\deg(f_s)>0$.
    \begin{itemize}
        \item When $V=0$, then $s=0$. In this case, $f_1$ is the minimal polynomial of $T$.
        \item The form consisting of a block diagonal matrix of companion matrices.
    \end{itemize}
    \item \textbf{Jordan canonical form}:
    \begin{itemize}
        \item Has to do with $p$-primary components!
    \end{itemize}
    \item There's one more canonical form, too.
    \item Since no one knows what canonical forms are and we very much need them for what Nori was planning to do, Nori will change his plans. No tensors in the last week, either.
    \item $p$-primary components: When $p=X-a$, $a\in F$.
    \item $(V,T)$ is \textbf{$p$-primary} if there exists an $n$ such that $(T-a)^nv=0$ for all $v\in V$.
    \item $1_V:V\to V$ is the identity.
    \item $a\cdot 1_V=a_V:V\to V$.
    \item $(T-a_v)^n=0\in\End_F(V)$.
    \item We're now doing generalized eigenspaces ?? lol.
    \item The $p$-primary component is as the generalized $a$-eigenspace.
    \begin{itemize}
        \item $(T-a)v=0$, i.e., $Tv=av$ is the $a$-eigenspace; the eigenspaces are components of the generalized eigenspaces.
    \end{itemize}
    \item Let $V=F[X]/(X-a)^n$. Let $v_1=1$, $v_2=\overline{X-a}$, \dots, $v_n=\overline{(X-a)^{n-1}}$.
    \begin{itemize}
        \item We know that $X(X-a)^r=(X-a+a)(X-a)^r=(X-a)^{r+1}+a(X-a)^r$.
        \item Nori writes Jordan blocks as
        \begin{equation*}
            \begin{pmatrix}
                a & 0 & 0 & 0\\
                1 & a & 0 & 0\\
                0 & 1 & a & 0\\
                0 & 0 & 1 & a\\
            \end{pmatrix}
        \end{equation*}
        not with 1's in the superdiagonal.
        \begin{itemize}
            \item Thus, the \emph{last} generalized eigenvector is an eigenvector here, instead of the \emph{first}.
        \end{itemize}
    \end{itemize}
\end{itemize}



\section{Office Hours (Nori)}
\begin{itemize}
    \item Midterm: We never covered the universal property of a quotient in class, did we?
    \begin{itemize}
        \item That's the special lemma from last office hours.
    \end{itemize}
    \item PSet 7: 7.3 and 7.4 typos.
    \item Wednesday lecture?
    \begin{itemize}
        \item Seventh week summary will suffice.
    \end{itemize}
    \item What do you need us to know about the rational canonical form? Should I still read \textcite{bib:DummitFoote}, Section 12.2 or is that no longer necessary?
    \begin{itemize}
        \item Nori will probably push ahead with 12.2. Thus I should read it. He's not sure what he'll do beyond that, though, since he doesn't want to jam tensors into the last week.
        \item I will need tensor products for representation theory, regardless, so if I want to take it, I should self-study it.
        \item No chance tensor products will be covered next quarter.
        \item Serre is a terrific mathematician whose wife is a super chemist, and that's why he wrote his book on representation theory (and wrote it in a less terse manner than usual).
        \item No tensors means no exterior algebra, too.
        \item Nori hasn't read any of \textcite{bib:DummitFoote}.
        \item The transfer theory of groups arises in a later chapter, and that's important for representation theory, though.
    \end{itemize}
    \item Nori doesn't think any teacher pays attention to what courses are supposed to cover as stated in the course catalog.
    \begin{itemize}
        \item We will never do modules, multilinear and quadratic forms.
        \item $p$-adic field and Galois theory.
        \item Nori thinks the proof of Theorem \ref{trm:12.4} is very difficult to follow for a first-timer.
        \item Solvable groups were supposed to be a MATH 25700 topic, but got cut because of 9-week quarters.
        \item Syclotomic fields have applications to the representation theory of finite groups; there are theorems of representation theory that you need syclotomic fields to prove.
        \item Emil Artin: Galois Theory is worth looking up.
        \item Gauss and constructions of 17-gons also needs syclotomic fields.
    \end{itemize}
\end{itemize}



\section{Office Hours (Nori)}
\begin{itemize}
    \item \marginnote{2/21:}Lecture 6.1: Proposition proof?
    \item Lecture 6.1: $(2)\subsetneq\Z$ example?
    \item Lecture 6.1: The end of the theorem proof.
    \item Lecture 6.2: Does the first theorem you proved not appear in the book until Chapter 12?
    \item Lecture 6.2: What is $A$ in the proof?
    \item Resources for the proofs in Week 6?
    \item Lecture 7.1: Quotient stuff.
    \item Lecture 7.2: Why does $\Ann(v)=(p^k)$, why not just $(p^k)\subset\Ann(v)$? Additionally, how does $p^kw'=0$ imply that $p^k\in\Ann(w)$?
    \begin{itemize}
        \item $R$ is a PID!
        \item $\Ann(w)$ should be $\Ann(w')$ in the centered line.
        \item We don't need to know the theorem from the book for a while (second year of graduate school at least).
        \item It's good to know the proofs from class just for going forward in math, but we probably will not be asked to reproduce them on an exam.
    \end{itemize}
    \item Lecture 7.3: RCF proof?
    \begin{itemize}
        \item It's not $m_i,1$, it's $m_{i,1}$!
        \item Rewrite the proof when I'm awake enough to understand it.
    \end{itemize}
\end{itemize}



\section{Chapter 12: Modules over Principal Ideal Domains}
\emph{From \textcite{bib:DummitFoote}.}
\setcounter{bookch}{12}
\setcounter{proposition}{11}
\subsection*{Section 12.2: The Rational Canonical Form}
\begin{itemize}
    \item As stated previously, we apply the results of Section 12.1 to $F[X]$-modules herein.
    \item Let $V$ be a finite dimensional vector space over $F$ of dimension $N$. Let $(V,T)$ be an $F[X]$-module.
    \item Since $V$ is finite dimensional, it is finitely generated as an $F$-module and hence also as an $F[X]$-module.
    \item If $V$ were free, it would be isomorphic to a direct sum of copies of $F[X]$ (by Theorem \ref{trm:12.5.1}) and hence be infinite dimensional.
    \begin{itemize}
        \item Thus, $V$ is a torsion $F[X]$-module.
        \item Theorem \ref{trm:12.5.3}: $V$ is isomorphic to the direct sum of cyclic, torsion $F[X]$-modules.
        \item This decomposition will allow us to choose a basis for $V$ with respect to which the matrix representation for the linear transformation $T$ is in a specific simple form.
    \end{itemize}
    \item \textbf{Rational canonical form} (of a matrix): The form obtained when we use the invariant factor decomposition of the relevant vector space.
    \item \textbf{Jordan canonical form} (of a matrix): The form obtained when we use the elementary divisor decomposition (and when $F$ contains all the eigenvalues of $T$).
    \item Theorem \ref{trm:12.9} ensures that the RCF and JCF are unique, justifying the labeling of them as \emph{canonical}.
    \item An application of canonical forms: Classifying distinct linear transformations.
    \begin{itemize}
        \item Two matrices that represent the same linear transformation (hence are similar) have the same RCF and JCF.
        \item This is another instance of the structure of the space being acted upon (e.g., the invariant factor decomposition of $V$) providing information on the algebraic objects (e.g., linear transformations) which are acting.
    \end{itemize}
    \item \textbf{Representation Theory of Groups}: The special case of algebraic objects acting on spaces concerning groups acting on vector spaces.
    % \item \textbf{Eigenvalue}: An element $\lambda\in F$ corresponding to a linear transformation $T$ such that there exists a nonzero vector $v\in V$ satisfying $Tv=\lambda v$. \emph{Denoted by} $\bm{\lambda}$.
    % \item \textbf{Eigenvector}: The vector $v$ in the above definition.
    % \item \textbf{Eigenspace} (of a linear transformation corresponding to an eigenvalue): The set of all eigenvectors of a linear transformation $T$ corresponding to a particular eigenvalue $\lambda$ of $T$. \emph{Given by}
    % \begin{equation*}
    %     \{v\in V:Tv=\lambda v\}
    % \end{equation*}
    \item \textbf{Eigenvalues}, \textbf{eigenvectors}, \textbf{eigenspaces}, and the \textbf{determinant} are defined for linear transformations and analogously for matrices.
    \item Properties of eigenvalues.
    \begin{proposition}\label{prp:12.12}
        TFAE.
        \begin{enumerate}
            \item $\lambda$ is an eigenvalue of $T$.
            \item $\lambda I-T$ is a singular linear transformation of $V$.
            \item $\det(\lambda I-T)=0$.
        \end{enumerate}
        \begin{proof}
            Given.
        \end{proof}
    \end{proposition}
    \item \textbf{Characteristic polynomial} (of a linear transformation): The polynomial defined as follows, where $T$ is the linear transformation in question. \emph{Denoted by} $\bm{c_T(X)}$. \emph{Given by}
    \begin{equation*}
        c_T(X) = \det(XI-T)
    \end{equation*}
    % \item \textbf{Characteristic polynomial} (of $A$): The polynomial defined as follows. \emph{Denoted by} $\bm{c_A(X)}$. \emph{Given by}
    % \begin{equation*}
    %     c_A(X) = \det(XI-A)
    % \end{equation*}
    \begin{itemize}
        \item Defined similarly for matrices $A$.
        \item A monic polynomial of degree $\dim V$.
        \item The eigenvalues are the roots.
    \end{itemize}
    \item \textbf{Minimal polynomial} (of of a linear transformation): The unique monic polynomial which generates the ideal $\Ann(V)$ in $F[X]$. \emph{Denoted by} $\bm{m_T(X)}$.
    \begin{itemize}
        \item Defined similarly for matrices $A$.
        \item We know that such a polynomial exists by Theorem \ref{trm:12.5.3}.
        \item Exercise 12.2.5: The degree of the minimal polynomial is at most $n^2$.
    \end{itemize}
    \item \textbf{Cayley-Hamilton Theorem}: The minimal polynomial for $T$ is a divisor of the characteristic polynomial for $T$.
    \begin{itemize}
        \item Thus, the degree of the minimal polynomial is at most $n$.
    \end{itemize}
    \item We now build up to the \textbf{rational canonical form}.
    \item Introduction.
    \begin{itemize}
        \item Theorem \ref{trm:12.5}: There exists an isomorphism
        \begin{equation}\label{eqn:12.1}
            V \cong F[X]/(a_1(X))\oplus\cdots\oplus F[X]/(a_m(X))
        \end{equation}
        \item The invariant factors $a_i$ are only determined up to units, but since $F[X]^\times=F-\{0\}$, we can make the $a_i$ unique by requiring them to be monic.
        \item Theorem \ref{trm:12.5.3} asserts that $(a_m(X))=\Ann(V)$.
    \end{itemize}
    \item The minimal polynomial and the invariant factors.
    \begin{proposition}\label{prp:12.13}
        The minimal polynomial $m_T(X)$ is the largest invariant factor of $V$. All of the invariant factors of $V$ divide $m_T(X)$.
    \end{proposition}
    \item We now build up to calculating the minimal polynomial of $T$ and the other invariant factors.
    \item Choosing a basis for each of the summands in Equation \ref{eqn:12.1}.
    \begin{itemize}
        \item Recall that the action of $T$ on $V$ is equivalent to the action of $X$ on each summand.
        \item Recall also (from the Example following Proposition \ref{prp:11.1}) that $1,\bar{X},\bar{X}^2,\dots,\bar{X}^{k-1}$ gives a basis of $F[X]/(a(X))$, where $a(X)=X^k+b_{k-1}X^{k-1}+\cdots+b_0$.
        \item With respect to this basis, the linear transformation $T=l_X$ acts via
        \begin{align*}
            1 &\mapsto \bar{X}\\
            \bar{X} &\mapsto \bar{X}^2\\
            \bar{X}^2 &\mapsto \bar{X}^3\\
            &\hspace{2.5mm}\vdots\\
            \bar{X}^{k-2} &\mapsto \bar{X}^{k-1}\\
            \bar{X}^{k-1} &\mapsto \bar{X}^k = -b_0-b_1\bar{X}-\cdots-b_{k-1}\bar{X}^{k-1}
        \end{align*}
        \begin{itemize}
            \item The last equality holds since $a(\bar{X})=0$ in $F[X]/(a(X))$.
        \end{itemize}
        \item With respect to this basis, the matrix for multiplication by $X$ is called the \textbf{companion matrix} of $a(X)$.
        \item Applying this procedure to each of the cyclic modules on the right side of Equation \ref{eqn:12.1} under an appropriate basis yields the \textbf{direct sum} of the companion matrices for the invariant factors as the matrix of $T$.
        \item Note that this matrix is uniquely determined by the invariant factors of the $F[X]$-module $V$. These invariant factors, in turn, uniquely determine $V$ up to isomorphism by Theorem \ref{trm:12.9}.
    \end{itemize}
    \item \textbf{First subdiagonal}: The set of entries in a matrix which lie directly below a diagonal entry. \emph{Also known as} \textbf{subdiagonal}.
    \item \textbf{Companion matrix} (of a polynomial): The $k\times k$ matrix, pertaining to the polynomial $a(X)=X^k+b_{k-1}X^{k-1}+\cdots+b_0$, which consists of 1's down the first subdiagonal, $-b_0,\dots,-b_{k-1}$ down the last column, and zeros elsewhere. \emph{Denoted by} $\bm{\mathcal{C}_{a(X)}}$. \emph{Given by}
    \begin{equation*}
        \mathcal{C}_{a(X)} =
        \begin{pmatrix}
            0      & 0      & \cdots & \cdots & \cdots & -b_0\\
            1      & 0      & \cdots & \cdots & \cdots & -b_1\\
            0      & 1      & \cdots & \cdots & \cdots & -b_2\\
            0      & 0      & \ddots &        &        & \vdots\\
            \vdots & \vdots &        & \ddots &        & \vdots\\
            0      & 0      & \cdots & \cdots & 1      & -b_{k-1}\\
        \end{pmatrix}
    \end{equation*}
    \item \textbf{Direct sum} (of matrices): The block diagonal matrix consisting of the component matrices.
    \begin{itemize}
        \item See the RCF example below.
    \end{itemize}
    \item \textbf{Rational canonical form} (of a matrix): A matrix that is the direct sum of companion matrices for monic polynomials $a_1(X),\dots,a_M(X)$ of degree at least one with $a_1(X)\mid a_2(X)\mid\cdots\mid a_m(X)$. \emph{Also known as} \textbf{RCF}. \emph{Given by}
    \begin{equation*}
        \begin{pmatrix}
            \mathcal{C}_{a_1(X)} &  &  & \\
             & \mathcal{C}_{a_2(X)} &  & \\
             &  & \ddots & \\
             &  &  & \mathcal{C}_{a_m(X)}\\
        \end{pmatrix}
    \end{equation*}
    \item \textbf{Invariant factors} (of the RCF): The polynomials $a_i$ in the above definition.
    \item Definition of a \textbf{block diagonal} matrix.
    \item \textbf{Rational canonical form} (of a linear transformation): The matrix representing $T$ which is in rational canonical form.
    \item \textcite{bib:DummitFoote} proves that the rational canonical form is unique by means of running the generation process in reverse.
    \begin{theorem}[Rational Canonical Form for Linear Transformations]\label{trm:12.14}
        Let $V$ be a finite dimensional vector space over the field $F$, and let $T$ be a linear transformation of $V$.
        \begin{enumerate}
            \item There is a basis for $V$ with respect to which the matrix for $T$ is in rational canonical form, i.e., is a block diagonal matrix whose diagonal blocks are the companion matrices for monic polynomials $a_1(X),\dots,a_m(X)$ of degree at least one with $a_1(X)\mid a_2(X)\mid\cdots\mid a_m(X)$.
            \item The rational canonical from for $T$ is unique.
        \end{enumerate}
    \end{theorem}
    \item Why the \emph{rational} canonical form?
    \begin{itemize}
        \item "Rational" refers to the fact that this canonical form is calculated entirely within the field $F$ and exists for any linear transformation $T$.
        \item This is not the case for the JCF, which only exists if the field $F$ contains the eigenvalues for $T$.
    \end{itemize}
    \item Similar matrices, modules, and the RCF.
    \begin{theorem}\label{trm:12.15}
        Let $S$ and $T$ be linear transformations of $V$. Then TFAE.
        \begin{enumerate}
            \item $S$ and $T$ are similar linear transformations.
            \item The $F[X]$-modules obtained from $V$ via $S$ and via $T$ are isomorphic $F[X]$-modules.
            \item $S$ and $T$ have the same rational canonical form.
        \end{enumerate}
        \begin{proof}
            Given.
        \end{proof}
    \end{theorem}
    \item Observation: Any $n\times n$ matrix $A$ with entries in $F$ arises as the matrix for some linear transformation $T$ of an $n$-dimensional vector space.
    \item This observation allows us to restate Theorems \ref{trm:12.14}-\ref{trm:12.15} in the language of matrices.
    \begin{theorem}[Rational Canonical Form for Matrices]\label{trm:12.16}
        Let $A$ be an $n\times n$ matrix over the field $F$.
        \begin{enumerate}
            \item The matrix $A$ is similar to a matrix in rational canonical form, i.e., there is an invertible $n\times n$ matrix $P$ over $F$ such that $P^{-1}AP$ is a block diagonal matrix whose diagonal blocks are the companion matrices for monic polynomials $a_1(X),\dots,a_m(X)$ of degree at least one with $a_1(X)\mid a_2(X)\mid\cdots\mid a_m(X)$.
            \item The rational canonical from for $A$ is unique.
        \end{enumerate}
    \end{theorem}
    \begin{theorem}\label{trm:12.17}
        Let $A,B$ be $n\times n$ matrices over the field $F$. Then $A,B$ are similar iff $A,B$ have the same RCF.
    \end{theorem}
    \item \textbf{Invariant factors} (of a matrix): The invariant factors of the matrix's RCF.
    \item RCF and similarity questions for $A$ do not depend on which field contains the entries of $A$.
    \begin{corollary}\label{cly:12.18}
        Let $A,B$ be two $n\times n$ matrices over a field $F$, and suppose $F$ is a subfield of the field $K$.
        \begin{enumerate}
            \item The rational canonical form of $A$ is the same whether it is computed over $K$ or over $F$. The minimal and characteristic polynomials and the invariant factors of $A$ are the same whether $A$ is considered as a matrix over $F$ or as a matrix over $K$.
            \item The matrices $A,B$ are similar over $K$ iff they are similar over $F$, i.e., there exists an invertible $n\times n$ matrix $P$ with entries from $K$ such that $B=P^{-1}AP$ iff there exists an (in general different) invertible $n\times n$ matrix $Q$ with entries from $F$ such that $B=Q^{-1}AQ$.
        \end{enumerate}
        \begin{proof}
            Given.
        \end{proof}
    \end{corollary}
    \item Takeaways from Corollary \ref{cly:12.18}.
    \begin{itemize}
        \item The RCF for $A$ is an $n\times n$ matrix with entries in the smallest field containing the entries of $A$.
        \item Further explanation of the word \emph{rational}: The RCF is the same matrix even if we allow conjugation of $A$ by nonsingular matrices whose entries come from larger fields.
    \end{itemize}
    \item Characteristic polynomials and invariant factors.
    \begin{lemma}\label{lem:12.19}
        Let $a(X)\in F[X]$ be any monic polynomial.
        \begin{enumerate}
            \item The characteristic polynomial of the companion matrix of $a(X)$ is $a(X)$.
            \item If $M$ is the block diagonal matrix
            \begin{equation*}
                M =
                \begin{pmatrix}
                    A_1 & 0 & \cdots & 0\\
                    0 & A_2 & \cdots & 0\\
                    \vdots & \vdots & \ddots & \vdots\\
                    0 & 0 & \cdots & A_k\\
                \end{pmatrix}
            \end{equation*}
            given by the direct sum of matrices $A_1,\dots,A_k$, then the characteristic polynomial of $M$ is the product of the characteristic polynomials of $A_1,\dots,A_k$.
        \end{enumerate}
        \begin{proof}
            See the exercises.
        \end{proof}
    \end{lemma}
    \begin{proposition}\label{prp:12.20}
        Let $A$ be an $n\times n$ matrix over the field $F$.
        \begin{enumerate}
            \item The characteristic polynomial of $A$ is the product of all the invariant factors of $A$.
            \item (The Cayley-Hamilton Theorem) The minimal polynomial of $A$ divides the characteristic polynomial of $A$.
            \item The characteristic polynomial of $A$ divides some power of the minimal polynomial of $A$. In particular, these polynomials have the same roots, not counting multiplicities.
        \end{enumerate}
        \begin{proof}
            Given.
        \end{proof}
    \end{proposition}
    \item The relations in Proposition \ref{prp:12.20} are frequently useful in determining the invariant factors of $A$, particularly for $\deg(A)$ small.
    \item \textbf{Elementary row and column operations}: The following three operations, where $A$ is an $n\times n$ matrix over the field $F$ and $XI-A$ is an $n\times n$ matrix with entries in $F[X]$. \emph{Given by}
    \begin{enumerate}[label={(\roman*)}]
        \item Interchanging two rows or columns.
        \item Adding a multiple (in $F[X]$) of one row or column to another.
        \item Multiplying any row or column by a unit in $F[X]$, i.e., by a nonzero element in $F$.
    \end{enumerate}
    \item \textbf{Smith Normal Form} (of a matrix): The following form of the $n\times n$ matrix $XI-A$ with entries from $F[X]$, where $a_1,\dots,a_m$ are the invariant factors of $A$. \emph{Given by}
    \begin{equation*}
        \begin{pmatrix}
            1\\
            & \ddots\\
            && 1\\
            &&& a_1(X)\\
            &&&& a_2(X)\\
            &&&&& \ddots\\
            &&&&&& a_m(X)\\
        \end{pmatrix}
    \end{equation*}
    \item Computing the invariant factors in general.
    \begin{theorem}\label{trm:12.21}
        Let $A$ be an $n\times n$ matrix over the field $F$. Using the three elementary row and column operations above, the $n\times n$ matrix $XI-A$ is with entries in $F[X]$ can be put into Smith Normal Form.
    \end{theorem}
    \item \textcite{bib:DummitFoote} provides algorithms for computing the invariant factor decomposition and the RCF. Return to later.
\end{itemize}




\end{document}