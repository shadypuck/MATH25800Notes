\documentclass[../notes.tex]{subfiles}

\pagestyle{main}
\renewcommand{\chaptermark}[1]{\markboth{\chaptername\ \thechapter\ (#1)}{}}
\setcounter{chapter}{4}

\begin{document}




\chapter{???}
\section{Prime Factorizations}
\begin{itemize}
    \item \marginnote{1/30:}Midterm next Monday.
    \begin{itemize}
        \item There's a list of topics on Canvas.
        \item Don't worry about quadratic fields (or any of the other examples in Chapter 7 of \textcite{bib:DummitFoote}). These are interesting, but will be saved for the absolute end of the course.
        \item After the midterm, Nori will start on modules.
        \item We've been talking about fields, which are contained in EDs, which are contained in PIDs. There probably will not be anything on EDs. Use the weakest definition for ED (the ones in class and the book differ). Which is this??
        \item PIDs are contained in UFDs, which are contained in integral domains, which are contained in commutative rings.
    \end{itemize}
    \item PIDs are nice! $\gcd(a,b)$ can be computed without factoring $a,b$. Review page 2 of Chapter 8, as referenced in a previous class.
    \item In PIDs, you can factor $a=qb+r$, but $q,r$ may not be specific; in EDs, these $q,r$ are unique.
    \begin{itemize}
        \item Is this correct??
    \end{itemize}
    \item Theorem: $R$ is a UFD implies $R[X]$ is a UFD.
    \item Corollary: $R$ is a UFD implies $R[X_1,\dots,X_n]$ is a UFD.
    \begin{proof}
        Use induction.
    \end{proof}
    \item Corollary: $R[X]$ is a field implies $R$ is a PID implies $R[X_1,\dots,X_n]$ is a UFD.
    \begin{itemize}
        \item Something about $\Z$, $F[[X]]$ where $F$ is a field??
    \end{itemize}
    \item Example: What are the irreducibles of $\Z[X]$?
    \begin{itemize}
        \item Prime numbers.
        \item Let $g\in\Q[X]$. Assume $g$ is monic. Then $g(X)=X^d+a_1X^{d-1}+\cdots+a_d$ for all $a_i\in\Q$. There exists $n\in\N$ such that $ng(X)\in\Z[X]$. Let $n$ be the least natural number for which this is true. It follows by our hypothesis that $n$ is the smallest such $n$ that the coefficients of $ng$ are relatively prime. Conclusion: $ng(X)$ is irreducible in $\Z[X]$.
    \end{itemize}
    \item Takeaway: There are two types of irreducibles (those from $\Z$ and the new ones).
    \begin{itemize}
        \item This statement has a clear parallel for every UFD.
    \end{itemize}
    \item Let $R$ be a UFD, and let $\mathcal{P}(R)\subset R\setminus\{0\}$ be such that\dots
    \begin{enumerate}[label={(\roman*)}]
        \item Every $\pi\in\mathcal{P}(R)$ is irreducible.
        \item For all $\alpha\in\R\setminus\{0\}$, $\alpha$ irreducible, there exists a unique $\pi\in\mathcal{P}(R)$ such that $(\alpha)=(\pi)$.
    \end{enumerate}
    \item Statement (*): Every nonzero element $\alpha\in R$ is uniquely expressible as
    \begin{equation*}
        \alpha = u\prod_{\pi\in\mathcal{P}(R)}\pi^{k(\pi)}
    \end{equation*}
    where $u\in R^\times$ and for all $\pi$, $k(\pi)\in\Zg$ and $|\{\pi\in\mathcal{P}(R)\mid k(\pi)>0\}|$ is finite.
    \begin{proof}
        $R$ is a UFD implies (*).
    \end{proof}
    \item Conversely, if $\mathcal{P}(R)$ is a subset of an integral domain $R$ such that (*) holds, then $R$ is a UFD.
    \begin{proof}
        Note that $\pi\in\mathcal{P}(R)$ implies $\pi$ is irreducible.\par
        Argument for something?? Let $\pi=ab$. Suppose $a=\pi^{m_0}\pi^{m_1}\cdots\pi_h^{m_h}u$ and $b=\pi^{n_0}\pi_1^{n_1}\cdots\pi_h^{n_h}$. Then $\pi=ab=\pi^{m_0+n_0}\pi^{m_1+n_1}\cdots$. But then because of unique factorization, we cannot have $\pi_1^{x_1}\cdots\pi_h^{x_h}$.
    \end{proof}
    \item \textbf{Content} (of $f\in R[X]$): The greatest common divisor of the coefficients of a nonzero $f=a_0+a_1X+a_2X^2\cdots$ in $R[X]$. \emph{Denoted by} $\bm{c(f)}$. \emph{Given by}
    \begin{equation*}
        c(f) = \gcd(a_0,a_1,a_2,\dots)
    \end{equation*}
    \item Let $c(f)=\prod_{\pi\in\mathcal{P}(R)}\pi^{k(\pi)}$.
    \item Gauss lemma: $f,g\in R[X]$ both nonzero implies that $c(fg)=c(f)c(g)$.
    \begin{proof}
        It suffices to prove the case where $c(f)=c(g)=1$.\par
        Let $\pi$ be irreducible (hence prime). Consider the canonical surjection $R\to R/(\pi)$. It gives rise to a ring homomorphism $\varphi:R[X]\to R/(\pi)[X]$ defined by
        \begin{equation*}
            \varphi(a_0+a_1X+\cdots+a_dX^d) = \bar{a}_0+\bar{a}_1X+\cdots+\bar{a}_dX^d
        \end{equation*}
        Notationally, if $a_i\in R$, then $\bar{a}_i$ is the image of $a_i$ in $R/(\pi)$ under the canonical surjection.\par
        $c(f)=1$ implies that there exists $i$ such that $a_i\neq 0$. Therefore, $\varphi(f)\neq 0$. Similarly, $c(g)=1$ implies that $\varphi(g)\neq 0$. It follows that $\varphi(fg)=\varphi(f)\varphi(g)$. Since $R/(\pi)$ is an integral domain and thus contains no zero divisors, we know that $\varphi(fg)=\varphi(f)\varphi(g)\neq 0$. It follows that $\pi\nmid c(fg)$. This is true for all irreducible $\pi\in R$. Indeed, it follows that $c(fg)=1$.
    \end{proof}
    \item This proof can be done by brute force without quotient rings, and elegantly with quotient rings. \textcite{bib:DummitFoote} does both and we should check this out. The above is Nori's cover of just the latter, elegant argument.
    \item Let $K$ be the fraction field of $R$. We know that $K[X]$ is a PID (hence a UFD, etc.). The primes are the irreducible monic polynomials. Let $g=a_0+a_1X+\cdots+a_{d-1}X^{d-1}+X^d\in K[X]$ be monic. Then there exists a nonzero $\alpha\in R$ such that $R[X]\subset K[X]$. It follows that $a_i=\alpha_i/\beta_i$ for some $\alpha_i,\beta_i\in R$ with $\beta_i\neq 0$ since $K=\Frac R$.
    \item Claim 1: There exists a unique $\beta\in R$, $\beta=\prod_{\pi\in\mathcal{P}(R)}\pi^{k(\pi)}$, such that $\beta g\in R[X]$ and $c(\beta g)=1$.
    \begin{proof}
        Denote $\beta g$ by $\tilde{g}$. Then the claim is that $\tilde{g}\in R[X]$ has content 1. Thus,
        \begin{equation*}
            \frac{\tilde{g}}{\ell(\tilde{g})} = g
        \end{equation*}
    \end{proof}
    \item Claim 2: $g\mapsto\tilde{g}$ is a monic polynomial in $K[X]$. Then $\tilde{g}\in R[X]$ with content 1 and
    \begin{equation*}
        \widetilde{gh} = \tilde{g}\cdot\tilde{h}
    \end{equation*}
    \begin{proof}
        Use the Gauss lemma.
    \end{proof}
    \item Statement (*) holds as a result.
    \item $\mathcal{P}(R[X])=\mathcal{P}(R)\sqcup\{\tilde{g}\mid g\in K[X]\text{ is monic and irreducible}\}$.
    \item Claim 3: (*) holds for $\mathcal{P}(R[X])$.
    \begin{proof}
        Scratch: Let $f\in R[X]$ be nonzero. Then $f/\ell(f)\in K[X]$ for each $g_i$ monic and irreducible.\par
        $\widetilde{\frac{f}{\ell(f)}}=\tilde{g}_1^{k_1}\cdots\tilde{g}_r^{k_r}$. We have $f,\tilde{g}_1^{k_1}\cdots\tilde{g}_r^{k_r}\in R[X]$. $f=\beta(\tilde{g}_1^{k_1}\cdots\tilde{g}_r^{k_r})$. $\beta\in R$.
    \end{proof}
    \item Two remaining lectures on rings: Factoring polynomials in $\Z[X]$ and $\R[X]$.
\end{itemize}



\section{Office Hours (Nori)}
\begin{itemize}
    \item Problem 4.1?
    \begin{itemize}
        \item See picture.
    \end{itemize}
    \item Lecture 2.2: "We need bijectivity because continuous functions don't necessarily have continuous inverses?"
    \begin{itemize}
        \item We can use "$f:R_1\to R_2$ is a ring homomorphism plus bijection" as the definition of isomorphism.
        \item An equivalent definition is, "there exists a ring homomorphism $g:R_2\to R_1$ such that $g\circ f=\id_{R_1}$ and $f\circ g=\id_{R_2}$."
        \item Even though the first is simpler, the reason people use the second is because in some contexts, there \emph{is} a difference between the definitions (such as with homeomorphisms, whose inverses need to be continuous [think proper]).
    \end{itemize}
    \item Lecture 2.2: We have only defined the finite sum of ideals, not an infinite sum, right?
    \begin{itemize}
        \item We defined an infinite sum, too.
        \item In particular, $\sum_{i\in I}M_i=\bigcup_{\substack{F\subset I}\\F\text{ is finite}}$.
        \item Note that in a more general sense, you can have infinitely generated ideals. For example, infinite polynomials.
    \end{itemize}
    \item Lecture 2.2: $IJ=I\cap J$ conditions.
    \begin{itemize}
        % \item The product is equal to the intersection in commutative rings, but not just for two-sided ideals.
        \item $IJ\subset I\cap J$ in commutative rings.
        \item Counterexample: $R=\Z$ and $I=(d)$ and $J=(d)$. Then $IJ=(d^2)\neq(d)=I\cap J$.
        \item Equality is meaningful.
    \end{itemize}
    \item To what extent are we covering Chapter 9, and to what extent will reading it help my understanding of the course content?
    \begin{itemize}
        \item Just the result that $F[X]$ is a PID (implies UFD).
        \item All we need from Chapter 8 for the midterm is ED implies PID, all we need from Chapter 9 for the midterm is PID implies UFD.
        \item Main examples of PIDs are $\Z$, $F[X]$, and $F[[X]]$.
    \end{itemize}
    \item Have we done anything outside Chapters 7-9, or if I understand them, am I good to go?
    \begin{itemize}
        \item The Euclidean algorithm for monic polynomials may not be in Chapter 8.
    \end{itemize}
    \item Lecture 3.1: Everything from creating $\C$ from $\R$, down.
    \begin{itemize}
        \item We use monic polynomials just so that we can apply the Euclidean algorithm (EA).
        \item We want to find ring homomorphisms $\varphi:R[X]\to A$ such that $\varphi(X^2+1)=0$. How do I get hold of a $\varphi$ and an $A$? There's exactly one way to do it. We use the universal property of a polynomial ring.
        \item We want $X^2+1\in\ker\psi$, so we define $R[X]/(X^2+1)$.
        \item $R[X]/(X^2+1)$ generalizes the construction of the complex numbers. Creating a new ring in which $X^2+1=0$ has a solution.
        \item Suppose $R$ is a ring such that $f(X)\in R[X]$ doesn't have a solution. Then it does have a solution in $R[X]/(f(X))$.
        \item We recover $\C$ as a special case of this more general construction, specifically the case where $f(X)=X^2+1$.
    \end{itemize}
    \item Lecture 3.2: Do I have it right that the only nontrivial ideals of $\Q$ are the diadic numbers, $\Z_{(2)}$, and $(2^n)$? Why is this? What about the triadics, for instance?
    \begin{itemize}
        \item In $\Z_{(2)}$, the only ideals are of the form $(2^n)$ for some $n$.
    \end{itemize}
    \item Lecture 3.2: What is the significance of the final theorem?
    \begin{itemize}
        \item That all rings with the $D$-to-units property bear a certain similarity to the ring of fractions.
    \end{itemize}
    \item Section 7.5: Difference between the rational functions and the field of rational functions?
    \item Lecture 4.1: What all is going on with $F[[X^{1/2^n}]]$?
    \begin{itemize}
        \item The idea is the irreducible elements of one ring can become reducible in the context of other rings. This is just a specific example; note how $X$ is the only irreducible element in the first ring, but it reduces to $X=(X^{1/2})^2$ in the next ring, and so on.
    \end{itemize}
    \item Lecture 4.3: Speech for PIDs over UFDs?
    \item Lecture 4.3: $R\setminus\{0\}$ or $R$ is an integral domain.
    \begin{itemize}
        \item Takeaway: You don't need to factor $a,b$ to get their gcd; indeed, you can just find a single generator of $(a,b)$.
    \end{itemize}
    \item Lecture 4.3: Products of commutative diagrams?
    \item Lecture 5.1: What is the weakest definition for an ED?
    \begin{itemize}
        \item The \emph{book} teaches the weakest one.
        \item \emph{We're} only interested in Euclidean domains with positive norms.
    \end{itemize}
    \item Lecture 5.1: Uniqueness condition in the Euclidean algorithm.
    \item Lecture 5.1: The thing about $\Z$ and $F[[X]]$.
    \begin{itemize}
        \item These are the only rings we've talked about that are PIDs. Gaussian integers are, too, but we haven't proved that yet.
    \end{itemize}
    \item Lecture 5.1: Argument for something --- is this part of the proof of the converse statement?
    \item Lecture 5.1: Correct notation?
    \item What is the set $\Z[X,Y,Z,W]_{XW-YZ}$ in Q4.6b?
    \begin{itemize}
        \item Like $R_f$.
    \end{itemize}
    \item What is the purpose of the commutative diagram in Q4.7?
    \item Where does $d$ come into play in Q4.10?
    \begin{itemize}
        \item We're gonna prove that the cardinality of the set is less than or equal to $d$. About the number of roots of a polynomial of a certain degree, like how $X^3+\cdots$ can't have more then 3 roots. The most relevant property is that $\R$ is an integral domain.
    \end{itemize}
\end{itemize}



\section{Factorization Techniques}
\begin{itemize}
    \item \marginnote{2/1:}Notes on HW4 Q4.1.
    \begin{itemize}
        \item A lot of people have asked questions about this.
        \item The point is to get used to universal properties.
        \item Universal properties are important because\dots
        \begin{itemize}
            \item They will come up time and time again;
            \item They will be especially important if/when we get to tensor products;
            \item Two objects that satisfy the same universal property are isomorphic.
        \end{itemize}
    \end{itemize}
    \item We've introduced a lot of theory at this point, but everything is getting used more and more.
    \item Today: Factoring polynomials. We will look at two methods to do so.
    \item Assumption for this lecture: Let $f=a_0X^n+a_1X^{n-1}+\cdots+a_n\in\Z[X]$ have $c(f)=1$.
    \item Factorization prep.
    \begin{itemize}
        \item Today's ring of interest: $\Z[X]$.
        \item We want to test reducibility. Recall from Lecture 5.1 that\dots
        \begin{itemize}
            \item If $\deg(f)>0$, then $f$ is irreducible in $\Z[X]$ iff $c(f)=1$ and $f$ is irreducible in $\Q[X]$.
            \item Why we need the latter condition even though I don't think it was mentioned last lecture (motivation via examples).
            \begin{itemize}
                \item Consider $X^2-1/4\in\Q[X]$. This polynomial reduces to $(X-1/2)(X+1/2)$. Thus, taking $n=4$, $4X^2-1$ is still reducible in $\Z[X]$ as it equals $(2X-1)(2X+1)$.
                \item Consider $X^2-1/3\in\Q[X]$. This polynomial reduces to $(X-1/\sqrt{3})(1+1/\sqrt{3})$ in $\R[X]$, but is irreducible in $\Q[X]$. Thus, taking $n=3$, $3X^2-1$ is still irreducible in $\Z[X]$.
            \end{itemize}
            \item If $\deg(f)=0$, then $f$ is irreducible in $\Z[X]$ iff $f$ is a prime integer.
        \end{itemize}
        \item Recall that $\ell(f)$ denotes the leading coefficient.
        \item If $f$ is irreducible in $\Q[X]$, then so is $f/\ell(f)$, but now $f/\ell(f)$ is monic.
        \item Consider $f\mapsto f/\ell(f)$. It sends
        \begin{equation*}
            \{f\in\Z[X]\mid f\text{ is irreducible and }\deg(f)>0\} \to \{\text{monic irreducible polynomials in }\Q[X]\}
        \end{equation*}
        \item The above is not a bijection as is, but if we treat $\pm f$ as the same, then it is. In other words,
        \begin{equation*}
            \pm\backslash\{f\in\Z[X]\mid f\text{ is irreducible and }\deg(f)>0\} \cong \{\text{monic irreducible polynomials in }\Q[X]\}
        \end{equation*}
        where the isomorphism is defined as above.
    \end{itemize}
    \item Factorization by monomials.
    \begin{itemize}
        \item How many $g(X)=aX+b$ are there in $\Z[X]$ that divide $f$?
        \item If $aX+b\mid f$, then $a\mid a_0$ and $b\mid a_n$.
        \item We know that $a_0>0$ by the definition of the $X^n$ term as the leading term. It may be either way with $a_n$.
        \begin{itemize}
            \item For the sake of continuing, we will assume that $a_n\neq 0$. Why??
            \item We also assume that $\gcd(a,b)=1$.
        \end{itemize}
        \item Because of the above constraint, we know that
        \begin{equation*}
            \{g\in\Z[X]\mid \deg g=1,\ g\mid f\} \subset \text{known finite set}
        \end{equation*}
        where the latter set consists of all monomials $g$ with $a\mid a_0$ and $b\mid a_n$.
        \item $aX+b\mid f$ in $\Z[X]$ iff $aX+b\mid f$ in $\Q[X]$ iff $f(-b/a)=0$.
        \item Note: If $\deg(f)\leq 3$ and $f$ is reducible, then there exists $g\in\Z[X]$ such that $\deg(g)=1$ and $g\mid f$.
        \begin{itemize}
            \item Let $f=gh$. We know that $3\geq\deg(f)=\deg(g)+\deg(h)$. Since $c(f)=1$ by hypothesis, $\deg(g)\neq 0\neq\deg(h)$. Thus, $1\leq\deg(g)\leq 3-\deg(h)\leq 2$ and a similar statement holds for $\deg(h)$. If $\deg(g)=1$, then we are done. If $\deg(g)=2$, then $\deg(h)=1$, and we are done.
            \item When we get to $\deg(f)=4$, the above argument obviously won't work (it would be perfectly acceptable to have $\deg(g)=\deg(h)=2$ here, for instance).
        \end{itemize}
    \end{itemize}
    \item We now move on to actual factorization techniques.
    \item Method 1: \textbf{Kronecker's method}.
    \begin{itemize}
        \item This method should be covered in the book somewhere.
    \end{itemize}
    \item Let $f$ have the same $n$-degree form as above.
    \item Let $1\leq d\leq n$. Does there exist $g\in\Z[X]$ with $c(g)=1$ and $\deg(g)=d$ such that $g\mid f$?
    \item Select $d+1$ distinct integers $c_0,\dots,c_d$.
    \item Easy lemma: Let $c_0,\dots,c_d\in F$ be distinct, and let
    \begin{equation*}
        P_d = \{g\in F[X]\mid\deg(g)\leq d\}
    \end{equation*}
    be a a $(d+1)$-dimensional vector space. Then $T:P_d\to F^{d+1}$ given by
    \begin{equation*}
        T(g) = (g(c_0),\dots,g(c_d))
    \end{equation*}
    is an isomorphism of $F$-vector spaces.
    \begin{proof}
        $P_d$ and $F^{d+1}$ both have the same dimension. Thus, to prove bijectivity of this linear transformation, it will suffice to prove injectivity. To do so, we will show that $\ker(T)=\{0\}$. Let $g\in\ker(T)$ be arbitrary. Then
        \begin{align*}
            T(g) &= 0\\
            (g(c_0),\dots,g(c_d)) &= (0,\dots,0)
        \end{align*}
        Thus, $g$ has $d+1$ distinct roots $c_0,\dots,c_d$. It follows that $g\in((X-c_0)\dots(X-c_d))$, meaning that $g=0$ or $\deg(g)\geq d+1$. However, $g\in P_d$ by hypothesis as well, meaning $\deg(g)\leq d$. Therefore, $g=0$, as desired.
    \end{proof}
    \item There is an alternative proof of this result that doesn't deal with any existence business but just gives you a formula for computing $T$.
    \item Corollary: Given $e_0,\dots,e_d\in F$ arbitrary, there exists a unique $g\in P_d$ such that $g(c_i)=e_i$ ($i=0,\dots,d$).
    \begin{itemize}
        \item Note that this is less a corollary and more a restatement of the lemma: A "unique" element of the domain speaks to bijectivity.
    \end{itemize}
    \item If such a $g$ exists, then $f=gh$ for some $h\in\Z[X]$. It follows that it is uniquely determined by its the values $g(c_0),\dots,g(c_d)$. But $g(c_i)\mid f(c_i)$ for all $i=0,\dots,d$. Note that if $f(c_i)=0$, then $X-c_i\mid f$ in $\Z[X]$.
    \item Now consider $S_i=\{u_i\in\Z:u_i\mid f(c_i)\}$. Then $S_0\times\cdots\times S_d\subset\Q^{d+1}$.
    \item Take $F=\Q$. Then $T:P_d\to\Q^{d+1}\supset S_0\times\cdots\times S_d$ where $T$ is an isomorphism.
    \item It follows that $g\in T^{-1}(S_0\times\cdots\times S_d)\cap\Z[X]\cap\{g:c(g)=1\}$. Thus, $g$ is an element of a finite set that is somewhat "known."
    \item Check whether or not $g\mid f$ (use the Euclidean Algorithm for monic polynomials).
    \item Then $f(X)=(X-c_0)\cdots(X-c_n)+b$
    \item Method 2.
    \begin{itemize}
        \item Basic philosophy: Given a monic polynomial over $\C$ and for which you know all of the coefficients, said coefficients yield an upper bound on the value of every root.
    \end{itemize}
    \item Lemma: Let $f(X)=a_0X^n+a_1X^{n-1}+\cdots+a_n\in\C[X]$ have $a_0\neq 0$. Define the number
    \begin{equation*}
        C = \max\left\{ \left| \frac{a_1}{a_0} \right|,\left| \frac{a_2}{a_0} \right|^{1/2},\dots,\left| \frac{a_n}{a_0} \right|^{1/n} \right\}
    \end{equation*}
    The elements in the max set are the coefficients of $1/\ell(f)$. If $z\in\C$ and $f(z)=0$, then $|z|\leq 2C$. Moreover,
    \begin{equation*}
        \frac{1}{2}+\frac{1}{4}+\cdots+\frac{1}{2^n} = 1
    \end{equation*}
    \begin{proof}
        If $C=0$, you're done. Thus, we assume that $C\neq 0$.\par
        WLOG, take $a_0=1$ so that $f$ is monic (if $a_0\neq 1$, divide through by $a_0$). It follows that
        \begin{align*}
            0 &= 1z^n+a_1z^{n-1}+\cdots+a_n\\
            -z^n &= a_1z^{n-1}+\cdots+a_n\\
            -1 &= a_1\frac{1}{z}+a_2\frac{1}{z^2}+\cdots+a_n\frac{1}{z^n}\\
            &= \left( \frac{a_1}{C} \right)\left( \frac{C}{z} \right)+\left( \frac{a_2}{C^2} \right)\left( \frac{C}{z} \right)^2+\cdots+\left( \frac{a_n}{C^n} \right)\left( \frac{C}{z} \right)^n
        \end{align*}
        By the definition of $C$, we have that
        \begin{equation*}
            |a_r|^{1/r} \leq C
        \end{equation*}
        Thus, $|a_r|\leq C^r$ and hence $|a_r/C^r|\leq 1$. We now can relate back to the above.\par
        If $|C/z|\leq 1/2$, this contradicts the triangle inequality (why??), so we must have $|C/z|>1/2$ or $|z/C|<2$ so $|z|<2C$.\par
        We now want $g\in\Z[X]$ with $c(g)=1$, $\deg(g)=d$, and $g\mid f$ in $\Z[X],\Q[X],\C[X]$.
        We have $g=b_0X^d+b_1X^{d-1}+\cdots+b_d$ ($b_i\in\Z$). Thus, $g/b_0=(X-z_1)\cdots(X-z_d)$ with $f(z_1)=\cdots=f(z_d)=0$. Then we have the following by expanding.
        \begin{equation*}
            = X^d-\left( \sum_{i=1}^dz_i \right)X^{d-1}+\left( \sum_{1\leq i\leq j\leq d}z_iz_j \right)X^{d-2}+\cdots
        \end{equation*}
        The second term is equal to $b_1/b_0$; the third is $b_2/b_0$; etc.
        We thus have an upper bound
        \begin{equation*}
            |b_r/b_0| \leq (2C)^r\binom{d}{r}
        \end{equation*}
        Note that $\ell(g)\mid\ell(f)$. The search for the coefficients is now limited to a finite space, and we are done. $a_0b_r/b_0\in\Z$ and we have an upper bound on its absolute value, specifically the following which, at this point, we can turn over the problem to someone with a computer to solve.
        \begin{equation*}
            |a_0b_r/b_0| \leq (2C)^r\binom{d}{r}(a_0)
        \end{equation*}
    \end{proof}
    \item A great technique for reducing polynomials modulo a prime number.
    \begin{itemize}
        \item Consider $0^2,1^2,2^2,3^2,4^2\pmod 5$. This is $\{0,\pm 1\}$. It follows that $m\equiv\pm 2\pmod 5$. $X^2-m\in\Z[X]$ is irreducible, but $(X^2-m)=(X-h)(X+h)$ implies that $X^2-h^2\equiv m\pmod 5$.
    \end{itemize}
\end{itemize}




\end{document}