\documentclass[../notes.tex]{subfiles}

\pagestyle{main}
\renewcommand{\chaptermark}[1]{\markboth{\chaptername\ \thechapter\ (#1)}{}}
\setcounter{chapter}{3}
\setcounter{proposition}{16}

\begin{document}




\chapter{???}
\section{Euclidean Domains and Reducibility}
\begin{itemize}
    \item \marginnote{1/23:}Notes to wrap up last time to start.
    \item Recall the theorem from last time: There is an injective ring homomorphism $\iota:R\to D^{-1}R$ such that for any $\varphi:R\to S$ such that $\varphi(D)\subset S^\times$, there exists a unique $\tilde{\varphi}:D^{-1}R\to S$ such that $\tilde{\varphi}\circ\iota=\varphi$.
    \begin{itemize}
        \item Callum redraws Figure \ref{fig:fracDecomp}.
    \end{itemize}
    \item Something Callum misstated last time: Diadic refers to 2-adic, not $p$-adic.
    \item Corollary: If $f\in R$ is not a zero divisor, then $R_f\cong R[X]/(fX-1)$.
    \begin{itemize}
        \item We can prove this using the universal property; it's on the HW.
    \end{itemize}
    \item \textbf{Subfield of $\bm{F}$ generated by $\bm{R}$}: The field defined as follows, where $F$ is a field and $R\subset F$ is an integral domain. \emph{Denoted by} $\bm{K}$. \emph{Given by}
    \begin{equation*}
        K = \bigcap_{\substack{R\subset F'\subset F\\F'\text{ a field}}}F'
    \end{equation*}
    \begin{itemize}
        \item Alternative definition: The smallest field inside $F$ that contains $R$.
    \end{itemize}
    \item Proposition: Let $R\subset F$ be an integral domain, where $F$ is a field. Then
    \begin{equation*}
        K \cong \Frac R
    \end{equation*}
    \begin{proof}
        Background: Consider the injection $R\to F$. It sends every element of $D=R\setminus\{0\}$ to a unit in $F$. Moreover, this function "factors through the fraction field" via Figure \ref{fig:fracDecomp} as per the theorem. We now begin the argument in earnest.\par
        To prove that $K\cong\Frac R$, we will use a bidirectional inclusion proof. For the forward direction, observe that $R\subset\Frac R\subset F$. Therefore, by the definition of $K$, $K\subset\Frac R$, as desired. For the backward direction, let $x/y\in\Frac R$ be arbitrary. To confirm that $x/y\in K$, it will suffice to verify that $x/y\in F'$ for all $R\subset F'\subset F$. Let $F'$ subject to said constraint be arbitrary. Since $x/y\in\Frac R$, $x,y\in R$. It follows since $R\subset F'$ that $x,y\in F'$. Thus, since $F'$ is a field and hence closed under multiplicative inverses, $1/y\in F'$. Finally, since $F'$ is closed under multiplication and $x,1/y\in F'$, we have that $x/y\in F'$, as desired.
    \end{proof}
    \item Example: Let $R=\Z[\sqrt{2}]=\Z[X]/(X^2-2)$. Then
    \begin{equation*}
        \Frac R = \Q[\sqrt{2}] = \frac{\Q[X]}{(X^2-2)}
    \end{equation*}
    \item That's it for rings of fractions. We now move onto Euclidean Domains (EDs), Principal Ideal Domains (PIDs), and Unique Factorization Domains (UFDs).
    \item An ED is a PID, and a PID is a UFD (hence, for example, an ED is both a PID and a UFD).
    \item \textbf{Norm}: A function from an integral domain $R$ to $\Zg$ that satisfies the following. \emph{Denoted by} $\bm{N}$. \emph{Constraints}
    \begin{enumerate}[label={(\roman*)}]
        \item Let $a\in R$. Then $N(a)=0$ iff $a=0$.
        \item $h,f\in R$ and $f\neq 0$ implies that there exists $q,r\in R$ such that $h=qf+r$ and $N(r)<N(f)$.
    \end{enumerate}
    \item \textbf{Euclidean domain}: An integral domain on which there exists a norm. \emph{Also known as} \textbf{ED}.
    \item Theorem: If $R$ is an ED, then $R$ is a PID.
    \begin{proof}
        % Identical to that for polynomials in one variable.
        % WTS: Let $I\subset R$ is a nonzero ideal. Let $d=\min\{N(a)\mid a\in I\setminus\{0\}\}$ (by assumption nonempty, and hence the min is well defined b/c of well-ordering). Choose $f\in I\setminus\{0\}$ such that $N(f)=d$. Since $I$ is an ideal and $f\in I$, $I\supset Rf$. Now let $h\in I$. Then $h=qf+r$ and $N(r)<N(f)=d$. Additionally, $r=h-qf$ where $h,qf\in I$ so $r\in I$. (Note that showing that $r\in I$ this way would not be acceptable in the HW??) It follows by the definition of $d$ that $r=0$, and hence that $h=qf$, and that's what we wanted to show, i.e., that everything is a multiple of $f$.


        This proof will use an analogous argument to that used in the proof that $F[X]$ is a PID from the end Lecture 3.1. Let's begin.\par
        To prove that $R$ is a PID, it will suffice show that for every ideal $I\subset R$, $I=(f)$ for some $f\in I$. Let $I\subset R$ be arbitrary. Let
        \begin{equation*}
            d = \min\{N(a)\mid a\in I\setminus\{0\}\}
        \end{equation*}
        Pick $f\in I\setminus\{0\}$ such that $N(f)=d$. We will now argue that $I=(f)$ via a bidirectional inclusion proof. In one direction, since $I$ is an ideal, $(f)=Rf\subset I$. In the other direction, let $h\in I$ be arbitrary. Then since $f\neq 0$ by assumption, the hypothesis that $R$ is an ED implies that there exist $q,r\in R$ such that $h=qf+r$ and $N(r)<N(f)$. It follows since $h,qf\in I$ that $r=h-qf\in I$. But since $N(r)<N(f)=d$, $r\in I$ implies by the definition of $d$ that necessarily $N(r)=0$ and hence $r=0$. Therefore, $h=qf$, as desired.
    \end{proof}
    \item Note that showing that $r\in I$ this way would not be acceptable in the HW??
    \item Examples of EDs:
    \begin{enumerate}
        \item $\Z$, $N(m)=|m|$.
        \begin{itemize}
            \item The norm is non-unique.
        \end{itemize}
        \item $F[X]$\footnote{Henceforth, "$F$" is assumed to denote a field.}, $N(f)=2^{\deg(f)}$.
        \begin{itemize}
            \item We define the norm in this way because then the degree of the zero polynomial being $-\infty$ makes $N(0)=2^{-\infty}=0$.
            \item Note that since $\deg(fg)=\deg(f)+\deg(g)$, $N(fg)=N(f)N(g)$ here.
        \end{itemize}
        \item $\Z[\sqrt{d}]=\{a+b\sqrt{d}\mid a,b\in\Z\}$ ($d$ is a \textbf{square-free integer}), $N(a+b\sqrt{d})=|(a+b\sqrt{d})(a-b\sqrt{d})|=|a^2-b^2d|$ for $a,b\in\Q$.
        \begin{itemize}
            \item Most famous example: $\Z[\sqrt{-1}]$, which are the \textbf{Gaussian integers}.
            \item Also interesting are $\Z[\sqrt{-2}]$, $\Z[\sqrt{2}]$, and $\Z[\frac{-1+\sqrt{-3}}{2}]\cong\Z[X]/(X^2+X+1)$.
            \begin{itemize}
                \item In the last example, the complex number in brackets is a cube root of unity equal to $\cos(120)+i\sin(120)$.
            \end{itemize}
            \item The reason why we define the norm on $\{a+b\sqrt{d}\}$ for $a,b\in\Q$ instead of $a,b\in\Z$.
            \begin{itemize}
                \item The number $\theta$ in $\Z[\theta]$ may not always be a radical or imaginary; it can be complex, too, as in the case of $\Z[\frac{-1+\sqrt{-3}}{2}]$.
                \item Let $\theta=\frac{-1+\sqrt{-3}}{2}$. In this case, we have
                \begin{equation*}
                    \left\{ \alpha+\beta\frac{-1+\sqrt{-3}}{2}\mid\alpha,\beta\in\Z \right\} \cong \left\{ a+b\sqrt{-3}\mid a,b\in\Q,\ a=\alpha-\frac{1}{2}\beta,\ b=\frac{1}{2}\beta,\ \alpha,\beta\in\Z \right\}
                \end{equation*}
            \end{itemize}
        \end{itemize}
    \end{enumerate}
    \item \textbf{Square-free integer}: An integer that is not divisible by the square of any integer.
    \item \textbf{Gaussian integers}: The Euclidean domain $\Z[\sqrt{-1}]$.
    \item \textbf{Unit}: An element $u\in R$ for which there exists $v\in R$ such that $uv=vu=1$.
    \item $\bm{R^\times}$: The set of all units of $R$.
    \begin{itemize}
        \item $(R^\times,\times)$ is a group.
    \end{itemize}
    \item Examples:
    \begin{enumerate}
        \item $F^\times=F\setminus\{0\}$.
        \item $F[X]^\times=F^\times$, i.e., is the nonzero constant polynomials.
        \begin{itemize}
            \item This is because any higher degree polynomial cannot be taken back down in degree --- multiplying polynomials adds degrees.
        \end{itemize}
        \item $\Z^\times=\{\pm 1\}$.
        \item $\Z[\sqrt{-1}]^\times=\{\pm 1,\pm i\}$.
        \item $R[X]^\times=R^\times$ ($R$ an integral domain).
        \item Suppose $R$ is not an integral domain. Then we get things like $a\neq 0\in R$ and $a^2=0$ (i.e., $a$ is a zero divisor) implies that $(1-aX)(1+aX)=1-a^2X^2=1$.
        \begin{itemize}
            \item We forbid this! It's nasty. Thus, we assume that rings of polynomials are taken over integral domains.
        \end{itemize}
    \end{enumerate}
    \item \textbf{Reducible} (element): A nonzero element $a\in R$ such that $a=bc$ and $b,c\notin R^\times$, where $R$ is an integral domain.
    \begin{itemize}
        \item Alternative definition: An element that is the product of two things, neither of which is a unit.
    \end{itemize}
    \item $R\setminus\{0\}$ is a disjoint union of\dots
    \begin{enumerate}[label={(\roman*)}]
        \item Units;
        \item Reducible elements;
        \item And irreducible elements.
    \end{enumerate}
    \begin{proof}
        Suppose for the sake of contradiction that $a\in R\setminus\{0\}$ is both reducible and a unit. Since $a$ is reducible, $a=bc$ where $b,c\notin R^\times$. Since $a$ is a unit, we may define $d=a^{-1}$. Then
        \begin{equation*}
            1 = ad
            = bcd
            = b(cd)
        \end{equation*}
        so $b\in R^\times$, a contradiction.
    \end{proof}
    \item Reducibility/irreducibility changes based on context.
    \item Example:
    \begin{itemize}
        \item Consider $F[[X]]$, where $X$ is taken to be irreducible.
        \begin{itemize}
            \item Here, all elements are of the form $uX^n$ for some $u\in F$ and $n\in\Zg$.
        \end{itemize}
        \item However, if we define $X=(X^{1/2})^2$, then $F[[X]]\subset F[[X^{1/2}]]$. In this larger context, $X$ is now reducible.
        \item We can continue the chain via
        \begin{equation*}
            \bigcup_{n=1}^\infty F[[X^{\frac{1}{2^n}}]]
        \end{equation*}
    \end{itemize}
    \item \textbf{Factorization} (of $a\in R$): A product of certain elements of $R$ that is equal to $a$, where $R$ is a ring; in particular, the product must consist of one unit $u$ and $r$ irreducible elements $\pi_1,\dots,\pi_r\in R$. \emph{Given by}
    \begin{equation*}
        a = u\pi_1\pi_2\cdots\pi_r
    \end{equation*}
    \item \textbf{Unique factorization domain}: A ring $R$ such that for every nonzero element $a\in R$, any two factorizations
    \begin{align*}
        a &= u\pi_1\pi_2\cdots\pi_r&
        a &= u'\pi_1'\pi_2'\cdots\pi_s'
    \end{align*}
    of $a$ satisfy the following conditions.
    \begin{enumerate}[label={(\roman*)}]
        \item $r=s$.
        \item There exists $\sigma\in S_r$ such that $\pi_i'=\pi_{\sigma(i)}u_i$ for all $1\leq i\leq r$, $u_i$ being a unit.
    \end{enumerate}
    \emph{Also known as} \textbf{UFD}.
    \item Wednesday: Show that a PID is a UFD.
\end{itemize}



\section{Unique Factorization Domains}
\begin{itemize}
    \item \marginnote{1/25:}Goal: UFDs.
    \item We review some definitions from last time to start.
    \item \textbf{Prime} (ideal): An ideal $P$ in a commutative ring $R$ for which $R/P$ is an integral domain.
    \begin{itemize}
        \item Equivalently, $1\notin P$ and $a,b\notin P$ imply $ab\notin P$, i.e., $R\setminus P$ is a multiplicative set.
    \end{itemize}
    \item Observation: Maximal ideals are prime ideals.
    \item From now on, $R$ denotes an integral domain.
    \item \textbf{Factorization} (of a nonzero element): A product $a=u\pi_1\pi_2\cdots\pi_r$, where $u\in R^\times$, each $\pi_i$ is irreducible, and $r=0$ is allowed.
    \item \textbf{Irreducible} (element): An element...
    \begin{itemize}
        \item Think of them a bit like primes, though this is very dangerous.
    \end{itemize}
    \item \textbf{Equivalent} (factorizations): Two factorizations $a=u\pi_1\pi_2\cdots\pi_r$ and $a=u'\pi_1'\pi_2'\cdots\pi_s'$ for which $r=s$ and there exists $\sigma\in S_r$ and $u_1,\dots,u_r\in R^\times$ such that $\pi_i'=u_i\pi_{\sigma(i)}$ ($i=1,\dots,r$) where $u\pi_1$ is also irreducible.
    \item \textbf{Unique factorization domain}: An integral domain $R$ for which every nonzero $a$ has a factorization and any factorizations of $a$ are equivalent to each other.
    \item \textbf{Prime} (element): A nonzero $\pi\in R$ for which $(\pi)$ is a prime ideal.
    \item Exercise: Prove that if $\pi$ is prime, then $\pi$ is irreducible.
    \begin{itemize}
        \item Note that $\pi$ irreducible does \emph{not} imply that $\pi$ is prime in general.
    \end{itemize}
    \item Lemma*: If every irreducible element of $R$ is prime, then any two factorizations of any nonzero $a\in R$ are equivalent.
    \begin{proof}
        % Thus, since anything that divides a unit is a unit, everything else goes away??
        % Example: $u=\alpha_1\alpha_2\cdots\alpha_m$ and $u\in R^\times$ implies that all $\alpha_i\in R^\times$. Thus, $r=0$ implies $s=0$, as desired.\par

        % Now suppose inductively that $\pi_1$ is irreducible implies $\pi_1$ is prime, i.e., $(\pi_1)$ is a prime ideal. Let $a\in(\pi_1)$ LHS. Then $u'\pi_1'\cdots\pi_s'$. Does not imply that $u'\in(\pi_1)$ since this would make $u'$ a unit?? Thus, $\pi_1'\in(\pi_1$). WLOG, let $\pi_1'\in(\pi_1)$. Then $\pi_1'=b\pi_1$. It follows that $\pi_1'$ irreducible implies $b\in R^\times$ or $\pi_1\in R^\times$, where the latter statement is a contradiction. Then $\pi_1'=b\pi_1$, $b\in R^\times$, where $u\pi_2\pi_3\cdots\pi_r=u'b\pi_2'\cdots\pi_s'$ and $b\in R^\times$. Finished by induction.


        We induct on the length $r\geq 0$ of factorizations.\par
        For the base case $r=0$, let $a\in R$ be arbitrary. Factor it into
        \begin{equation*}
            a = u\prod_{i=1}^r\pi_i
            = u\prod_{i=1}^0\pi_i
            = u
        \end{equation*}
        It follows that $a$ is a unit. Therefore, there exists $b\in R$ such that $ab=1$. Now suppose for the sake of contradiction that we also have
        \begin{equation*}
            a = u'\pi_1'\cdots\pi_s'
        \end{equation*}
        It follows that
        \begin{equation*}
            1 = (u'\pi_1'\cdots\pi_s')b
            = \pi_1'(u'\pi_2'\cdots\pi_s'b)
        \end{equation*}
        Thus, $\pi_1'$ is a unit, contradicting the hypothesis that $\pi_1'$ is irreducible. Therefore, $s=0$ and $u'=u$, as desired.\par
        Now suppose inductively that we have proven the claim for $r-1$; we now wish to prove it for $r$. Let
        \begin{align*}
            a &= u\pi_1\cdots\pi_r&
            a &= u'\pi_1'\cdots\pi_s'
        \end{align*}
        be two factorizations of an arbitrary $a\in R$. By the definition of a factorization, $\pi_1$ is irreducible. Thus, by hypothesis, $\pi_1$ is prime and hence $(\pi_1)$ is a prime ideal. Additionally, we have that
        \begin{equation*}
            a = u\pi_1\cdots\pi_r
            = (u\pi_2\cdots\pi_r)\pi_1
            \in R\pi_1
            = (\pi_1)
        \end{equation*}
        Thus, we must have $u'\pi_1'\cdots\pi_s'\in(\pi_1)$ as well. It follows that one of the elements in the product $u'\pi_1'\cdots\pi_s'$ is equal to $\pi_1b$ for some $b\in R$. Suppose for the sake of contradiction that this element is $u'$. Then $u'=\pi_1b$. But since $u'$ is a unit, there exists $c\in R$ such that $1=u'c$. It follows via substitution that
        \begin{equation*}
            1 = u'c
            = \pi_1bc
            = \pi_1(bc)
        \end{equation*}
        i.e., that $\pi_1$ is a unit, contradicting the hypothesis that it's irreducible. Therefore, $u'\notin(\pi_1)$. It follows that one of the $\pi_i'\in(\pi_1)$. WLOG, let $\pi_1'\in(\pi_1)$. Then $\pi_1'=u_1\pi_1$ for some $u_1\in R$. In particular, since $\pi_1'$ is irreducible, then either $u_1\in R^\times$ or $\pi_1\in R^\times$. But we can't have the second case since $\pi_1$ is irreducible (and hence not a unit) by assumption. Thus $u_1\in R^\times$. It follows that
        \begin{align*}
            a &= a\\
            u\pi_1\cdots\pi_r &= u'\pi_1'\cdots\pi_s'\\
            u\pi_1\cdots\pi_r &= u'u_1\pi_1\pi_2'\cdots\pi_s'\\
            u\pi_2\cdots\pi_r &= u'u_1\pi_2'\cdots\pi_s'
        \end{align*}
        where we apply the cancellation lemma in the last step, as permitted by the facts that $R$ is an integral domain and $\pi_1$ is irreducible (hence nonzero). Thus, by the induction hypothesis, the factorizations $u\pi_2\cdots\pi_r$ and $u'u_1\pi_2'\cdots\pi_s'$ are equivalent. It follows that $r=s$ and there exists $\sigma\in S_{[2:r]}$ and units $u_2,\dots,u_r\in R^\times$ such that $\pi_i'=u_i\pi_{\sigma(i)}$ ($i=2,\dots,r$). Extend $\sigma$ to $S_r$ by defining $\sigma(1)=1$. Thus, taking $\sigma\in S_r$ and $u_1,\dots,u_r\in R^\times$, we know that $\pi_i'=u_i\pi_i$ ($i=1,\dots,r$). Therefore, $u\pi_1\cdots\pi_r$ and $u'\pi_1'\cdots\pi_s'$ are equivalent factorizations of $a$, as desired.
    \end{proof}
    \item To prove that something is a UFD, it is all important to show that irreducible...??
    \item Notation: $a\mid b$ iff $b\in(a)$.
    \item \textbf{Greatest common divisor}: The number pertaining to $a,b\in R$ both nonzero which satisfies the following two constraints. \emph{Denoted by} $\bm{d}$, $\bm{\gcd(a,b)}$, $\bm{\textbf{g.c.d.}\,(a,b)}$. \emph{Constraints}
    \begin{enumerate}[label={(\roman*)}]
        \item $d\mid a$ and $d\mid b$.
        \item $d'\mid a$ and $d'\mid b$ implies $d'|d$.
    \end{enumerate}
    \item $d$ is well-defined up to multiplication by $u\in R^\times$.
    \begin{itemize}
        \item Example: We commonly think of $\gcd(6,9)=3$, but in $\Z$, it could also be $-3=-1\cdot 3$ where $-1\in\Z^\times=\{\pm 1\}$.
    \end{itemize}
    \item Essay: $d\mid a$ implies $a=bd$ and the factors of $d$ are a subset of the factors of $a$. Let $a=u\pi_1\cdots\pi_r\cdot\pi_1'\pi_2'\cdots\pi_h'$ and $b=u'\pi_1\cdots\pi_r\cdot\pi_1''\pi_2''\cdots\pi_g''$. For all $i\leq h$, $j\leq g$: $\pi_i\nmid\pi_j''$.
    \begin{itemize}
        \item I.e., the factors of $a,b$ that don't multiply out to $\gcd(a,b)=d$ are all relatively prime.
    \end{itemize}
    \item Let $d=\pi_1\cdots\pi_r=\gcd(a,b)R$.
    \item Existence of factorization in a PID.
    \item Example: $F[X]$.
    \begin{itemize}
        \item Recall that $F[X]$ is a PID.
        \item Let $f\in F[X]$ have $\deg(f)>0$.
        \item Then since PIDs are UFDs, $f=uf_1\cdots f_r$ where $u\in F[X]^\times=F^\times$ and each $f_i$ is irreducible.
        \item We have that $\deg f=\deg f_1+\cdots+\deg f_r\geq r$.
        \item This is the Fundamental Theorem of Algebra!
    \end{itemize}
    \item We now attempt a rigorous proof of existence in PIDs. Without a good norm (as we have in EDs), we need this proof.
    \begin{itemize}
        \item Suppose that $a\in R$ nonzero is not a unit.
        \item Then $a=bc$ where $b,c\notin R^\times$.
        \item If $b,c$ have a factorization, then $a=bc$ has a factorization.
        \item WLOG, let $b$ have a factorization.
        \item Let $a=b_1a_2$, where $b_1\notin R^\times$ and $a_2$ does not admit a factorization. Therefore, $a_2=b_2a_3$, where $b_2$ is not a unit and $a_3$ does not admit a factorization.
        \item We can go on forever: $a_n=b_na_{n+1}$ where $b_n\notin R^\times$ and $a_{n+1}\cdots$.
        \item It follows that $(a_n)\subset(a_{n+1})$ and $b_n\notin R^\times$ implies $(a_n)\neq(a_{n+1})$.
        \item All ideals $I_1\subset I_2\subset I_3\subset\cdots$. Is $\bigcup_{n=1}^\infty I_n$ an ideal? Yes, it is. Let's call it $I$.
        \item $R$ is a PID implies that $I=(\alpha)$.
        \item There exists $n$ such that $\alpha\in I_n$, and $(\alpha)\subset I_n\subsetneq I_{n+1}\subset\cdots\subset(\alpha)$.
        \item See the proof in the book for clarification: Theorem \ref{trm:8.14} on \textcite[287-89]{bib:DummitFoote}.
    \end{itemize}
    \item Last theorem to prove.
    \item Theorem: $R$ is a PID implies $R$ is a UFD.
    \begin{itemize}
        \item Existence, we've done.
        \item Equivalence: By Lemma*, we only need irreducible $\pi\in R$ to be prime.
        \item $a$ is reducible. $a=bc$, $b\notin R^\times$ and $c\notin R^\times$ implies $(a)\subsetneq(b)\subsetneq R$.
        \item Thus, $a$ is irreducible. It follows that $(a)$ is maximal and hence $(a)$ is prime. All these concepts are equivalent in a PID.
    \end{itemize}
    \item Examples: $\Z$, $F[X]$, $F[[X]]$.
    \item Let $a_n=b_na_{n+1}$. Then $(a_n)\subset(a_{n+1})$. and $b_n\notin R^\times$.
    \item If $(a_n)=(a_{n+1})$, then $a_{n+1}=ca_n$, $a_n=b_n\subset a_n$, $1=b_nc$.
\end{itemize}



\section{Office Hours (Callum)}
\begin{itemize}
    \item What kind of stuff from the recent lectures do we need to use in HW3?
    \begin{itemize}
        \item It is mostly content from before Wednesday of Week 3.
        \item The Euclidean algorithm will crop up in a few places, and some more recent/advanced stuff may be needed to solve the last problem.
    \end{itemize}
    \item Do we need to provide rationale for our answers to Q3.1?
    \begin{itemize}
        \item Yes.
        \item We can just give a general proof once in the first one.
    \end{itemize}
    \item Is Q3.2 a rote check of the definition? Are there any other factors to worry about?
    \begin{itemize}
        \item It is straight from the definition.
    \end{itemize}
    \item Is Q3.3(iii) too difficult?
    \begin{itemize}
        \item The forward inclusion $I_1I_2\subset I_1\cap I_2$ always holds. The backwards one needs coprime ideals (i.e., the fact that $(m)+(n)=\Z$ if $m,n$ are coprime).
    \end{itemize}
    \item Q3.5?
    \begin{itemize}
        \item No complications; just consecutive applications of the universal property of $R[X]$ should yield the desired result.
    \end{itemize}
    \item Is Q3.6 discussing evaluation functions?
    \begin{itemize}
        \item Yes, even though they're denoted $\phi$ there.
        \item See the Corollary from Lecture 3.1 for help on this problem.
    \end{itemize}
    \item Hint for Q3.6(ii)?
    \begin{itemize}
        \item This is a "you either see it or you don't" problem.
        \item It shouldn't take that long to do once you see it, but it could take a long time to see it.
    \end{itemize}
    \item For Q3.7, do we just have to define an inverse $\psi$ and check $\phi\circ\psi=\psi\circ\phi=\id$, or do we need to conduct a broader set of isomorphism checks, such as bijectivity, ring homomorphism ones, etc.?
    \begin{itemize}
        \item Cite Q3.5 for proving that the inverse is a ring homomorphism. Other than that, not really --- it is mainly about focusing on the inverse condition.
    \end{itemize}
    \item What is meant by "type" in Q3.8? Does the argument have to be a monomial of the given form, or are higher order polynomials allowed, too? Do you more broadly mean evaluation-based functions?
    \begin{itemize}
        \item Exactly the same monomial evaluation. The only degrees of freedom are $a,b$.
    \end{itemize}
    \item Is $\F_p=\Z/p\Z$?
    \begin{itemize}
        \item Yes.
        \item Note: Don't use $q$ as a dummy variable because $\F_q$ is something else.
    \end{itemize}
    \item In Q3.9(ii), how do I prove that there are always two $a$'s that go to $a^2$? Can I just show that $a^2=1^2a^2$ or something?
    \begin{itemize}
        \item Don't use (i) to prove (ii); just use similar reasoning.
        \item I've already made the big observation by noting that its $\pm a$ that both square to the same number. Rest should be smooth sailing.
    \end{itemize}
    \item Thoughts on Q3.10?
    \begin{itemize}
        \item By far the hardest question.
        \item Tips: Show that $X^2-\theta^2$ is a maximal ideal in the polynomial ring. If $f$ is irreducible, then $(f)$ is maximal. Check that $X^2-\theta^2$ is irreducible.
        \item Like 5 problems in 1 problem. Takes a bunch of techniques. The case where the square is zero is not hard. Write down four distinct rings and then use this to prove that you can't get any other ones. Keep them all in the quotient form? One is a product of two cyclic groups; that's a product of fields. You're allowed to multiply differently when they're rings, not groups. 2 groups, but 4 rings.
    \end{itemize}
\end{itemize}




\end{document}