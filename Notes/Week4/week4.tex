\documentclass[../notes.tex]{subfiles}

\pagestyle{main}
\renewcommand{\chaptermark}[1]{\markboth{\chaptername\ \thechapter\ (#1)}{}}
\setcounter{chapter}{3}
\setcounter{proposition}{16}

\begin{document}




\chapter{???}
\section{Euclidean Domains and Reducibility}
\begin{itemize}
    \item \marginnote{1/23:}Notes to wrap up last time to start.
    \item Recall the theorem from last time: There is an injective ring homomorphism $\iota:R\to D^{-1}R$ such that for any $\varphi:R\to S$ such that $\varphi(D)\subset S^\times$, there exists a unique $\tilde{\varphi}:D^{-1}R\to S$ such that $\tilde{\varphi}\circ\iota=\varphi$.
    \begin{itemize}
        \item Callum redraws Figure \ref{fig:fracDecomp}.
    \end{itemize}
    \item Something Callum misstated last time: Diadic refers to 2-adic, not $p$-adic.
    \item Corollary: If $f\in R$ is not a zero divisor, then $R_f\cong R[X]/(fX-1)$.
    \begin{itemize}
        \item We can prove this using the universal property; it's on the HW.
    \end{itemize}
    \item \textbf{Subfield of $\bm{F}$ generated by $\bm{R}$}: The field defined as follows, where $F$ is a field and $R\subset F$ is an integral domain. \emph{Denoted by} $\bm{K}$. \emph{Given by}
    \begin{equation*}
        K = \bigcap_{\substack{R\subset F'\subset F\\F'\text{ a field}}}F'
    \end{equation*}
    \begin{itemize}
        \item Alternative definition: The smallest field inside $F$ that contains $R$.
    \end{itemize}
    \item Proposition: Let $R\subset F$ be an integral domain, where $F$ is a field. Then
    \begin{equation*}
        K \cong \Frac R
    \end{equation*}
    \begin{proof}
        Background: Consider the injection $R\to F$. It sends every element of $D=R\setminus\{0\}$ to a unit in $F$. Moreover, this function "factors through the fraction field" via Figure \ref{fig:fracDecomp} as per the theorem. We now begin the argument in earnest.\par
        To prove that $K\cong\Frac R$, we will use a bidirectional inclusion proof. For the forward direction, observe that $R\subset\Frac R\subset F$. Therefore, by the definition of $K$, $K\subset\Frac R$, as desired. For the backward direction, let $x/y\in\Frac R$ be arbitrary. To confirm that $x/y\in K$, it will suffice to verify that $x/y\in F'$ for all $R\subset F'\subset F$. Let $F'$ subject to said constraint be arbitrary. Since $x/y\in\Frac R$, $x,y\in R$. It follows since $R\subset F'$ that $x,y\in F'$. Thus, since $F'$ is a field and hence closed under multiplicative inverses, $1/y\in F'$. Finally, since $F'$ is closed under multiplication and $x,1/y\in F'$, we have that $x/y\in F'$, as desired.
    \end{proof}
    \item Example: Let $R=\Z[\sqrt{2}]=\Z[X]/(X^2-2)$. Then
    \begin{equation*}
        \Frac R = \Q[\sqrt{2}] = \frac{\Q[X]}{(X^2-2)}
    \end{equation*}
    \item That's it for rings of fractions. We now move onto Euclidean Domains (EDs), Principal Ideal Domains (PIDs), and Unique Factorization Domains (UFDs).
    \item An ED is a PID, and a PID is a UFD (hence, for example, an ED is both a PID and a UFD).
    \item \textbf{Norm}: A function from an integral domain $R$ to $\Zg$ that satisfies the following. \emph{Denoted by} $\bm{N}$. \emph{Constraints}
    \begin{enumerate}[label={(\roman*)}]
        \item Let $a\in R$. Then $N(a)=0$ iff $a=0$.
        \item $h,f\in R$ and $f\neq 0$ implies that there exists $q,r\in R$ such that $h=qf+r$ and $N(r)<N(f)$.
    \end{enumerate}
    \item \textbf{Euclidean domain}: An integral domain on which there exists a norm. \emph{Also known as} \textbf{ED}.
    \item Theorem: If $R$ is an ED, then $R$ is a PID.
    \begin{proof}
        % Identical to that for polynomials in one variable.
        % WTS: Let $I\subset R$ is a nonzero ideal. Let $d=\min\{N(a)\mid a\in I\setminus\{0\}\}$ (by assumption nonempty, and hence the min is well defined b/c of well-ordering). Choose $f\in I\setminus\{0\}$ such that $N(f)=d$. Since $I$ is an ideal and $f\in I$, $I\supset Rf$. Now let $h\in I$. Then $h=qf+r$ and $N(r)<N(f)=d$. Additionally, $r=h-qf$ where $h,qf\in I$ so $r\in I$. (Note that showing that $r\in I$ this way would not be acceptable in the HW??) It follows by the definition of $d$ that $r=0$, and hence that $h=qf$, and that's what we wanted to show, i.e., that everything is a multiple of $f$.


        This proof will use an analogous argument to that used in the proof that $F[X]$ is a PID from the end Lecture 3.1. Let's begin.\par
        To prove that $R$ is a PID, it will suffice show that for every ideal $I\subset R$, $I=(f)$ for some $f\in I$. Let $I\subset R$ be arbitrary. Let
        \begin{equation*}
            d = \min\{N(a)\mid a\in I\setminus\{0\}\}
        \end{equation*}
        Pick $f\in I\setminus\{0\}$ such that $N(f)=d$. We will now argue that $I=(f)$ via a bidirectional inclusion proof. In one direction, since $I$ is an ideal, $(f)=Rf\subset I$. In the other direction, let $h\in I$ be arbitrary. Then since $f\neq 0$ by assumption, the hypothesis that $R$ is an ED implies that there exist $q,r\in R$ such that $h=qf+r$ and $N(r)<N(f)$. It follows since $h,qf\in I$ that $r=h-qf\in I$. But since $N(r)<N(f)=d$, $r\in I$ implies by the definition of $d$ that necessarily $N(r)=0$ and hence $r=0$. Therefore, $h=qf$, as desired.
    \end{proof}
    \item Note that showing that $r\in I$ this way would not be acceptable in the HW??
    \item Examples of EDs:
    \begin{enumerate}
        \item $\Z$, $N(m)=|m|$.
        \begin{itemize}
            \item The norm is non-unique.
        \end{itemize}
        \item $F[X]$\footnote{Henceforth, "$F$" is assumed to denote a field.}, $N(f)=2^{\deg(f)}$.
        \begin{itemize}
            \item We define the norm in this way because then the degree of the zero polynomial being $-\infty$ makes $N(0)=2^{-\infty}=0$.
            \item Note that since $\deg(fg)=\deg(f)+\deg(g)$, $N(fg)=N(f)N(g)$ here.
        \end{itemize}
        \item $\Z[\sqrt{d}]=\{a+b\sqrt{d}\mid a,b\in\Z\}$ ($d$ is a \textbf{square-free integer}), $N(a+b\sqrt{d})=|(a+b\sqrt{d})(a-b\sqrt{d})|=|a^2-b^2d|$ for $a,b\in\Q$.
        \begin{itemize}
            \item Most famous example: $\Z[\sqrt{-1}]$, which are the \textbf{Gaussian integers}.
            \item Also interesting are $\Z[\sqrt{-2}]$, $\Z[\sqrt{2}]$, and $\Z[\frac{-1+\sqrt{-3}}{2}]\cong\Z[X]/(X^2+X+1)$.
            \begin{itemize}
                \item In the last example, the complex number in brackets is a cube root of unity equal to $\cos(120)+i\sin(120)$.
            \end{itemize}
            \item The reason why we define the norm on $\{a+b\sqrt{d}\}$ for $a,b\in\Q$ instead of $a,b\in\Z$.
            \begin{itemize}
                \item The number $\theta$ in $\Z[\theta]$ may not always be a radical or imaginary; it can be complex, too, as in the case of $\Z[\frac{-1+\sqrt{-3}}{2}]$.
                \item Let $\theta=\frac{-1+\sqrt{-3}}{2}$. In this case, we have
                \begin{equation*}
                    \left\{ \alpha+\beta\frac{-1+\sqrt{-3}}{2}\mid\alpha,\beta\in\Z \right\} \cong \left\{ a+b\sqrt{-3}\mid a,b\in\Q,\ a=\alpha-\frac{1}{2}\beta,\ b=\frac{1}{2}\beta,\ \alpha,\beta\in\Z \right\}
                \end{equation*}
            \end{itemize}
        \end{itemize}
    \end{enumerate}
    \item \textbf{Square-free integer}: An integer that is not divisible by the square of any integer.
    \item \textbf{Gaussian integers}: The Euclidean domain $\Z[\sqrt{-1}]$.
    \item \textbf{Unit}: An element $u\in R$ for which there exists $v\in R$ such that $uv=vu=1$.
    \item $\bm{R^\times}$: The set of all units of $R$.
    \begin{itemize}
        \item $(R^\times,\times)$ is a group.
    \end{itemize}
    \item Examples:
    \begin{enumerate}
        \item $F^\times=F\setminus\{0\}$.
        \item $F[X]^\times=F^\times$, i.e., is the nonzero constant polynomials.
        \begin{itemize}
            \item This is because any higher degree polynomial cannot be taken back down in degree --- multiplying polynomials adds degrees.
        \end{itemize}
        \item $\Z^\times=\{\pm 1\}$.
        \item $\Z[\sqrt{-1}]^\times=\{\pm 1,\pm i\}$.
        \item $R[X]^\times=R^\times$ ($R$ an integral domain).
        \item Suppose $R$ is not an integral domain. Then we get things like $a\neq 0\in R$ and $a^2=0$ (i.e., $a$ is a zero divisor) implies that $(1-aX)(1+aX)=1-a^2X^2=1$.
        \begin{itemize}
            \item We forbid this! It's nasty. Thus, we assume that rings of polynomials are taken over integral domains.
        \end{itemize}
    \end{enumerate}
    \item \textbf{Reducible} (element): A nonzero element $a\in R$ such that $a=bc$ and $b,c\notin R^\times$, where $R$ is an integral domain.
    \begin{itemize}
        \item Alternative definition: An element that is the product of two things, neither of which is a unit.
    \end{itemize}
    \item $R\setminus\{0\}$ is a disjoint union of\dots
    \begin{enumerate}[label={(\roman*)}]
        \item Units;
        \item Reducible elements;
        \item And irreducible elements.
    \end{enumerate}
    \begin{proof}
        Suppose for the sake of contradiction that $a\in R\setminus\{0\}$ is both reducible and a unit. Since $a$ is reducible, $a=bc$ where $b,c\notin R^\times$. Since $a$ is a unit, we may define $d=a^{-1}$. Then
        \begin{equation*}
            1 = ad
            = bcd
            = b(cd)
        \end{equation*}
        so $b\in R^\times$, a contradiction.
    \end{proof}
    \item Reducibility/irreducibility changes based on context.
    \item Example:
    \begin{itemize}
        \item Consider $F[[X]]$, where $X$ is taken to be irreducible.
        \begin{itemize}
            \item Here, all elements are of the form $uX^n$ for some $u\in F$ and $n\in\Zg$.
        \end{itemize}
        \item However, if we define $X=(X^{1/2})^2$, then $F[[X]]\subset F[[X^{1/2}]]$. In this larger context, $X$ is now reducible.
        \item We can continue the chain via
        \begin{equation*}
            \bigcup_{n=1}^\infty F[[X^{\frac{1}{2^n}}]]
        \end{equation*}
    \end{itemize}
    \item \textbf{Factorization} (of $a\in R$): A product of certain elements of $R$ that is equal to $a$, where $R$ is a ring; in particular, the product must consist of one unit $u$ and $r$ irreducible elements $\pi_1,\dots,\pi_r\in R$. \emph{Given by}
    \begin{equation*}
        a = u\pi_1\pi_2\cdots\pi_r
    \end{equation*}
    \item \textbf{Unique factorization domain}: A ring $R$ such that for every nonzero element $a\in R$, any two factorizations
    \begin{align*}
        a &= u\pi_1\pi_2\cdots\pi_r&
        a &= u'\pi_1'\pi_2'\cdots\pi_s'
    \end{align*}
    of $a$ satisfy the following conditions.
    \begin{enumerate}[label={(\roman*)}]
        \item $r=s$.
        \item There exists $\sigma\in S_r$ such that $\pi_i'=\pi_{\sigma(i)}u_i$ for all $1\leq i\leq r$, $u_i$ being a unit.
    \end{enumerate}
    \emph{Also known as} \textbf{UFD}.
    \item Wednesday: Show that a PID is a UFD.
\end{itemize}




\end{document}