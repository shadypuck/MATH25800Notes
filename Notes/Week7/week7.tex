\documentclass[../notes.tex]{subfiles}

\pagestyle{main}
\renewcommand{\chaptermark}[1]{\markboth{\chaptername\ \thechapter\ (#1)}{}}
\setcounter{chapter}{6}

\begin{document}




\chapter{Modules Over PIDs}
\section{Zorn's Lemma and Intro to Modules Over PIDs}
\begin{itemize}
    \item \marginnote{2/13:}Picking up from last time with Zorn's lemma.
    \item \textbf{Partially ordered set}: A set together with a binary relation indicating that, for certain pairs of elements in the set, one of the elements precedes the other in the ordering. \emph{Also known as} \textbf{poset}. \emph{Denoted by} $\bm{P}$.
    \begin{itemize}
        \item The domain of the \textbf{partial order} may be a proper subset of $P\times P$.
    \end{itemize}
    \item \textbf{Partial order}: The binary relation on a poset.
    \item \textbf{Maximal} ($f\in P$): An element $f\in P$ such that for all $q\in P$, the statement $q>f$ is false.
    \item Example.
    \begin{itemize}
        \item Let $X$ be a set with $|X|\geq 2$\footnote{Nori denotes cardinality by $\#X$.}.
        \item Define a poset $P=\{A\subsetneq X\}$ with corresponding partial order defined by taking subsets. In particular, if $A\subset B$, write $A\leq B$.
        \item For any $x\in X$, $X-\{x\}$ is a maximal element of $P$.
    \end{itemize}
    \item \textbf{Chain}: A subset of a poset $P$ such that if $c_1,c_2$ are in said subset, then implies $c_1\leq c_2$ or $c_2\leq c_1$. \emph{Denoted by} $\bm{C}$.
    \begin{itemize}
        \item In other words, a chain is a subset of a poset that is a \textbf{totally ordered set}.
    \end{itemize}
    \item \textbf{Totally ordered set}: A set together with a binary relation indicating that, for any pair of elements in the set, one of the elements precedes the other in the ordering.
    \item Observation: If $F$ is a subset of a nonempty finite chain $C$, then there exists $c\in F$ such that $c\geq q$ for all $q\in F$.
    \item \textbf{Upper bound} (of $C$): An element $p\in P$ such that $p\geq c$ for all $c\in C$.
    \item \textbf{Zorn's lemma}: Let $P$ be a poset that satisfies
    \begin{enumerate}[label={(\roman*)}]
        \item $P\neq\emptyset$;
        \item Every chain $C\subset P$ has an upper bound.
    \end{enumerate}
    Then $P$ has a maximal element.
    \item We will not prove Zorn's lemma. It rarely if ever gets proven in an undergraduate course, maybe in a logic course.
    \begin{itemize}
        \item And by "prove" we mean "deduce Zorn's lemma from the Axiom of Choice."
    \end{itemize}
    \item We now investigate a situation in which Zorn's lemma gets applied.
    \item Let $M$ be a finitely generated $A$-module.
    \begin{itemize}
        \item Let $v_1,\dots,v_r\in M$ be elements such that such that $M=Av_1+\cdots+Av_r$.
        \item Before we prove the proposition that requires Zorn's lemma, we will need one more definition: that of a \textbf{maximal submodule}.
    \end{itemize}
    \item \textbf{Maximal submodule} (of $M$): A submodule of $M$ that is a maximal element of the poset
    \begin{equation*}
        P = \{N\subsetneq M:N\text{ is an }A\text{-submodule}\}
    \end{equation*}
    \item Proposition: Every nonzero finitely generated $A$-module $M$ has a maximal submodule.
    \begin{proof}
        % Apply Zorn's lemma: First, check that $P$ is nonempty. $M\neq\{0\}$ (by hypothesis) implies that $\{0\}\in P$ implies that $P$ is nonempty. Now let $C\subset P$ be a chain. Let $\mathcal{N}=\bigcup\{N:N\in C\}$. We will check that $\mathcal{N}$ is a submodule and that it's in $P$. It's a submodule since: Let $n_1,n_2\in\mathcal{N}$. Then $n_i\in N_i$ for $i=1,2$ and $N_1,N_2\in C$. WLOG, assume that $N_1\subset N_2$. Then $n_1,n_2\in N_2$. It follows that $a_1n_1+a_2n_2\in N_2\subset\mathcal{N}$ for all $a_1,a_2\in A$. This completes the proof that $\mathcal{N}$ is a submodule of $M$.\par
        % Now we address if it's possible that $\mathcal{N}\in P$. Suppose for the sake of contradiction that $\mathcal{N}=M$. We have $v_1,\dots,v_r\in M$ such that $M=Av_1+\cdots+Av_r$. Because $\mathcal{N}=M$, $v_i\in\mathcal{N}$ for all $i$. Thus, there exist $N_i\in C$ such that $v_i\in N_i$ for all $i=1,\dots,r$. So there exists $i\in\{1,\dots,r\}$ such that for all $j\in\{1,\dots,r\}$, $N_j\subset N_i$. Therefore, $v_j\in N_j\subset N_i$ for all $j=1,\dots,r$. But $N_i$ is a module, so $M=Av_1+\cdots+Av_r\subset N_i\subset M$. So $N_i=M$, which contradicts the assumption that $N_i$ is proper (because $N_i\in P$). Therefore, $\mathcal{N}\in P$.\par
        % We now apply Zorn's lemma to get the desired result.

        To prove that $M$ has a maximal submodule, it will suffice show that there exists a maximal element of the poset
        \begin{equation*}
            P = \{N\subsetneq M:N\text{ is an }A\text{-submodule}\}
        \end{equation*}
        To do this, Zorn's lemma tells us that it will suffice to confirm that $P\neq\emptyset$ and that every chain $C\subset P$ has an upper bound. Let's begin.\par
        We first confirm that $P\neq\emptyset$. By hypothesis, $M$ is nonzero. Thus, the zero $A$-submodule is a proper subset of $M$, so $0\in P$ and hence $P$ is nonempty.\par
        We now confirm that every chain $C\subset P$ has an upper bound. Let $C\subset P$ be an arbitrary chain. Define
        \begin{equation*}
            \mathcal{N}_C = \bigcup\{N:N\in C\}
        \end{equation*}
        We will first verify that $\mathcal{N}_C\in P$, and then we will show that $\mathcal{N}_C$ is an upper bound of $C$. Let's begin. To verify that $\mathcal{N}_C\in P$, it will suffice to demonstrate that $\mathcal{N}_C$ is an $A$-submodule of $M$ and that $\mathcal{N}_C\subsetneq M$.\par
        To demonstrate that $\mathcal{N}_C$ is an $A$-submodule, Proposition \ref{prp:10.1} tells us that it will suffice to show that $\mathcal{N}_C\neq\emptyset$ and $n_1+an_2\in\mathcal{N}_C$ for all $a\in A$ and $n_1,n_2\in\mathcal{N}_C$. Since $P$ is nonempty, $\mathcal{N}_C$ is nonempty by definition, as desired. Additionally, let $n_1,n_2\in\mathcal{N}_C$ be arbitrary. It follows by the definition of $\mathcal{N}_C$ that there exist $N_1,N_2\in C$ such that $n_i\in N_i$ ($i=1,2$). WLOG, assume $N_1\subset N_2$. Then $n_1,n_2\in N_2$. It follows since $N_2$ is an $A$-submodule that $n_1+an_2\in N_2\subset\mathcal{N}_C$ for all $a\in A$, as desired.\par
        We know that $\mathcal{N}_C\subset M$. Thus, if $\mathcal{N}_C\not\subsetneq M$, then we must have $\mathcal{N}_C=M$. Suppose for the sake of contradiction that $\mathcal{N}_C=M$. Recall that $M=Av_1+\cdots+Av_r$. Since the $v_i$ are elements of $M$ and $\mathcal{N}_C=M$, it follows that $v_i\in\mathcal{N}_C$ ($i=1,\dots,r$). Thus, as before, there must exist $N_1,\dots,N_r\in C$, not necessarily distinct, such that $v_i\in N_i$ ($i=1,\dots,r$). It follows by the observation from earlier that there is an $i\in[r]$ such that for all $j\in[r]$, $N_j\subset N_i$. Consequently, $v_j\in N_j\subset N_i$ ($j=1,\dots,r$). But $N_i$ is an $A$-submodule, so $M=Av_1+\cdots+Av_r\subset N_i\subset M$. But this means that $N_i=M$, contradicting the assumption that $N_i\subsetneq P$ (since $N_i\in P$). Therefore, $\mathcal{N}_C\subsetneq M$, as desired.\par
        It follows that $\mathcal{N}_C\in P$, as desired. Lastly, we have by its definition that $N\subset\mathcal{N}_C$ for all $N\in C$, meaning that $\mathcal{N}_C$ is an upper bound of $C$ by definition. Therefore, by Zorn's lemma, $P$ has a maximal element, and hence $M$ has a maximal submodule, as desired.
    \end{proof}
    \item Corollary: Every nonzero commutative ring $R$ has a maximal ideal.
    \begin{proof}
        % $A$ can be considered a module over itself.\par
        % In particular, take $M=A$, $A\neq 0$, and let it be generated by 1.

        Consider $R$ as an $R$-module. Then $R=(1)$ is finitely generated. This combined with the fact that it is nonzero by hypothesis allows us to invoke the above proposition, learning that $R$ has a maximal submodule $N$. But by the observation from Lecture 6.1, $N$ is a left ideal, which is equivalent to a two-sided ideal in a commutative ring. Maximality transfers over as well (as we can confirm), proving that $N$ is the desired maximal ideal of $R$.
    \end{proof}
    \item Remark: Suppose that $J$ is a two-sided ideal of $A$. Let $M$ be an $A$-module such that for all $a\in J$ and $m\in M$, we have $am=0$. Then $M$ may be regarded as an $(A/J)$-module in a natural manner.
    \begin{itemize}
        \item In particular, we may take $\rho:A\to\End(M,+)$ to be a ring homomorphism.
        \item We can factor $\rho=\bar{\rho}\circ\pi$, where $\pi:A\to A/J$ and $\bar{\rho}:A/J\to\End(M,+)$. It follows that $\bar{\rho}$ is a ring homomorphism. Therefore, $M$ is an $A/J$-module.
        \item This remark will be used!
        \item Review annihilators from Section 10.1!
    \end{itemize}
    \item Remark: Given a left ideal $I\subset A$ and an $A$-module $M$, we get a whole lot of modules because each element of $M$ generates one. In particular, we note that $Im\subset Am\subset M$, where both $Im,Am$ are submodules for all $m\in M$.
    \item \textbf{Product} (of modules): The $A$-submodule of $M$ defined as follows. \emph{Denoted by} $\bm{IM}$. \emph{Given by}
    \begin{equation*}
        IM = \sum_{m\in M}Im
    \end{equation*}
    \item It follows that $M/IM$ is an $A$-module, but also one with a special property: $a(M/IM)=0$ for all $a\in I$.
    \begin{itemize}
        \item If $A$ is commutative, then $M/IM$ is an $A/I$-module.
    \end{itemize}
    \item Proposition: Let $R$ be a nonzero commutative ring. If $R^m\cong R^n$ as $R$-modules, then $m=n$.
    \begin{proof}
        Let $I\subset R$ be a maximal ideal. (We know that one exists by the above corollary.) If $f:R^m\to R^n$ is an isomorphism of $R$-modules, then $f$ restricts to $I(R^m)\to I(R^n)$. This gives rise to the isomorphism $\bar{f}:R^m/I(R^m)\to R^n/I(R^n)$ of $R$-modules, in fact of $R/I$ modules. It follows that $R/I$ is a field, so $m=n$.
    \end{proof}
    \item Classifying modules up to isomorphism under commutative rings.
    \begin{itemize}
        \item This is a hard problem, and there are still many open problems in this field today.
        \item We will not go into this, though.
    \end{itemize}
    \item We now move on to modules over PIDs.
    \begin{itemize}
        \item Nori will go \emph{much} slower than the book.
        \item Do you have any recommended resources??
        \item Do we need to read and understand Chapters 10-11 to start on Chapter 12??
    \end{itemize}
    \item Objective: Let $R$ be a PID. Classify all finitely generated $R$-modules up to isomorphism.
    \begin{itemize}
        \item Our first result in this field was that submodules of $R^n$ are equal to $R^m$ for $m\leq n$.
        \item Where this is applicable: $\Z$ and $F[X]$.
        \begin{itemize}
            \item Go back and check out $\Z$-modules and $F[X]$-modules in Section 10.1!
        \end{itemize}
    \end{itemize}
    \item \textbf{Torsion module}: An $R$-module $M$ such that for all $m\in M$, there exists $0\neq a\in R$ such that $am=0$.
    \item \textbf{Torsion-free module}: An $R$-module $M$ such that for all nonzero $m\in M$ and for all nonzero $a\in R$, we have $am\neq 0$.
    \item Theorem: If $M$ is a finitely generated torsion-free $R$-module, then $M\cong R^n$ for some $n$.
    \begin{itemize}
        \item With a little work, we could prove this. But Nori will postpone it.
    \end{itemize}
    \item \textbf{$\bm{p}$-primary} (module): An $R$-module $M$ such that for all $m\in M$, there exists $k\geq 0$ for which $p^km=0$, where $p$ is prime in $R$.
    \item We want to classify these up to isomorphism.
    \begin{itemize}
        \item Nori can state these today, but will not have time to prove it until another day.
        \item Something that gets annihilated by $p$ is a $\Z/(p)$-module. The moment you go from $k=1$ to $k=2$, things get interesting.
    \end{itemize}
    \item Examples: $R/(p^{n_1})\oplus\cdots\oplus R/(p^{n_k})$, where $n_1\geq\cdots\geq n_k\geq 1$.
    \begin{itemize}
        \item Note that $k=0$ is allowed.
    \end{itemize}
    \item Uniqueness will take some time, but existence can be given as an exercise now.
    \item $M/pM$ is an $R/(p)$-vector space. $pM/p^2M$ is an $R/(p)$-vector space as well. So is $p^kM/p^{k+1}M$.
    \begin{itemize}
        \item Use $d_0,d_1,\dots,d_k$ to denote the dimensions of the vector spaces.
        \item $d_0,\dots,d_k$ is a decreasing sequence of nonnegative integers.
    \end{itemize}
\end{itemize}



\section{Office Hours (Nori)}
\begin{itemize}
    \item Homework questions.
    \begin{itemize}
        \item See pictures + unnumbered lemma.
        \item Example of the kernel being bigger than $(f)$.
        \item A ring homomorphism $\Z[X]\to\R$ must be evaluation by the universal property of polynomial rings.
        \item Factoring enables a constraint on $a$.
    \end{itemize}
    \item Lecture 6.1: Proposition proof?
    \item Lecture 6.1: $(2)\subsetneq\Z$ example?
    \item Lecture 6.1: The end of the theorem proof.
    \item Lecture 6.2: Does the first theorem you proved not appear in the book until Chapter 12?
    \item Lecture 6.2: What is $A$ in the proof?
    \item Resources for the proofs in Week 6?
    \item Lecture 7.1: Quotient stuff.
    \item Recommended resources for modules over PIDs? Chapter 12?
    \begin{itemize}
        \item We should be able to read chapter 12, since chapter 11 is just vector spaces.
        \item Nori's doing Chapter 12 in the classical manner (pre-1970). \textcite{bib:DummitFoote} just does it in the first few pages as the \textbf{elementary divisor theorem}.
    \end{itemize}
    \item HW6: So you want us to solve 1, 10, 13 for our own edification, but we don't need to write up a solution? Will we ever be responsible for the content therein?
    \begin{itemize}
        \item We'll need to understand them to move forward.
        \item Q6.4-Q6.5 are particularly important (good for number theory).
    \end{itemize}
\end{itemize}



\section{Office Hours (Ray)}
\begin{itemize}
    \item Universal properties save you from having to do pages upon pages of ring homomorphism checks (think Q3.10).
    \item Algebra: Chapter 0 by Paolo Aluffi for learning quotienting by polynomials.
    \begin{itemize}
        \item Universal properties show up on page 30.
        \item Read stuff before as needed.
        \item Has a chapter called universal properties of polynomial rings. Universal properties of quotients, too.
    \end{itemize}
    \item Direct sums and direct products.
    \begin{itemize}
        \item Let $M,N$ be $R$-modules. Then $M\times N$ is an $R$-module defined by the Cartesian product of the sets and with \textbf{diagonal} module action $r(m,n)=(rm,rn)$ (diagonal meaning we just act on two elements).
        \item $M\oplus N=M\times N$.
        \item For infinite sets, we get a difference. Indeed, $\prod_{i=1}^\infty M_i\neq\bigoplus_{i=1}^\infty M_i$.
    \end{itemize}
\end{itemize}



\section{Classifying Modules Over PIDs}
\begin{itemize}
    \item \marginnote{2/15:}We pick up from yesterday, classifying finitely generated $R$-modules $M$ up to isomorphism when $R$ is a PID.
    \item In particular, we begin with a further investigation of the properties of torsion modules.
    \item \textbf{Lift} (of $x\in M/M'$): The choice of an element $y\in M$ such that $\pi(y)=x$.
    \item Lemma:
    \begin{enumerate}[label={(\roman*)}]
        \item $\Tor(M)$ is an $R$-submodule of $M$.
        \begin{proof}
            To prove that $\Tor(M)$ is an $R$-submodule of $M$, Proposition \ref{prp:10.1} tells us that it will suffice to show that $\Tor(M)\neq\emptyset$ and that $x+ry\in\Tor(M)$ for all $r\in R$, $x,y\in\Tor(M)$. Consider $0\in M$. By definition, $r\cdot 0=0$. Thus, $0\in\Tor(M)$ as desired. Additionally, let $r\in R$ and $x,y\in\Tor(M)$ be arbitrary. Since $x,y\in\Tor(M)$, there exist nonzero $a,b\in R$ such that $ax=0$ and $by=0$. Because $R$ is an integral domain (as a PID), $a,b$ nonzero implies that $ab\neq 0$. Thus, since
            \begin{equation*}
                ab(x+ry) = abx+abry
                = b(ax)+ar(by)
                = b(0)+ar(0)
                = 0
            \end{equation*}
            we have that $x+ry\in\Tor(M)$, as desired.
        \end{proof}
        \item The quotient module $M/\Tor(M)$ is torsion-free.
        \begin{proof}
            To prove that $M/\Tor(M)$ is torsion-free, it will suffice to show that every torsion element of $M/\Tor(M)$ is 0. Let's begin. Let $v\in M/\Tor(M)$ be an arbitrary torsion element. Then there exists $a\in R$ nonzero such that $av=0$. Now lift $v\in M/\Tor(M)$ to $w\in M$. The constraint $av=0=0+\Tor(M)$ from the quotient module implies that $0=a\pi(w)=\pi(aw)$, hence $aw\in\Tor(M)$. Thus, there exists $b\in R$ nonzero such that $b(aw)=0$. It follows that $(ba)w=0$, where $ba\neq 0$ since $a,b\neq 0$ by the fact that $R$ is an integral domain. Thus, $w\in\Tor(M)$, and hence $v=\pi(w)=0$, as desired.
        \end{proof}
    \end{enumerate}
    \item We now give some claims that will be useful later today, but whose proofs we will delay until next lecture.
    \item The first one pertains to the properties of finitely generated torsion-free modules over an integral domain.
    \item Lemma: Let $R$ be an integral domain, and let $M$ be a finitely generated $R$-module. Then there exists a submodule $M'\subset M$ such that\dots
    \begin{enumerate}[label={(\roman*)}]
        \item $M'\cong R^h$ for some $h\geq 0$;
        \item There exists a nonzero $a\in R$ such that $aM\subset M'$ (equivalently, $a(M/M')=0$).
    \end{enumerate}
    \item The next two pertain to the properties of finitely generated modules over a PID.
    \item Corollary: Every finitely generated torsion-free module $M$ over a PID $R$ is isomorphic to $R^h$ for some $h\in\Zg$.
    \item Theorem: Let $M$ be a finitely generated $R$-module, where $R$ is a PID. Then\dots
    \begin{enumerate}[label={(\roman*)}]
        \item $\Tor(M)\oplus R^h\cong M$ for some $h\geq 0$;
        \item $\Tor(M)$ is finitely generated.
    \end{enumerate}
    \item \textbf{Rank} (of a module): The number $h$ pertaining to an $R$-module $M$, where $M/\Tor(M)\cong R^h$. \emph{Denoted by} $\bm{\rank(M)}$.
    \begin{itemize}
        \item It follows by the proposition from last lecture (Lecture 7.1) that rank is well-defined.
    \end{itemize}
    \item Corollary: Finitely generated $R$-modules $M_1$ and $M_2$ are isomorphic to each other iff
    \begin{enumerate}[label={(\roman*)}]
        \item $M_1$ and $M_2$ have the same rank;
        \item $\Tor(M_1)$ is isomorphic to $\Tor(M_2)$.
    \end{enumerate}
    \begin{proof}
        Suppose first that $\phi:M_1\to M_2$ is an isomorphism. Then naturally they will have the same ranks and torsion submodules.\par
        On the other hand, if $\rank(M_1)=\rank(M_2)$, then $M_1/\Tor(M_1)\cong M_2/\Tor(M_2)$. This combined with the hypothesis that $\Tor(M_1)\cong\Tor(M_2)$ implies that
        \begin{align*}
            \Tor(M_1)\oplus M_1/\Tor(M_1) &\cong \Tor(M_2)\oplus M_2/\Tor(M_2)\\
            M_1 &\cong M_2
        \end{align*}
        where the second line follows from the preceding theorem.
    \end{proof}
    \item The classification of finitely generated $R$-modules ($R$ a PID) is completed by the following results.
    \item \textbf{$\bm{p}$-primary component} (of a module): The submodule of a module $M$ consisting of those $m\in M$ such that $p^km=0$ for some $k\in\Zg$. \emph{Denoted by} $\bm{M_{(p)}}$.
    \begin{itemize}
        \item Showing that $M_{(p)}$ is a submodule of $M$ can be accomplished with the submodule criterion (Proposition \ref{prp:10.1}), just like in the first lemma proven today.
    \end{itemize}
    \item Notation and observations.
    \begin{enumerate}
        \item Let $M_1,\dots,M_k$ be submodules of $M$. Then $T:\prod_{i=1}^kM_i\to M$ defined by
        \begin{equation*}
            T(m_1,\dots,m_k) = m_1+\cdots+m_k
        \end{equation*}
        is not injective in general.
        \begin{itemize}
            \item For example, if $k=2$, then $\ker(T)\cong M_1\cap M_2$ in general.
            \item Thus, some care is required in our selection of submodules if we want $\ker(T)=0$.
        \end{itemize}
        \item Obtaining a natural $R$-module homomorphism $T:\oplus_{i\in I}M_i\to M$ defined as above.
        \begin{itemize}
            \item We have that $\oplus_{i\in I}M_i\subset\prod_{i\in I}M_i$ in general. Here's why:
            \item Given a finite subset $F\subset I$, we may regard $\prod_{i\in F}M_i$ as a submodule of $\prod_{i\in I}M_i$ by taking the entries in the $i^\text{th}$ place to be zero for all $i\notin F$.
            \item The direct sum is simply the union of the submodules $\prod_{i\in F}M_i$ taken over all finite $F\subset I$.
            \item We define $T$ on the overall direct sum one submodule $\prod_{i\in F}M_i$ at a time.
        \end{itemize}
    \end{enumerate}
    \item Proposition: The natural $R$-module homomorphism $T:\oplus_{(p)}M_{(p)}\to\Tor(M)$ is an isomorphism, where the direct sum is indexed by the set of nonzero prime ideals of $R$.
    \begin{proof}
        Let $F$ be a set of $r$ distinct primes $p_1,\dots,p_r$ (i.e., the prime ideals $(p_1),\dots,(p_r)$ are pairwise distinct sets). Let $(m_1,\dots,m_r)\in\prod_{(p)\in F}M_{(p)}$. Then as per the notation and observations section above, $T$ is defined such that
        \begin{equation*}
            T(m_1,\dots,m_r) = m_1+\cdots+m_r
        \end{equation*}
        We first prove that $T$ is injective. Let $(m_1,\dots,m_r)\in\ker(T)$ be arbitrary. Then $T(m_1,\dots,m_r)=m_1+\cdots+m_r=0$. By hypothesis, there exist $k_1,\dots,k_r$ such that $p_i^{k_i}m_i=0$ ($i=1,\dots,r$). Define $a=p_2^{k_2}\cdots p_r^{k_r}$. It follows that $am_2=\cdots=am_r=0$. Thus,
        \begin{align*}
            a(0) &= 0\\
            a(m_1+\cdots+m_r) &= 0\\
            am_1+\cdots+am_r &= 0\\
            am_1 &= -(am_2+\cdots+am_r)\\
            &= -(0+\cdots+0)\\
            &= 0
        \end{align*}
        Additionally, $\gcd(a,p_1^{k_1})=1$ by definition, so $1\in(a,p_1^{k_1})$. It follows that there exist $b,c\in R$ such that $ba+cp_1^{k_1}=1$. This combined with the facts that $am_1=0$ and $p_1^{k_1}m_1=0$ implies that
        \begin{equation*}
            m_1 = 1\cdot m_1
            = (ba+cp_1^{k_1})m_1
            = b(am_1)+c(p_1^{k_1}m_1)
            = b(0)+c(0)
            = 0
        \end{equation*}
        A symmetric argument shows that all $m_i=0$, i.e., $(m_1,\dots,m_r)=(0,\dots,0)$. Therefore, $\ker(T)=0$, as desired.\par
        We now prove that $T$ is surjective. Let $m\in\Tor(M)$ be arbitrary. Consider the submodule $N=Am\subset M$. To prove that $m$ is the sum of elements, each from a $p$-primary component of $M$, it will suffice to prove that stronger condition that every element in $N$ is the sum of elements, each from a $p$-primary component of $M$. Equivalently, it will suffice to show that $N$ is the isomorphic to the sum of its $p$-primary components, since the $p$-primary components of $N$ are contained in those of $M$. Define $I=\{a\in R:am=0\}$. Notice that $I=\ker(l_a)$, where $l_a:R\to N$ is the left multiplication homomorphism. It follows by the FIT that there exists an isomorphism $\overline{l_a}:R/I\to N$. Thus, we need only show that $R/I$ is isomorphic to the direct sum of its $p$-primary components. But the \hyperref[trm:7.17]{Chinese Remainder Theorem} takes care of this for us since $I$ is a nonzero ideal.
    \end{proof}
    \item In view of the last proposition, our final task will be to classify finitely generated $p$-primary modules.
    \item We begin with some definitions.
    \item \textbf{$\bm{p}$-primary} (module): An $R$-module $M$ such that $M=M_{(p)}$ for some prime $p\in R$.
    \item \textbf{Annihilator} (of a module): The set of all $a\in R$ such that $am=0$ for all $m\in M$. \emph{Denoted by} $\bm{\Ann(M)}$. \emph{Given by}
    \begin{equation*}
        \Ann(M) = \{a\in R:am=0\ \forall\ m\in M\}
    \end{equation*}
    \item \textbf{Annihilator} (of an element): The set of all $a\in R$ such that $am=0$ pertaining to a specific $m\in M$. \emph{Denoted by} $\bm{\Ann(m)}$. \emph{Given by}
    \begin{equation*}
        \Ann(m) = \{a\in R:am=0\}
    \end{equation*}
    \item Consider $l_m:R\to M$ defined by $l_m(a)=am$.
    \begin{itemize}
        \item By the FIT, there exists a module isomorphism $\overline{l_m}:R/\Ann(m)\to Rm$.
        \item $\ker(l_m)=\Ann(m)$.
    \end{itemize}
    \item \textbf{Cyclic} (module): An $R$-module $M$ for which there exists $m\in M$ such that $M=Rm$.
    \begin{itemize}
        \item Cyclic modules are isomorphic to $R/\Ann(m)$ for a similar reason to the above ($Rm=M$ here).
    \end{itemize}
    \item With these definitions out of the way, we seek to show that every finitely generated $R$-module is the direct sum of cyclic modules.
    \item To prove this result, we will need the following lemma.
    \item Lemma: Let $M'=Re$ be a cyclic submodule of $M$, where $R$ is a PID. We assume that\dots
    \begin{enumerate}[label={(\roman*)}]
        \item $\Ann(e)=(p^n)$;
        \item $p^nM=0$.
    \end{enumerate}
    Then every $v\in M/M'$ has a lift $w\in M$ such that $\Ann(w)=\Ann(v)$.
    \begin{proof}
        Let $v\in M/M'$ be arbitrary. We first characterize the annihilator of $v$\footnote{Steps like the following will be performed often in subsequent proofs without elaboration, so this paragraph serves to go through everything in full detail once.}. Since $p^nM=0$, we know that $p^n(M/M')=0$. Thus, we absolutely know that $p^n$ annihilates $v\in M/M'$. However, it is possible that some power $k\leq n$ of $p$ also annihilates the specific element $v$ of $M/M'$. Let $k$ be the smallest power of $p$ such that $p^kv=0$. Then $p^k\in\Ann(v)$. In particular, since the annihilator is an ideal (any element of the annihilator times any other element of $R$ [multiplied left or right] is also in the annihilator by the assumed commutativity of $R$) and $R$ is a PID, we know that $\Ann(v)$ is principal and its generator must divide $p^k$ (i.e., be a power of $p$). But by the assumption that $k$ is the smallest integer such that $p^k\in\Ann(v)$, we have that $\Ann(v)=(p^k)$.\par
        We now begin the bidirectional inclusion argument in earnest. Our strategy is thus: We will construct a lift $w'$ of $v$, prove that $\Ann(v)\subset\Ann(w')$, and then prove that $\Ann(w')\subset\Ann(v)$. Let's begin.\par
        Pick any lift $w\in M$ of $v$. By hypothesis $p^kv=0$, so $p^kw\in M'$. It follows since $M'$ is cyclic that $p^kw=\alpha e$ for some $\alpha\in R$. Additionally, since $p^nM=0$ by hypothesis, we know that $p^nw=0$. Thus, since $n\geq k$, we have that
        \begin{equation*}
            0 = p^nw = p^{n-k}p^kw = p^{n-k}\alpha e
        \end{equation*}
        Thus, $p^{n-k}\alpha\in\Ann(e)$. It follows since $\Ann(e)=(p^n)$ by hypothesis that
        \begin{align*}
            p^{n-k}\alpha &= p^n\beta\\
            \alpha &= p^k\beta
        \end{align*}
        for some $\beta\in R$. Now define $w'=w-\beta e$. Note that $w'$ is still a lift of $v$ since we only added the element $-\beta e$ of $M'=Ae$ to it.\par
        In particular, we have that
        \begin{equation*}
            p^kw' = p^kw-p^k\beta e
            = p^kw-\alpha e
            = 0
        \end{equation*}
        This proves that $p^k\in\Ann(w')$. Since annihilators are ideals, as discussed above, it follows that $\Ann(v)=(p^k)\subset\Ann(w')$.\par
        To finish the proof, it will just suffice to show that $\Ann(w')\subset\Ann(v)$. Let $a\in\Ann(w')$ be arbitrary. Then $aw'=0$. It follows that $0=\pi(aw')=a\pi(w')=av$. Therefore, $a\in Ann(v)$ as well.
    \end{proof}
    \item Proposition: For every finitely generated $p$-primary module $M$, there exist $e_1,\dots,e_s$ such that $M$ is the direct sum of the cyclic submodules $Re_i$.
    \begin{proof}
        Since $M$ is finitely generated, we know that $M=Rv_1+\cdots+Rv_r$. We induct on $r$.\par
        For the base case $r=1$, $M$ is cyclic by definition.\par
        Now suppose that we have proven the claim for $r-1$; we now seek to prove it for $r$. Assume WLOG that $(p^n)=\Ann(v_1)\subset\Ann(v_i)$ for all $i=1,\dots,r$. Essentially, what we are doing here is just relabeling the generators so that $v_1$ is the generator of $M$ with the smallest annihilator, i.e., the one with the highest power of $p$ as generator. In particular, since $n$ is the largest of its kind, we know that $p^nM=0$. Now let $e=v_1$ and $M'=Re$. Then by the properties of the canonical \emph{surjection}, $M/M'$ is generated by $\bar{v}_1,\dots,\bar{v}_r$. But since $\bar{v}_1=0$ by the definition of $M'$, we have that $M/M'$ is generated by $\bar{v}_2,\dots,\bar{v}_r$.\par
        Therefore, by the induction hypothesis, there exist $e_1,\dots,e_s$ such that $M$ is the direct sum of the cyclic submodules $\bigoplus_{i=1}^sRe_i$. Another way of phrasing this is that the natural homomorphism $T'':Re_1\oplus\cdots\oplus Re_s\to M/M'$ is an isomorphism. It follows by the preceding lemma that there exist lifts $w_1,\dots,w_s\in M$ of $e_1,\dots,e_s$, respectively, such that $\Ann(w_i)=\Ann(e_i)$ for all $i=1,\dots,s$.\par
        We wish to deduce that the natural homomorphism $T:Re\oplus Rw_1\oplus\cdots\oplus Rw_s\to M$ is also an isomorphism. For surjectivity, let $N=Rw_1+\cdots+Rw_s$. It follows logically that the image of the composite homomorphism $N\hookrightarrow M\to M/M'$ is just $Re_1+\cdots+Re_s$. This set is, in fact, all of $M/M'$ by the surjectivity of $T''$. Thus, $M'+N=M$, as desired. For injectivity, let $a,a_1,\dots,a_s$ be such that $ae+a_1w_1+\cdots+a_sw_s=0$. Then we have the equation $a_1e_1+\cdots+a_se_s=0$ in $M/M'$. It follows by the injectivity of $T''$ that $a_i\in\Ann(e_i)$ for all $i=1,\dots,r$. Since $\Ann(e_i)=\Ann(w_i)$ by the above, it follows that $a_iw_i=0$ ($i=1,\dots,s$). Thus,
        \begin{align*}
            0 = ae+a_1w_1+\cdots+a_sw_s
            = ae+0+\cdots+0
            = ae
        \end{align*}
        Therefore, since $ae\in Re$ is zero and is the last remaining term, $\ker(T)=0$.
    \end{proof}
\end{itemize}



\section{Rational Canonical Form and Proofs of Earlier Lemmas}
\begin{itemize}
    \item \marginnote{2/17:}Theorem: Every finitely generated $R$-module $M$ (where $R$ is a PID) is isomorphic to $\Tor(M)\oplus R^h$ for some $h\in\Zg$, where $h=\rank(M)$.
    \item Recall the following theorem.
    \item Theorem: Let $R$ be a PID. Then
    \begin{enumerate}[label={(\arabic*)}]
        \item Every finitely generated $p$-primary $R$-module is a finite direct sum of cyclic modules (which are isomorphic to $R/p^hR$ for some $h\in\N$).
        \item Every torsion module $M$ is the direct sum of its $p$-primary components.
    \end{enumerate}
    \item Corollary: Every finitely generated torsion $R$-module is isomorphic to the finite direct sum of cyclic $p$-primary modules where $p$ is an element of a finite set of primes.
    \emph{picture}
    \item $M$ finitely generated implies that $M_{(p)}$ is finitely generated.
    \item Said aloud that only finite primes $p$ satisfy $M_{(p)}\neq 0$.
    \item Theorem (Rational canonical form): Let $R$ be a PID. Then every finitely generated $R$-torsion module is isomorphic to
    \begin{equation*}
        R/(a_1)\oplus\cdots\oplus R/(a_\ell)
    \end{equation*}
    where $a_2\mid a_1$, $a_3\mid a_2$, \dots, $a_\ell\mid a_{\ell-1}$.
    \item Observe: The principal ideal $(a_1)$ is exactly the annihilator of $M$, i.e.,
    \begin{equation*}
        (a_1) = \{\alpha\in R:\alpha m=0\ \forall\ m\in M\}
    \end{equation*}
    \begin{itemize}
        \item Later, $(a_1)$ will play the role of a minimal polynomial, and the product will play the role of the characteristic polynomial.
    \end{itemize}
    \begin{proof}[Proof of theorem]
        % Let $p_1,\dots,p_\ell$ be the set of distinct primes for which $M_{(p)}\neq 0$. Let
        % \begin{equation*}
        %     M_{(p_i)} \cong R/(p_i^{m_{i,1}})\times R/(p_i^{m_{i,2}})\times\cdots
        % \end{equation*}
        % where $m_{i,1}\geq m_{i,2}\geq\cdots$. are such that there exists $N$ for which $m_{i,N}=0$. Then
        % \begin{equation*}
        %     M/(p_j) \cong R/(p_j^{m_{j,1}})^\times\times R/(p_j^{m_{j,2}})^\times
        % \end{equation*}
        % Then we apply the Chinese Remainder Theorem. Define
        % \begin{equation*}
        %     a_r = \prod_{i=1}^\ell p_i^{m_{i,r}}
        % \end{equation*}
        % where $a_{r+1}\mid a_r$ because $m_{i,j}$ is ?? in $j$. Use the CRT to imply that
        % \begin{equation*}
        %     \prod_{i=1}^\ell R/(p_i^{m_i,r}) \cong R/(a_r)
        % \end{equation*}

        % Strategy: Break it all the way down to basic primes, then build primes back up into certain numbers, sorted by power.

        % The finititude of primes: We actually need the direct sum consideration! Take your generators $v_1,\dots,v_n$ of $\Tor(M)=M$. Map these isomorphically into your direct sum. Each of the images will be a direct sum nonzero elements from only finitely (say $m_1$) many $M_{(p)}$. Then the total number of primes is $\sum_{i=1}^nm_i$. And that must generate the direct sum, so the direct sum must only contain finitely many $p$-primary components. Define $\ell=\sum_{i=1}^nm_i$ and move on.


        Let $M$ be an arbitrary finitely generated $R$-torsion module. Since $M=\Tor(M)$, a proposition from last lecture implies that
        \begin{equation*}
            M = \Tor(M) \cong \bigoplus_{(p)}M_{(p)}
        \end{equation*}
        We will first show that the above direct sum is only taken over finitely many primes. Let $v_1,\dots,v_n$ be a finite generating set of $M$. By the above isomorphism, each of these elements of $M$ maps to a direct sum of nonzero elements from some subset of the $M_{(p)}$'s. Importantly, the image of $v_i$ must be a \emph{finite} direct sum by the infinite generalization definition of the direct sum. Let $w_i$ denote the number of $M_{(p)}$'s that donate a nonzero element to the direct sum defining the image of $v_i$ under the isomorphism. Then the total number of $M_{(p)}$'s which donate a nonzero element is \emph{at most} $w_1+\cdots+w_n$, a finite number, so we can eliminate all other $M_{(p)}$'s from the direct sum and know that an isomorphism still holds (because $N\cong N\oplus\{0\}$ in general).\par
        Having established the finiteness of the involved primes, let $p_1,\dots,p_\ell$ be the set of distinct primes for which $M_{(p)}\neq 0$. Then
        \begin{equation*}
            M \cong M_{(p_1)}\oplus\cdots\oplus M_{(p_\ell)}
        \end{equation*}
        Consider some $M_{(p_i)}$ in the above direct sum. Since it is finitely generated (because the isomorphism is natural) and $p$-primary (by definition), we have by another proposition from last time that
        \begin{equation*}
            M_{(p_i)} \cong Re_1\oplus\cdots\oplus Re_{s_i}
        \end{equation*}
        We know (again from last lecture) that each cyclic submodule $Re_j$ is isomorphic to $R/\Ann(e_j)$. Since $M_{(p_i)}$ is $p_i$-primary and $e_j\in M_{(p_i)}$, we know that there exists (a minimal) $m_{i,j}$ such that $p_i^{m_{i,j}}e_j=0$. Thus, since $R$ is a PID, $\Ann(e_j)=(p_i^{m_{i,j}})$. Replacing every element in the above direct sum with our new form reveals that
        \begin{equation*}
            M_{(p_i)} \cong R/(p_i^{m_{i,1}})\oplus\cdots\oplus R/(p_i^{m_{i,s_i}})
        \end{equation*}
        WLOG, let $m_{i,1}\geq\cdots\geq m_{i,s_i}$. Define
        \begin{equation*}
            a_r = \prod_{i=1}^\ell p_i^{m_{i,r}}
        \end{equation*}
        for all $r=1,\dots,s_i$. It follows by the construction that $a_{r+1}\mid a_r$ ($r=1,\dots,s_i-1$). Additionally, we have by the \hyperref[trm:7.17]{Chinese Remainder Theorem} that for each $r=1,\dots,s_i$,
        \begin{equation*}
            R/(a_r) \cong \prod_{i=1}^\ell R/(p_i^{m_{i,r}}) = \bigoplus_{i=1}^\ell R/(p_i^{m_{i,r}})
        \end{equation*}
        WLOG, let $s_\ell\geq s_i$ ($i=1,\dots,\ell$). Therefore, putting everything together, we have that
        \begin{align*}
            M &\cong M_{(p_1)}\oplus\cdots\oplus M_{(p_\ell)}\\
            &\cong \left( \bigoplus_{j=1}^{s_1}R/(p_1^{m_{1,j}}) \right)\oplus\cdots\oplus\left( \bigoplus_{j=1}^{s_\ell}R/(p_\ell^{m_{\ell,j}}) \right)\\
            &\cong \left( \bigoplus_{i=1}^\ell R/(p_i^{m_i,1}) \right)\oplus\cdots\oplus\left( \bigoplus_{i=1}^\ell R/(p_i^{m_i,s_\ell}) \right)\\
            &\cong R/(a_1)\oplus\cdots\oplus R/(a_{s_\ell})
        \end{align*}
        as desired.
    \end{proof}
    \item The previous theorem but over all modules instead of just torsion modules.
    \item Proposition: Every finitely generated $R$-module, where $R$ is a PID, is isomorphic to
    \begin{equation*}
        R/I_1\oplus R/I_2\oplus\cdots
    \end{equation*}
    for a unique increasing sequence of ideals $I_1\subset I_2\subset\cdots$ which have the property that $I_n=R$ for some $n$.
    \begin{proof}
        {\color{white}hi}
        \begin{itemize}
            \item 2.4: $M\cong R^h\oplus\Tor(M)$ for some $h\geq 0$.
            \item RCF: $\Tor(M)\cong R/(a_1)\oplus\cdots\oplus R/(a_\ell)$ where $a_\ell\mid a_{\ell-1}\mid\cdots\mid a_1$.
            \item $R^h\cong Re_1\oplus\cdots\oplus Re_h\cong R/\Ann(e_1)\oplus\cdots\oplus R/\Ann(e_h)$.
            \item $R$ is a PID: $\Ann(e_j)=(a_{\ell+j})$ for some $a_{\ell+j}$ and all $j=1,\dots,h$.
            \item Let $I_i=(a_i)$.
            \item WLOG, order them. How do I guarantee the subset condition??
            \item Then $M\cong R/I_1\oplus\cdots\oplus R/I_{\ell+h}$.
            \item If no $I_i=R$, define $I_{\ell+h+1},I_{\ell+h+2},\dots$ to be equal to $R$.
        \end{itemize}

        Consider the $\Ann(R^h)$. It is a principal ideal since $R$ is a PID.

        Can we take $R^h\cong R\oplus\cdots\oplus R=R/(0)\oplus\cdots\oplus R/(0)$, $h$ times?
    \end{proof}
    \item That concludes torsion modules over PIDs; we now do torsion modules over fields, which should be easier.
    \item \textbf{$\bm{R}$-linearly independent} (elements of $M$): A set of elements $u_1,\dots,u_\ell\in M$ such that the constraints
    \begin{align*}
        (a_1,\dots,a_\ell) &\in R^\ell&
        \sum_{i=1}^\ell a_iu_i &= 0
    \end{align*}
    imply that $(a_1,\dots,a_\ell)=0$. Equivalently, $H:R^\ell\to M$ defined by
    \begin{equation*}
        H(a_1,\dots,a_\ell) = \sum_{i=1}^\ell a_iu_i
    \end{equation*}
    is 1-1, i.e., $R^\ell\cong H(M)$.
    \item Lemma: Let $R$ be an integral domain, and let $M$ be a finitely generated $R$-module. Then there exists a submodule $M'\subset M$ such that\dots
    \begin{enumerate}[label={(\roman*)}]
        \item $M'\cong R^h$ for some $h\geq 0$;
        \begin{proof}
            Let $S\subset M$ be a finite generating set. Select $T\subset S$ such that (i) $T$ is linearly independent and (ii) $T\subsetneq W\subset S$ implies that $W$ is \emph{not} linearly independent. In other words, we are picking $T$ to be a maximal linear independence set. Now suppose $|T|=h$ so that $T=\{u_1,\dots,u_h\}$. Then by definition,
            \begin{equation*}
                M' = \sum_{i=1}^hRu_i
                \cong R^h
            \end{equation*}
            where the latter isomorphism follows from Proposition \ref{prp:10.5}.
        \end{proof}
        \item There exists a nonzero $a\in R$ such that $aM\subset M'$ (equivalently, $a(M/M')=0$).
        \begin{proof}
            Pick $w\in S$ such that $w\notin T$. Then since we picked $T$ to be a \emph{maximal} linear independence set, $T\cup\{w\}$ is linearly \emph{dependent}. It follows that there exists a nonzero $(a_1,\dots,a_{h+1})\in R^{h+1}$ such that
            \begin{equation*}
                a_1u_1+\cdots+a_hu_h+a_{h+1}w = 0
            \end{equation*}
            If $a_{h+1}=0$, then $(a_1,\dots,a_h)\neq 0$ makes $a_1u_1+\cdots+a_hu_h=0$, contradicting the assumed linear independence of $T$. Thus, $a_{h+1}\neq 0$. It follows that
            \begin{equation*}
                a_{h+1}w = -\sum_{i=1}^ha_iu_i \in M'
            \end{equation*}
            We may repeat this process for any $w\in S-T$ to obtain a nonzero $a_w$ such that $a_ww\in M'$. Additionally, if $w\in T$, take $a_w=1$. Now define
            \begin{equation*}
                a = \prod_{w\in S}a_w
            \end{equation*}
            Since $R$ is an integral domain by hypothesis and each $a_w$ in the above product is nonzero, $a$ is nonzero. Moreover, by its construction, $aw\in M'$ for all $w\in S$. Therefore,
            \begin{equation*}
                aM = a\left( \sum_{s\in S}As \right)
                \subset M'
            \end{equation*}
            as desired.
        \end{proof}
    \end{enumerate}
    \item Note that you can make stronger statements than the above; you'll just have to use Zorn's lemma to do so.
    \item We now return to PID-land.
    \item Corollary: Every finitely generated torsion-free module $M$ over a PID $R$ is isomorphic to $R^h$ for some $h\in\Zg$.
    \begin{proof}
        % Let $0\neq a\in R$ and $M'\subset M$ as in the lemma. Consider $H:M\to M'$ given by $H(m)=am$. It is an $R$-module homomorphism (where $R$ is commutative) and $\ker H=0$ because $M$ is torsion-free (i.e., the only torsion element is 0). Therefore, $M\cong H(M)\subset M'\cong R^h$. Since $R$ is a PID, $H(M)\cong R^n$ for some $0\leq h\leq n$. It follows that some submodule of $R^h$ is isomorphic to $R^n$ by a theorem proven earlier in the course.

        Apply the lemma to obtain a submodule $M'$ of $M$ such that $M'\cong R^h$ and a nonzero $a\in R$ such that $aM\subset M'$. Consider $H:M\to M'$ defined by $H(m)=am$. Since $H$ is just left-multiplication, $H$ is an $R$-module homomorphism. Additionally, since $M$ is torsion free, $am=0$ iff $m=0$ so we have $\ker H=0$. Thus, since $H$ is injective, $M\cong H(M)\subset M'\cong R^h$. Furthermore, since $R$ is a PID, the submodule $H(M)$ of $R^h$ must be isomorphic to $R^n$ for some $0\leq n\leq h$ by the Theorem from Week 6. It follows by transitivity that $M\cong H(M)\cong R^n$, as desired.
    \end{proof}
    \item Takeaway: The torsion-free part is far easier to handle than the torsion part.
    \item Theorem: Let $M$ be a finitely generated $R$-module, where $R$ is a PID. Then\dots
    \begin{enumerate}[label={(\roman*)}]
        \item $\Tor(M)\oplus R^h\cong M$ for some $h\geq 0$;
        \begin{proof}
            % Claim (may have been proven last time): $M/\Tor(M)$ is torsion-free. We assume this and proceed.\par
            % We have $M/\Tor(M)\cong R^h=Re_1\oplus\cdots\oplus Re_h$. Lift $\varphi(e_i)$ to $\tilde{e}_i\in M$. Then $T:\Tor(M)\oplus R^h\to M$ defined by $T(m,(a_1,\dots,a_h))=m+\sum a_i\tilde{e}_i$ is an isomorphism.

            To prove that $\Tor(M)\oplus R^h\cong M$, the second theorem from Lecture 6.3 tells us that it will suffice to show that $M/\Tor(M)\cong R^h$ for some $h\geq 0$. By part (ii) of the lemma from last time (Lecture 7.2), we have that $M/\Tor(M)$ is torsion-free. This combined with the fact that $M/\Tor(M)$ is a finitely generated (since $M$ is finitely generated) module over a PID allows us to invoke the above corollary, yielding the desired result.\par
            Note that the isomorphism $T:\Tor(M)\oplus R^h\to M$ is given by
            \begin{equation*}
                T(m,(a_1,\dots,a_h)) = m+\sum a_ie_i
            \end{equation*}
            where $e_1,\dots,e_h$ generate $R^h$.
        \end{proof}
        \item $\Tor(M)$ is finitely generated.
        \begin{proof}
            % The projection is a surjection, which implies that $\Tor(M)$ is finitely generated.

            Since $M$ is finitely generated, part (i) implies that $\Tor(M)\oplus R^h$ is finitely generated. Now consider the projection $\pi:\Tor(M)\oplus R^h\to\Tor(M)$. Since it is a surjection, the (finite number of) images of the generators of $\Tor(M)\oplus R^h$ generate $\Tor(M)$.
        \end{proof}
    \end{enumerate}
    \item Nori reproves the claim that $M/\Tor(M)$ is torsion-free (see the first lemma from last lecture).
    % \begin{proof}
    %     Let $v\in M/\Tor(M)$ be a torsion element. Then there exists a nonzero $a\in R$ such that $av=0$. Thus, if $w\in M$ is a lift of $v$, then $aw\in\Tor(M)$. By the definition of $\Tor(M)$, there exists a nonzero $b\in R$ such that $b(aw)=0$. It follows that $w\in\Tor(M)$. But $w$ was supposed to be a lift of $v$, so $v=0$.
    % \end{proof}
    \item If $\pi:M\to M/M'$ and $S:M/M'\to R_h$ is an isomorphism, then there exists $\varphi:R^h\to M$ such that the diagram commutes, i.e., $S\pi\varphi=\id_{R^h}$.
    \item Next week is going to be straight linear algebra.
    \item Nori would try to do tensors in one week (the last week), but it'd be ridiculous to do something on Friday and put it on a test on Tuesday.
    \item Imaginary quadratic fields, curves, Dedekind domains, etc.
    \item Content from this week in the book.
    \begin{itemize}
        \item Section 12.1.
        \begin{itemize}
            \item The material before Theorem \ref{trm:12.5} is OMITTED from the course.
            \item Theorem \ref{trm:12.9.2} is also OMITTED from the course.
            \item The rest of this section will be covered.
            \item The main theorems are: The existence theorem (Theorem \ref{trm:12.5}) and the uniqueness theorem (Theorem \ref{trm:12.9.1})
        \end{itemize}
        \item Section 12.2 deals with the PID $F[X]$ and its applications to linear algebra; this will be covered on Monday next week.
    \end{itemize}
\end{itemize}



\section{Office Hours (Callum)}
\begin{itemize}
    \item Problem 6.5?
    \begin{itemize}
        \item Go with the explicit route, not the universal property of the ring of fractions route.
        \item Explicit: Define
        \begin{equation*}
            F(v) = \frac{1}{a}f(av)
        \end{equation*}
        \item We need to prove that $1/af(av)=1/bf(bv)$ for valid $a,b$. Multiply both sides by $ab$ and use commutativity. Thus, $F(v)$ is well defined.
    \end{itemize}
    \item Problem 6.8?
    \begin{itemize}
        \item The hardest one. Doesn't really use any of the previous parts.
        \item Define $\phi:A\oplus M\to A^2$ to be the isomorphism. Consider $(1,0)\in A\oplus M$. In particular, let $\phi(1,0)=(a,b)$. We know that it will generate a copy of $A$ in $A^2$. Essentially, $A(a,b)=A^2$. We know that $\phi^{-1}:A^2\to A\oplus M$ and $P:A\oplus M\to A$. Suppose $P\circ\phi^{-1}:(1,0)\mapsto c$ and $(0,1)\mapsto d$.
        \item Consider
        \begin{equation*}
            A\hookrightarrow A\oplus M\xrightarrow{\phi}A^2\xrightarrow{\phi^{-1}}A\oplus M\xrightarrow{P}A
        \end{equation*}
        which is the identity on $A$. Then
        \begin{equation*}
            1\mapsto (1,0)\mapsto (a,b)=a(1,0)+b(0,1)\mapsto ac+bd
        \end{equation*}
        so $ac+bd=1$.
        \item Consider the matrix
        \begin{equation*}
            \begin{pmatrix}
                a & d\\
                b & c\\
            \end{pmatrix}
        \end{equation*}
        \begin{itemize}
            \item Determinant??
            \item $(-d,c)$
            \item So thus, $M=A(-d,c)$??
        \end{itemize}
        \item $(-d,c)\in A^2$ defines a map from $A^2\to M$ with kernel $A$. $(-d,c)\in\ker(P\circ\phi^{-1})$. Thus, $\phi^{-1}(-d,c)\in\{0\}\oplus M\cong M$.
        \item Thus, at this point, we may define a map
        \begin{equation*}
            A\hookrightarrow A^2\xrightarrow{\phi^{-1}}A\oplus M\xrightarrow{P}M
        \end{equation*}
        by
        \begin{equation*}
            1\mapsto(-d,c)
        \end{equation*}
        and this should be an isomorphism.
        \item $(-d,c)$ generates a submodule of $A^2$ that is isomorphic to $M$.
        \item Injectivity follows from that of all of the components.
        \item Surjectivity: Pull $m$ back to $(0,m)$ and then $\phi(0,m)\in A^2$. The subset of $A^2$ equal to all $\phi(0,m)$ is equal to
        \begin{equation*}
            \{(u,v)\in A^2:\phi^{-1}(u,v)\in 0\oplus M\} = \{(u,v)\in A^2:uc+vd=0\}
        \end{equation*}
        \item We want to find $k\in A$ such that $(u,v)=k(-d,c)$. In other words, we want $u=-kd$ and $v=kc$. $ua=-kda=k(1-bc)=k-kbc=k-bv$. Thus, $k=ua+bv$. Now we have to substitute that back in and show that it works.
        \item Thus, we have that
        \begin{equation*}
            kc = ua+bvc
            = uac+b(1-ad)
            = v+uac-vad
            = v+a(bc-ad)
        \end{equation*}
        \item Saying $A\cong M$ is kind of like saying that there's a change of basis. That's why matrices keep coming up.
        \item Summary of what we did.
        \begin{enumerate}
            \item We have
            \begin{equation*}
                A\hookrightarrow A\oplus M\xrightarrow{\phi}A^2\xrightarrow{\phi^{-1}}A\oplus M\xrightarrow{P}A
            \end{equation*}
            and this is the identity.
            \item We define $(1,0)\mapsto(a,b)$, which will generate a copy of $A$ in $A^2$.
            \item We now need to find a basis vector corresponding to $M$ (which we hope is $A$).
            \item $\{(1,0),(0,1)\}$ is the standard basis for $A^2$.
            \item We need to solve for $x,y$ such that
            \begin{equation*}
                \begin{pmatrix}
                    a & x\\
                    b & y\\
                \end{pmatrix}
            \end{equation*}
            is invertible.
            \item $\{\phi^{-1}(1,0),\phi^{-1}(0,1)\}$ is another basis of $A^2$.
            \item We want $ac+bd=1$.
        \end{enumerate}
    \end{itemize}
\end{itemize}



\section{Chapter 11: Vector Spaces}
\emph{From \textcite{bib:DummitFoote}.}
\setcounter{bookch}{11}
\subsection*{Section 11.1: Definitions and Basic Theory}
\begin{itemize}
    \item \marginnote{2/20:}Reviewing \textcite{bib:LADRNotes} is probably a good idea.
    \begin{itemize}
        \item Many of \textcite{bib:DummitFoote}'s proofs more elegant, though.
    \end{itemize}
    \item Goal of this chapter:
    \begin{itemize}
        \item Brief overview of results that will be used later on; more in-depth (even introductory level) linear algebra topics, such as Gauss-Jordan elimination, row echelon forms, etc., will not be covered.
        \item Only finite-dimensional vector spaces are discussed in the text; some stuff on infinite dimensional vector spaces is included in the exercises.
        \item Characteristic polynomials and eigenvalues: Next chapter.
    \end{itemize}
    \item Module terminology vs. vector space terminology.
    \begin{table}[h!]
        \centering
        \small
        \renewcommand{\arraystretch}{1.2}
        \begin{tabular}{ll}
            \textbf{Terminology for $\bm{R}$ any Ring} & \textbf{Terminology for $\bm{R}$ a Field}\\
            \hline
            $M$ is an $R$-module & $M$ is a vector space over $R$\\
            $m$ is an element of $M$ & $m$ is a vector in $M$\\
            $\alpha$ is a ring element & $\alpha$ is a scalar\\
            $N$ is a submodule of $M$ & $N$ is a subspace of $M$\\
            $M/N$ is a quotient module & $M/N$ is a quotient space\\
            $M$ is a free module of rank $n$ & $M$ is a vector space of dimension $n$\\
            $M$ is a finitely generated module & $M$ is a finite dimensional vector space\\
            $M$ is a nonzero cyclic module & $M$ is a 1-dimensional vector space\\
            $\varphi:M\to N$ is an $R$-module homomorphism & $\varphi:M\to N$ is a linear transformation\\
            $M$ and $N$ are isomorphic as $R$-modules & $M$ and $N$ are isomorphic vector spaces\\
            The subset $A$ of $M$ generates $M$ & The subset $A$ of $M$ spans $M$\\
            $M=RA$ & \begin{tabular}{@{}l@{}}Each element of $M$ is a linear combination\\of elements of $A$, i.e., $M=\Span(A)$\end{tabular}\\
        \end{tabular}
        \caption{Module vs. vector space terminology.}
        \label{tab:moduleVecTerms}
    \end{table}
    \item In this chapter, $F$ denotes a field and $V$ denotes a vector space over $F$.
    \item \textbf{Linearly independent} (subset $S\subset V$): A subset $S$ of $V$ for which the equation $\alpha_1v_1+\cdots+\alpha_nv_n=0$ with $\alpha_1,\dots,\alpha_n\in F$ and $v_1,\dots,v_n\in S$ implies $\alpha_1=\cdots=\alpha_n=0$.
    \item \textbf{Basis}: An ordered set of linearly independent vectors which span $V$. \emph{Also known as} \textbf{ordered basis}.
    \begin{itemize}
        \item In particular, two bases will be considered different even if one is simply a rearrangement of the other.
    \end{itemize}
    \item Examples.
    \begin{enumerate}
        \item $V=F[X]$.
        \begin{itemize}
            \item Basis: $1,X,X^2,\dots$ is linearly independent by definition since a polynomial is zero iff all of its coefficients are 0.
        \end{itemize}
        \item The collection of solutions of a linear, homogeneous, constant coefficient differential equation over $\C$.
        \begin{itemize}
            \item A vector space since differentiation is a linear operator.
            \item Elements are linearly independent if they are linearly independent as functions.
            \begin{itemize}
                \item Example: $\e[t],\e[2t]$ are easily seen to be solutions of the equation $y''-3y'+2y=0$.
                \item They are linearly independent since $a\e[t]+b\e[2t]=0$ implies $a+b=0$ ($t=0$) and $a\e+b\e[2]=0$ ($t=1$), and the only solution to this system of two equations is $a=b=0$.
                \item It is a theorem of differential equations that these elements span the set of solutions of this equation.
            \end{itemize}
        \end{itemize}
    \end{enumerate}
    \item Vector spaces are free modules.
    \begin{proposition}\label{prp:11.1}
        Assume the set $\mathcal{A}=\{v_1,\dots,v_n\}$ spans the vector space $V$ but no proper subset of $\mathcal{A}$ spans $V$. Then $\mathcal{A}$ is a basis of $V$. In particular, any finitely generated (i.e., finitely spanned) vector space over $F$ is a free $F$-module.
        \begin{proof}
            Given.
        \end{proof}
    \end{proposition}
    \item Example.
    \begin{enumerate}
        \item Consider $F[X]/(f)$, where $f=X^n+a_{n-1}X^{n-1}+\cdots+a_0$.
        \begin{itemize}
            \item $(f)$ is a subspace of $F[X]$.
            \item Euclidean Algorithm: Every $a\in F[X]$ can be written uniquely in the form $qf+r$ where $0\leq\deg(r)\leq n-1$. Thus, every element of the quotient is represented by a polynomial $r$ of degree $\leq n-1$.
            \item It follows that $\overline{1},\overline{X},\overline{X^2},\dots,\overline{X^{n-1}}$ spans $F[X]/(f)$.
        \end{itemize}
    \end{enumerate}
    \item Spanning sets contain bases.
    \begin{corollary}\label{cly:11.2}
        Assume the finite set $\mathcal{A}$ spans the vector space $V$. Then $\mathcal{A}$ contains a basis of $V$.
        \begin{proof}
            Given.
        \end{proof}
    \end{corollary}
    \item A new property of bases.
    \begin{theorem}[Replacement Theorem]\label{trm:11.3}
        Assume $\mathcal{A}=\{a_1,\dots,a_n\}$ is a basis for $V$ containing $n$ elements and $\{b_1,\dots,b_m\}$ is a set of linearly independent vectors in $V$. Then there is an ordering $a_1,\dots,a_n$ such that for each $k\in\{1,\dots,m\}$, the set
        \begin{equation*}
            \{b_1,\dots,b_k,a_{k+1},\dots,a_n\}
        \end{equation*}
        is a basis of $V$. In other words, the elements $b_1,\dots,b_m$ can be used to successively replace the elements of the basis $\mathcal{A}$, still retaining a basis. In particular, $n\geq m$.
        \begin{proof}
            Given.
        \end{proof}
    \end{theorem}
    \item Linear independence, span, and cardinality.
    \begin{corollary}\label{cly:11.4}\leavevmode
        \begin{enumerate}
            \item Suppose $V$ has a finite basis with $n$ elements. Any set of linearly independent vectors has $\leq n$ elements. Any spanning set has $\geq n$ elements.
            \item If $V$ has some finite basis, then any two bases of $V$ have the same cardinality.
        \end{enumerate}
        \begin{proof}
            Given.
        \end{proof}
    \end{corollary}
    \item \textbf{Dimension}: The cardinality of any basis of $V$. \emph{Denoted by} $\bm{\dim_FV}$, $\bm{\dim V}$.
    \item \textbf{Finite dimensional} (vector space): A vector space $V$ that is finitely generated.
    \item \textbf{Infinite dimensional} (vector space): A vector space $V$ that is not finitely generated.
    \begin{itemize}
        \item We write $\dim V=\infty$ for these.
    \end{itemize}
    \item Examples.
    \begin{enumerate}
        \item The dimension of the solution space to $y''-3y'+2y=0$ is 2.
        \begin{itemize}
            \item Recall from above that a basis is $\e[t],\e[2t]$.
            \item In general, it is a theorem in differential equations that the space of solutions of an $n^\text{th}$ order linear, homogeneous, constant coefficient differential equation of degree $n$ over $\C$ is a vector space over $\C$ of dimension $n$.
        \end{itemize}
        \item The dimension of $F[X]/(f)$ is $\deg(f)$.
        \begin{itemize}
            \item $F[X]$ and $(f)$ are infinite dimensional vector spaces.
        \end{itemize}
    \end{enumerate}
    \item Linearly independent lists and bases.
    \begin{corollary}[Building-Up Lemma]\label{cly:11.5}
        If $A$ is a set of linearly independent vectors in the finite dimensional space $V$, then there exists a basis of $V$ containing $A$.
        \begin{proof}
            Given.
        \end{proof}
    \end{corollary}
    \item Characterizing finite dimensional vector spaces.
    \begin{theorem}\label{trm:11.6}
        If $V$ is an $n$-dimensional vector space over $F$, then $V\cong F^n$. In particular, any two finite dimensional vector spaces over $F$ of the same dimension are isomorphic.
        \begin{proof}
            Given.
        \end{proof}
    \end{theorem}
    \item Examples.
    \begin{enumerate}
        \item Bases of ${\F_q}^k$.
        \begin{itemize}
            \item \textcite{bib:DummitFoote} justifies that the number of distinct bases of ${\F_q}^k$ is
            \begin{equation*}
                (q^k-1)(q^k-q)(q^k-q^2)\cdots(q^k-q^{k-1})
            \end{equation*}
            \item For every vector $v\in{\F_q}^k$, there are $q-1$ other linearly dependent vectors (corresponding to the $q$ $\F$-multiples of it).
        \end{itemize}
        \item Subspaces of ${\F_q}^n$.
        \begin{itemize}
            \item \textcite{bib:DummitFoote} justifies that the number of distinct $k$-dimensional subspaces of ${\F_q}^n$ is
            \begin{equation*}
                \frac{(q^n-1)(q^n-q)\cdots(q^n-q^{k-1})}{(q^k-1)(q^k-q)\cdots(q^k-q^{k-1})}
            \end{equation*}
        \end{itemize}
    \end{enumerate}
    \item Dimension of the quotient space.
    \begin{theorem}\label{trm:11.7}
        Let $V$ be a vector space over $F$, and let $W$ be a subspace of $V$. Then $V/W$ is a vector space with $\dim V=\dim W+\dim V/W$ (where if one side is infinite, then both are).
        \begin{proof}
            Given.
        \end{proof}
    \end{theorem}
    \item Dimension of the kernel and image of a linear transformation.
    \begin{corollary}\label{cly:11.8}
        Let $\varphi:V\to U$ be a linear transformation of vector spaces over $F$. Then $\ker\varphi$ is a subspace of $V$, $\varphi(V)$ is a subspace of $U$, and $\dim V=\dim\ker\varphi+\dim\varphi(V)$.
        \begin{proof}
            Given.
        \end{proof}
    \end{corollary}
    \item Classifying isomorphic operator.
    \begin{corollary}\label{cly:11.9}
        Let $\varphi:V\to W$ be a linear transformation of vector spaces of the same finite dimension. Then the following are equivalent.
        \begin{enumerate}
            \item $\varphi$ is an isomorphism.
            \item $\varphi$ is injective, i.e., $\ker\varphi=0$.
            \item $\varphi$ is surjective, i.e., $\varphi(V)=W$.
            \item $\varphi$ sends a basis of $V$ to a basis of $W$.
        \end{enumerate}
        \begin{proof}
            Given.
        \end{proof}
    \end{corollary}
    \item \textbf{Null space} (of a linear transformation): The kernel of the linear transformation.
    \item \textbf{Nullity} (of a linear transformation): The dimension of the kernel of the linear transformation.
    \item \textbf{Rank} (of a linear transformation): The dimension of the image of the linear transformation.
    \item \textbf{Nonsingular} (linear transformation): A linear transformation $\varphi$ for which $\ker\varphi=0$.
    \item \textbf{General linear group}: The group of all nonsingular linear transformations from $V\to V$ under the group operation of composition. \emph{Denoted by} $\bm{GL(V)}$.
    \begin{itemize}
        \item \textcite{bib:DummitFoote} justifies that if $V={\F_q}^n$, then
        \begin{equation*}
            |GL(V)| = (q^n-1)(q^n-q)\cdots(q^n-q^{n-1})
        \end{equation*}
    \end{itemize}
\end{itemize}

\subsubsection*{Exercises}
\begin{enumerate}[label={\textbf{\arabic*.}},ref={11.1.\arabic*},start=4]
    \item \label{exr:11.1.4}Prove that the space of real-valued functions on the closed interval $[a,b]$ is an infinite dimensional vector space over $\R$, where $a<b$.
    \item \label{exr:11.1.5}Prove that the space of continuous real-valued functions on the closed interval $[a,b]$ is an infinite dimensional vector space over $\R$, where $a<b$.
    \setcounter{enumi}{9}
    \item \label{exr:11.1.10}Prove that any vector space $V$ has a basis (by convention, the null set is the basis for the zero space). \emph{Hint}: Let $\mathcal{S}$ be the set of subsets of $V$ consisting of linearly independent vectors, partially ordered under inclusion; apply Zorn's Lemma to $\mathcal{S}$ and show that a maximal element of $\mathcal{S}$ is a basis.
    \item \label{exr:11.1.11}Refine your argument in the preceding exercise to prove that any set of linearly independent vectors of $V$ is contained in a basis of $V$.
    \item \label{exr:11.1.12}If $F$ is a field with a finite or countable number of elements and $V$ is an infinite dimensional vector space over $F$ with basis $\mathcal{B}$, prove that the cardinality of $V$ equals the cardinality of $\mathcal{B}$. Deduce in this case that any two bases of $V$ have the same cardinality.
    \item \label{exr:11.1.13}Prove that as vector spaces over $\Q$, $\R^n\cong\R$ for all $n\in\Z^+$. Note that, in particular, this means that $\R^n$ and $\R$ are isomorphic as additive abelian groups.
    \item \label{exr:11.1.14}Let $\mathcal{A}$ be a basis for the infinite dimensional vector space $V$. Prove that $V$ is isomorphic to the direct sum of copies of the field $F$ indexed by the set $\mathcal{A}$. Prove that the direct product of copies of $F$ indexed by $\mathcal{A}$ is a vector space over $F$ and it has strictly larger dimension than the dimension of $V$ (see the exercises in Section 10.3 for the definitions of direct sum and direct product over infinitely many modules).
\end{enumerate}


\subsection*{Section 11.2: The Matrix of a Linear Transformation}
\begin{itemize}
    \item Assumptions for this section.
    \begin{itemize}
        \item $V,W$ are vector spaces over the field $F$.
        \item $\mathcal{B}=\{v_1,\dots,v_n\}$ is an (ordered) basis of $V$, and $\mathcal{E}=\{w_1,\dots,w_m\}$ is an (ordered) basis of $W$.
        \item $\varphi\in\Hom(V,W)$.
    \end{itemize}
    \item \textbf{Matrix} (of $\varphi$ with respect to the bases $\mathcal{B},\mathcal{E}$): The $m\times n$ matrix whose $i,j$ entry is $\alpha_{ij}$, where
    \begin{equation*}
        \varphi(v_j) = \sum_{i=1}^m\alpha_{ij}w_i
    \end{equation*}
    \emph{Denoted by} $\bm{\mat{B}{E}{\varphi}}$.
    \item \textcite{bib:DummitFoote} reviews how to recover $\varphi$ from $\mat{B}{E}{\varphi}$.
    \begin{itemize}
        \item The equivalence of matrix multiplying and linear transforming is sometimes denoted
        \begin{equation*}
            [\varphi(v)]_\mathcal{E} = \mat{B}{E}{\varphi}[v]_\mathcal{B}
        \end{equation*}
    \end{itemize}
    \item \textbf{Representation} (of $\varphi$ with respect to the bases $\mathcal{B},\mathcal{E}$): The matrix $A=(a_{ij})$ associated with $\varphi$.
    \item Examples.
    \begin{enumerate}
        \item Computing a matrix with respect to the standard bases of $\R^3,\R^2$.
        \item The matrix of the differentiation operator $\varphi:V\to V$ on the 2-dimensional space of solutions $V$ to $y''-3y'+2y=0$.
        \begin{itemize}
            \item Since
            \begin{align*}
                \varphi(v_1) &= \dv{t}(\e[t]) = \e[t] = v_1&
                \varphi(v_2) &= \dv{t}(\e[2t]) = 2\e[2t] = 2v_2
            \end{align*}
            the representation of $\varphi$ is
            \begin{equation*}
                \begin{pmatrix}
                    1 & 0\\
                    0 & 2\\
                \end{pmatrix}
            \end{equation*}
        \end{itemize}
        \item Computing a matrix with respect to the standard bases of $\Q^3,\Q^3$.
    \end{enumerate}
    \item Isomorphism between the space of linear transformations and the space of matrices.
    \begin{theorem}\label{trm:11.10}
        Let $V$ be a vector space over $F$ of dimension $n$ and let $W$ be a vector space over $F$ of dimension $m$, with respective bases $\mathcal{B},\mathcal{E}$. Then the map $\Hom_F(V,W)\to M_{m\times n}(F)$ from the space of linear transformations from $V$ to $W$ to the space of $m\times n$ matrices with coefficients in $F$ defined by $\varphi\mapsto\mat{B}{E}{\varphi}$ is a vector space isomorphism. In particular, there is a bijective correspondence between linear transformations and their associated matrices with respect to a fixed choice of bases.
        \begin{proof}
            Given.
        \end{proof}
    \end{theorem}
    \item There is no \emph{natural} isomorphism between $\Hom_F(V,W)$ and $M_{m\times n}(F)$.
    \begin{itemize}
        \item This is because the choices of bases are arbitrary (there is no natural choice of them).
    \end{itemize}
    \item Dimension of the space of linear transformations.
    \begin{corollary}\label{cly:11.11}
        The dimension of $\Hom_F(V,W)$ is $(\dim V)(\dim W)$.
        \begin{proof}
            Given.
        \end{proof}
    \end{corollary}
    \item \textbf{Nonsingular} (matrix): An $m\times n$ matrix $A$ such that $Ax=0$ with $x\in F^n$ implies that $x=0$. \emph{Also known as} \textbf{invertible}.
    \item Nonsingular linear transformations vs. nonsingular matrices.
    \begin{itemize}
        \item Independent of the choice of bases, a matrix is nonsingular iff the corresponding linear transformation is nonsingular.
    \end{itemize}
    \item \textcite{bib:DummitFoote} uses the definition of the matrix to deduce the formula for matrix multiplication.
    \item Relating matrix multiplication to linear transformation composition.
    \begin{theorem}\label{trm:11.12}
        Let $U,V,W$ be finite dimensional vector spaces over $F$ with ordered bases $\mathcal{D},\mathcal{B},\mathcal{E}$, and assume $\psi:U\to V$ and $\varphi:V\to W$ are linear transformations. Then
        \begin{equation*}
            \mat{D}{E}{\varphi\circ\psi} = \mat{B}{E}{\varphi}\mat{D}{B}{\psi}
        \end{equation*}
        In words, the product of the matrices representing the linear transformations $\varphi,\psi$ is the matrix representing the composite linear transformation $\varphi\circ\psi$.
    \end{theorem}
    \item Properties of matrix multiplication.
    \begin{corollary}\label{cly:11.13}
        Matrix multiplication is associative and distributive (whenever the dimensions are such as to make products defined). An $n\times m$ matrix $A$ is nonsingular if and only if it is invertible.
        \begin{proof}
            Given.
        \end{proof}
    \end{corollary}
    \item Ring-like properties of $M_n(F)$, as induced by those of $\Hom_F(V,V)$.
    \begin{corollary}\label{cly:11.14}\leavevmode
        \begin{enumerate}
            \item If $\mathcal{B}$ is a basis of the $n$-dimensional space $V$, the map $\varphi\mapsto\mat{B}{B}{\varphi}$ is a ring and a vector space isomorphism of $\Hom_F(V,V)$ onto the space $M_n(F)$ of $n\times n$ matrices with coefficients in $F$.
            \item $GL(V)\cong GL_n(F)$, where $\dim V=n$. In particular, if $F$ is a finite field, the order of the finite group $GL_n(F)$ (which equals $|GL(V)|$) is given by the formula at the end of Section 11.1.
        \end{enumerate}
        \begin{proof}
            Given.
        \end{proof}
    \end{corollary}
    \item \textbf{Row rank} (of a matrix): The maximal number of linearly independent rows of the matrix, where the rows are considered as vectors in affine $m$-space.
    \item \textbf{Column rank} (of a matrix): The maximal number of linearly independent columns of the matrix, where the columns are considered as vectors in affine $n$-space.
    \item Relating ranks.
    \begin{itemize}
        \item The rank of $\psi$ equals the column rank of $\mat{B}{E}{\psi}$.
    \end{itemize}
    \item \textbf{Similar} (matrices): Two $n\times n$ matrices $A,B$ for which there exists an invertible $n\times n$ matrix $P$ such that $P^{-1}AP=B$.
    \item \textbf{Similar} (linear transformations): Two linear transformations $\varphi,\psi:V\to V$ for which there exists a nonsingular linear transformation $\xi$ such that $\xi^{-1}\varphi\xi=\psi$.
    \begin{itemize}
        \item This is an equivalence relation whose equivalence classes are the orbits of $GL(V)$ acting by conjugation on $\Hom_F(V,V)$.
    \end{itemize}
    \item \textbf{Transition} (matrix from $\mathcal{B}$ to $\mathcal{E}$): The matrix defined as follows, where $I$ is the identity transformation. \emph{Also known as} \textbf{change of basis} (matrix). \emph{Denoted by} $\bm{P}$. \emph{Given by}
    \begin{equation*}
        P = \mat{B}{E}{I}
    \end{equation*}
    \begin{itemize}
        \item $P=\mat{B}{E}{I}$ satisfies $P^{-1}\mat{B}{B}{I}P=\mat{E}{E}{\varphi}$.
        \begin{itemize}
            \item If $\mathcal{B}\neq\mathcal{E}$, then $P$ is not the identity matrix.
        \end{itemize}
        \item Note that we need \emph{ordered} bases to have a unique $P=\mat{B}{E}{I}$!
    \end{itemize}
    \item \textbf{Change of basis}: The similarity action of $\mat{B}{E}{I}$ on $\mat{B}{B}{\varphi}$.
    \item \textcite{bib:DummitFoote} proves that any two similar matrices represent the same linear transformation with respect to two different choices of bases.
    \item Example of similarity given.
    \item \textbf{Canonical forms}: The study of the simplest possible matrix representing a given linear transformation (and which basis to choose to realize it).
    \item We now move on to linear transformations on tensor products of vector spaces.
    \item Return to later.
    \setcounter{proposition}{17}
    \item \textbf{Idempotent} (linear transformation): A linear transformation $\psi$ satisfying $\psi^2=\psi$.
    \begin{itemize}
        \item Characterized in Exercise 11.2.11.
    \end{itemize}
\end{itemize}


\subsection*{Section 11.3: Dual Vector Spaces}
\begin{itemize}
    \item \textbf{Dual space} (of a vector space): The space of linear transformations from $V$ to $F$. \emph{Denoted by} $\bm{V^*}$.
    \item \textbf{Linear functional}: An element of $V^*$.
    \item \textbf{Dual basis} (to a basis of $V$): The basis related to a basis $\{v_1,\dots,v_n\}$ of $V$ by
    \begin{equation*}
        v_i^*(v_j) =
        \begin{cases}
            1 & i=j\\
            0 & i\neq j
        \end{cases}
    \end{equation*}
    for $1\leq j\leq n$. \emph{Denoted by} $\bm{\{v_1^*,\ldots,v_n^*\}}$.
    \item The dual basis to a basis of $V$ is a basis of $V^*$.
    \begin{proposition}\label{prp:11.18}
        With notations as above, $\{v_1^*,\dots,v_n^*\}$ is a basis of $V^*$. In particular, if $V$ is finite dimensional, then $V^*$ has the same dimension as $V$.
        \begin{proof}
            Given.
        \end{proof}
    \end{proposition}
    \item If $V$ is infinite dimensional, then $\dim V<\dim V^*$.
    \item \textbf{Algebraic} (dual space to $V$): The dual space $V^*$ taken for $V$ of arbitrary dimension.
    \item If $V$ has additional structure (e.g., a topology), we can get other types of dual spaces, such as the following.
    \item \textbf{Continuous} (dual of $V$): A dual of $V$ in which the linear functionals must be continuous.
    \item Example.
    \begin{enumerate}
        \item Let $V=C([a,b],\R)$.
        \begin{itemize}
            \item If $a<b$, then $V$ is infinite dimensional.
            \item For each $g\in V$, the function $\varphi_g:V\to\R$ defined by
            \begin{equation*}
                \varphi_g(f) = \int_a^bf(t)g(t)\dd{t}
            \end{equation*}
            is a linear functional on $V$.
        \end{itemize}
    \end{enumerate}
    \item \textbf{Double dual} (of $V$): The dual of $V^*$. \emph{Also known as} \textbf{second dual}. \emph{Denoted by} $\bm{V^{**}}$.
    \item For finite dimensional $V$, $\dim V=\dim V^{**}$ and hence $V\cong V^{**}$.
    \begin{itemize}
        \item There is a \textbf{natural} (i.e., basis independent/coordinate free) isomorphism.
        \begin{itemize}
            \item More detail on this is given.
        \end{itemize}
        \item This is different for infinite dimensional $V$, as per the above.
    \end{itemize}
    \item Existence of a natural map $V\to V^{**}$.
    \begin{theorem}\label{trm:11.19}
        There is a natural injective linear transformation from $V$ to $V^{**}$. If $V$ is finite dimensional, then this linear transformation is an isomorphism.
        \begin{proof}
            Given.
        \end{proof}
    \end{theorem}
    \item $\bm{\varphi^*}$: The induced function from $W^*\to V^*$ defined by
    \begin{equation*}
        f \mapsto f\circ\varphi
    \end{equation*}
    \begin{itemize}
        \item This is just the \textbf{pullback} or \textbf{dual map}.
    \end{itemize}
    \item Pullback: Linearity and matrix.
    \begin{theorem}\label{trm:11.20}
        With notations as above, $\varphi^*$ is a linear transformation from $W^*$ to $V^*$ and $\mat{E^*}{B^*}{\varphi^*}$ is the transpose of the matrix $\mat{B}{E}{\varphi}$.
        \begin{proof}
            Given.
        \end{proof}
    \end{theorem}
    \item A partial statement of the rank-nullity theorem.
    \begin{corollary}\label{cly:11.21}
        For any matrix $A$, the row rank of $A$ equals the column rank of $A$.
        \begin{proof}
            Given.
        \end{proof}
    \end{corollary}
    \item \textbf{Annihilator} (of $S$ in $V$): The set of all $v\in V$ for which $f(v)=0$ for all $f\in S\subset V^*$. \emph{Denoted by} $\bm{\Ann(S)}$. \emph{Given by}
    \begin{equation*}
        \Ann(S) = \{v\in V:f(v)=0\ \forall\ f\in S\}
    \end{equation*}
\end{itemize}
\setcounter{proposition}{0}



\section{Chapter 12: Modules over Principal Ideal Domains}
\emph{From \textcite{bib:DummitFoote}.}
\setcounter{bookch}{12}
\subsection*{Introduction}
\begin{itemize}
    \item Goal of this chapter.
    \begin{itemize}
        \item Characterize the structure of finitely generated modules over PIDs.
        \item This is an example of the ideal structure of a ring being reflected in the structure of its modules.
    \end{itemize}
    \item \textbf{Fundamental Theorem of Finitely Generated Abelian Groups}: Any finitely generated abelian group is isomorphic to the direct sum of cyclic abelian groups (either $\Z$ or $\Z/n\Z$ for some $n>0$).
    \begin{itemize}
        \item See Chapter 5.
    \end{itemize}
    \item Applying this theorem when the PID is $\Z$ proves the Fundamental Theorem of Finitely Generated Abelian Groups.
    \begin{itemize}
        \item The relation: Abelian groups are $\Z$-modules!
        \item In the language of modules, this theorem states that "any finitely generated $\Z$-module is the direct sum of modules of the form $\Z/I$ where $I$ is an ideal of $\Z$" \parencite[456]{bib:DummitFoote}.
        \begin{itemize}
            \item We will also need a uniqueness statement for the direct sum.
        \end{itemize}
    \end{itemize}
    \item Applying this theorem when the PID is $F[X]$ leads to the rational and Jordan canonical forms for a matrix.
    \begin{itemize}
        \item Recall that $F[X]$-modules require the specification of a linear transformation $T$.
        \item Thus, applying this theorem to $F[X]$-modules can be walked backwards to obtain information about $T$.
        \item The Jordan canonical form requires that $F$ contains all eigenvalues of $T$; the rational canonical form does not.
        \item Similarity will somehow be involved here.
    \end{itemize}
    \item Example of JCF.
    \begin{itemize}
        \item Mirrors the example from the end of Section 11.2.
    \end{itemize}
    \item Section 12.1 gives some definitions and then states and proves the Fundamental Theorem of Finitely Generated Modules over a PID.
    \item Section 12.2-12.3 cover the applications of the Fundamental Theorem to canonical forms, specifically the rational and Jordan ones, respectively.
    \item The application to abelian groups mentioned above will not be discussed further herein (it was discussed in Chapter 5).
    \item Note that an alternate and computationally useful proof of the Fundamental Theorem valid for Euclidean Domains (so also $\Z$ and $F[X]$ in particular) along the lines of row and column operations is outlined in Exercises 16-22 of Section 12.1.
\end{itemize}


\subsection*{Section 12.1: The Basic Theory}
\begin{itemize}
    \item \textbf{Ascending chain condition of submodules}: The condition pertaining to a module $M$ that no infinite increasing chain of submodules $N_i\subset M$ exists, that is, whenever
    \begin{equation*}
        N_1 \subset N_2 \subset \cdots
    \end{equation*}
    is an increasing chain of submodules of $M$, then there is a positive integer $m$ such that for all $k\geq m$, $M_k=M_m$ (so the chain becomes stationary at stage $m$: $M_m=M_{m+1}=\cdots$). \emph{Also known as} \textbf{ACC of submodules}.
    \begin{itemize}
        \item There exist analogous notions of the ACC on right and two-sided ideals in a (possibly noncommutative) ring $R$.
    \end{itemize}
    \item \textbf{Noetherian} ($R$-module): A left $R$-module $M$ that satisfies that ACC on submodules.
    \item \textbf{Noetherian} (ring): A ring $R$ that is Noetherian as a left module over itself.
    \item Characterizing Noetherian modules.
    \begin{theorem}\label{trm:12.1}
        Let $R$ be a ring and let $M$ be a left $R$-module. Then TFAE.
        \begin{enumerate}
            \item $M$ is a Noetherian $R$-module.
            \item Every nonempty set of submodules of $M$ contains a maximal element under inclusion.
            \item Every submodule of $M$ is finitely generated.
        \end{enumerate}
        \begin{proof}
            Given.
        \end{proof}
    \end{theorem}
    \item PIDs are Noetherian.
    \begin{corollary}\label{cly:12.2}
        If $R$ is a PID, then every nonempty set of ideals of $R$ has a maximal element and $R$ is a Noetherian ring.
        \begin{proof}
            Given.
        \end{proof}
    \end{corollary}
    \item Recall that finitely generated modules need not have finitely generated submodules; see Example 2 from Section 10.3.
    \begin{itemize}
        \item Thus, the Noetherian condition is stronger in general than the finite generation condition.
    \end{itemize}
    \item A useful linear dependence result.
    \begin{proposition}\label{prp:12.3}
        Let $R$ be an integral domain, and let $M$ be a free $R$-module of rank $n<\infty$. Then any $n+1$ elements of $M$ are $R$-linearly dependent, i.e., for any $y_1,\dots,y_{n+1}\in M$, there are elements $r_1,\dots,r_{n+1}\in R$, not all zero, such that
        \begin{equation*}
            r_1y_1+\cdots+r_{n+1}y_{n+1} = 0
        \end{equation*}
        \begin{proof}
            Given.
        \end{proof}
    \end{proposition}
    \item \textbf{The torsion submodule} (of $M$): The submodule of a $R$-module $M$, where $R$ is an integral domain, equal to all elements of $M$ such that $rx=0$ for some nonzero $r\in R$. \emph{Denoted by} $\bm{\Tor(R)}$. \emph{Given by}
    \begin{equation*}
        \Tor(M) = \{x\in M:rx=0\text{ for some nonzero }r\in R\}
    \end{equation*}
    \item \textbf{A torsion submodule} (of $M$): Any submodule of $\Tor(M)$.
    \item \textbf{Torsion module}: A module $M$ for which $\Tor(M)=M$.
    \item \textbf{Torsion-free} (module): A module $M$ for which $\Tor(M)=0$.
    \item \textbf{Annihilator} (of a submodule): The ideal of $R$ defined as follows, where $M$ is an $R$-module and $N$ is the submodule of $M$ in question. \emph{Denoted by} $\bm{\Ann(N)}$. \emph{Given by}
    \begin{equation*}
        \Ann(N) = \{r\in R:rn=0\ \forall\ n\in N\}
    \end{equation*}
    \begin{itemize}
        \item If $N$ is not a torsion submodule of $M$, then $\Ann(N)=0$.
        \item $N\subset L$ submodules of $M$ implies $\Ann(L)\subset\Ann(N)$.
        \item $R$ a PID, $N\subset L\subset M$, $\Ann(N)=(a)$, and $\Ann(L)=(b)$ implies that $a\mid b$.
        \begin{itemize}
            \item This follows from Lagrange's theorem when $R=\Z$.
        \end{itemize}
    \end{itemize}
    \item \textbf{Rank} (of a module): The maximum number of $R$-linearly independent elements of $M$.
    \begin{itemize}
        \item Proposition \ref{prp:12.3} states that for a free $R$-module $M$ over an integral domain, the rank of a submodule is bounded by the rank of $M$.
        \item This definition agrees with the previous one over fields: If $R=F$ is a field, then the rank of any $R$-module $M$ is the dimension of $M$ since any maximal set of $F$-linearly independent elements is a basis.
        \item Note that general modules over integral domains need not have a basis, i.e., need not be free even if they are torsion-free.
    \end{itemize}
    \item Relating free modules, PIDs, rank, and generators.
    \begin{theorem}\label{trm:12.4}
        Let $R$ be a PID, let $M$ be a free $R$-module of finite rank $n$, and let $N$ be a submodule of $M$. Then\dots
        \begin{enumerate}
            \item $N$ is free of rank $m\leq n$;
            \item There exists a basis $y_1,\dots,y_n$ of $M$ such that $a_1y_1,\dots,a_my_m$ is a basis of $N$ where $a_1,\dots,a_m$ are nonzero elements of $R$ that satisfy the divisibility relations
            \begin{equation*}
                a_1\mid a_2\mid\cdots\mid a_m
            \end{equation*}
        \end{enumerate}
        \begin{proof}
            Given.
        \end{proof}
    \end{theorem}
    \item Warm-up to the Fundamental Theorem: The special case of \emph{cyclic} (not finitely generated) $R$-modules.
    \begin{itemize}
        \item Let $C$ be a cyclic $R$-module. Then $C=Rx$ for some $x\in C$.
        \item Define $\pi:R\to C$ by $\pi(r)=rx$.
        \item $\pi$ is surjective by the assumption that $C=Rx$. Thus, by the FIT, $R/\ker\pi\cong C$.
        \item We are assuming that $R$ is a PID, so we must have $\ker\pi=(a)$ for some $a\in R$. In particular, note that $(a)=\Ann(C)$ by definition.
        \item Essentially, $C\cong R/(a)$, and the classification is complete.
    \end{itemize}
    \item We now treat the broader case of finite generation.
    \begin{theorem}[Fundamental Theorem, Existence: Invariant Factor Form]\label{trm:12.5}
        Let $R$ be a PID and let $M$ be a finitely generated $R$-module. Then\dots
        \begin{enumerate}[ref={\thetheorem(\arabic*)}]
            \item \label{trm:12.5.1}$M$ is isomorphic to the direct sum of finitely many cyclic modules. More precisely,
            \begin{equation*}
                M \cong R^r\oplus R/(a_1)\oplus\cdots\oplus R/(a_m)
            \end{equation*}
            for some integer $r\geq 0$ and nonzero elements $a_1,\dots,a_m\in R$ which are not units in $R$ and which satisfy the divisibility relations
            \begin{equation*}
                a_1\mid a_2\mid\cdots\mid a_m
            \end{equation*}
            \item \label{trm:12.5.2}$M$ is torsion-free iff $M$ is free.
            \item \label{trm:12.5.3}In the decomposition in part (1),
            \begin{equation*}
                \Tor(M) \cong R/(a_1)\oplus\cdots\oplus R/(a_m)
            \end{equation*}
            In particular, $M$ is a torsion module iff $r=0$ and in this case, the annihilator of $M$ is the ideal $(a_m)$.
        \end{enumerate}
        \begin{proof}
            Given.
        \end{proof}
    \end{theorem}
    \item We will shortly prove that the decomposition in Theorem \ref{trm:12.5.1} is unique; this proof will rely heavily on the divisibility condition.
    \item \textbf{Free rank}: The integer $r$ in Theorem \ref{trm:12.5}. \emph{Also known as} \textbf{Betti number}.
    \item \textbf{Invariant factors}: The elements $a_1,\dots,a_m\in R$ in Theorem \ref{trm:12.5}.
    \item Applying the \hyperref[trm:7.17]{Chinese Remainder Theorem} allows us to decompose $R/(a)$ further (and to do so uniquely).
    \begin{itemize}
        \item This gives $M$ as the direct sum of cyclic modules whose annihilators are as simple as possible.
    \end{itemize}
    \item The above idea is summarized by the following theorem.
    \begin{theorem}[Fundamental Theorem, Existence: Elementary Divisor Form]\label{trm:12.6}
        Let $R$ be a PID and let $M$ be a finitely generated $R$-module. Then $M$ is the direct sum of a finite number of cyclic modules whose annihilators are either $(0)$ or are generated by powers of primes in $R$, i.e.,
        \begin{equation*}
            M \cong R^r\oplus R/(p_1^{\alpha_1})\oplus\cdots\oplus R/(p_t^{\alpha_t})
        \end{equation*}
        where $r\geq 0$ is an integer and $p_1^{\alpha_1},\dots,p_t^{\alpha_t}$ are positive powers of (not necessarily distinct) primes in $R$.
    \end{theorem}
    \item \textbf{Elementary divisor}: A prime power $p_i^{\alpha_i}$ (defined up to multiplication by units in $R$), where $R$ is a PID and $M$ is a finitely generated $R$-module as in Theorem \ref{trm:12.6}.
    \item Grouping together all cyclic factors corresponding to the same prime $p_i$ shows that $M$ can be written as a direct sum $M=N_1\oplus\cdots\oplus N_n$ where $N_i$ consists of all the elements of $M$ which are annihilated by some power of the prime $p_i$.
    \item Summarizing the above idea.
    \begin{theorem}[The Primary Decompostion Theorem]\label{trm:12.7}
        Let $R$ be a PID and let $M$ be a nonzero torsion $R$-module (not necessarily finitely generated) with nonzero annihilator $a$. Suppose the factorization of $a$ into distinct prime powers in $R$ is
        \begin{equation*}
            a = up_1^{\alpha_1}\cdots p_n^{\alpha_n}
        \end{equation*}
        and let $N_i=\{x\in M:p_i^{\alpha_i}x=0\}$ ($1\leq i\leq n$). Then $N_i$ is a submodule of $M$ with annihilator $p_i^{\alpha_i}$ and is the submodule of $M$ of all elements annihilated by some power of $p_i$. In particular, we have
        \begin{equation*}
            M = N_1\oplus\cdots\oplus N_n
        \end{equation*}
        If $M$ is finitely generated, then each $N_i$ is the direct sum of finitely many cyclic modules whose annihilators are divisors of $p_i^{\alpha_i}$.
        \begin{proof}
            Given.
        \end{proof}
    \end{theorem}
    \item \textbf{$\bm{p_i}$-primary component} (of $M$): The submodule of $M$ of all elements annihilated by some power of $p_i$.
    \item We now prove the uniqueness statement of the Fundamental theorem.
    \begin{lemma}\label{lem:12.8}
        Let $R$ be a PID and let $p$ be a prime in $R$. Let $F$ denote the field $R/(p)$.
        \begin{enumerate}[ref={\thelemma(\arabic*)}]
            \item \label{lem:12.8.1}Let $M=R^r$. Then $M/pM\cong F^r$.
            \item \label{lem:12.8.2}Let $M=R/(a)$ where $a$ is a nonzero element of $R$. Then
            \begin{equation*}
                M/pM \cong
                \begin{cases}
                    F & p\mid a\\
                    0 & p\nmid a
                \end{cases}
            \end{equation*}
            \item \label{lem:12.8.3}Let $M=R/(a_1)\oplus\cdots\oplus R/(a_k)$ where each $a_i$ is divisible by $p$. Then $M/pM\cong F^k$.
        \end{enumerate}
        \begin{proof}
            Given.
        \end{proof}
    \end{lemma}
    \begin{theorem}[Fundamental Theorem, Uniqueness]\label{trm:12.9}
        Let $R$ be a PID.
        \begin{enumerate}[ref={\thetheorem(\arabic*)}]
            \item \label{lem:12.9.1}Two finitely generated $R$-modules $M_1$ and $M_2$ are isomorphic iff they have the same free rank and the same list of invariant factors.
            \item \label{lem:12.9.2}Two finitely generated $R$-modules $M_1$ and $M_2$ are isomorphic iff they have the same free rank and the same list of elementary divisors.
        \end{enumerate}
        \begin{proof}
            Given.
        \end{proof}
    \end{theorem}
    \item Further classification.
    \begin{corollary}\label{cly:12.10}
        Let $R$ be a PID and let $M$ be a finitely generated $R$-module. Then\dots
        \begin{enumerate}
            \item The elementary divisors of $M$ are the prime power factors of the invariant factors of $M$.
        \end{enumerate}
        \begin{proof}
            Given.
        \end{proof}
    \end{corollary}
    \item Restatement of Theorem 5.3 and 5.5.
    \begin{corollary}[The Fundamental Theorem of Finitely Generated Abelian Groups]\label{cly:12.11}\leavevmode
        \begin{enumerate}
            \item 5.3: Let $G$ be a finitely generated abelian group. Then\dots
            \begin{enumerate}
                \item $G\cong\Z^r\times\Z_{n_1}\times\cdots\times\Z_{n_s}$ for some integers $r,n_1,n_2,\dots,n_s$ satisfying the following conditions.
                \begin{enumerate}[label={(\roman*)}]
                    \item $r\geq 0$ and $n_j\geq 2$ for all $j$.
                    \item $n_{i+1}\mid n_i$ for $1\leq i\leq s-1$.
                \end{enumerate}
                \item The expression in part (1) is unique, i.e., if $G\cong\Z^t\times\Z_{m_1}\times\cdots\times Z_{m_u}$, where $t$ and $m_1,\dots,m_u$ satisfy $a,b$ (i.e., $g\geq 0$, $m_j\geq 2$ for all $j$ and $m_{i+1}\mid m_i$ for all $1\leq i\leq u-1$), then $t=r$, $u=s$, and $m_i=n_i$ for all $i$.
            \end{enumerate}
            \item 5.5: Let $G$ be an abelian group of order $n>1$ and let the unique factorization into distinct prime powers be
            \begin{equation*}
                n = p_1^{\alpha_1}\cdots p_k^{\alpha_k}
            \end{equation*}
            Then\dots
            \begin{enumerate}
                \item $G\cong A_1\times\cdots\times A_k$, where $|A_i|=p_i^{\alpha_i}$;
                \item For each $A\in\{A_1,\dots,A_k\}$ with $|A|=p^\alpha$,
                \begin{equation*}
                    A \cong Z_{p^{\beta_1}}\times\cdots\times Z_{p^{\beta_t}}
                \end{equation*}
                with $\beta_1\geq\cdots\geq\beta_t\geq 1$ and $\beta_1+\cdots+\beta_t=\alpha$ (where $t$ and $\beta_1,\dots,\beta_t$ depend on $i$).
                \item The decompositions in part (1) and (2) are unique, i.e., if $G\cong B_1\times\cdots\times B_m$ with the factors $|B_i|=p_i^{\alpha_i}$ for all $i$, then $B_i\cong A_i$ and $B_i,A_i$ have the same invariant factors.
            \end{enumerate}
        \end{enumerate}
        \begin{proof}
            Given.
        \end{proof}
    \end{corollary}
    \item More on the relationship between elementary divisors and invariant factors can be found in Chapter 5.
    \item Eye ahead: If a finitely generated module is written as a direct sum of cyclic modules of the form $R/(a)$, then the ideals $(a)$ which occur are not in general unique unless some additional conditions are imposed.
    \begin{itemize}
        \item To decide whether two modules are isomorphic, we must first write them in \emph{canonical} form.
    \end{itemize}
\end{itemize}
\setcounter{proposition}{0}




\end{document}