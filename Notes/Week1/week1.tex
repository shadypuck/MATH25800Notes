\documentclass[../notes.tex]{subfiles}

\pagestyle{main}
\renewcommand{\chaptermark}[1]{\markboth{\chaptername\ \thechapter\ (#1)}{}}

\begin{document}




\chapter{Rings Intro}
\section{Rings, Subrings, and Ring Homomorphisms}
\begin{itemize}
    \item \marginnote{1/4:}Intro to the course.
    \item What will be covered: Most of Chapters 7-12 in \textcite{bib:DummitFoote}.
    \begin{itemize}
        \item Mostly rings, a bit of modules.
        \begin{itemize}
            \item Modules tend to get more complicated.
        \end{itemize}
        \item The topics covered in class will all be in the book, but not necessarily in the same order.
        \item Some of Nori's definitions will be different from those used in the book.
        \begin{itemize}
            \item Different enough, in fact, to get us the wrong answers in PSet and Exam questions.
            \item We should use his, though.
            \item He diverges from the book because his is the mathematical literature standard.
            \item Three main differences: Definition of a ring, subring, and ring homomorphism.
        \end{itemize}
    \end{itemize}
    \item Homework will be due every Wednesday.
    \begin{itemize}
        \item The first will be due next week (on Wednesday, 1/11).
        \item Rings, subrings, and ring homomorphisms, only, are needed for the first HW.
    \end{itemize}
    \item Grading breakdown.
    \begin{itemize}
        \item HW (30\%).
        \item Midterm (30\%) --- third or fourth week.
        \item Final (40\%).
    \end{itemize}
    \item Office hours for Nori in Eckhart 310.
    \begin{itemize}
        \item M (3:00-4:30).
        \item Tu (3:30-5:00).
        \item Th (3:00-4:30).
    \end{itemize}
    \item Callum is our TA; Ray is for the other section. Their OH are TBA.
    \item All important course info will be in Files on Canvas.
    \item There will be course notes provided for the course.
    \item If we think something Nori writes down looks suspicious, feel free to ask!
    \item We now start the course content.
    \item \textbf{Ring}\footnote{Definition from \textcite{bib:DummitFoote}.}: A triple $(R,+,\times)$ comprising a set $R$ equipped with binary operations $+$ and $\times$ that satisfies the following three properties.
    \begin{enumerate}[label={(\roman*)}]
        \item $(R,+)$ is an abelian group.
        \item $(R,\times)$ is associative, i.e.,
        \begin{equation*}
            a\times(b\times c) = (a\times b)\times c
        \end{equation*}
        for all $a,b,c\in R$.
        \item The left and right distributive laws hold, i.e.,
        \begin{align*}
            a\times(b+c) &= (a\times b)+(a\times c)&
            (b+c)\times a &= (b\times a)+(c\times a)
        \end{align*}
        for all $a,b,c\in R$.
    \end{enumerate}
    \item Misc comments.
    \begin{itemize}
        \item The parentheses on the RHSs in (iii) indicate the "standard" order of operations.
        \item We still often drop the $\times$ in favor of $a\cdot b$ or simply $ab$.
        \item We haven't postulated multiplicative inverses. That makes things more tricky :)
    \end{itemize}
    \item We define left- and right-multiplication functions for every element $a\in R$.
    \begin{itemize}
        \item These are denoted $l_a:R\to R$ and $r_a:R\to R$. In particular,
        \begin{align*}
            l_a(b) &= a\times b&
            r_a(b) &= b\times a
        \end{align*}
        for all $b\in R$.
        \item The statement "$l_a,r_a$ are group homomorphisms\footnote{Since we will soon introduce other types of homomorphisms (e.g., ring homomorphisms) beyond the one type with which we are familiar, we now have to specify that a homomorphism of the type dealt with in MATH 25700 is a \emph{group} homomorphism.} from $(R,+)$ to itself, i.e.,
        \begin{equation*}
            l_a(b+c) = l_a(b)+l_a(c)
        \end{equation*}
        for all $b,c\in R$" is equivalent to (iii).
    \end{itemize}
    \item \textbf{Additive identity} (of $R$): The unique element of $R$ that satisfies the following constraint. \emph{Denoted by} $\bm{0_R}$.
    \begin{equation*}
        0_R+a = a+0_R = a
    \end{equation*}
    for all $a\in R$.
    \begin{itemize}
        \item The existence and uniqueness of $0_R$ follows from property (i) of rings (groups must have an identity element, which in this case is the \emph{additive} identity since it corresponds to the addition operation).
    \end{itemize}
    \item Similarly, we know that unique additive inverses exist for all $a\in R$. We denote these by $\bm{-a}$.
    \item Since $l_a$ is a group homomorphism, this must mean that
    \begin{align*}
        l_a(0_R) &= 0_R&
            l_a(-b) &= -l_a(b)\\
        a\times 0_R &= 0_R&
            a\times(-b) &= -(a\times b)
    \end{align*}
    for all $a,b\in R$.
    \begin{itemize}
        \item The same holds for $r_a$/positions interchanged.
        \item These are consequences of the distributive law.
    \end{itemize}
    \item In Part 1, \textcite{bib:DummitFoote} defines rings as above.
    \begin{itemize}
        \item In Part 2, \textcite{bib:DummitFoote} takes $R$ to be \textbf{commutative}.
        \item In Part 3, \textcite{bib:DummitFoote} takes $R$ to be a \textbf{ring with identity}.
    \end{itemize}
    \item \textbf{Commutative ring}: A ring $R$ such that
    \begin{equation*}
        a\times b = b\times a
    \end{equation*}
    for all $a,b\in R$.
    \item \textbf{Ring with identity}: A ring $R$ containing a 2-sided identity, i.e., an element $e\in R$ such that
    \begin{equation*}
        e\times a = a\times e = a
    \end{equation*}
    for all $a\in R$.
    \item We now justify that it's ok to denote the 2-sided identity with a single letter.
    \item Exercise: The identity is unique.
    \begin{proof}
        If $e'$ is also a 2-sided identity, then
        \begin{equation*}
            e = e\times e' = e'
        \end{equation*}
    \end{proof}
    \item In this course, we will always take "ring" to mean "ring with identity." That is, we will always assume that our rings contain a 2-sided identity $e=1_R$.
    \item Examples of rings.
    \begin{enumerate}
        \item $\N\subset\Z\subset\Q\subset\R\subset\C$ all have two binary operations, but are they all rings?
        \begin{itemize}
            \item $\N$ is not a ring since $(\N,+)$ is not an abelian group (or even a group --- no additive inverses).
            \item The rest are rings. In fact, they are commutative rings.
            \item $\Q,\R,\C$ are also \textbf{fields}.
        \end{itemize}
        \item Let $X$ be a set, and $f,g:X\to\R$. We can define $f+g:X\to\R$ by $(f+g)(x)=f(x)+g(x)$ and $f\times g:X\to\R$ by $(f\times g)(x)=f(x)g(x)$.
        \begin{itemize}
            \item Thus, the set of all functions from $X\to\R$ --- denoted $\Fun(X;\R)$ or $\R^X$ --- has two binary operations and is a ring.
            \item This follows from the fact that the real numbers form a ring.
        \end{itemize}
        \item More generally, let $X$ be a set and let $R$ be a ring. Then $\Fun(X;R)=R^X$ is a ring.
        \begin{itemize}
            \item The constant function taking the value $1_R\in R$ is the identity of $R^X$.
        \end{itemize}
        \item Let $X=\{1,2\}$. Then $R^X\cong R\times R$.
        \begin{itemize}
            \item Correct topology:
            \begin{align*}
                (a_1,a_2)+(b_1,b_2) &= (a_1+b_1,a_2+b_2)&
                (a_1,a_2)\times(b_1,b_2) &= (a_1\times b_1,a_2\times b_2)
            \end{align*}
            \item Implication: The same "formula" shows that if $R_1,R_2$ are rings, then $R_1\times R_2$ is a ring.
        \end{itemize}
        \item If $R_i$ is a ring for all $i\in I$, where $I$ could be any indexing set (e.g., $\N$, but need not be countable), then $\prod_{i\in I}R_i$ is also a ring.
        \begin{itemize}
            \item The identity is $(e_i,e_j,\dots)$.
        \end{itemize}
    \end{enumerate}
    \item \textbf{Field}: A commutative ring $R$ with multiplicative inverses for every element except $0_R$.
    \item In the context of groups, we've discussed subgroups, group homomorphisms, the fact that the inclusion of a subgroup into a bigger group is a group homomorphism, and the fact that the image of a group homomorphism is a subgroup.
    \item Today, let's define subrings and ring homomorphisms and make sure that the corresponding properties remain true.
    \item Intuitively, a \textbf{subring} should be a subset of a ring that is itself a ring under the restricted operations.
    \item \textbf{Subring}: A subset $S$ of a ring $R$ such that\dots
    \begin{enumerate}[label={(\roman*)}]
        \item For all $a,b\in S$, both $a+b,ab\in S$. For all $a\in S$, $-a\in S$.
        \item $1_R\in S$.
    \end{enumerate}
    \item Check that these conditions are sufficient!
    \item \textbf{Ring homomorphism}: A function $f:A\to B$, where $A,B$ are rings, such that
    \begin{align*}
        f(a_1+a_2) &= f(a_1)+f(a_2)\\
        f(a_1\times a_2) &= f(a_1)\times f(a_2)\\
        f(1_A) &= 1_B
    \end{align*}
    for all $a_1,a_2\in A$.
    \item Note that we need the third constraint because we are not postulating the existence of multiplicative inverses.
    \item Examples:
    \begin{enumerate}
        \item If $S$ is a subring of a ring $R$ and $i:S\to R$ is the inclusion map, then it is a ring homomorphism.
        \item $R_1,R_2$ are rings. Then $\pi:R_1\times R_2\to R_1$ defined by $\pi(a_1,a_2)=a_1$ for all $(a_1,a_2)\in R_1\times R_2$ is a ring homomorphism.
        \item $i:R_1\to R_1\times R_2$ defined by $i(a)=(a,0)$ is not a ring homomorphism unless $R_2$ is trivial since $i(1_{R_1})=(1_{R_1},0)\neq (1_{R_1},1_{R_2})=1_{R_1\times R_2}$.
        \item $f:M_2(\R)\to M_3(\R)$ defined by inclusion in the upper lefthand corner is not a ring homomorphism for the same reason as the above. To be clear, the functional relation considered here is
        \begin{equation*}
            \begin{pmatrix}
                a & b\\
                c & d\\
            \end{pmatrix}
            \mapsto
            \begin{pNiceArray}{cc|c}
                a & b & 0\\
                c & d & 0\\ \hline
                0 & 0 & 0\\
            \end{pNiceArray}
        \end{equation*}
    \end{enumerate}
    \item The integers have no subrings except for itself.
    \begin{itemize}
        \item Consider $\Z/10\Z$, for instance. Doesn't work because we postulate the existence of an identity, but $1\notin\Z/10\Z$.
    \end{itemize}
    \item Subrings of $\Q$:
    \begin{itemize}
        \item $\Z,\Q$, the $p$-adic rationals $\{a/p^n:a\in\Z,n=0,1,\dots\}$, $\{a/(p_1p_2\cdots p_r)^n:a\in\Z,n=0,1,\dots\}$, arbitrary subsets of primes in the denominator.
        \item Exercise: There's a bijective correspondence between the subrings of $\Q$ and the power set of the prime numbers.
    \end{itemize}
\end{itemize}



\section{Office Hours (Nori)}
\begin{itemize}
    \item \marginnote{1/5:}Is $\Z$ a commutative ring?
    \begin{itemize}
        \item Yes it is.
    \end{itemize}
    \item Can you clarify the statement of Problem 1.4?
    \begin{itemize}
        \item For any ring $R$, define a function $\Delta:R\to R\times R$ by
        \begin{equation*}
            \Delta(a) = (a,a)
        \end{equation*}
        \item Clearly $\Delta$ is a ring homomorphism.
        \item Then consider the image $\Delta(R)\subset R\times R$.
        \item We are asked to show that if $\Delta(\Q)\subset B\subset\Q\times\Q$ for $B$ a subring of $\Q\times\Q$, then either $B=\Delta(\Q)$ or $B=\Q\times\Q$.
    \end{itemize}
\end{itemize}



\section{Polynomial Rings and Power Series Rings}
\begin{itemize}
    \item \marginnote{1/6:}End of last time: The subrings of $\Q$.
    \item Today: The subrings an arbitrary ring $R$.
    \item Question 1: Let $R$ a ring, $x\in R$ arbitrary. What is the "smallest" subring $M\subset R$ such that $x\in M$?
    \begin{itemize}
        \item We know that $1_R\in M$. Thus, $1_R+1_R=2_R\in M$. It follows by induction that
        \begin{equation*}
            n_R \in M
        \end{equation*}
        for all $n\in\Z$.
        \item Moving on, $x\in M$ implies that $n_Rx,xn_R\in M$. Is it true that $n_Rx=xn_R$? Yes it is. Here's why.
        \begin{itemize}
            \item Let $C=\{c\in R:cx=xc\}$, where $x$ is the element we've been talking about.
            \item We can prove that $C$ is a subring of $R$; this is Exercise 7.1.9 of \textcite{bib:DummitFoote}; see HW2.
            \item If $C$ is a subring, then $1_R\in C$ implies $1_R+1_R=2_R\in C$, implies $n_R\in C$. Therefore,
            \begin{equation*}
                n_Rx=xn_R \in M
            \end{equation*}
            for all $n\in\Z$.
        \end{itemize}
        \item The above and additive closure:
        \begin{equation*}
            \{a_R+b_Rx:a,b\in\Z\} \subset M
        \end{equation*}
        \item Multiplicative closure: $x\cdot x=x^2\in M$. In general, defining $x^n$ in the usual way (i.e., inductively), shows that
        \begin{equation*}
            x^n \in M
        \end{equation*}
        for all $n\in\Zg$.
        \begin{itemize}
            \item To be explicit, the inductive definition of $x^n$ is $x^0=1_R$ and $x^{n+1}=x\cdot x^n$.
        \end{itemize}
        \item Multiplicative closure and $n_Ry=yn_R$ for $y\in R$ arbitrary (see above argument):
        \begin{equation*}
            a_Rx^n=xa_Rx^{n-1}=\cdots=x^na_R \in M
        \end{equation*}
        for all $a\in\Z$, $n\in\Zg$.
        \item Additive closure:
        \begin{equation*}
            (a_0)_R+(a_1)_Rx+\cdots+(a_n)_Rx^n \in M
        \end{equation*}
        for all $a_0,a_1,\dots,a_n\in\Z$ and $n\in\Zg$.
        \begin{itemize}
            \item Naturally, terms of this form are called \textbf{polynomials}.
            \item As the set of polynomials is at last closed under $+,\times$, $M$ must be a \textbf{polynomial ring}.
        \end{itemize}
    \end{itemize}
    \pagebreak
    \item \textbf{Polynomial ring} (over $\Z$): The ring defined as follows. \emph{Denoted by} $\bm{\pmb{\Z}[X]}$. \emph{Given by}
    \begin{equation*}
        \Z[X] = \bigcup_{m=0}^\infty\{a_0+a_1X+\cdots+a_mX^m:a_0,a_1,\dots,a_m\in\Z\}
    \end{equation*}
    \begin{itemize}
        \item Note that we \emph{insist} on using uppercase for the indeterminate. The motivation for doing so is illustrated by the next example.
    \end{itemize}
    \item $\Z[X]$ induces\footnote{Recall that the terminology "induce" means that to every $R'[X]$, we can assign a set of ring homomorphisms of the given form. In other words, the set of polynomial rings over rings $R'$ is in bijective correspondence with the set of collections of functions $\phi_x$.} a collection of ring homomorphisms $\phi_x:\Z[X]\to R$, one for every $R$ and $x\in R$. These are defined by
    \begin{equation*}
        \phi_x(f) = f(x)
    \end{equation*}
    where $f=a_0+a_1X+\cdots+a_mX^m$, $f(x)=(a_0)_R+(a_1)_Rx+\cdots+(a_m)_Rx^m$, and all $a_i\in\Z$.
    \item Implication.
    \begin{itemize}
        \item For any $R$ and any $x\in R$, $\phi_x(\Z[X])\subset R$.
        \item In layman's terms, the set of all polynomials of a single element of any ring is necessarily a subset of the ring overall.
    \end{itemize}
    \item Question 2: Let $R\subset B$ be rings, and let $x\in B$. Find the smallest subring $M\subset B$ such that $R\subset M$ and $x\in M$.
    \begin{itemize}
        \item Last time, we only knew that $1_R$ had to be in $M$. This time, we have a whole set of elements $R$ to choose from!
        \item Let $a\in R$ be arbitrary. We see that $a,x\in M$; this means that $ax,xa\in M$. But we may not have $ax=xa$ as we did so nicely for the integers $n_R$, so we have to postulate commutativity if we want to avoid a messy answer.
        \item Henceforth, we assume
        \begin{equation*}
            ax=xa \in M
        \end{equation*}
        for all $a\in R$.
        \item As in Question 1, $ax=xa$ implies
        \begin{equation*}
            ax^m=x^ma \in M
        \end{equation*}
        for all $a\in R$, $m\in\Zg$.
        \item Thus,
        \begin{equation*}
            a_0+\cdots+a_mx^m \in M
        \end{equation*}
        for $a_0,\dots,a_m\in R$, $m\in\Zg$.
        \item This set of polynomials is already a subring. Thus, it is not only contained in $M$, but must also equal $M$.
        \item Difference between these polynomials and the ones from Question 1: These are the polynomials with coefficients in $R\supset\Z$, where this containment is homomorphic (not necessarily injective).
        \begin{itemize}
            \item Therefore, we need to define a broader type of polynomial ring.
        \end{itemize}
    \end{itemize}
    \item \textbf{Polynomial ring} (over $R$): The ring defined as follows. \emph{Denoted by} $\bm{R[X]}$. \emph{Given by}
    \begin{equation*}
        R[X] = \bigcup_{m=0}^\infty\{a_0+a_1X+\cdots+a_mX^m:a_0,a_1,\dots,a_m\in R\}
    \end{equation*}
    \begin{itemize}
        \item We do not require that $R$ is commutative.
        \item Note that $R[X]$ will be commutative, however, owing to the way it's defined.
    \end{itemize}
    \item We now seek to generalize polynomial rings to \textbf{power series rings}.
    \item To do so, we'll need to get more precise than the infinite unions we've been using.
    \begin{itemize}
        \item Consider the set of nonnegative integers $\Zg=\{0,1,2,\dots\}$.
        \begin{itemize}
            \item This is a \textbf{monoid} under both addition and multiplication.
        \end{itemize}
        \item Let $(R,+)$ be an abelian group.
        \item Then $(R^\Zg,+)$ is also an abelian group.
        \begin{itemize}
            \item As per last class, all elements $a\in(R^\Zg,+)$ are functions $a:\Zg\to R$.
            \item We write that $a:n\mapsto a_n$, i.e., the value of $a$ at $n$ will be denoted $a_n$, not $a(n)$.
        \end{itemize}
        \item Every element $a\in R^\Zg$ will be represented by $\sum_{n=0}^\infty a_nX^n$.
        \begin{itemize}
            \item This is allowable because there is a natural bijective correspondence between each $a$ and each power series $\sum_{n=0}^\infty a_nX^n$.
            \item Essentially, what we are doing here is using the rigorously defined set of functions $R^\Zg$ to theoretically stand in for the intuitive concept of a power series. This is acceptable since both objects have very similar properties, especially as pertains to adding and multiplying them.
            \item This is like defining the real numbers (intuitive) in terms of Dedekind cuts (rigorous).
            \item Note that alternatively, we could introduce the entire sequences/series analytical framework from Honors Calculus IBL to logically underpin power series, but this technique will be much less bulky and suit our purposes just fine.
        \end{itemize}
        \item We define addition and multiplication on $R^\Zg$ as follows.
        \begin{gather*}
            \left( \sum_{n=0}^\infty a_nX^n \right)+\left( \sum_{n=0}^\infty b_nX^n \right) = \sum_{n=0}^\infty(a_n+b_n)X^n\\
            \left( \sum_{p=0}^\infty a_pX^p \right)\left( \sum_{q=0}^\infty b_qX^q \right) = \sum_{\substack{p\geq 0,\\ q\geq 0}}a_pb_qX^{p+q}
                = \sum_{r=0}^\infty\left( \sum_{p=0}^ra_pb_{r-p} \right)X^r
        \end{gather*}
        \item This is the \textbf{power series ring}.
    \end{itemize}
    \item \textbf{Monoid}: A set equipped with an associative binary operation and an identity element.
    \item \textbf{Power series ring} (over $R$): The ring defined as follows, with $+,\times$ defined as above. \emph{Denoted by} $\bm{(R[[X]],+,\times)}$. \emph{Given by}
    \begin{equation*}
        R[[X]] = R^\Zg
    \end{equation*}
    \item Note that the definitions of addition and multiplication for $R[[X]]$ are precisely the ones needed for $R[X]$, too, (just the finite version) even though we didn't state them earlier.
    \item Two observations about power series rings which will also hold for polynomial rings.
    \begin{enumerate}
        \item $R$ is a subring of $R[[X]]$ with the inclusion ring homomorphism $a\mapsto a1+0X^1+0X^2+\cdots$.
        \item Additionally, we can map $X\in R$ to $0X^0+1X^1+0X^2+\cdots\in R[[X]]$.
    \end{enumerate}
    \item $aX=Xa$ for all $a\in R$.
    \begin{itemize}
        \item Why?? Ask in OH.
    \end{itemize}
    \item Alternate definition of $R[X]$: The subring of $R[[X]]$ given by
    \begin{equation*}
        R[X] = \left\{ \sum_{m=0}^\infty a_mX^m\in R[[X]] \middle| |\{m\in\Zg:a_m\neq 0\}|<\infty \right\}
    \end{equation*}
    \item Theorem (Universal Property of a Polynomial Ring): Let $R$ be a ring, $\alpha:R\to B$ a ring homomorphism, and $x\in B$. Assume that $x\cdot\alpha(a)=\alpha(a)\cdot x$ for all $a\in R$. Then there is a unique ring homomorphism $\beta:R[X]\to B$ such that $\beta(a)=\alpha(a)$ for all $a\in R$ and $\beta(X)=x$.
    \begin{proof}
        % Scratch: $\beta(X)=x$ implies $\beta(X^m)=x^m$ for all $m=0,1,\dots$. Taking $a_m\in R$, then $\beta(a_m)=\alpha(a_m)$ given $\beta(a_mX^m)=\beta(a_m)\beta(X^m)=\alpha(a_m)x^m$.\par
        % $\beta(a_0+\cdots+a_mX^m)=\sum_{i=0}^m\beta(a_iX^i)=\sum_{i=0}^m\alpha(a_i)x^i$.

        We first prove that such a ring homomorphism exists. Then we address uniqueness.\par
        Let $\beta(X)=x$. Then if $\beta$ is to be a ring homomorphism, we must have
        \begin{equation*}
            \beta(X^m) = x^m
        \end{equation*}
        for all $m\in\Zg$. We also require that $\beta(a_m)=\alpha(a_m)$ for all $a_m\in R$ (at this point, $a_m$ is just suggestive notation). Again, if $\beta$ is to be a ring homomorphism, it must follow that
        \begin{equation*}
            \beta(a_mX^m) = \beta(a_m)\beta(X^m)
            = \alpha(a_m)x^m
        \end{equation*}
        for all $a_m\in R$, $m\in\Z$. Lastly, if $\beta$ is to be a ring homomorphism, it must follow that
        \begin{equation*}
            \beta\left( \sum_{i=0}^ma_iX^i \right) = \sum_{i=0}^m\beta(a_iX^i)
            = \sum_{i=0}^m\alpha(a_i)x^i
        \end{equation*}
        But then by its construction, $\beta$ is defined on every element in $R[X]$ and is a ring homomorphism satisfying the desired properties.\par
        Suppose $\beta,\beta':R[X]\to B$ are ring homomorphisms satisfing $\beta(a)=\beta'(a)=\alpha(a)$ for all $a\in R$ and $\beta(X)=\beta'(X)=x$. Let $\sum_{i=0}^ma_iX^i\in R[X]$ be arbitrary. Then
        \begin{equation*}
            \beta\left( \sum_{i=0}^ma_iX^i \right) = \sum_{i=0}^m\alpha(a_i)x^i
            = \beta'\left( \sum_{i=0}^ma_iX^i \right)
        \end{equation*}
        as desired.
    \end{proof}
    \item The idea of the theorem.
    \begin{itemize}
        \item Evaluation of a function ($f\in R[X]$) at a point ($x\in B$): If $R\subset B$ and $\alpha(a)=a$ for all $a\in R$, then $\beta(f)=f(x)$. Recall the $\phi_x$ from earlier.
        \item $\alpha$ is like a coordinate change function, allowing us to evaluate variants of each $f$.
        \item In fact, this idea is highly related to the linear algebra concept that specifying the action of a map on a basis specifies its action on all elements.
        \begin{itemize}
            \item However, here we are dealing with a \textbf{module homomorphism}, not a linear transformation.
        \end{itemize}
    \end{itemize}
\end{itemize}



\section{Chapter 7: Introduction to Rings}
\emph{From \textcite{bib:DummitFoote}.}
\setcounter{bookch}{7}
\subsection*{A Word on Ring Theory}
\begin{itemize}
    \item \marginnote{1/7:}Plan for Part II: Ring theory.
    \begin{itemize}
        \item Study analogues of group-related objects, such as "subrings, quotient rings, ideals (which are the analogues of normal subgroups), and ring homomorphisms" \parencite[222]{bib:DummitFoote}.
        \item Answer questions about general rings, leading to fields and finite fields.
        \item Arithmetic over general rings, and applications of these results to polynomial rings.
    \end{itemize}
    \item Part II grounds the remaining four parts of the book.
    \begin{itemize}
        \item Part III is modules (ring actions).
        \item Part IV is fields and polynomial equations over them (applications of ring structure theory).
        \item Part V is ring applications.
        \item Part VI is specific kinds of rings and the objects on which they act.
    \end{itemize}
\end{itemize}


\subsection*{Section 7.1: Basic Definitions and Examples}
\begin{itemize}
    \item Definition of a \textbf{ring} \parencite[223]{bib:DummitFoote}.
    \item Motivation for requiring $(R,+)$ to be abelian.
    \begin{itemize}
        \item If $R$ is a ring with identity, then the distributive laws imply commutativity of addition anyway, as follows.\footnote{Thus, our definition of a ring in class is somewhat redundant. Indeed, if we're defining a ring to be a ring with identity, then we can omit the abelian condition and know that the distributive laws will still imply it.}
        \item Let $a,b\in R$ be arbitrary. We have from the ring axioms that
        \begin{alignat*}{6}
            (1+1)(a+b) &= 1(a+b)+1(a+b)&
                &= 1a+1b+1a+1b&
                &= a+b+a+b\\
            (1+1)(a+b) &= (1+1)a+(1+1)b&
                &= 1a+1a+1b+1b&
                &= a+a+b+b
        \end{alignat*}
        \item Thus, by transitivity and the cancellation law,
        \begin{equation*}
            b+a = a+b
        \end{equation*}
    \end{itemize}
    \item One of the most important examples of a ring is a \textbf{field}.
    \item \textbf{Division ring}: A ring $R$ with identity $1\neq 0$ such that every nonzero element $a\in R$ has a multiplicative inverse, i.e., there exists $b\in R$ such that $ab=ba=1$. \emph{Also known as} \textbf{skew field}.
    \item \textbf{Field}: A commutative division ring.
    \item \textbf{Trivial ring}: A ring $R$ for which $a\times b=0$ for all $a,b\in R$.
    \begin{itemize}
        \item So named because "although trivial rings have two binary operations, multiplication adds no new structure to the additive group, and the theory of rings goves no information which could not already be obtained from (abelian) group theory" \parencite[224]{bib:DummitFoote}.
    \end{itemize}
    \item \textbf{Zero ring}: The trivial ring where $R=\{0\}$. \emph{Denoted by} $\bm{R=0}$.
    \item Excluding the zero ring, trivial rings do not contain a multiplicative identity.
    \begin{itemize}
        \item Suppose for the sake of contradiction that there exists $1\in R$ trivial and nonzero. Let $a$ be a nonzero element of $R$. Then
        \begin{equation*}
            a = 1\times a = 0
        \end{equation*}
        a contradiction.
    \end{itemize}
    \item $\Z/n\Z$ is a commutative ring with identity under modular arithmetic.
    \item \textbf{Hamilton Quaternions}: The set of elements of the form
    \begin{equation*}
        a+bi+cj+dk
    \end{equation*}
    where $a,b,c,d\in\R$, under componentwise addition
    \begin{equation*}
        (a+bi+cj+dk)+(a'+b'i+c'j+d'k) = (a+a')+(b+b')i+(c+c')j+(d+d')k
    \end{equation*}
    and distributive noncommutative multiplication subject to the relations
    \begin{align*}
        i^2 &= j^2 = k^2 = -1&
        ij &= -ji = k&
        jk &= -kj = i&
        ki &= -ik = j
    \end{align*}
    \emph{Also known as} \textbf{real Hamilton Quaternions}. \emph{Denoted by} $\pmb{\mathbb{H}}$.
    \begin{itemize}
        \item \textcite{bib:DummitFoote} provides an example multiplication.
        \item $\mathbb{H}$ is a ring, specifically a \emph{noncommutative} ring with identity ($1=1+0i+0j+0k$).
        \item Historically, it was one of the first noncommutative rings discovered.
        \begin{itemize}
            \item Sir William Rowan Hamilton discovered it in 1843.
            \item Quaternions have been very influential in the development of mathematics and continue to be important in certain areas of mathematics and physics.
        \end{itemize}
        \item The Quaternions form a division ring with
        \begin{equation*}
            (a+bi+cj+dk)^{-1} = \frac{a-bi-cj-dk}{a^2+b^2+c^2+d^2}
        \end{equation*}
        \item We can also define the rational Hamilton Quaternions by only taking $a,b,c,d\in\Q$.
    \end{itemize}
    \item $R=A^X$ is commutative iff $A$ is commutative.
    \begin{itemize}
        \item $R$ has 1 iff $A$ has 1 (in which case $1_R:X\to A$ sends $x\mapsto 1_A$ for all $x\in X$).
    \end{itemize}
    \item $C([a,b],\R)$ is a ring with identity, though we need limit theorems to prove this.
    \item Basic properties of arbitrary rings.
    \begin{proposition}\label{prp:7.1}
        Let $R$ be a ring. Then
        \begin{enumerate}
            \item $0a=a0=a$ for all $a\in R$;
            \item $(-a)b=a(-b)=-(ab)$ for all $a,b\in R$;
            \item $(-a)(-b)=ab$ for all $a,b\in R$;
            \item If $R$ has an identity 1, then the identity is unique and $-a=(-1)a$.
        \end{enumerate}
        \begin{proof}
            Given.
        \end{proof}
    \end{proposition}
    \item \textbf{Zero divisor}: A nonzero element $a\in R$ to which there corresponds a nonzero element $b\in R$ such that either $ab=0$ or $ba=0$.
    \item \textbf{Unit} (in $R$ a nonzero ring with identity): An element $u\in R$ to which there corresponds some $v\in R$ such that $uv=vu=1$.
    \begin{itemize}
        \item As the phrasing of the term implies, the property of being a unit depends on the ring in which an element is viewed. For example, 2 is not a unit in $\Z$, but 2 is a unit in $\Q$.
    \end{itemize}
    \item \textbf{Group of units} (of $R$): The set of all units in $R$. \emph{Denoted by} $\bm{R^\times}$, $\bm{R^*}$.
    \begin{itemize}
        \item As the name implies, $R^\times$ is a group under multiplication.
    \end{itemize}
    \item Alternate definition of field: A commutative ring $F$ with identity $1\neq 0$ in which every nonzero element is a unit, i.e., $F^\times=F-\{0\}$.
    \item A zero divisor can never be a unit.
    \begin{itemize}
        \item Suppose for the sake of contradiction that $a$ is a unit in $R$ and $ab=0$ for some nonzero $b\in R$. Then $va=1$ for some $v\in R$. It follows that
        \begin{equation*}
            b = 1b = (va)b = v(ab) = v0 = 0
        \end{equation*}
        a contradiction. The argument is symmetric if we assume $ba=0$.
        \item It follows that fields contain no zero divisors.
    \end{itemize}
    \item Examples of zero divisors and units.
    \begin{enumerate}
        \item $\Z$.
        \begin{itemize}
            \item No zero divisors and $\Z^\times=\{\pm 1\}$.
        \end{itemize}
        \item $\Z/n\Z$.
        \begin{itemize}
            \item The elements $\bar{u}$ for which $u,n$ are relatively prime are units (see proof in Chapter 8).
            \item If $a,n$ are not relatively prime, then $\bar{a}$ is a zero divisor in $\Z/n\Z$ ($a\cdot n/a=0$).
            \item Thus, every nonzero element of $\Z/n\Z$ is either a unit or a zero divisor.
            \item $\Z/n\Z$ is a field iff $n$ is prime (every nonzero element is a unit iff they are all relatively prime to $n$).
        \end{itemize}
        \item $\R^{[0,1]}$.
        \begin{itemize}
            \item The units are all functions that are nonzero on the entire domain.
            \item $f$ not a unit and nonzero implies $f$ is a zero divisor: Choose
            \begin{equation*}
                g(x) =
                \begin{cases}
                    0 & f(x)\neq 0\\
                    1 & f(x)=1
                \end{cases}
            \end{equation*}
        \end{itemize}
        \item $C([0,1],\R)$.
        \begin{itemize}
            \item There exist units (same as above), zero divisors (consider a function that is nonzero on $[0,0.5)$ and zero on $[0.5,1]$), and functions that are neither (consider a function that is only zero at $x=0.5$; then its complement would necessarily be discontinuous at $x=0.5$).
        \end{itemize}
        \item \textbf{Quadratic fields} (see Section 13.2).
    \end{enumerate}
    \item \textbf{Quadratic field}: A ring of the following form, where $D$ is a rational number and not a perfect square in $\Q$. \emph{Denoted by} $\bm{\pmb{\Q}(\sqrt{D})}$. \emph{Given by}
    \begin{equation*}
        \Q(\sqrt{D}) = \{a+b\sqrt{D}:a,b\in\Q\}
    \end{equation*}
    \begin{itemize}
        \item Addition is componentwise and multiplication is "as expected" based on the notation, i.e.,
        \begin{align*}
            (a+b\sqrt{D})+(c+d\sqrt{D}) &= (a+c)+(b+d)\sqrt{D}\\
            (a+b\sqrt{D})\times(c+d\sqrt{D}) &= (ac+bdD)+(ad+bc)\sqrt{D}
        \end{align*}
        \begin{itemize}
            \item It follows that multiplication is commutative; hence, $\Q(\sqrt{D})$ is a commutative ring.
        \end{itemize}
        \item $\Q(\sqrt{D})$ is a subring of $\C$.
        \begin{itemize}
            \item If $D>0$, then it is a subring of $\R$.
        \end{itemize}
        \item The assumption that $D$ is not a perfect square implies that every element in $\Q(\sqrt{D})$ can be written uniquely in the form $a+b\sqrt{D}$.
        \begin{itemize}
            \item Consequence: $a^2-Db^2\neq 0$ if $a,b$ are nonzero.
        \end{itemize}
        \item Since $(a+b\sqrt{D})(a-b\sqrt{D})=a^2-Db^2$, the inverse of $a+b\sqrt{D}\neq 0$ is
        \begin{equation*}
            \frac{a-b\sqrt{D}}{a^2-Db^2}
        \end{equation*}
        \item Thus, all nonzero elements in $\Q(\sqrt{D})$ are units; hence, $\Q(\sqrt{D})$ is a field.
    \end{itemize}
    \item \textbf{Squarefree part} (of $D\in\Q$): The unique integer $D'$ that is not divisible by the square of any integer greater than 1 and such that $D=f^2D'$ for some $f\in\Q$.
    \begin{itemize}
        \item Since $\sqrt{D}=f\sqrt{D'}$, we may take $D$ to be a squarefree integer in the definition of $\Q(\sqrt{D})$ in general and WLOG.
        \item Indeed, we just combine $f$ into $b$.
    \end{itemize}
    \item \textbf{Integral domain}: A commutative ring with identity $1\neq 0$ that has no zero divisors.
    \begin{itemize}
        \item $\Z$ is the prototypical integral domain.
    \end{itemize}
    \item Properties of integral domains.
    \begin{proposition}[Cancellation law]\label{prp:7.2}
        Assume $a,b,c$ are elements of any ring with $a$ not a zero divisor. If $ab=ac$, then either $a=0$ or $b=c$ (i.e., if $a\neq 0$, then we can cancel the $a$'s).\par
        In particular, if $a,b,c$ are any elements of an integral domain and $ab=ac$, then either $a=0$ or $b=c$.
        \begin{proof}
            $ab=ac$ implies $a(b-c)=0$. Thus, since $a$ is not a zero divisor, either $a=0$ or $b-c=0$ (equivalently, $b=c$).
        \end{proof}
    \end{proposition}
    \begin{corollary}\label{cly:7.3}
        Any finite integral domain is a field.
        \begin{proof}
            Let $R$ be a finite integral domain, and $a$ be an arbitrary, nonzero element of $R$. We seek to find $b$ such that $ab=1$, which will imply that $a$ (i.e., every element) is a unit in $R$.\par
            Define the map $x\mapsto ax$. By the cancellation law, this map is injective. Injectivity plus the fact that $R$ is finite proves that this map is surjective. Thus, there exists $b\in R$ such that $ab=1$, as desired.
        \end{proof}
    \end{corollary}
    \item Wedderburn: A finite division ring is necessarily commutative, i.e., is a field.
    \begin{itemize}
        \item See Exercise 13.6.13 for a proof.
    \end{itemize}
    \item "Every nonzero element of a commutative ring that is not a zero divisor has a multiplicative inverse in some larger ring" \parencite[228]{bib:DummitFoote}.
    \begin{itemize}
        \item See Section 7.5.
    \end{itemize}
    \item \textbf{Subring} (of $R$): A subgroup of $R$ that is closed under multiplication.
    \item To confirm that $S\subset R$ is a subring, check that is is nonempty, closed under subtraction, and closed under multiplication.
    \item The property "is a subring of" is transitive.
    \item "If $R$ is a subring of a field $F$ that contains the identity of $F$, then $R$ is an integral domain. The converse of this is also true, namely any integral domain is contained in a field" \parencite[229]{bib:DummitFoote}.
    \begin{itemize}
        \item See Section 7.5.
    \end{itemize}
    \item \textbf{Ring of integers} (in the quadratic field $\Q(\sqrt{D})$): The subring defined as follows. \emph{Denoted by} $\bm{\mathcal{O}}$, $\bm{\mathcal{O}_{\pmb{\Q}(\sqrt{D})}}$. \emph{Given by}
    \begin{equation*}
        \mathcal{O} = \Z[\omega]
        = \{a+b\omega:a,b\in\Z\}
    \end{equation*}
    where
    \begin{equation*}
        \omega =
        \begin{cases}
            \sqrt{D} & D\equiv 2,3\mod 4\\
            \frac{1+\sqrt{D}}{2} & D\equiv 1\mod 4
        \end{cases}
    \end{equation*}
    \begin{itemize}
        \item Etymology: Elements of the subring $\mathcal{O}$ in the field $\Q(\sqrt{D})$ have many analogous properties to those of the of the subring $\Z$ in the field $\Q$.
        \item $\mathcal{O}$ is the \textbf{integral closure} of $\Z$ in $\Q(\sqrt{D})$ --- see Section 15.3.
    \end{itemize}
    \item \textbf{Gaussian integers}: The ring of integers in the quadratic field $\Q(\sqrt{-1})$. \emph{Denoted by} $\bm{\pmb{\Z}[i]}$.
    \begin{itemize}
        \item Gauss originally introduced these in 1800 to state the \textbf{biquadratic reciprocity law}.
    \end{itemize}
    \item \textbf{Biquadratic reciprocity law}: A statement dealing with the "beautiful relations that exist among fourth powers modulo primes" \parencite[229]{bib:DummitFoote}.
    \item \textbf{Field norm}: The function from $\Q(\sqrt{D})\to\Q$ defined as follows. \emph{Denoted by} $\bm{N}$. \emph{Given by}
    \begin{equation*}
        N(a+b\sqrt{D}) = (a+b\sqrt{D})(a-b\sqrt{D}) = a^2-Db^2
    \end{equation*}
    \begin{itemize}
        \item $N$ is nonzero when $a+b\sqrt{D}\neq 0$ (see above).
        \item Measures "size" --- for example, if $D=-1$, then $N(a+bi)=a^2+b^2$, which is the length of this complex number considered as a vector in the complex plane.
        \item Useful for establishing many properties of $\mathcal{O}$.
        \item $N$ is multiplicative: $N(\alpha\beta)=N(\alpha)N(\beta)$ for all $\alpha,\beta\in\Q(\sqrt{D})$.
        \item Defining $N$ on $\mathcal{O}$ shows that $N(\alpha)$ is an \emph{integer} for every $\alpha\in\mathcal{O}$.
    \end{itemize}
    \item $\alpha\in\mathcal{O}^\times$ iff $N(\alpha)=\pm 1$.
    \begin{itemize}
        \item \textcite{bib:DummitFoote} proves this from the definition.
    \end{itemize}
    \item \textbf{Pell's equation}: The following equation, where $x,y,D\in\Z$. \emph{Given by}
    \begin{equation*}
        x^2-Dy^2 = \pm 1
    \end{equation*}
    \begin{itemize}
        \item Finding solutions is equivalent to finding units in $\mathcal{O}$.
    \end{itemize}
    \item Proves via Pell's equation that
    \begin{align*}
        \Z[i]^\times &= \{\pm 1,\pm i\}&
        \Z\left[ \frac{1+\sqrt{-3}}{2} \right] &= \{\pm 1,\pm\rho,\pm\rho^2\}
    \end{align*}
    where $\rho=(1+\sqrt{-3})/2$.
    \begin{itemize}
        \item When $D<0$ and $D\neq -1,-3$, $\mathcal{O}^\times=\{\pm 1\}$.
        \item When $D>0$, $\mathcal{O}^\times$ is infinite.
    \end{itemize}
    \item This whole discussion on the ring of integers in a quadratic field is highly related to HW4 Q4.3-4.4.
    \item \textbf{Nilpotent} (element): An element $x\in R$ such that $x^m=0$ for some $m\in\N$.
\end{itemize}


\subsection*{Section 7.2: Examples -- Polynomial Rings, Matrix Rings, and Group Rings}
\begin{itemize}
    \item \textbf{Polynomial rings}, \textbf{matrix rings}, and \textbf{group rings} are often related.
    \begin{itemize}
        \item Example: The group ring of a group $G$ over the complex numbers $\C$ is a direct product of matrix rings over $\C$.
    \end{itemize}
    \item Example applications of these three classes of rings.
    \begin{itemize}
        \item Study them in their own right.
        \item Polynomial rings help prove classification theorems for matrices which, in particular, determine when a matrix is similar to a diagonal matrix.
        \item Group rings help study group actions and prove additional classification theorems.
    \end{itemize}
    \item We begin with polynomial rings.
    \item Fix a commutative ring $R$ with identity.
    \item \textbf{Indeterminate}: The "variable" $X$.
    \item \textbf{Polynomial} (in $X$ with coefficients $a_i$ in $R$): The formal sum
    \begin{equation*}
        a_nX^n+a_{n-1}X^{n-1}+\cdots+a_1X+a_0
    \end{equation*}
    with $n\geq 0$ and each $a_i\in R$.
    \item \textbf{Degree $\bm{n}$} (polynomial): A polynomial for which $a_n\neq 0$.
    \item \textbf{Leading term}: The $a_nX^n$ term.
    \item \textbf{Leading coefficient}: The $a_n$ coefficient.
    \item \textbf{Monic} (polynomial): A polynomial for which $a_n=1$.
    \item Definition of $R[X]$ \parencite[234]{bib:DummitFoote}.
    \item \textbf{Constant polynomials}: The set of polynomials $R\subset R[X]$.
    \item It follows from its construction that $R[X]$ is a commutative ring with identity (specifically $1_R$).
    \item Definition of $\Z[X],\Q[X]$.
    \item We can also define polynomial rings like $\Z/3\Z[X]$.
    \begin{itemize}
        \item This ring consists of the set of polynomials with coefficients $0,1,2$ and calculations on the coefficients performed modulo 3.
        \item Example: If $p(X)=X^2+2X+1$ and $q(X)=X^3+X+2$, then $p(X)+q(X)=X^3+X^2$.
    \end{itemize}
    \item The ring in which the coefficients are taken makes a substantial difference in the polynomials' behavior.
    \begin{itemize}
        \item Example: $X^2+1$ is not a perfect square in $\Z[X]$, but is in $\Z/2\Z[X]$ since here,
        \begin{equation*}
            (X+1)^2 = X^2+2X+1 = X^2+1
        \end{equation*}
    \end{itemize}
    \item Properties of polynomials over integral domains.
    \begin{proposition}\label{prp:7.4}
        Let $R$ be an integral domain and let $p(X),q(X)$ be nonzero elements of $R[X]$. Then
        \begin{enumerate}
            \item $\deg p(X)q(X)=\deg p(X)+\deg q(X)$;
            \begin{proof}
                If $p(X),q(X)$ are polynomials with leading terms $a_nX^n,b_mX^m$, respectively, then the leading term of $p(X)q(X)$ is $a_nb_mX^{n+m}$, provided $a_nb_m\neq 0$. But since $a_n,b_m\neq 0$ (as leading coefficients) and $R$ has no zero divisors (as an integral domain), we have that $a_nb_m\neq 0$. Applying the definition of degree completes the proof.
            \end{proof}
            \item The units of $R[X]$ are just the units of $R$;
            \begin{proof}
                Suppose $p(X)\in R[X]$ is a unit. Then $p(X)q(X)=1$ for some $q(X)\in R[X]$. It follows by part (1) that
                \begin{equation*}
                    \deg p(X)+\deg q(X) = \deg p(X)q(X) = 0
                    \quad\Longleftrightarrow\quad
                    \deg p(X)=\deg q(X) = 0
                \end{equation*}
                Therefore, $p(X),q(X)\in R$ and hence are units of $R$, as desired.
            \end{proof}
            \item $R[X]$ is an integral domain.
            \begin{proof}
                We have already established that the commutativity and identity of $R[X]$ follow from $R$. As to no zero divisors, this constraint follows from part (1).
            \end{proof}
        \end{enumerate}
    \end{proposition}
    \item If $R$ has zero divisors, then so does $R[X]$.
    \begin{itemize}
        \item If $f\in R[X]$ is a zero divisor, then $cf=0$ for some nonzero $c\in R$ (see Exercise \ref{exr:7.2.2}).
    \end{itemize}
    \item If $S$ is a subring of $R$, then $S[X]$ is a subring of $R[X]$.
    \begin{itemize}
        \item Think back to the definition.
    \end{itemize}
    \item More on polynomial rings in Chapter 9.
    \item \marginnote{1/9:}We now move onto matrix rings.
    \item \textbf{Matrix ring} (over $R$): The set of all $n\times n$ matrices $(a_{ij})$ with entries from $R$ under componentwise addition and matrix multiplication, where $R$ is an arbitrary ring and $n\in\N$. \emph{Denoted by} $\bm{M_n(R)}$.
    \item $M_n(R)$ is \emph{not} commutative for all nontrivial $R$ and $n\geq 2$.
    \begin{proof}
        Since $R$ is nontrivial, we may pick $a,b\in R$ such that $ab\neq 0$. Let $A$ be the matrix with $a_{1,1}=a$ and zeroes elsewhere, and let $B$ be the matrix with $b_{1,2}=b$ and zeroes elsewhere. Then $ab$ is the nonzero entry in position $1,2$ of $AB$ whereas $BA=0$.
    \end{proof}
    \item The matrices defined in the above proof are also zero divisors.
    \begin{itemize}
        \item Thus, $M_n(R)$ has zero divisors for all nonzero rings $R$ where $n\geq 2$.
    \end{itemize}
    \item \textbf{Scalar matrix}: An element $(a_{ij})\in M_n(R)$ such that
    \begin{equation*}
        a_{ij} = a\cdot\delta_{ij}
    \end{equation*}
    for some $a\in R$ and all $i,j\in\{1,\dots,n\}$.
    \begin{itemize}
        \item The scalar matrices form a subring of $M_n(R)$, specifically one that is isomorphic to $R$.
        \item We have that
        \begin{align*}
            \diag(a)+\diag(b) &= \diag(a+b)&
            \diag(a)\cdot\diag(b) &= \diag(a\cdot b)
        \end{align*}
        \item If $R$ is commutative, the scalar matrices commute with all elements of $M_n(R)$.
    \end{itemize}
    \item \textbf{Identity matrix}: The scalar matrix for which $a=1$, where 1 is the identity of $R$.
    \begin{itemize}
        \item Only exists if $R$ is a ring with identity.
        \item If it exists, this matrix is the 1 of $M_n(R)$.
        \item The existence of a 1 in $M_n(R)$ allows us to define the units in $M_n(R)$, as follows.
    \end{itemize}
    \item \textbf{General linear group} (of degree $n$): The group of units of $M_n(R)$. \emph{Denoted by} $\bm{GL_n(R)}$.
    \begin{itemize}
        \item Alternative definition: The set of $n\times n$ invertible matrices with entries in $R$.
    \end{itemize}
    \item If $S$ is a subring of $R$, then $M_n(S)$ is a subring of $M_n(R)$.
    \item \textbf{Upper triangular matrix}: The set of all matrices $(a_{ij})$ for which $a_{pq}=0$ whenever $p>q$.
    \begin{itemize}
        \item The set of upper triangular matrices is a subring of $M_n(R)$.
    \end{itemize}
    \item Lastly, we address group rings.
    \item \textbf{Group ring} (of $G$ with coefficients in $R$): The set of all formal sums
    \begin{equation*}
        a_1g_1+\cdots+a_ng_n
    \end{equation*}
    under componentwise addition
    \begin{equation*}
        (a_1g_1+\cdots+a_ng_n)+(b_1g_1+\cdots+b_ng_n) = (a_1+b_1)g_1+\cdots+(a_n+b_n)g_n
    \end{equation*}
    and multiplication defined by the distributive law as well as $(ag_i)(bg_j)=(ab)g_k$ (where $g_k=g_ig_j$) such that the coefficient of $g_k$ in the product $(a_1g_1+\cdots+a_ng_n)\times(b_1g_1+\cdots+b_ng_n)$ is
    \begin{equation*}
        \sum_{g_ig_j=g_k}a_ib_j
    \end{equation*}
    where $a_i\in R$, a commutative ring with identity $1\neq 0$, and $g_i\in G$, a finite group with group operation written multiplicatively, for all $1\leq i\leq n$. \emph{Denoted by} $\bm{RG}$.
    \begin{itemize}
        \item Note that the commutativity of $R$ is not technically needed.
        \item The associativity of multiplication follows from the associativity of the group operation in $G$.
        \item $RG$ is commutative iff $G$ is abelian.
        \item If $g_1\in G$ is the identity of $G$, then we denote $a_1g_1$ by $a_1$.
        \item Similarly, if $1\in R$ is the multiplicative identity of $R$, then we denote $1g_i$ by $g_i$.
    \end{itemize}
    \item \textcite{bib:DummitFoote} gives an example sum and product evaluation in $\Z D_8$.
    \item $R$ appears in $RG$ as the "constant" formal sums, that is, the $R$-multiples of the identity of $G$.
    \begin{itemize}
        \item You can check that addition and multiplication on $RG$ when restricted to these elements is just addition and multiplication on $R$.
        \item These "elements of $R$" commute with all elements of $RG$.
        \item The identity of $R$ is the identity of $RG$.
    \end{itemize}
    \item $G$ appears in $RG$ as the elements $1g_i$.
    \begin{itemize}
        \item Multiplication in $RG$ when restricted to these elements is just the group operation of $G$.
    \end{itemize}
    \item Consequence: Each "element of $G$" has a multiplicative in $RG$ (namely, its inverse in $G$).
    \begin{itemize}
        \item Thus, $G$ is a subgroup of the group of units of $RG$.
    \end{itemize}
    \item If $|G|>1$, then $RG$ always has zero divisors.
    \begin{proof}
        Pick $g\in G$ of order $m>1$. Then
        \begin{equation*}
            (1-g)(1+g+\cdots+g^{m-1}) = 1-g^m
            = 1-1
            = 0
        \end{equation*}
        so $1-g$, for example, is a zero divisor.
    \end{proof}
    \item If $S$ is a subring of $R$, then $SG$ is a subring of $RG$.
    \item \textbf{Integral group ring} (of $G$): The group ring of $G$ with coefficients in $\Z$. \emph{Denoted by} $\bm{\pmb{\Z}G}$.
    \item \textbf{Rational group ring} (of $G$): The group ring of $G$ with coefficients in $\Q$. \emph{Denoted by} $\bm{\pmb{\Q}G}$.
    \item If $H\leq G$, then $RH$ is a subring of $RG$.
    \item Note that $\R Q_8\neq\mathbb{H}$.
    \begin{itemize}
        \item One difference is that $\R Q_8$ necessarily contains zero divisors, while $\mathbb{H}$ is a division ring and hence cannot contain zero divisors.
    \end{itemize}
    \item Group rings over fields will be studied extensively in Chapter 18.
\end{itemize}

\subsubsection*{Exercises}
\begin{enumerate}[label={\textbf{\arabic*.}},ref={7.2.\arabic*}]
    \stepcounter{enumi}
    \item \label{exr:7.2.2}\marginnote{1/7:}Let $p(x)=a_nx^n+a_{n-1}x^{n-1}+\cdots+a_1x+a_0$ be an element of the polynomial ring $R[X]$. Prove that $p(x)$ is a zero divisor in $R[X]$ iff there is a nonzero $b\in R$ such that $bp(x)=0$. \emph{Hint}: Let $g(x)=b_mx^m+b_{m-1}x^{m-1}+\cdots+b_1x+b_0$ be a nonzero polynomial of minimal degree such that $g(x)p(x)=0$. Show that $b_ma_n=0$ and so $a_ng(x)$ is a polynomial of degree less than $m$ that also gives 0 when multiplied by $p(x)$. Conclude that $a_ng(x)=0$. Apply a similar argument to show by induction on $i$ that $a_{n-i}g(x)=0$ for $i=0,1,\dots,n$ and show that this implies $b_mp(x)=0$.
\end{enumerate}




\end{document}