\documentclass[../notes.tex]{subfiles}

\pagestyle{main}
\renewcommand{\chaptermark}[1]{\markboth{\chaptername\ \thechapter\ (#1)}{}}

\begin{document}




\chapter{Rings Intro}
\section{Rings, Subrings, and Ring Homomorphisms}
\begin{itemize}
    \item \marginnote{1/4:}Intro to the course.
    \item What will be covered: Most of Chapters 7-12 in \textcite{bib:DummitFoote}.
    \begin{itemize}
        \item Mostly rings, a bit of modules.
        \begin{itemize}
            \item Modules tend to get more complicated.
        \end{itemize}
        \item The topics covered in class will all be in the book, but not necessarily in the same order.
        \item Some of Nori's definitions will be different from those used in the book.
        \begin{itemize}
            \item Different enough, in fact, to get us the wrong answers in PSet and Exam questions.
            \item We should use his, though.
            \item He diverges from the book because his is the mathematical literature standard.
            \item Three main differences: Definition of a ring, subring, and ring homomorphism.
        \end{itemize}
    \end{itemize}
    \item Homework will be due every Wednesday.
    \begin{itemize}
        \item The first will be due next week (on Wednesday, 1/11).
        \item Rings, subrings, and ring homomorphisms, only, are needed for the first HW.
    \end{itemize}
    \item Grading breakdown.
    \begin{itemize}
        \item HW (30\%).
        \item Midterm (30\%) --- third or fourth week.
        \item Final (40\%).
    \end{itemize}
    \item Office hours for Nori in Eckhart 310.
    \begin{itemize}
        \item M (3:00-4:30).
        \item Tu (3:30-5:00).
        \item Th (3:00-4:30).
    \end{itemize}
    \item Callum is our TA; Ray is for the other section. Their OH are TBA.
    \item All important course info will be in Files on Canvas.
    \item There will be course notes provided for the course.
    \item If we think something Nori writes down looks suspicious, feel free to ask!
    \item We now start the course content.
    \item \textbf{Ring}\footnote{Definition from \textcite{bib:DummitFoote}.}: A triple $(R,+,\times)$ comprising a set $R$ equipped with binary operations $+$ and $\times$ that satisfies the following three properties.
    \begin{enumerate}[label={(\roman*)}]
        \item $(R,+)$ is an abelian group.
        \item $(R,\times)$ is associative, i.e.,
        \begin{equation*}
            a\times(b\times c) = (a\times b)\times c
        \end{equation*}
        for all $a,b,c\in R$.
        \item The left and right distributive laws hold, i.e.,
        \begin{align*}
            a\times(b+c) &= (a\times b)+(a\times c)&
            (b+c)\times a &= (b\times a)+(c\times a)
        \end{align*}
        for all $a,b,c\in R$.
    \end{enumerate}
    \item Misc comments.
    \begin{itemize}
        \item The parentheses on the RHSs in (iii) indicate the "standard" order of operations.
        \item We still often drop the $\times$ in favor of $a\cdot b$ or simply $ab$.
        \item We haven't postulated multiplicative inverses. That makes things more tricky :)
    \end{itemize}
    \item We define left- and right-multiplication functions for every element $a\in R$.
    \begin{itemize}
        \item These are denoted $l_a:R\to R$ and $r_a:R\to R$. In particular,
        \begin{align*}
            l_a(b) &= a\times b&
            r_a(b) &= b\times a
        \end{align*}
        for all $b\in R$.
        \item The statement "$l_a,r_a$ are group homomorphisms\footnote{Since we will soon introduce other types of homomorphisms (e.g., ring homomorphisms) beyond the one type with which we are familiar, we now have to specify that a homomorphism of the type dealt with in MATH 25700 is a \emph{group} homomorphism.} from $(R,+)$ to itself, i.e.,
        \begin{equation*}
            l_a(b+c) = l_a(b)+l_a(c)
        \end{equation*}
        for all $b,c\in R$" is equivalent to (iii).
    \end{itemize}
    \item \textbf{Additive identity} (of $R$): The unique element of $R$ that satisfies the following constraint. \emph{Denoted by} $\bm{0_R}$.
    \begin{equation*}
        0_R+a = a+0_R = a
    \end{equation*}
    for all $a\in R$.
    \begin{itemize}
        \item The existence and uniqueness of $0_R$ follows from property (i) of rings (groups must have an identity element, which in this case is the \emph{additive} identity since it corresponds to the addition operation).
    \end{itemize}
    \item Similarly, we know that unique additive inverses exist for all $a\in R$. We denote these by $\bm{-a}$.
    \item Since $l_a$ is a group homomorphism, this must mean that
    \begin{align*}
        l_a(0_R) &= 0_R&
            l_a(-b) &= -l_a(b)\\
        a\times 0_R &= 0_R&
            a\times(-b) &= -(a\times b)
    \end{align*}
    for all $a,b\in R$.
    \begin{itemize}
        \item The same holds for $r_a$/positions interchanged.
        \item These are consequences of the distributive law.
    \end{itemize}
    \item In Part 1, \textcite{bib:DummitFoote} defines rings as above.
    \begin{itemize}
        \item In Part 2, \textcite{bib:DummitFoote} takes $R$ to be \textbf{commutative}.
        \item In Part 3, \textcite{bib:DummitFoote} takes $R$ to be a \textbf{ring with identity}.
    \end{itemize}
    \item \textbf{Commutative ring}: A ring $R$ such that
    \begin{equation*}
        a\times b = b\times a
    \end{equation*}
    for all $a,b\in R$.
    \item \textbf{Ring with identity}: A ring $R$ containing a 2-sided identity, i.e., an element $e\in R$ such that
    \begin{equation*}
        e\times a = a\times e = a
    \end{equation*}
    for all $a\in R$.
    \item We now justify that it's ok to denote the 2-sided identity with a single letter.
    \item Exercise: The identity is unique.
    \begin{proof}
        If $e'$ is also a 2-sided identity, then
        \begin{equation*}
            e = e\times e' = e'
        \end{equation*}
    \end{proof}
    \item In this course, we will always take "ring" to mean "ring with identity." That is, we will always assume that our rings contain a 2-sided identity $e=1_R$.
    \item Examples of rings.
    \begin{enumerate}
        \item $\N\subset\Z\subset\Q\subset\R\subset\C$ all have two binary operations, but are they all rings?
        \begin{itemize}
            \item $\N$ is not a ring since $(\N,+)$ is not an abelian group (or even a group --- no additive inverses).
            \item The rest are rings. In fact, they are commutative rings.
            \item $\Q,\R,\C$ are also \textbf{fields}.
        \end{itemize}
        \item Let $X$ be a set, and $f,g:X\to\R$. We can define $f+g:X\to\R$ by $(f+g)(x)=f(x)+g(x)$ and $f\times g:X\to\R$ by $(f\times g)(x)=f(x)g(x)$.
        \begin{itemize}
            \item Thus, the set of all functions from $X\to\R$ --- denoted $\Fun(X;\R)$ or $\R^X$ --- has two binary operations and is a ring.
            \item This follows from the fact that the real numbers form a ring.
        \end{itemize}
        \item More generally, let $X$ be a set and let $R$ be a ring. Then $\Fun(X;R)=R^X$ is a ring.
        \begin{itemize}
            \item The constant function taking the value $1_R\in R$ is the identity of $R^X$.
        \end{itemize}
        \item Let $X=\{1,2\}$. Then $R^X\cong R\times R$.
        \begin{itemize}
            \item Correct topology:
            \begin{align*}
                (a_1,a_2)+(b_1,b_2) &= (a_1+b_1,a_2+b_2)&
                (a_1,a_2)\times(b_1,b_2) &= (a_1\times b_1,a_2\times b_2)
            \end{align*}
            \item Implication: The same "formula" shows that if $R_1,R_2$ are rings, then $R_1\times R_2$ is a ring.
        \end{itemize}
        \item If $R_i$ is a ring for all $i\in I$, where $I$ could be any indexing set (e.g., $\N$, but need not be countable), then $\prod_{i\in I}R_i$ is also a ring.
        \begin{itemize}
            \item The identity is $(e_i,e_j,\dots)$.
        \end{itemize}
    \end{enumerate}
    \item \textbf{Field}: A commutative ring $R$ with multiplicative inverses for every element except $0_R$.
    \item In the context of groups, we've discussed subgroups, group homomorphisms, the fact that the inclusion of a subgroup into a bigger group is a group homomorphism, and the fact that the image of a group homomorphism is a subgroup.
    \item Today, let's define subrings and ring homomorphisms and make sure that the corresponding properties remain true.
    \item Intuitively, a \textbf{subring} should be a subset of a ring that is itself a ring under the restricted operations.
    \item \textbf{Subring}: A subset $S$ of a ring $R$ such that\dots
    \begin{enumerate}[label={(\roman*)}]
        \item For all $a,b\in S$, both $a+b,ab\in S$. For all $a\in S$, $-a\in S$.
        \item $1_R\in S$.
    \end{enumerate}
    \item Check that these conditions are sufficient!
    \item \textbf{Ring homomorphism}: A function $f:A\to B$, where $A,B$ are rings, such that
    \begin{align*}
        f(a_1+a_2) &= f(a_1)+f(a_2)\\
        f(a_1\times a_2) &= f(a_1)\times f(a_2)\\
        f(1_A) &= f(1_B)
    \end{align*}
    for all $a_1,a_2\in A$.
    \item Note that we need the third constraint because we are not postulating the existence of multiplicative inverses.
    \item Examples:
    \begin{enumerate}
        \item If $S$ is a subring of a ring $R$ and $i:S\to R$ is the inclusion map, then it is a ring homomorphism.
        \item $R_1,R_2$ are rings. Then $\pi:R_1\times R_2\to R_1$ defined by $\pi(a_1,a_2)=a_1$ for all $(a_1,a_2)\in R_1\times R_2$ is a ring homomorphism.
        \item $i:R_1\to R_1\times R_2$ defined by $i(a)=(a,0)$ is not a ring homomorphism unless $R_2$ is trivial since $i(1_{R_1})=(1_{R_1},0)\neq (1_{R_1},1_{R_2})=1_{R_1\times R_2}$.
        \item $f:M_2(\R)\to M_3(\R)$ defined by inclusion in the upper lefthand corner is not a ring homomorphism for the same reason as the above. To be clear, the functional relation considered here is
        \begin{equation*}
            \begin{pmatrix}
                a & b\\
                c & d\\
            \end{pmatrix}
            \mapsto
            \begin{pNiceArray}{cc|c}
                a & b & 0\\
                c & d & 0\\ \hline
                0 & 0 & 0\\
            \end{pNiceArray}
        \end{equation*}
    \end{enumerate}
    \item The integers have no subrings except for itself.
    \begin{itemize}
        \item Consider $\Z/10\Z$, for instance. Doesn't work because we postulate the existence of an identity, but $1\notin\Z/10\Z$.
    \end{itemize}
    \item Subrings of $\Q$:
    \begin{itemize}
        \item $\Z,\Q$, the $p$-adic rationals $\{a/p^n\mid a\in\Z,n=0,1,\dots\}$, $\{a/(p_1p_2\cdots p_r)^n\mid a\in\Z,n=0,1,\dots\}$, arbitrary subsets of primes in the denominator.
        \item Exercise: There's a bijective correspondence between the subrings of $\Q$ and the power set of the prime numbers.
    \end{itemize}
\end{itemize}



\section{Office Hours (Nori)}
\begin{itemize}
    \item \marginnote{1/5:}Is $\Z$ a commutative ring?
    \begin{itemize}
        \item Yes it is.
    \end{itemize}
    \item Can you clarify the statement of Problem 1.4?
    \begin{itemize}
        \item For any ring $R$, define a function $\Delta:R\to R\times R$ by
        \begin{equation*}
            \Delta(a) = (a,a)
        \end{equation*}
        \item Clearly $\Delta$ is a ring homomorphism.
        \item Then consider the image $\Delta(R)\subset R\times R$.
        \item We are asked to show that if $\Delta(\Q)\subset B\subset\Q\times\Q$ for $B$ a subring of $\Q\times\Q$, then either $B=\Delta(\Q)$ or $B=\Q\times\Q$.
    \end{itemize}
\end{itemize}



\section{Polynomial Rings and Power Series Rings}
\begin{itemize}
    \item \marginnote{1/6:}End of last time: The subrings of $\Q$.
    \item Today: The subrings an arbitrary ring $R$.
    \item Question 1: Let $R$ a ring, $x\in R$ arbitrary. What is the "smallest" subring $M\subset R$ such that $x\in M$?
    \begin{itemize}
        \item We know that $1_R\in M$. Thus, $1_R+1_R=2_R\in M$. It follows by induction that
        \begin{equation*}
            n_R \in M
        \end{equation*}
        for all $n\in\Z$.
        \item Moving on, $x\in M$ implies that $n_Rx,xn_R\in M$. Is it true that $n_Rx=xn_R$? Yes it is. Here's why.
        \begin{itemize}
            \item Let $C=\{c\in R\mid cx=xc\}$, where $x$ is the element we've been talking about.
            \item We can prove that $C$ is a subring of $R$; this is Exercise 7.1.9 of \textcite{bib:DummitFoote}.
            \item If $C$ is a subring, then $1_R\in C$ implies $1_R+1_R=2_R\in C$, implies $n_R\in C$. Therefore,
            \begin{equation*}
                n_Rx=xn_R \in M
            \end{equation*}
            for all $n\in\Z$.
        \end{itemize}
        \item The above and additive closure:
        \begin{equation*}
            \{a_R+b_Rx\mid a,b\in\Z\} \subset M
        \end{equation*}
        \item Multiplicative closure: $x\cdot x=x^2\in M$. Moreover, defining $x^n$ in the usual way (i.e., inductively),
        \begin{equation*}
            x^n \in M
        \end{equation*}
        for all $n\in\Zg$.
        \begin{itemize}
            \item To be explicit, the inductive definition of $x^n$ is $x^0=1_R$ and $x^{n+1}=x\cdot x^n$.
        \end{itemize}
        \item Multiplicative closure and $n_Ry=yn_R$ for $y\in R$ arbitrary (see above argument):
        \begin{equation*}
            a_Rx^n=xa_Rx^{n-1}=\cdots=x^na_R \in M
        \end{equation*}
        for all $a\in\Z$, $n\in\Zg$.
        \item Additive closure:
        \begin{equation*}
            (a_0)_R+(a_1)_Rx+\cdots+(a_n)_Rx^n \in M
        \end{equation*}
        for all $a_0,a_1,\dots,a_n\in\Z$ and $n\in\Zg$.
        \begin{itemize}
            \item Naturally, terms of this form are called \textbf{polynomials}.
            \item As the set of polynomials is at last closed under $+,\times$, $M$ must be a \textbf{polynomial ring}.
        \end{itemize}
    \end{itemize}
    \pagebreak
    \item \textbf{Polynomial ring} (over $\Z$): The ring defined as follows. \emph{Denoted by} $\bm{\pmb{\Z}[X]}$. \emph{Given by}
    \begin{equation*}
        \Z[X] = \bigcup_{m=0}^\infty\{a_0+a_1X+\cdots+a_mX^m\mid a_0,a_1,\dots,a_m\in\Z\}
    \end{equation*}
    \begin{itemize}
        \item Note that we \emph{insist} on using uppercase for the indeterminate. The motivation for doing so is illustrated by the next example.
    \end{itemize}
    \item $\Z[X]$ induces\footnote{Recall that the terminology "induce" means that to every $R'[X]$, we can assign a set of ring homomorphisms of the given form. In other words, the set of polynomial rings over rings $R'$ is in bijective correspondence with the set of collections of functions $\phi_x$.} a collection of ring homomorphisms $\phi_x:\Z[X]\to R$, one for every $R$ and $x\in R$. These are defined by
    \begin{equation*}
        \phi_x(f) = f(x)
    \end{equation*}
    where $f=a_0+a_1X+\cdots+a_mX^m$, $f(x)=(a_0)_R+(a_1)_Rx+\cdots+(a_m)_Rx^m$, and all $a_i\in\Z$.
    \item Implication.
    \begin{itemize}
        \item For any $R$ and any $x\in R$, $\phi_x(\Z[X])\subset R$.
        \item In layman's terms, the set of all polynomials of a single element of any ring is necessarily a subset of the ring overall.
    \end{itemize}
    \item Question 2: Let $R\subset B$ be rings, and let $x\in B$. Find the smallest subring $M\subset B$ such that $R\subset M$ and $x\in M$.
    \begin{itemize}
        \item Last time, we only knew that $1_R$ had to be in $M$. This time, we have a whole set of elements $R$ to choose from!
        \item Let $a\in R$ be arbitrary. We see that $a,x\in M$; this means that $ax,xa\in M$. But we may not have $ax=xa$ as we did so nicely for the integers $n_R$, so we have to postulate commutativity if we want to avoid a messy answer.
        \item Henceforth, we assume
        \begin{equation*}
            ax=xa \in M
        \end{equation*}
        for all $a\in R$.
        \item As in Question 1, $ax=xa$ implies
        \begin{equation*}
            ax^m=x^ma \in M
        \end{equation*}
        for all $a\in R$, $m\in\Zg$.
        \item Thus,
        \begin{equation*}
            a_0+\cdots+a_mx^m \in M
        \end{equation*}
        for $a_0,\dots,a_m\in R$, $m\in\Zg$.
        \item This set of polynomials is already a subring. Thus, it is not only contained in $M$, but must also equal $M$.
        \item Difference between this set of polynomials and the ones from Question 1: These are the polynomials with coefficients in $R\supset\Z$.
        \begin{itemize}
            \item Therefore, we need to define a broader type of polynomial ring.
        \end{itemize}
    \end{itemize}
    \item \textbf{Polynomial ring} (over $R$): The ring defined as follows. \emph{Denoted by} $\bm{R[X]}$. \emph{Given by}
    \begin{equation*}
        R[X] = \bigcup_{m=0}^\infty\{a_0+a_1X+\cdots+a_mX^m\mid a_0,a_1,\dots,a_m\in R\}
    \end{equation*}
    \begin{itemize}
        \item We do not require that $R$ is commutative.
        \item Note that $R[X]$ will be commutative, however, owing to the way it's defined.
    \end{itemize}
    \item We now seek to generalize polynomial rings to \textbf{power series rings}.
    \item To do so, we'll need to get more precise than the infinite unions we've been using.
    \begin{itemize}
        \item Consider the set of nonnegative integers $\Zg=\{0,1,2,\dots\}$.
        \begin{itemize}
            \item This is a \textbf{monoid} under both addition and multiplication.
        \end{itemize}
        \item Let $(R,+)$ be an abelian group.
        \item Then $(R^\Zg,+)$ is also an abelian group.
        \begin{itemize}
            \item As per last class, all elements $a\in(R^\Zg,+)$ are functions $a:\Zg\to R$.
            \item We write that $a:n\mapsto a_n$, i.e., the value of $a$ at $n$ will be denoted $a_n$, not $a(n)$.
        \end{itemize}
        \item Every element $a\in R^\Zg$ will be represented by $\sum_{n=0}^\infty a_nX^n$.
        \begin{itemize}
            \item This is allowable because there is a natural bijective correspondence between each $a$ and each power series $\sum_{n=0}^\infty a_nX^n$.
            \item Essentially, what we are doing here is using the rigorously defined set of functions $R^\Zg$ to theoretically stand in for the intuitive concept of a power series. This is acceptable since both objects have very similar properties, especially as pertains to adding and multiplying them.
            \item This is like defining the real numbers (intuitive) in terms of Dedekind cuts (rigorous).
            \item Note that alternatively, we could introduce the entire sequences/series analytical framework from Honors Calculus IBL to logically underpin power series, but this technique will be much less bulky and suit our purposes just fine.
        \end{itemize}
        \item We define addition and multiplication on $R^\Zg$ as follows.
        \begin{gather*}
            \left( \sum_{n=0}^\infty a_nX^n \right)+\left( \sum_{n=0}^\infty b_nX^n \right) = \sum_{n=0}^\infty(a_n+b_n)X^n\\
            \left( \sum_{p=0}^\infty a_pX^p \right)\left( \sum_{q=0}^\infty b_qX^q \right) = \sum_{\substack{p\geq 0,\\ q\geq 0}}a_pb_qX^{p+q}
                = \sum_{r=0}^\infty\left( \sum_{p=0}^ra_pb_{r-p} \right)X^r
        \end{gather*}
        \item This is the \textbf{power series ring}.
    \end{itemize}
    \item \textbf{Monoid}: A set equipped with an associative binary operation and an identity element.
    \item \textbf{Power series ring} (over $R$): The ring defined as follows, with $+,\times$ defined as above. \emph{Denoted by} $\bm{(R[[X]],+,\times)}$. \emph{Given by}
    \begin{equation*}
        R[[X]] = R^\Zg
    \end{equation*}
    \item Note that the definitions of addition and multiplication for $R[[X]]$ are precisely the ones needed for $R[X]$, too, (just the finite version) even though we didn't state them earlier.
    \item Two observations about power series rings which will also hold for polynomial rings.
    \begin{enumerate}
        \item $R$ is a subring of $R[[X]]$ with the inclusion ring homomorphism $a\mapsto a1+0X^1+0X^2+\cdots$.
        \item Additionally, we can map $X\in R$ to $0X^0+1X^1+0X^2+\cdots\in R[[X]]$.
    \end{enumerate}
    \item $aX=Xa$ for all $a\in R$.
    \begin{itemize}
        \item Why?? Ask in OH.
    \end{itemize}
    \item Alternate definition of $R[X]$: The subring of $R[[X]]$ given by
    \begin{equation*}
        R[X] = \left\{ \sum_{m=0}^\infty a_mX^m\in R[[X]] \middle| |\{m\in\Zg\mid a_m\neq 0\}|<\infty \right\}
    \end{equation*}
    \item Theorem (Universal Property of a Polynomial Ring): Let $R$ be a ring, $\alpha:R\to B$ a ring homomorphism, and $x\in B$. Assume that $x\cdot\alpha(a)=\alpha(a)\cdot x$ for all $a\in R$. Then there is a unique ring homomorphism $\beta:R[X]\to B$ such that $\beta(a)=\alpha(a)$ for all $a\in R$ and $\beta(X)=x$.
    \begin{proof}
        % Scratch: $\beta(X)=x$ implies $\beta(X^m)=x^m$ for all $m=0,1,\dots$. Taking $a_m\in R$, then $\beta(a_m)=\alpha(a_m)$ given $\beta(a_mX^m)=\beta(a_m)\beta(X^m)=\alpha(a_m)x^m$.\par
        % $\beta(a_0+\cdots+a_mX^m)=\sum_{i=0}^m\beta(a_iX^i)=\sum_{i=0}^m\alpha(a_i)x^i$.

        We first prove that such a ring homomorphism exists. Then we address uniqueness.\par
        Let $\beta(X)=x$. Then if $\beta$ is to be a ring homomorphism, we must have
        \begin{equation*}
            \beta(X^m) = x^m
        \end{equation*}
        for all $m\in\Zg$. We also require that $\beta(a_m)=\alpha(a_m)$ for all $a_m\in R$ (at this point, $a_m$ is just suggestive notation). Again, if $\beta$ is to be a ring homomorphism, it must follow that
        \begin{equation*}
            \beta(a_mX^m) = \beta(a_m)\beta(X^m)
            = \alpha(a_m)x^m
        \end{equation*}
        for all $a_m\in R$, $m\in\Z$. Lastly, if $\beta$ is to be a ring homomorphism, it must follow that
        \begin{equation*}
            \beta\left( \sum_{i=0}^ma_iX^i \right) = \sum_{i=0}^m\beta(a_iX^i)
            = \sum_{i=0}^m\alpha(a_i)x^i
        \end{equation*}
        But then by its construction, $\beta$ is defined on every element in $R[X]$ and is a ring homomorphism satisfying the desired properties.\par
        Suppose $\beta,\beta':R[X]\to B$ are ring homomorphisms satisfing $\beta(a)=\beta'(a)=\alpha(a)$ for all $a\in R$ and $\beta(X)=\beta'(X)=x$. Let $\sum_{i=0}^ma_iX^i\in R[X]$ be arbitrary. Then
        \begin{equation*}
            \beta\left( \sum_{i=0}^ma_iX^i \right) = \sum_{i=0}^m\alpha(a_i)x^i
            = \beta'\left( \sum_{i=0}^ma_iX^i \right)
        \end{equation*}
        as desired.
    \end{proof}
    \item The idea of the theorem.
    \begin{itemize}
        \item Evaluation of a function ($f\in R[X]$) at a point ($x\in B$): If $R\subset B$ and $\alpha(a)=a$ for all $a\in R$, then $\beta(f)=f(x)$.
        \item $\alpha$ is like a coordinate change function, allowing us to evaluate variants of each $f$.
        \item In fact, this idea is highly related to the linear algebra concept that specifying the action of a map on a basis specifies its action on all elements.
        \begin{itemize}
            \item However, here we are dealing with a \textbf{module homomorphism}, not a linear transformation.
        \end{itemize}
    \end{itemize}
\end{itemize}




\end{document}