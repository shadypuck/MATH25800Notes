\documentclass[../notes.tex]{subfiles}

\pagestyle{main}
\renewcommand{\chaptermark}[1]{\markboth{\chaptername\ \thechapter\ (#1)}{}}

\begin{document}




\chapter{???}
\section{Rings, Subrings, and Ring Homomorphisms}
\begin{itemize}
    \item \marginnote{1/4:}Intro to the course.
    \item What will be covered: Most of Chapters 7-12 in \textcite{bib:DummitFoote}.
    \begin{itemize}
        \item Mostly rings, a bit of modules.
        \begin{itemize}
            \item Modules tend to get more complicated.
        \end{itemize}
        \item The topics covered in class will all be in the book, but not necessarily in the same order.
        \item Some of Nori's definitions will be different from those used in the book.
        \begin{itemize}
            \item Different enough, in fact, to get us the wrong answers in PSet and Exam questions.
            \item We should use his, though.
            \item He diverges from the book because his is the mathematical literature standard.
            \item Three main differences: Definition of a ring, subring, and ring homomorphism.
        \end{itemize}
    \end{itemize}
    \item Homework will be due every Wednesday.
    \begin{itemize}
        \item The first will be due next week (on Wednesday, 1/11).
        \item Rings, subrings, and ring homomorphisms, only, are needed for the first HW.
    \end{itemize}
    \item Grading breakdown.
    \begin{itemize}
        \item HW (30\%).
        \item Midterm (30\%) --- third or fourth week.
        \item Final (40\%).
    \end{itemize}
    \item Office hours for Nori in Eckhart 310.
    \begin{itemize}
        \item M (3:00-4:30).
        \item Tu (3:30-5:00).
        \item Th (3:00-4:30).
    \end{itemize}
    \item Callum is our TA; Ray is for the other section. Their OH are TBA.
    \item All important course info will be in Files on Canvas.
    \item There will be course notes provided for the course.
    \item If we think something Nori writes down looks suspicious, feel free to ask!
    \item We now start the course content.
    \item \textbf{Ring}\footnote{Definition from \textcite{bib:DummitFoote}.}: A triple $(R,+,\times)$ comprising a set $R$ equipped with binary operations $+$ and $\times$ that satisfies the following three properties.
    \begin{enumerate}[label={(\roman*)}]
        \item $(R,+)$ is an abelian group.
        \item $(R,\times)$ is associative, i.e.,
        \begin{equation*}
            a\times(b\times c) = (a\times b)\times c
        \end{equation*}
        for all $a,b,c\in R$.
        \item The left and right distributive laws hold, i.e.,
        \begin{align*}
            a\times(b+c) &= (a\times b)+(a\times c)&
            (b+c)\times a &= (b\times a)+(c\times a)
        \end{align*}
        for all $a,b,c\in R$.
    \end{enumerate}
    \item Misc comments.
    \begin{itemize}
        \item The parentheses on the RHSs in (iii) indicate the "standard" order of operations.
        \item We still often drop the $\times$ in favor of $a\cdot b$ or simply $ab$.
        \item We haven't postulated multiplicative inverses. That makes things more tricky :)
    \end{itemize}
    \item We define left- and right-multiplication functions for every element $a\in R$.
    \begin{itemize}
        \item These are denoted $l_a:R\to R$ and $r_a:R\to R$. In particular,
        \begin{align*}
            l_a(b) &= a\times b&
            r_a(b) &= b\times a
        \end{align*}
        for all $b\in R$.
        \item The statement "$l_a,r_a$ are group homomorphisms\footnote{Since we will soon introduce other types of homomorphisms (e.g., ring homomorphisms) beyond the one type with which we are familiar, we now have to specify that a homomorphism of the type dealt with in MATH 25700 is a \emph{group} homomorphism.} from $(R,+)$ to itself, i.e.,
        \begin{equation*}
            l_a(b+c) = l_a(b)+l_a(c)
        \end{equation*}
        for all $b,c\in R$" is equivalent to (iii).
    \end{itemize}
    \item \textbf{Additive identity} (of $R$): The unique element of $R$ that satisfies the following constraint. \emph{Denoted by} $\bm{0_R}$.
    \begin{equation*}
        0_R+a = a+0_R = a
    \end{equation*}
    for all $a\in R$.
    \begin{itemize}
        \item The existence and uniqueness of $0_R$ follows from property (i) of rings (groups must have an identity element, which in this case is the \emph{additive} identity since it corresponds to the addition operation).
    \end{itemize}
    \item Similarly, we know that unique additive inverses exist for all $a\in R$. We denote these by $\bm{-a}$.
    \item Since $l_a$ is a group homomorphism, this must mean that
    \begin{align*}
        l_a(0_R) &= 0_R&
            l_a(-b) &= -l_a(b)\\
        a\times 0_R &= 0_R&
            a\times(-b) &= -(a\times b)
    \end{align*}
    for all $a,b\in R$.
    \begin{itemize}
        \item The same holds for $r_a$/positions interchanged.
        \item These are consequences of the distributive law.
    \end{itemize}
    \item In Part 1, \textcite{bib:DummitFoote} defines rings as above.
    \begin{itemize}
        \item In Part 2, \textcite{bib:DummitFoote} takes $R$ to be \textbf{commutative}.
        \item In Part 3, \textcite{bib:DummitFoote} takes $R$ to be a \textbf{ring with identity}.
    \end{itemize}
    \item \textbf{Commutative ring}: A ring $R$ such that
    \begin{equation*}
        a\times b = b\times a
    \end{equation*}
    for all $a,b\in R$.
    \item \textbf{Ring with identity}: A ring $R$ containing a 2-sided identity, i.e., an element $e\in R$ such that
    \begin{equation*}
        e\times a = a\times e = a
    \end{equation*}
    for all $a\in R$.
    \item We now justify that it's ok to denote the 2-sided identity with a single letter.
    \item Exercise: The identity is unique.
    \begin{proof}
        If $e'$ is also a 2-sided identity, then
        \begin{equation*}
            e = e\times e' = e'
        \end{equation*}
    \end{proof}
    \item In this course, we will always take "ring" to mean "ring with identity." That is, we will always assume that our rings contain a 2-sided identity $e=1_R$.
    \item Examples of rings.
    \begin{enumerate}
        \item $\N\subset\Z\subset\Q\subset\R\subset\C$ all have two binary operations, but are they all rings?
        \begin{itemize}
            \item $\N$ is not a ring since $(\N,+)$ is not an abelian group (or even a group --- no additive inverses).
            \item The rest are rings. In fact, they are commutative rings.
            \item $\Q,\R,\C$ are also \textbf{fields}.
        \end{itemize}
        \item Let $X$ be a set, and $f,g:X\to\R$. We can define $f+g:X\to\R$ by $(f+g)(x)=f(x)+g(x)$ and $f\times g:X\to\R$ by $(f\times g)(x)=f(x)g(x)$.
        \begin{itemize}
            \item Thus, the set of all functions from $X\to\R$ --- denoted $\Fun(X;\R)$ or $\R^X$ --- has two binary operations and is a ring.
            \item This follows from the fact that the real numbers form a ring.
        \end{itemize}
        \item More generally, let $X$ be a set and let $R$ be a ring. Then $\Fun(X;R)=R^X$ is a ring.
        \begin{itemize}
            \item The constant function taking the value $1_R\in R$ is the identity of $R^X$.
        \end{itemize}
        \item Let $X=\{1,2\}$. Then $R^X\cong R\times R$.
        \begin{itemize}
            \item Correct topology:
            \begin{align*}
                (a_1,a_2)+(b_1,b_2) &= (a_1+b_1,a_2+b_2)&
                (a_1,a_2)\times(b_1,b_2) &= (a_1\times b_1,a_2\times b_2)
            \end{align*}
            \item Implication: The same "formula" shows that if $R_1,R_2$ are rings, then $R_1\times R_2$ is a ring.
        \end{itemize}
        \item If $R_i$ is a ring for all $i\in I$, where $I$ could be any indexing set (e.g., $\N$, but need not be countable), then $\prod_{i\in I}R_i$ is also a ring.
        \begin{itemize}
            \item The identity is $(e_i,e_j,\dots)$.
        \end{itemize}
    \end{enumerate}
    \item \textbf{Field}: A commutative ring $R$ with multiplicative inverses for every element except $0_R$.
    \item In the context of groups, we've discussed subgroups, group homomorphisms, the fact that the inclusion of a subgroup into a bigger group is a group homomorphism, and the fact that the image of a group homomorphism is a subgroup.
    \item Today, let's define subrings and ring homomorphisms and make sure that the corresponding properties remain true.
    \item Intuitively, a \textbf{subring} should be a subset of a ring that is itself a ring under the restricted operations.
    \item \textbf{Subring}: A subset $S$ of a ring $R$ such that\dots
    \begin{enumerate}[label={(\roman*)}]
        \item For all $a,b\in S$, both $a+b,ab\in S$. For all $a\in S$, $-a\in S$.
        \item $1_R\in S$.
    \end{enumerate}
    \item Check that these conditions are sufficient!
    \item \textbf{Ring homomorphism}: A function $f:A\to B$, where $A,B$ are rings, such that
    \begin{align*}
        f(a_1+a_2) &= f(a_1)+f(a_2)\\
        f(a_1\times a_2) &= f(a_1)\times f(a_2)\\
        f(1_A) &= f(1_B)
    \end{align*}
    for all $a_1,a_2\in A$.
    \item Note that we need the third constraint because we are not postulating the existence of multiplicative inverses.
    \item Examples:
    \begin{enumerate}
        \item If $S$ is a subring of a ring $R$ and $i:S\to R$ is the inclusion map, then it is a ring homomorphism.
        \item $R_1,R_2$ are rings. Then $\pi:R_1\times R_2\to R_1$ defined by $\pi(a_1,a_2)=a_1$ for all $(a_1,a_2)\in R_1\times R_2$ is a ring homomorphism.
        \item $i:R_1\to R_1\times R_2$ defined by $i(a)=(a,0)$ is not a ring homomorphism unless $R_2$ is trivial since $i(1_{R_1})=(1_{R_1},0)\neq (1_{R_1},1_{R_2})=1_{R_1\times R_2}$.
        \item $f:M_2(\R)\to M_3(\R)$ defined by inclusion in the upper lefthand corner is not a ring homomorphism for the same reason as the above. To be clear, the functional relation considered here is
        \begin{equation*}
            \begin{pmatrix}
                a & b\\
                c & d\\
            \end{pmatrix}
            \mapsto
            \begin{pNiceArray}{cc|c}
                a & b & 0\\
                c & d & 0\\ \hline
                0 & 0 & 0\\
            \end{pNiceArray}
        \end{equation*}
    \end{enumerate}
    \item The integers have no subrings except for itself.
    \begin{itemize}
        \item Consider $\Z/10\Z$, for instance. Doesn't work because we postulate the existence of an identity, but $1\notin\Z/10\Z$.
    \end{itemize}
    \item Subrings of $\Q$:
    \begin{itemize}
        \item $\Z,\Q$, the $p$-adic rationals $\{a/p^n\mid a\in\Z,n=0,1,\dots\}$, $\{a/(p_1p_2\cdots p_r)^n\mid a\in\Z,n=0,1,\dots\}$, arbitrary subsets of primes in the denominator.
        \item Exercise: There's a bijective correspondence between the subrings of $\Q$ and the power set of the prime numbers.
    \end{itemize}
\end{itemize}




\end{document}